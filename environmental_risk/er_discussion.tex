In this pilot study, we found that maternal immune activation (MIA) triggers significant gene expression differences in the embryonic brain. 
These results also contribute Proof of Concept evidence that exposure to MIA in early gestation impacts upon genes implicated in neurodevelopmental conditions such as schizophrenia and autism. 

\subsection{MIA acts upon genes necessary for synaptic structure and function}
We discovered that GABA- and glutamate- related genes such as \gene{Gad1}, \gene{Gad2}, \gene{Gabra3}, \gene{Slc32a1} and \gene{Slc1a2} were significantly down-regulated following MIA; and most of the differentially expressed genes (DEGs) were functionally related to neurogenesis and neurotransmitter regulation.
Glutamate and GABA are known to play critical roles in neurogenesis, neuronal migration and synaptogenesis during development\cite{Cameron1998,LaMantia1995,LoTurco1995} and disruption of the GABAergic and glutamatergic systems leading to excitation/inhibition imbalance has been implicated in neurodevelopmental psychiatric disorders, such as schizophrenia\cite{Wassef2003} and autism\cite{Rubenstein2003}.

For example, the expression of \gene{Dlg4} (discs, large homolog 4), also known as postsynaptic density protein 95 (\gene{PSD95})\cite{DeBartolomeis2014}, was significantly lower in the PolyI:C-exposed fetal brains. 
\gene{PSD95} is a critical regulator of the neurexin-neuroligin-SHANK pathway implicated in autism spectrum disorders\cite{Feyder2010}.
\gene{PSD95} abnormalities are therefore thought to alter the balance of excitation and inhibition, and variations in this balance might change, not only local circuit function, but also connectivity patterns between brain regions, leading to developmental and behavioral deficits\cite{Cline2005}. 
Consistent with this, \gene{PSD95} have been associated with autism\cite{Risch1999} and schizophrenia\cite{Hahn2006}. 

\subsection{MIA acts on microglial/inflammatory genes}
The genes influenced by MIA included multiple microglia markers such as \gene{Cx3cr1} (chemokine receptor 1) and \gene{Csf1r} (colony stimulating factor 1 receptor) which were both down-regulated in the PolyI:C-exposed fetal brain. 
Microglia are the resident macrophages of the brain and the first and main form of active immune response within the central nervous system (CNS).
One of the main functions of microglia in brain development is synaptic pruning\cite{Paolicelli2011}, and thus deregulation of microglia might be expected to have a prolonged effect on brain development. 
Indeed, a recent study of the chemokine receptor \gene{Cx3cr1} knock out mouse found that a decrease in microglia was associated with a decrease in synaptic multiplicity and also autism-like behaviors such as increased repetitive behaviors and impaired social interactions\cite{Rogers2011}. 
\citet{Erblich2011} also reported enlarged lateral ventricles and reduced volume of olfactory area without changes of overall brain volume in the microglia marker \gene{Csf1r} knock-out mice, which is similar to our previous findings in the MIA mouse model\cite{Li2009c}. 
%If there isn't enough pages, we can elaborate here
Thus a similar \gene{Csf1r} deficit might be acting in both models of neurodevelopmental disorder.
	
\subsection{Overview}
Synaptic and microglial mechanisms are not independent and it is increasingly recognized that brain excitatory-inhibitory balance is a product of interplay between neuronal and glial systems\cite{Zhan2014}. 
Moreover, neurodevelopmental disorders are increasingly understood to be influenced by synaptic and/or neuroinflammatory pathology and this complexity likely maps to the heterogeneous nature of these conditions\cite{Garay2010}.

\subsection{Limitations}
%This has to be more detail analysis of the study. cannot be the same as the one in the paper
We first acknowledge that the sample size of current study is modest. 
That said, the effect sizes observed were large and multimodal measures acquired from the same animals were consistent with predictions. 
Second, we examined only male fetal brains in the current study. 
The decision to direct experimental resources to males was made because males with neurodevelopmental disorders such as autism out-number females; and there is evidence that the male fetus is more vulnerable to environmental exposures such as inflammation in prenatal life\cite{Bergeron2013}\cite{Lein2007}. 


