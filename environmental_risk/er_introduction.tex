Briefly say that environmental risk are interesting to study because one might prevent said environment to prevent the disorder. 
\subsection{Maternal Immune activation}
Environmental insults, such as infections during the prenatal period, have a negative impact on the normal course of fetal brain development.
The consequences of postnatal brain dysmaturation\cite{Meyer2007a} include an increased risk of neurodevelopmental conditions such as  schizophrenia and autism \cite{Fatemi2009a,Bale2010}. 
Prenatal infection and Maternal immune activation (MIA) can be modeled in the rodent; specifically the viral analogue polyriboinosinic-polyribocytidilic acid (PolyI:C) precipitates a brain and behavioral phenotype in rodent offspring which mirrors that observed in schizophrenia and related neurodevelopmental conditions\cite{Li2009c,Meyer2008b,Li2010a,Ashwood2006}.

Recent studies of global gene expression patterns in MIA-exposed rodent fetal brains\cite{Garbett2012,Oskvig2012} suggest that the post-pubertal onset of schizophrenic and other psychosis-related phenotypes might stem from attempts of the brain to counteract the environmental stress induced by MIA during its early development\cite{Garbett2012}.
To date, all these studies have focused on the changes elicited by a mid-to-late gestation exposure (e.g. Gestation Day (GD) 12.5 for mouse, or GD15 for rat). 
However, although we and others\cite{Meyer2007a,Li2009c,Li2010a} have reported that MIA early in gestation event might exert a more extensive impact on the phenotype of offspring, gene expression changes soon after early MIA have not been examined. 

This early time point was therefore the focus of the present pilot study. 
We exploited recent advances in transcriptome analysis facilitated by RNA Sequencing (RNA-Seq) to obtain a profile of all expressed transcripts in embryonic brains of mice exposed to early MIA - an approach considered to be more accurate and reliable compared to conventional microarrays\cite{Wang2009b}.
Also, as Proof of Concept, we examined whether the gene expression changes in the MIA model mapped to gene implicated in schizophrenia and autism in humans.

\subsubsection{Immediate effect of maternal immune activation to gene expression pattern in foetal brain}
Here we further discuss how we focus on the current study: The immediate effect of MIA and also the effect of MIA during early gestation. 
Can follow some of the logic from the paper

\subsection{RNA Sequencing}
Discuss the use of RNA Sequencing, comparing with micro-array. 
\subsubsection{Different dimension of the transcriptom}
Should discuss all possible dimention of the RNA Seq data. 
Didn't include miRNA as they are not captured.
Should also mention Alternative splicing, fusion genes and denovo assembly.
\paragraph{Differential Gene Expression}
\paragraph{Alternative Splicing}
\paragraph{Fusion Genes}
\paragraph{Denovo transcripts}
Then should mention the problem of the analysis of Alternative splicing, fusion genes and denovo assembly:
Require large amount of wet lab validation in order to make sense of the data.
Our study is only a pilot study, therefore focus only on the differential expression analysis. 
\subsection{Aim}

