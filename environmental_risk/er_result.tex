\subsection{Quality Control and Alignment}
Figure \ref{fig:schematicMIA} summarizes the overall study design and results obtained.
On average, 39 million reads were generated for each sample. 
Over 90$\%$ of reads were uniquely mapped to the \textit{Mus musculus} reference genome (mm10, Ensembl GRCm38.74) and the ERCC spike-in control reference using STAR\cite{Dobin2013}. 
Read counts of the External RNA Controls Consortium (ERCC) spike-in control were highly correlated with their true concentration (Pearson’s R $\le$ 0.956)(Table \ref{tab:ercc_correlation}), indicating that the RNA Seq read counts were representative of the true mRNA concentration.
\begin{figure}[htbp]
	\centering
	\includegraphics[width=0.9\textwidth]{environmental_risk/image/er_flowchart.png}
	\caption[Schematic of the experimental flow]{Schematic of the experimental flow. 
		Fetal mouse brains were extracted and used for RNA Sequencing. 
		DESeq2 were used to generate the differential expressed gene (DEG) list and also stabilize the variance of the gene expression counts. 
		Functional enrichment analysis were then performed on the DEG list using GOrilla, REViGO and the userListEnrichment function. 
		Stabilzed gene counts were then merged with the external microarray control using ComBat where the differential expression analysis and functional enrichment were again performed}\label{fig:schematicMIA}
\end{figure}
\begin{table}[h]
	\centering
	\caption[Correlation between concentration and counts]{Correlation between the true concentration and the normalized RNA Seq read counts.
		The high correlation indicates that the RNA Seq read counts are representative of the true concentration.}
	\label{tab:ercc_correlation}
	\begin{tabular}{rr}
		\toprule
		& \textbf{True Concentration} \\
		\midrule
		\textbf{Case 1} & 0.964 \\
		\textbf{Case 2} & 0.969 \\
		\textbf{Case 3} & 0.968 \\
		\textbf{Case 4} & 0.960 \\
		\textbf{Case 5} & 0.963 \\
		\textbf{Control 1} & 0.967 \\
		\textbf{Control 2} & 0.964 \\
		\textbf{Control 3} & 0.960 \\
		\textbf{Control 4} & 0.959 \\
		\textbf{Control 5} & 0.966 \\
		\bottomrule
	\end{tabular}%
\end{table}
\\
\\



\subsection{Differential gene expression analysis}
A total of 16,015 genes passed QC, of which 3,430 genes were differentially expressed (1,775 up-regulated and 1,655 down-regulated) with a Bonferroni-corrected p-value $<$ 0.05. 
A volcano plot of PolyI:C versus Saline for all genes is presented in Figure \ref{subfig1:deseqVoc}, with the significantly up- and down-regulated genes colored in red and blue, respectively. 
Many of our differentially expressed genes (DEGs) were within the list of schizophrenia-related candidate genes identified by the Psychiatric Genomics Consortium (PGC)\cite{Greenwood2011} previously (hyper geometric p-value $= 1.12\times 10^{-11}$) (Table \ref{tab:candidateGeneTable}), including \textit{Dlg4}\cite{Balan2013} and \textit{Disc1}\cite{Morris2003}.
Some DEGs were also known to be associated with autism (like \textit{Gabrb3}\cite{Fatemi2009}) or other brain diseases (like \textit{Chrnb2} in epilepsy\cite{Conti2015}).
\begin{figure}[!h]
	\centering     
	\subfloat[Using only RNA Seq data\label{subfig1:deseqVoc}]{%
		\includegraphics[width=0.45\textwidth]{environmental_risk/image/er_deseq_volcano.png}
	}
	\subfloat[Combining RNA Seq and microarray data\label{subfig2:combatVoc}]{%
		\includegraphics[width=0.45\textwidth]{environmental_risk/image/er_combat_volcano.png}
	}
	\caption[Volcano Plot]{Volcano plot of Case (PolyI:C) vs Control (Saline). 
		Genes that were significantly down-regulated are colored in blue whereas genes that were significantly up-regulated are colored in red. 
		a) Using only RNA-Seq data; 
		b) Using combination of RNA-Seq and microarray data.
	}
\end{figure}


\subsubsection{Gene Ontology(GO) enrichment}
A total of 28 GO terms were significantly overrepresented in the list of DEGs, with 10 and 18 GO terms represented by up- and down-regulated genes respectively, at a Bonferroni corrected p-value $<$ 0.05 (Table \ref{tab:geneOntologyFull}). 
Among the significant GO terms, a number of them, such as ``Developmental process'' (p-value $= 9.75\times 10^{-6}$), ``Regulation of neurogenesis'' (p-value $= 7.08\times 10^{-17}$) and ``Regulation of neurotransmitter levels'' (p-value $= 1.05\times 10^{-9}$) are related to brain function and development.

\subsubsection{Gene set enrichment}
Of the 149 gene sets tested, 27 were significantly enriched by the DEGs at a Bonferroni corrected p-value $<$ 0.05, including post-synaptic density protein and gene sets related to neurodevelopmental disorders, such as schizophrenia probable and autism associated modules (Table \ref{tab:fullGeneSetEnrichment}).

\subsection{Differential gene expression analysis with external controls}
We incorporated the external micro-array data as additional controls.
After quality control (QC), there are a total of 13,112 genes that can be found in both our RNA Seq data and the microarray chips.
Among the 13,112 genes, 868 were found to be differentially expressed (398 up-regulated and 470 down-regulated) with a Bonferroni-corrected p-value $<$ 0.05.
A volcano plot of PolyI:C versus Saline for all genes passing QC is presented in Figure \ref{subfig2:combatVoc}. 851 of the DEGs (98.0$\%$) overlapped with those discovered in our previous analysis without these external control data. 
Seven of the candidate schizophrenia genes previously identified by the PGC remained in our DEG list, including the GABA enzyme, \textit{Gad1} and \textit{Gad2}; \textit{Gabra3}, \textit{Ctnna2} and the dopamine-related gene, \textit{Th} (Table \ref{tab:candidateGeneTable}). 

\subsubsection{Gene Ontology enrichment}
The GO enrichment analysis found a total of 24 significantly enriched GO terms, of which 8 were enriched by the up-regulated genes and 16 were enriched by those down-regulated.
21 of them (87.5$\%$) overlapped with those found in the previous analysis without these external control data. ``Regulation of neurogenesis'' (enrichment p-value $= 4.94\times10^{-5}$) and ``Regulation of neurotransmitter levels'' (enrichment p-value $= 1.4\times10^{-7}$) were still among the enriched GO terms. 
\subsubsection{Gene set enrichment}
The gene-set enrichment analysis identified 5 enriched brain-related gene sets in the DEGs (Table \ref{tab:targetGeneSetEnrichment}), which were all identified in the previous analysis without these external control data. 
% Table generated by Excel2LaTeX from sheet 'Sheet1'
\afterpage{
\setlength\LTcapwidth{\textwidth}
\begin{longtable}{rrrrrr}
	\caption[Candidate Genes for Schizophrenia]{Candidate genes for schizophrenia. We compared the candidate genes generated from the Psychiatric Genomic Consortium \cite{Greenwood2011} with the differentially expressed genes list from our study. It was found that majority (hyper geometric p-value $= 1.12\times 10^{-11}$) of the candidate genes were differentially expressed upon exposed to early maternal immune activation. Seven of the candidate genes remained in our differentially expressed gene list even after including the external control. Here only those that were significantly differentiated were shown}\label{tab:candidateGeneTable}\\
    \toprule
    \multicolumn{1}{c}{\multirow{2}[4]{*}{\textbf{Gene Name}}} & \multicolumn{3}{r}{Without external controls} &       & With external controls* \\
    \multicolumn{1}{c}{} & baseMean & Log2 Fold Change & Bonferroni Corrected P &       & Bonferroni Corrected P \\
    \midrule
    \textit{Adra2a} & 132.01 & -1.41 & $1.32\times 10^{-9}$ &       & 0.269 \\
    \textit{Adrbk2} & 400.52 & -0.343 & $1.77\times 10^{-3}$ &       & 1 \\
    \textit{Akt1} & 8537.11 & 0.0816 & 0.0364 &       & 1 \\
    \textit{Aspm} & 4875.21 & -0.179 & 0.017 &       & 1 \\
    \textit{Cacng2} & 30.78 & -1.53 & $1.21\times 10^{-8}$ &       & 0.233 \\
    \textit{Camk2a} & 357.99 & -0.583 & $6.25\times 10^{-8}$ &       & 1 \\
    \textit{Chrna4} & 844.89 & -0.278 & 0.0456 &       & 1 \\
    \textit{Chrnb2} & 336.29 & -1.1  & $7.66\times 10^{-11}$ &       & 0.0122 \\
    \textit{Crhr2} & 15.57 & 0.844 & 0.0146 &       & 1 \\
    \textit{Ctnna2} & 825.55 & -0.732 & $1.23\times 10^{-8}$ &       & 0.649 \\
    \textit{Dbh} & 31.55 & -2.49 & $5.15\times 10^{-15}$ &       & 0.0768 \\
    \textit{Dgcr2} & 3710.98 & 0.228 & $1.22\times 10^{-5}$ &       & 1 \\
    \textit{Disc1} & 54.78 & -0.58 & $1.05\times 10^{-3}$ &       & NA \\
    \textit{Dlg4} & 1626.55 & -0.353 & $2.45\times 10^{-11}$ &       & 0.0596 \\
    \textit{Drd2} & 12.38 & -0.974 & $7.79\times 10^{-3}$ &       & 1 \\
    \textit{Dtnbp1} & 1088.18 & 0.268 & $5.70\times 10^{-5}$ &       & 1 \\
    \textit{Ebf2} & 1003.45 & -0.451 & 0.025 &       & 1 \\
    \textit{Eea1} & 983.87 & -0.245 & $2.08\times 10^{-3}$ &       & 1 \\
    \textit{Erbb4} & 291.96 & -0.836 & $1.12\times 10^{-7}$ &       & 0.0912 \\
    \textit{Fez1} & 2783.05 & -0.581 & $4.68\times 10^{-11}$ &       & 1 \\
    \textit{Gabra3} & 204.37 & -0.634 & $1.67\times 10^{-8}$ &       & 0.00294 \\
    \textit{Gabrb2} & 150.74 & -1.137 & $4.73\times 10^{-17}$ &       & 0.0618 \\
    \textit{Gad1} & 80.15 & -2.96 & $2.79\times 10^{-19}$ &       & 0.00127 \\
    \textit{Gad2} & 183.28 & -1.66 & $1.25\times 10^{-13}$ &       & 0.0398 \\
    \textit{Grid2} & 15.99 & -0.933 & $2.76\times 10^{-3}$ &       & 1 \\
    \textit{Grik4} & 92.8  & -0.675 & $8.44\times 10^{-6}$ &       & 1 \\
    \textit{Grin3a} & 49.81 & -0.524 & $8.88\times 10^{-3}$ &       & 1 \\
    \textit{Grm2} & 250.43 & -0.673 & $3.66\times 10^{-4}$ &       & 1 \\
    \textit{Grm3} & 32.98 & -0.822 & $1.22\times 10^{-3}$ &       & 1 \\
    \textit{Grm4} & 164.28 & -0.659 & $7.32\times 10^{-5}$ &       & 1 \\
    \textit{Gsk3b} & 5068.93 & -0.197 & $6.37\times 10^{-4}$ &       & 1 \\
    \textit{Htr1b} & 121.93 & 0.448 & $9.74\times 10^{-4}$ &       & 1 \\
    \textit{Kcnh2} & 835.89 & -0.18 & 0.016 &       & 1 \\
    \textit{Ncam1} & 728.17 & -0.832 & $1.57\times 10^{-7}$ &       & 1 \\
    \textit{Nde1} & 6430.74 & 0.107 & $1.81\times 10^{-3}$ &       & 1 \\
    \textit{Ndel1} & 1195.39 & 0.114 & 0.0293 &       & 1 \\
    \textit{Nos1ap} & 109.8 & -0.579 & $3.86\times 10^{-5}$ &       & 1 \\
    \textit{Notch4} & 532.12 & -0.387 & $1.38\times 10^{-5}$ &       & 1 \\
    \textit{Ppp3cc} & 120.61 & -0.485 & $4.38\times 10^{-5}$ &       & 1 \\
    \textit{Prodh} & 550.47 & 0.699 & $7.59\times 10^{-13}$ &       & 0.807 \\
    \textit{Rgs4} & 164.1 & -0.958 & 0.0127 &       & 1 \\
    \textit{Slc1a2} & 777.36 & -1.28 & $4.82\times 10^{-21}$ &       & $6.92\times 10^{-3}$ \\
    \textit{Slc32a1} & 44.4  & -4.56 & $1.03\times 10^{-27}$ &       & $1.04\times 10^{-3}$ \\
    \textit{Slc6a1} & 126.37 & -1.88 & $3.32\times 10^{-18}$ &       & 0.341 \\
    \textit{Slc6a4} & 15.86 & -1.12 & $5.41\times 10^{-3}$ &       & 1 \\
    \textit{Sp4} & 910.49 & -0.407 & $4.15\times 10^{-11}$ &       & 0.063 \\
    \textit{Th} & 87.29 & -2.16 & $1.34\times 10^{-16}$ &       & $6.59\times 10^{-3}$ \\
    \textit{Ywhae} & 36516.86 & 0.258 & $1.32\times 10^{-6}$ &       & 1 \\
    \bottomrule
    \multicolumn{6}{l}{* Genes not found in the external control data were marked with NA} \\
    
\end{longtable}%
}

% Table generated by Excel2LaTeX from sheet 'Sheet2'
\afterpage{
\begin{landscape}
\setlength\LTcapwidth{\textwidth}
\begin{longtable}{cp{3cm}cccccp{5cm}}
  \caption[GO enrichment Result]{\gls{go} enrichment Result. GOrilla and REViGO were used to perform \gls{go} enrichment analysis. Similar GO terms were clustered together and represented by a single representative GO terms. Here, only the representative GO terms were shown. }\label{tab:geneOntologyFull}\\
    \toprule
    \multicolumn{1}{c}{} & \multicolumn{3}{r}{\textit{\textbf{RNA-seq gene level}}} &  & \multicolumn{3}{r}{\textit{\textbf{Microarray and RNA-seq combined}}} \\
    \midrule
    \multicolumn{1}{c}{} & \multicolumn{3}{r}{\textit{\textbf{GO functional enrichment}}} & \textit{\textbf{}} & \multicolumn{3}{r}{\textit{\textbf{GO functional enrichment}}} \\
    \textit{\textbf{GO ID}} & \textit{\textbf{GO term}} & \textit{\textbf{P-value}} & \textit{\textbf{Direction}} & \textit{\textbf{}} & \textit{\textbf{Minimum P-value}} & \textit{\textbf{Direction}} & \textit{\textbf{Represented by}} \\
    GO:0007610 & behavior & $9.67\times 10^{-19}$ & Down  &       & $6.76\times 10^{-8}$ & Down & defense response \\
    GO:0022610 & biological adhesion & $4.65\times 10^{-10}$ & Down  &       & $8.66\times 10^{-5}$ & Down  &  \\
    GO:0065007 & biological regulation & $2.15\times 10^{-18}$ & Down  &       & $1.97\times 10^{-8}$ & Down  &  \\
    GO:0001775 & cell activation & $5.56\times 10^{-7}$ & Down  &       & $4.96\times 10^{-10}$ & Down  &  \\
    GO:0007155 & cell adhesion & $6.12\times 10^{-10}$ & Down  &       & $7.97\times 10^{-5}$ & Down  &  \\
    GO:0007049 & cell cycle & $9.99\times 10^{-3}$ & Up    &       & -     & -     &  \\
    GO:0008283 & cell proliferation & $5.53\times 10^{-3}$ & Up    &       & $1.83\times 10^{-3}$ & Up    &  \\
    GO:0008037 & cell recognition & $6.05\times 10^{-3}$ & Down  &       & $5.28\times 10^{-3}$ & Down  &  \\
    GO:0007267 & cell-cell signaling & $5.9\times 10^{-6}$ & Down  &       & -     & -     &  \\
    GO:0006928 & cellular component movement & $1.40\times 10^{-7}$ & Down  &       & $1.74\times 10^{-5}$ & Down  &  \\
    GO:0071840 & cellular component organization or biogenesis & $4.01\times 10^{-6}$ & Up    &       & -     & -     &  \\
    GO:0009987 & cellular process & $9.23\times 10^{-13}$ & Up    &       & $1.57\times 10^{-3}$ & Up    &  \\
    GO:0006974 & cellular response to DNA damage stimulus & $4.76\times 10^{-3}$ & Up    &       & -     & -     &  \\
    GO:0032502 & developmental process & $1.23\times 10^{-5}$ & Down  &       & -     & -     &  \\
    GO:0002376 & immune system process & $1.16\times 10^{-8}$ & Down  &       & $4.39\times 10^{-11}$ & Down  &  \\
    GO:0051703 & intraspecies interaction between organisms & $1.48\times 10^{-3}$ & Down  &       & -     & -     &  \\
    GO:0040011 & locomotion & $2.10\times 10^{-9}$ & Down  &       & $7.23\times 10^{-6}$ & Down  &  \\
    GO:1903047 & mitotic cell cycle process & $1.94\times 10^{-3}$ & Up    &       & -     & -     &  \\
    GO:0032501 & multicellular organismal process & $1.61\times 10^{-11}$ & Down  &       & -     & -     &  \\
    GO:0051704 & multi-organism process & $6.29\times 10^{-3}$ & Down  &       & -     & -     &  \\
    GO:0001890 & placenta development & $9.63\times 10^{-5}$ & Up    &       & $4.01\times 10^{-5}$ & Up    & osteoblast differentiation \\
    GO:0042391 & regulation of membrane potential & $1.79\times 10^{-9}$ & Down  &       & $1.40\times 10^{-7}$ & Down  & regulation of neurotransmitter levels \\
    GO:0050767 & regulation of neurogenesis & $3.54\times 10^{-17}$ & Down  &       & $6.82\times 10^{-11}$ & Down  & cell projection organization, cognition, immune effector process, regulation of multicellular organismal process \\
    \multicolumn{1}{c}{GO:0022613} & ribonucleoprotein complex biogenesis & \multicolumn{1}{c}{$1.02\times 10^{-27}$} & \multicolumn{1}{c}{Up} & \multicolumn{1}{c}{} & \multicolumn{1}{c}{$6.37\times 10^{-13}$} & \multicolumn{1}{c}{Up} &  \\
    \multicolumn{1}{c}{GO:0006396} & RNA processing & \multicolumn{1}{c}{$2.25\times 10^{-50}$} & \multicolumn{1}{c}{Up} & \multicolumn{1}{c}{} & \multicolumn{1}{c}{$5.18\times 10^{-15}$} & \multicolumn{1}{c}{Up} & translation \\
    \multicolumn{1}{c}{GO:0050658} & RNA transport & \multicolumn{1}{c}{$3.21\times 10^{-10}$} & \multicolumn{1}{c}{Up} & \multicolumn{1}{c}{} & \multicolumn{1}{c}{$1.17\times 10^{-3}$} & \multicolumn{1}{c}{Up} &  \\
    \multicolumn{1}{c}{GO:0023052} & signaling & \multicolumn{1}{c}{$4.36\times 10^{-9}$} & \multicolumn{1}{c}{Down} & \multicolumn{1}{c}{} & \multicolumn{1}{c}{-} & \multicolumn{1}{c}{-} &  \\
    \multicolumn{1}{c}{GO:0071353} & cellular response to interleukin-4 & \multicolumn{1}{c}{-} & \multicolumn{1}{c}{-} & \multicolumn{1}{c}{} & \multicolumn{1}{c}{$2.20\times 10^{-5}$} & \multicolumn{1}{c}{Up} &  \\
    \multicolumn{1}{c}{GO:0000082} & G1/S transition of mitotic cell cycle & \multicolumn{1}{c}{-} & \multicolumn{1}{c}{-} & \multicolumn{1}{c}{} & \multicolumn{1}{c}{$6.14\times 10^{-3}$} & \multicolumn{1}{c}{Up} &  \\
    \multicolumn{1}{c}{GO:0030001} & metal ion transport & \multicolumn{1}{c}{-} & \multicolumn{1}{c}{-} & \multicolumn{1}{c}{} & \multicolumn{1}{c}{$1.25\times 10^{-7}$} & \multicolumn{1}{c}{Down} &  \\
    \multicolumn{1}{c}{GO:0050896} & response to stimulus & \multicolumn{1}{c}{-} & \multicolumn{1}{c}{-} & \multicolumn{1}{c}{} & \multicolumn{1}{c}{$9.89\times 10^{-7}$} & \multicolumn{1}{c}{Down} &  \\
    \bottomrule
    \end{longtable}%
    
\end{landscape}
}
% Table generated by Excel2LaTeX from sheet 'Sheet3'
\begin{landscape}
\begin{table}[htbp]
  \centering
  \caption{Gene set enrichment results of differential expressed genes combining microarray and RNA Seq data.
  Enrichment results mapping to \textit{de-novo} mutations and common variants are also shown.
  It was noted that the Post-synaptic density protein (PSD) gene sets were enriched with \textit{de-novo} mutations from 3 out of 4 studies and only the microglia gene set was enriched by the common variants observed in the schizophrenia Genome-wide association study (GWAS) conducted by the Psychiatric Genomic Consortium\cite{Ripke2013}.}
    \begin{tabular}{cccccccc}
    \multicolumn{1}{c}{\multirow{3}[3]{*}{\textbf{Gene Set}}} & \multicolumn{1}{c}{\multirow{3}[3]{*}{\textbf{RNA Seq}}} & \textbf{Denovo } & \textbf{} & \textbf{} & \textbf{} & \textbf{GWAS} & \textbf{} \\
    \cline{3-8}
    
    \multicolumn{1}{c}{} & \multicolumn{1}{c}{} & 
    \multicolumn{1}{p{2.3cm}}{\multirow{2}[2]{*}{\textbf{Fromer et al.}}} & \multicolumn{1}{p{2cm}}{\multirow{2}[2]{*}{\textbf{Neale et al.}}} & \multicolumn{1}{p{2.4cm}}{\multirow{2}[2]{*}{\textbf{Sanders et al.}}} & \multicolumn{1}{p{2.3cm}}{\multirow{2}[2]{*}{\textbf{O'Roak et al.}}} & \multicolumn{1}{p{2.3cm}}{\multirow{2}[2]{*}{\textbf{Anney et al.}}} & \multicolumn{1}{p{2.3cm}}{\multirow{2}[2]{*}{\textbf{Ripke et al.}}} \\[0.25cm]
    \multicolumn{1}{c}{} & \multicolumn{1}{c}{} & \multicolumn{1}{c}{\textbf{Scz\cite{Fromer2014}}} & \multicolumn{1}{c}{\textbf{ASD\cite{Neale2012}}} & \multicolumn{1}{c}{\textbf{ASD\cite{Sanders2012}}} & \multicolumn{1}{c}{\textbf{ASD\cite{ORoak2012}}} & \multicolumn{1}{c}{\textbf{ASD\cite{Anney2010a}}} & \multicolumn{1}{c}{\textbf{Scz\cite{Ripke2013}}} \\
    \midrule
    Neuron definite\cite{Cahoy2008} & $2.66\times 10^{-4}$ & 0.358 & 0.381 & 1     & $3.83\times10^{-3}$ & 0.614 & 0.188 \\
    Neuron probable\cite{Cahoy2008} & $5.77\times10^{-3}$ & $7.14\times10^{-9}$ & 0.373 & 1     & $2.51\times10^{-7}$ & 0.763 & 0.323 \\
    Microglia (Type1)\cite{Miller2010} & $3.17\times10^{-8}$ & 1     & 1     & 1     & 1     & 0.938 & $9.64\times10^{-3}$ \\
    PSD proteins\cite{Bayes2011} & $3.54\times10^{-5}$ & $3.07\times10^{-10}$ & 0.0732 & $2.63\times10^{-3}$ & $3.35\times10^{-4}$ & 0.461 & 0.548 \\
    Ribosome\cite{Miller2010} & $9.41\times10^{-4}$ & 1     & 1     & 1     & 1     & 0.489 & 0.317 \\
    \bottomrule
    \end{tabular}%
  \label{tab:targetGeneSetEnrichment}%
\end{table}%
\end{landscape}


\begin{figure}
	\caption[rtPCR Results]{real time PCR validation of RNA Seq results.
		The delta CT values are highly correlated with the RNA-Seq counts (case: pearson correlation = -0.968, control: pearson correlation = 0.952), suggesting that the RNA-Seq count is a true representation of the RNA concentration.
		The difference in detla CT between the cases and controls also support the differential gene expression results of the RNA Sequencing analysis.}\label{fig:rtResult}
		\vspace{-20pt}
	\begin{center}
		\includegraphics[trim=0cm 2cm 0cm 1cm, clip=true, width=0.7\textwidth]{environmental_risk/image/er_rtPCR.png}
	\end{center}
\end{figure}

\subsection{Test for burden of genetic risk variants in enriched brain-related gene sets in schizophrenia and autism cohorts }
Among the 27 enriched brain-related gene sets identified in the analysis without external control (Table \ref{tab:fullGeneSetEnrichment}), three (``Post-synaptic Density Proteins'', ``Neuron Probable'' and ``Down with Alzheimer's'') were enriched with rare \textit{de-novo} mutations in more than one study. 
In addition, the ``Oligodendrocyte Marker'' gene set was enriched with both rare \textit{de-novo} mutations and common genetic variants in schizophrenia studies. 

When combined with external control, only 5 gene sets remained \ref{tab:targetGeneSetEnrichment}. 
Of the 5 remaining gene sets, only the microglia gene set from \citet{Miller2010} was found to contain significantly more common genetic variations associated with schizophrenia in the PGC genome wide association study (GWAS).
Whereas the significant ``Neuron Probable'' and ``Post-synaptic Density Proteins'' were among the 5 remaining gene sets.

\subsection{RT-PCR validation of RNA-Seq data}
The CT value is the unit used during rtPCR to indicates the mRNA concentration.
A high CT value means more PCR cycle is required to detect the mRNA, therefore indicating a low mRNA concentration.
Thus, a high CT value represent a low mRNA concentration and a low CT value represent a high mRNA concentration, as oppose to the RNA Seq count value.

In the rtPCR, the CT values were highly negatively correlated with the RNA-Seq (Case: r  = -0.968, p = $2.98\times10^{-15}$; Control: r  = -0.952, p= $2.834\times10^{-13}$), suggesting that the RNA-Seq counts truly reflected the true mRNA concentration (Figure \ref{fig:rtResult}). 
Simple T-test performed on the relative CT values confirmed that all genes tested were significantly differentially expressed (Figure \ref{fig:rtResult}).


	
\subsection{Alternative splicing}
\subsection{De-novo Assembly}
