	\begin{figure}
			\centering
			\subfloat[SHREK]{
				\scalebox{.4}{\includegraphics{figure/he_summary/cc_500c/shrek_CC_Random_mean.png}}
				\label{fig:shrekCC500RandMean}
			}
			\subfloat[GCTA]{
				\scalebox{.4}{\includegraphics{figure/he_summary/cc_500c/gcta_CC_Random_mean.png}}
				\label{fig:gctaCC500RandMean}
			}\\
			\subfloat[LDSC with fix intercept]{
				\scalebox{.4}{\includegraphics{figure/he_summary/cc_500c/ldsc_CC_Random_mean.png}}
				\label{fig:ldscCC500RandMean}
			}
			\subfloat[LDSC with intercept estimation]{
				
				\scalebox{.4}{\includegraphics{figure/he_summary/cc_500c/ldscIn_CC_Random_mean.png}}
				\label{fig:ldscInCC500RandMean}
			}
			\caption[Case Control with Random Effect Size Simulation Result(Mean)]
			{Mean of results from case control simulation with random effect size simulation.
				The performance of \gls{gcta} was as suggested by \citet{Golan2014} where there was an underestimation as prevalence decreases.
				On the other hand, \gls{ldsc} were upwardly biased when a fixed intercept was used and this bias was corrected when an estimation of intercept was allowed.
				\gls{shrek} does not seems to as sensitive to change in prevalence and the estimation were relatively robust.
				} 
			\label{fig:CC500RandMean}
		\end{figure}
		
		\begin{figure}
			\centering
			\subfloat[SHREK]{
				\scalebox{.4}{\includegraphics{figure/he_summary/cc_500c/shrek_CC_Random_sd.png}}
				\label{fig:shrekCC500RandVar}
			}
			\subfloat[GCTA]{
				\scalebox{.4}{\includegraphics{figure/he_summary/cc_500c/gcta_CC_Random_sd.png}}
				\label{fig:gctaCC500RandVar}
			}\\
			\subfloat[LDSC with fix intercept]{
				\scalebox{.4}{\includegraphics{figure/he_summary/cc_500c/ldsc_CC_Random_sd.png}}
				\label{fig:ldscCC500RandVar}
			}
			\subfloat[LDSC with intercept estimation]{
				
				\scalebox{.4}{\includegraphics{figure/he_summary/cc_500c/ldscIn_CC_Random_sd.png}}
				\label{fig:ldscInCC500RandVar}
			}
			\caption[Case Control with Random Effect Size Simulation Result(Variance)]
			{Variance of results from case control simulation with random effect size simulation.
				It was clear that the prevalence affects the variance of estimation where a larger variance tends to increase the variance of estimation.
				Again, \gls{gcta} has the lowest variance, however, unlike in the quantitative trait simulation, \gls{shrek} has a lower average variance when compared to \gls{ldsc} with fixed intercept.
				Nonetheless, it was important to remember that in case control simulation, a much smaller amount of \glspl{SNP} was used, thus the results was not directly comparable to results from the quantitative simulation.
			} 
			\label{fig:CC500RandVar}
		\end{figure}
		
		
		\begin{figure}
			\centering
			\subfloat[SHREK]{
				\scalebox{.4}{\includegraphics{figure/he_summary/cc_500c/shrek_CC_Random_sdCom.png}}
				\label{fig:shrekCC500RandVarCom}
			}
			\subfloat[GCTA]{
				\scalebox{.4}{\includegraphics{figure/he_summary/cc_500c/gcta_CC_Random_sdCom.png}}
				\label{fig:gctaCC500RandVarCom}
			}\\
			\subfloat[LDSC with fix intercept]{
				\scalebox{.4}{\includegraphics{figure/he_summary/cc_500c/ldsc_CC_Random_sdCom.png}}
				\label{fig:ldscCC500RandVarCom}
			}
			\subfloat[LDSC with intercept estimation]{
				
				\scalebox{.4}{\includegraphics{figure/he_summary/cc_500c/ldscIn_CC_Random_sdCom.png}}
				\label{fig:ldscInCC500RandVarCom}
			}
			\caption[Case Control with Random Effect Size Simulation Result(Estimated Variance)]
			{Estimated variance of results from case control simulation with random effect size simulation when compared to empirical variance.
				From the quantitative trait simulation with random effect size(\cref{fig:QtRandVarCom}), it was observed that the variance estimation of \gls{shrek} and \gls{gcta} were rater accurate.
				Similarly, in the case control simulation with 100 causal \glspl{SNP}, it was observed that the variance estimation of \gls{shrek} and \gls{gcta} were close to the empirical variance with slight bias.
				A large up-ward bias was observed for \gls{ldsc} with fixed intercept estimation but the bias was less when \gls{ldsc} was allowed to estimate the intercept.s
			} 
			\label{fig:CC500RandVarCom}
		\end{figure}