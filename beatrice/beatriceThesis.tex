\documentclass[12pt]{report}
\usepackage{amsmath}
\usepackage{bm}
\usepackage{colortbl}
\usepackage{graphicx}
\usepackage{fullpage}
\usepackage{multirow}
\usepackage{setspace}
\usepackage{booktabs}
\usepackage{gensymb}
\usepackage{xr}
\usepackage[inline]{enumitem}
\usepackage[natbib=true, style=numeric-comp,subentry,backend=biber,sorting=none]{biblatex}
\addbibresource{C:/Users/swchoi/Desktop/thesis/citation/Thesis.bib}
\usepackage[hidelinks]{hyperref}
\usepackage{fancyhdr}
\pagestyle{fancy}
\fancyhead[LE,RO]{\itshape \nouppercase \rightmark}
\fancyhead[LO,RE]{\itshape \nouppercase Chapter \arabic{chapter}}
\setlength{\headheight}{40.0pt}
\setlength{\headsep}{0.2in}
\addtolength{\topmargin}{-4\baselineskip}
\title{Title}
\date{\today}
\author{Wu Hei Man}
\newcommand{\gene}[1]{\textit{#1}}
\renewcommand*\contentsname{Table of Content}
\begin{document}
	\maketitle
	\chapter*{Declaration}
	\chapter*{Acknowledgements}
	\tableofcontents
	
	\doublespacing
	
	\printbibliography
	\chapter{Literature Review I}
	\chapter{Literature Review II}
		\section{Antipsychotic pharmacogenomics}
		\subsection{Background}
			Given there is no cure for schizophrenia, antipsychotic medication is the mainstay for management of schizophrenia.
			To date, antipsychotic drugs remain to be the only effective therapeutic agent for treatment of schizophrenia. 
			Symptom reduction and relapse prevention are the ultimate goal of antipsychotic medication for treatment of schizophrenia. 
			Despite antipsychotics are consistently shown to be more effective than placebo, there is large variability between individuals in their response to antipsychotic in terms of drug efficacy and occurrence of adverse effects to the medication. 
			The factors that influence the variation in response to antipsychotic treatment have not been well-elucidated, making it difficult to develop diagnostic tests to provide the most effective and safe treatment tailored to individual patients. 
			Given several case series of monozygotic twins have reported concordant response to second-generation antipsychotics, these suggest a possible role of genetic variation underlying individual differences in response to antipsychotic treatment.  
		\section{Antipsychotic medications}
		\subsection{First-generation antipsychotics}
			Prior to the development of antipsychotic medications, the majority of patients with schizophrenia were hospitalized because the symptoms experienced were severe and prevent them to live a productive and independent life in the community.
			The first antipsychotic drug –chlorpromazine was introduced in 1951[1].
			In that decade, a number of first generation antipsychotics (FGAs), also known as “typical antipsychotics” were developed.
			FGAs are effective in reducing the positive symptoms such as hallucinations, delusion, disorganized speech and behavior, but ineffective in treating negative symptoms[2].
			In the 1970s, chlorpromazine became the most commonly used antipsychotic drug and it was reported that 50$\%$ of patients with schizophrenia demonstrated significant clinical improvement [3].
			However, one of the common adverse drug effects associated with use of FGAs is extrapyramidal motor symptoms (EPS), which include pseudoparkinsonism, akathisia, acute dystonia, and tardive dyskinesia[4].
			Nonetheless, the introduction of first generation antipsychotics has been described as one of the greatest revolutions in the care and treatment of schizophrenia in psychiatric field[5, 6]. 
		\subsection{Second-generation antipsychotics}
			The search for antipsychotic drug that are effective in treating both positive and negative symptoms led to the emergence of second-generation antipsychotics(SGAs) in the 1980s. 
			Clozapine was the first introduced SGAs, which is often referred to as the gold standard for treatment-refractory schizophrenia. 
			Clozapine has a high affinity for dopamine D4 receptors and serotonin 5-HT2A receptor antagonism and relatively modest affinity for DRD2 [7]. 
			Clozapine is effective not only in treating positive symptoms, but also negative and cognitive symptoms[8]. 
			Apart from its superior efficacy in treatment-refractory schizophrenia, use of clozapine has a lower risk of EPS. 
			However, clozapine is associated with a severe and potentially lethal adverse drug effect, known as agranulocytosis and this fatal side effect occurs in up to 1-2$\%$ of patients administered with clozapine for 4 weeks or longer[9]. 
			Therefore, clozapine is not used as a first line treatment of schizophrenia even though it remains superior to other available antipsychotics in terms of treatment efficacy[10]. 
			After the introduction of clozapine, pharmaceutical companies started to develop drugs that are pharmacologically and structurally similar to clozapine, attempting to mimic clozapine’s superior effectiveness without its hepatic toxicity. 
			A number of SGAs including risperidone, olanzapine, quetiapine, ziprasidone and aripiprazole have been developed[11]. 
			Unfortunately, although these drugs have a lower risk of EPS, they are associated with a higher rate of prolactin elevation and metabolic side effects including weight gain, diabetes mellitus and hyperlipidemia[12]. 
			Nonetheless, the SGAs have largely replaced FGAs as first line therapy for treatment of schizophrenia for a number of reasons. 
			Firstly, SGAs have a relatively more favorable side effect profile compared to significant movement side effect induced by FGAs[13]. 
			Secondly, it is generally thought that SGAs have a broader efficacy spectrum, particularly on negative symptoms and depression[6].
			
			Even though SGAs are thought to be more effective in treating broader spectrum of symptoms, there is debate regarding the efficacy of SGAs on negative symptoms [14, 15]. 
			Clinical trials on SGAs that have been criticized for recruiting patients with positive and negative symptoms, so that the improvement of negative symptoms reported could possibly due to indirect effect of the drugs[14].  
			Recent study reported that there is no clear evidence in treatment efficacy in terms of negative symptoms or cognitive impairment in any of the SGAs [16]. 
		
			\section{Antipsychotics: Mechanism of action}
			\subsection{Role of dopamine D2 receptor blockage}
				Although the exact mechanism of action of antipsychotic drugs is unknown, Dopamine D2 receptor is the primary target that all FGAs and SGAs bind [17]. 
				According to the dopamine hypothesis of schizophrenia, positive symptoms are result of mesolimbic dopamine hyperactivity. 
				To achieve antipsychotic effect, antipsychotics exert their clinical effect primarily through antagonism at dopamine D2 receptor in order to reduce dopaminergic activity.  
				Imaging studies have been adopted to study receptor binding affinity profiles of antipsychotic drugs.
				Generally, FGAs have a higher occupancy of striatal D2 receptor compared to SGAs, with 60-80$\%$ and 50-60$\%$ D2 receptor occupancies observed in FGAs and SGAs respectively[18, 19]. 
				Furthermore, it is suggested that most antipsychotics are effective when 65-70$\%$ of D2 receptors are blocked in the striatum [20-22]. 
				Therefore, antagonism at dopamine D2 receptor is necessary and sufficient for the antipsychotic effect[23]. 
				In addition, it was found that further increase in D2 receptor blockage does not result in additional antipsychotic efficacy, but result in a higher risk of excessive D2 receptor blockage associated side effects such as hyperprolactinemia and EPS[24].  
				It has been suggested that 65-80$\%$ of  blockage of D2 receptors is the optimal therapeutic target to maximize antipsychotics efficacy with minimal side effects [22, 25, 26].  
				Nonetheless, it should be noted that some patients do not respond to antipsychotics despite the occupancy of DRD2 is adequate [27].
			\subsection{Role of Non-D2 Receptor blockage}
				Apart from lower D2 binding affinity is observed in SGAs, they generally have additional affinities for a larger spectrum of receptor types. 
				The SGAs not only bind to dopaminergic receptors, but also serotonergic, cholinergic and adrenergic receptors[28]. 
				SGAs have a relatively higher affinity for the serotonin 5-HT2A receptor  compared to affinity for the D2 receptor, and this feature is the basis for the difference between FGAs and SGAs [29].  
				However, antagonism at the 5-HT2A receptor alone does not mediate antipsychotic effect, given SGAs such as risperidone and olanzapine can only be therapeutically effective when 65$\%$ of D2 receptors are blocked. 
				It has been suggested that the balance between D2 and 5-HT2A receptors is thought to be important for side effect liability and enhanced efficacy to treat negative and cognitive symptoms of schizophrenia[30, 31].  
				Contrary to this view, FGAs such as chlorpromazine and loxapine with binding affinity for 5-HT2A receptor have high incidence of EPS. 
				Plus, high dose of risperidone produces EPS, indicating antagonism at the 5-HT2A receptor along cannot eliminate EPS induced by high occupancy at D2 receptor. 
				Therefore, it remains unclear that low EPS liability observed in SGAs is related to the effect of the 5-HT2A receptor antagonism[32]. 
			\subsection{Preferential Limbic dopamine D2 receptor blockage}
				Given that clozapine exerts superior antipsychotic efficacy in treatment-resistant schizophrenia and its low incidence of EPS, the underlying mechanism for clozapine has been intensively studied. 
				In addition to its high affinity for 5-HT2A receptor over D2 receptor, it has shown to have a preferential action in dopamine system. 
				It has been suggested that clozapine selectively targeted the mesolimbic dopamine system while spared the nigrostriatal dopamine system[33]. 
				The hyperactivity in mesolimbic dopamine system is thought to be involved in positive symptoms of schizophrenia, while the drug-induced EPS is mediated by the blockage of dopamine receptor in the nigrostriatal dopamine system [34]. 
				In contrast, FGAs such as haloperidol act on both mesolimbic and nigrostriatal dopamine system equipotently and EPS during the administration of  haloperidol have been reported frequently [35].
				Based on these findings, it has been suggested the lower EPS liability of SGAs could be explained by their preferential modulation on the dopamine system [36].  
				However, a number of studies failed to show the preferential modulation of SGAs [14, 37].
			\section{Treatment response to antipsychotics medication}
				Although antipsychotic drugs are the mainstay for treatment of schizophrenia, large individual differences in treatment response exist among patients. 
				Up to 40$\%$ of patients do not response adequately to antipsychotic medication and even greater proportion (60-70$\%$) of patients experience adverse drug effects [38-41]. 
				Currently, there is no predictive test to determine which of the existing medication will be more efficacious and bring the least side effects for a given patients, finding the optimal treatment can only be achieved by a process of trial and error. 
				Patients who fail to respond to or experience intolerable side effects by their initial antipsychotic treatment are switched to another antipsychotic drug [42]. 
				The trail-and-error process may take months before the optimal treatment can be given[43].
				
				Moreover, treatment failure has a substantial clinical and economic cost. 
				Delay in giving optimal treatment leaves the patients vulnerable to continuing social dysfunction, suicide, unnecessary adverse effects and chances of recovery [44, 45]. 
				In terms of economic cost, around 2,500 pound, which accounts for 13$\%$ of British health expenditure, is spent for the treatment of refractory schizophrenia patients in the United Kingdom[46]. 
				Given these strong implications, identification of biological markers in association with antipsychotic response and accurate prediction test of individual response are essential for the better management of schizophrenia. 
			\subsection{Scales for assessing response to antipsychotics}
				Measuring treatment response is important to determine whether a specific target symptom is modified by a given medication. 
				To measure efficacy of antipsychotic medication, the most commonly use symptoms rating scales include the Brief Psychiatric Rating Scale (BPRS), Clinical Global Impression scale (CGI) and Positive and negative Symptom Scale (PANSS).  
				The change of a relative symptom severity from the baseline is typically used to reflect treatment efficacy.
				These symptom rating scales were not initially designed to measure efficacy of antipsychotic medication.
				Nevertheless, they have been widely used in this purpose[47]. 
				Non-response is typically defined as less than 20$\%$ improvement in PANSS or BPRS or post-treatment CGI-Severity score of ≥ 3 [48]. 
			\subsubsection{BPRS}
				Depending on the version, the BPRS includes a total of 18-24 items that address a range of psychotic and affective symptoms, including somatic concern, anxiety, depression, suicidality, guilt, hostility, elated mood, grandiosity, suspiciousness, hallucinations, unusual thought content, bizarre behavior, self-neglect, disorientation, conceptual disorganization, blunted affect, emotional withdrawal, motor retardation, tension, uncooperativeness, excitement, distractibility, motor hyperactivity, mannerisms, and posturing[49].  
				Each symptoms is rated on 1(not present) - 7(extremely severe) scale, with higher score indicating worse symptoms. 
				One major criticism is its low coverage of negative symptom assessment because there are only three items rating the negative symptoms[47]. 
				
			\subsubsection{CGI}
				The original version of CGI (CGI-S) is a 7-point scale which rates the overall severity of any mental disorder. 
				The assessment is entirely rated on clinician’s judgment about the severity of the illness, on a scale from 1 (healthy) to 7 (among the most severely ill). 
				Another version of for global improvement (CGI-I) for the comparison of patient’s current condition with his/her baseline condition is also a 7-point scale, with higher score indicating worsening of symptoms. 
				The scale is sensitive enough to distinguish non-responders from responders in clinical trials in depression[50]. 
				However, low specificity and lack of standard definitions become its weakness. 
				
			\subsubsection{PANSS}
				The PANSS includes all the items from the BPRS and selected items from the Psychopathology Rating Scale. 
				It consists of 7 positive symptoms items, 7 negative negative symptoms item and 16 general psychopathology symptoms items. 
				Each item is precisely defined and rated from 1 (absent) to 7 (extreme). 
				The PANSS provides a short description to severity of each symptom. Since its introduction, it has become perhaps the most widely used in clinical and research studies and is regarded as a reliable tool of symptom assessment[51]. 
				Since percentage PANSS reduction is usually used to define response, the percentage must be accurately calculated. 
				Owing to the scoring system in PANSS, an individual without any mental ill health will have a PANSS of 30. 
				This means that the potential minimum of 30 points must be subtracted from patient’s PANSS to get a meaningful understanding. 
				For example, a reduction of PANSS from 100 at baseline to 50 after treatment is a 71.4$\%$ reduction (after subtracting the minimum 30 points), but not 50$\%$ reduction. A 100$\%$ PANSS reduction can only be achieved when the 30 minimum points is subtracted[52]. 
				Moreover, factor analyses revealed five-factor model of the PANSS. 
				The five factors are often cited as positive, negative, anxiety/depression, hostility/excitement and cognitive/disorganization. 
				And it has been suggested that a five-factor model better captures the original PANSS structure in schizophrenia [53]. 
			\subsection{Factors influencing antipsychotic response}
				The variability in response to antipsychotics is a complex trait, it is influenced by environmental, clinical and genetic factors. 
				The following section will review evidence on the influence of these factors. 
			\subsubsection{Environmental factors}
				Factors such as smoking, diet, drug use, alcohol and demographic contribute to variation of antipsychotic response. 
				Cigarette smoking, diet, alcohol and drug use can induce or inhibit metabolic pathways, subsequently affecting plasma level of antipsychotic metabolites[54]. 
				These lift style habits might be subject to demographic variation. 
				Moreover, knowledge of diet and smoking habits can be served as an indicator for dosage adjustment[55]. 
			\subsubsection{Clinical factors}
				A number of premorbid and postmorbid factors have been reported to associate with antipsychotic outcome in schizophrenia. 
				It has been reported that female gender[56, 57], being married at the onset of the illness[56, 58], a diagnosis of paranoid schizophrenia, shorter duration of untreated psychosis, later onset, acute onset[56] and the presence of a psychological stress before the onset[59] are associated with better antipsychotic treatment outcome. 
				On the other hand, family history of schizophrenia[60], poor premorbid functioning[58, 61], lack of insight[61], severe negative symptoms[57], long duration of untreated psychosis[62] and substance misuse[63] are associated with poorer outcome. 
				
				For example, poor response to clozapine has been reported to be associated with early age of onset and male gender[64].  
				Also, it has been found that shorter duration of untreated psychosis with good premorbid function predict better treatment response to olanzapine and haloperidol [60, 62]. 
				However, it is worth noting that only small to modest effects were observed in the aforementioned factors. 
				Notably, several studies failed to replicate these findings or even detected the opposite direction of effect[65].
				Taken together, a better comprehension of these factors relating to antipsychotic response is needed to confirm their clinical usefulness. 
			\subsubsection{Genetic factors} 
				To investigate the role of genetic factors in a trait, epidemiology studies involving twin, family and adoption studies are needed[38]. 
				The heritability of schizophrenia is estimated to be up to 80$\%$ based on twin studies, indicating schizophrenia has a strong genetic component. 
				However, such systematic epidemiological study on antipsychotic response is lacking due to the fact that it is difficult to collect monozygotic twins concordant for both the disease and treatment. 
				Lack of twin data makes it difficult to quantify to what extent genetics influence the response to antipsychotics.
				Nonetheless, the role of genetics in antipsychotic response variability is evident from a number of case reports of twins and family studies with small sample size (summarized in table X\ref{tab:sampleSizeMaf}). 
				To summarize, monozygotic twins demonstrated the same adverse drug reactions with different antipsychotics medication and greater similarity in terms of weight gain to the same antipsychotics medication when compared to same-sex siblings.
				These studies indicated that genetic factor plays a substantial role in response variability. 
			\section{Pharmacogenetics and pharmacogenomics of antipsychotic treatment}
				Given that genetics contribute substantially to variation of antipsychotic response, identification of genetic variation in association with treatment response is important for better care of schizophrenia. 
				The term pharmacogenetics refers to the influence of single gene on the response to a drug, while the term pharmacogenomics refers to the influence of genes in the entire genome on the response of drug response.
				However, these two terms are often used interchangeably [66]. 
				Initially, pharmacogenetic studies aimed at finding mutations in enzyme genes in association with drug metabolism. 
				Later, the field expanded to investigate genes involving pharmacodynamics processes such as the dopaminergic and serotonergic systems to validate therapeutic target of antipsychotic drugs[38].
				However, the development of pharmacogenetics was relatively slow over the years due to lack of reproducibility of results. 
				Recent research has become more ambitious and attempted to combine genetic markers for prediction of antipsychotic response. 
				The ultimate goal of pharmacogenomics is to replace the current trial-and-error treatment paradigm with tailored drug treatment (the right dose of the right drug) based on one’s genetic profile[67].
				 
				Pharmacogenomics can be further divided into efficacy pharmacogenomics and safety pharmacogenomics, which corresponds to the search of genetic variants contributing to variability in treatment response and the development of adverse drug reactions, respectively[68]. 
				Increasing effort has been made in the last decades in studies to identify the influence of genetic factors on antipsychotic response. 
				The following sections provide an overview of the current state of pharmacogenetics/pharmacogenomics in schizophrenia treatment. 
			\subsection{Candidate gene studies} 
				The starting point pharmacogenetic study of treatment response began with investigating genetic variants of one or a small number of candidate genes. 
				Choice of candidate genes is hypothesis-driven, of which genes are selected with prior knowledge of their functional role in pharmacokinetic and pharmacodynamics aspects of drug action. 
				The term pharmacokinetic refers to what the body does to the drug (e.g. metabolism of the drug), while the pharmacodynamics refers to what the drug does to the body (e.g. drug target). 
			\subsubsection{Pharmacokinetic candidates}
				There is strong evidence of the contribution of pharmacokinetic candidates to antipsychotic treatment of which the cytochrome P450 (CYP) enzyme family in the liver is responsible for metabolism of many antipsychotic drugs, therefore influencing the antipsychotic treatment response. 
				Genetic variants in these enzymes have consistently associated with drug bioavailability (the fraction of antipsychotic drug that is absorbed and available to act on drug target), bioactivation and clearance[69]. 
				For example, defective, normal or duplicate copies of CYP genes were found to be correspond to different catalytic activity, classified as ‘poor metabolizer’, ‘extensive metabolizer’ and “ultrarapid metabolizer” respectively.
				 
				Among the cytochrome P450 enzyme family, CYP2D6 is the main metabolic enzyme for a number of FGAs and its genotype influence the bioavailability of antipsychotic drugs. 
				The gene encoding CYP2D6 is highly polymorphic, with more than 100 genetic variants including single nucleotide substitution, insertion and deletion reported[70]. 
				Among these variants, four variants (*3, *4, *5 and *6) account for 98$\%$ of poor metabolizer in Caucasian. 
				The frequencies of these variants are subject to population variation, with higher frequencies in Caucasians and lower frequencies in Asians, approximately 7-8$\%$ of Caucasian and 1-2$\%$ of Asians are poor metabolizer[71]. 
				A study has demonstrated that the metabolism of the haloperidol is significantly lowered in poor metabolizers. 
				Since poor metabolizers tend to accumulate higher dug in blood, lower therapeutic dose is recommended accordingly[38]. 
				Moreover, gene duplication with the presence of CYP2D6*1XN, *2XN and *35XN duplications are responsible for ultrarapid metabolism. 
				Ultrarapid metabolizers are rare and only found 1$\%$ of the population. 
				As opposed to poor metabolizer, ultrarapid metabolizers require higher therapeutic dose to compensate for the rapid elimination of drug metabolites[72]. 
				Owing to increasing evidence indicating the role of CYP2D6 in metabolic activity, it has been hypothesized that CYP2D6 metabolic status also plays an important role in antipsychotic efficacy.
				However, most studies failed to find such evidence[38]. 
				Nonetheless, CYP2D6 is the only Food and Drug Administration approved pharcmacogenetic test for antipsychotic medication, it is used to guide dosage of aripiprazole, iloperidone, and pimozide[67]. 
				In addition, a few studies have found significant relationship between CYP2D6 metabolic status and adverse drug reactions[73].
				 
				CYP1A2 is another cytochrome P450 enzyme which is the main metabolic pathway of clozapine and olanzapine. 
				Over 20 polymorphisms have been reported in the CYP1A2 gene, with three variants associated with reduced metabolic activity. 
				In addition to the influence of genetic variant on the metabolic activity, CYP1A2 activity is also inducible by environmental factors such as smoking and caffeine[70].
				 
				Finally, CYP3A4 is also involved in the metabolism of most antipsychotic drugs. 
				A few variants of the CYP3A4 gene were found to be associated with catalytic activity[74]. 
				However, most studies failed to detect association of these functional variants with antipsychotic variability, except one study reporting that CTP3A4*1B can predict better treatment outcome[75]. 
				Since the role of CYP1A2 and CYP3A4 in antipsychotic variability remains largely unclear, further research is required for provide evidence of their functional role.
				
				In addition to the CYP enzymes, drug transporters in the blood-brain barrier transporter may contribute to antipsychotic pharmacokinetics and treatment response. 
				P-glycoprotein is a transmember protein which regulates the absorption, distribution and elimination of xenobiotics in the blood-brain barrier[38]. 
				The gene encoding P-glyprotein, MDR1 (also known as ABCB1) is highly polymorphic. 
				Three SNPs (rs1045642, rs2032582 and rs1128503) in this gene have been extensively investigated and some studies reported positive association with response to SGA treatment[67]. 
				
			\subsubsection{Pharmacodynamic candidates}
				 Pharmacogenetic research on dynamics factors started as a strategy to validate therapeutic target of antipsychotic. 
				 Schizophrenia has been associated with dysregulation of neurotransmitter systems, therefore, the altered neurotransmitter systems have been used as therapeutic targets for antipsychotic drugs. 
				 The existing antipsychotic drugs display affinity for a variety of receptors, including dopamine, serotonin, histanube, muscarine, glutamate and adrenergic receptors. 
				 Variations in genes encoding targets of antipsychotic drug action, e.g. neurotransmitter receptor, might potentially affect binding affinity and efficiency of signal induction, resulting variability in treatment response among patients. 
				 If genetic variants in a given receptor are associated with antipsychotic efficacy, this could confirm the therapeutic value of receptor.
			\paragraph{Dopamine}
				Dopamine alterations in schizophrenia patients were the first pathological observation in schizophrenia, since then, the development for treatment has been based on targeting dopamine pathways. 
				\gene{DRD2} antagonism is the common feature of both FGAs and SGAs.  
				As discussed in earlier section, \gene{DRD2} antagonism is required for antipsychotic effect and associated with pyramidal side effects. 
				For these reasons, polymorphisms of \gene{DRD2} and the related dopamine genes have been extensively studied.
				The 141 Ins/Del polymorphism (rs1799732) located in the 5' promoter region of \gene{DRD2} has been reported to be associated with antipsychotic response, with the del allele associated with poorer outcome[76, 77]. 
				Despite several studies failed to detect such association, this finding has been confirmed by a meta-analysis[78].  
				The SNP rs1800477 (\gene{Taq1A}) is located 10kb downstream of \gene{DRD2}. 
				Its A1 allele is associated with reduced expression level of \gene{DRD2} and therefore hypothesized to contribute to variability of antipsychotic response. 
				However, previous studies have produced inconsistent results[79]. 
				
				Given most antipsychotic drugs display similar affinity for dopamine D3 receptor (\gene{DRD3})   and the number of \gene{DRD3} increase in striatum following antipsychotic treatment, \gene{DRD3} has been a candidate gene for investigating antipsychotic response. 
				The SNPs rs6280 (Ser9Gly) is a missense variants and associated with dopamine binding affinity[67]. 
				Although some studies reported better treatment outcome in association with Ser allele[80, 81], a recent meta-analysis failed to replicate this finding[78].  
			\paragraph{Serotonin}
				Serotonin dysregulations have been associated with depression, anxiety, eating disorders and negative symptoms of schizophrenia. 
				Traditionally, higher binding affinity for serotonin receptors relative to dopamine receptor is a main characteristic of SGAs. 
				Several serotonin genes have been reported to be associated with antipsychotic response. 
				
				Among serotonin receptors, the serotonin 2A receptor (\gene{HTR2A}) has been studied extensively because it is the major target of SGAs. 
				A number of studies have reported positive association between the SNP rs6313 with improvement in negative and cognitive symptoms. 
				Although the SNP rs6313 is a synonymous variant, it is in complete linkage disequilibrium (LD) with another function SNP rs6311 in the promoter region of \gene{HTR2A} in Caucasian population.  
				Given the complete LD between rs6313 and rs9311, the later can be partly explained the association with antipsychotic response by affecting promoter activity and thus the density of \gene{HTR2A} in some brain regions. 
				A lower expression of \gene{HTR2A} may decrease the efficacy of the dopamine-serotonin modulation exerted by antipsychotics. 
				Finally, a missense variant (rs6314, His452Tyr) located in exon 3 of \gene{HTR2A} was found to be robustly associated with antipsychotic efficacy and binding affinity of \gene{HTR2A}.
				Although some studies failed to replicate this finding, an early meta-analysis of five independent studies showed a clear association of the Tyr allele with poorer response to clozapine (OR=5.55). 
				The Tyr allele is associated with reduced level of calcium release and reduced ability to activate phospholipases, resulting in reduced signal and indirectly lower antipsychotic efficacy.
				Apart from high affinity for \gene{HTR2A}, most SGAs except quetiapine have binding affinity for serotonin 2C receptor (\gene{HTR2C}). 
				The most studied polymorphism of \gene{HTR2A} is a missense variant (rs6318, Cys23Ser) and it has been associated with negative symptoms in clozapine on study. 
				However, five other studies failed to replicate this finding. 
				
				Finally, serotonin 1A receptor (\gene{HTR1A}) may influence antipsychotic response. 
				The SNP rs6295 has been reported to be associated with improvement in negative symptoms. 
				This SNP is located in the promoter region of \gene{HTR1A} and affects a transcription binding site, resulting in change in gene expression level. 		
				
				In addition to genetic variants in receptor, alterations in serotonin cleft levels may also influence antipsychotic response. 
				The serotonin transport gene (\gene{SLC6A4} or \gene{5HTT}) is responsible to transporting serotonin from synapses from presynaptic neurons and terminating the action of serotonin, and its polymorphisms have been studied in relation to drug response.  
				An ins/del polymorphism of 44 bp in the promoter of \gene{SCL6A4}, \gene{5-HTTLPR} has been shown to be associated with response to clozapine in Han Chinese patients in two independent studies.
				Functionally, this polymorphism is associated with promoter activity, expression level of \gene{SCL6A4} and serotonin reuptake. 
				However, two other studies in Caucasians and Japanese did not confirm this finding. 
			\paragraph{Glutamatergic system}
				Since the symptoms associated with schizophrenia can be reproduced by administrating NMDA receptor antagonist such as ketamine and phencyclidine in healthy subjects, evidence of alteration in glutermatergic system in schizophrenia has been increasing. 
				Some studies hypothesized that dopamine dysfunction may be secondary to malfunction of glutamatergic system[82].  
				In addition, compounds modulating NMDA receptor complex are reported to reduce negative symptoms and cognitive impairment of schizophrenia[83]. 
				Taken together, these finding indicate the potential role of NMDA in pathophysiology and treatment of schizophrenia. 
				
				Compared to the well-studied dopaminergic and serotonergic genes, the pharmacogenetic of the glutamatergic remains understudied. 
				The type-3 metabotropic glutamate receptor gene (\gene{GRM3}) encodes for mGluR3 protein which is involved in modulating signaling though NMDA receptors. 
				It has been found that \gene{GRM3} variants influence antipsychotic effects on and negative symptom and cognition [84-86]. 
				However, for another glutamate receptor gene, glutamate receptor, ionotropic, N-methyl D-aspartate 2B (\gene{GRIN2B}), study has failed to find an association between its genetic variants and clozapine response[87]. 
				In summary, there is no clear association established between genetic variants of glutamatergic genes and antipsychotic response, probably due to the result of insufficient studies. 
				
			\paragraph{Developmental and regulatory genes.} 
				In addition to neurotransmitter genes, it has been suggested that genes involved regulation of neuronal genesis, plasticity and interactions might also contribute to symptom variability and antipsychotic response[88].
				Although these regulatory genes are not the direct target of antipsychotic drugs, genetic variants in these genes might lead to different pathology and symptomatology, resulting in patient subgroups that might respond to antipsychotic drugs differently. 
				Given this hypothesis, the pharmacogenetic studies have expanded to investigate genes involved in regulating neuronal metabolism, development and functionality[38]. 
				
				Given dopamine system has been regarded as a crucial target for antipsychotics, genes regulating dopamine functionality or availability might influence antipsychotic response. 
				The Catechol-O-methyltransferase (COMT) enzyme is involved in the main pathway of dopamine degradation. 
				A missense variant, Val108Met (rs4680) can cause substantial variations in enzyme activity. 
				It has found that the Met/Met genotype results in threefold to fourfold lower enzyme activity, when compared to the Val/Val genotype.  
				As a result of reduced enzyme activity, dopamine levels in synapse are increased due to ineffective degradation[89]. 
				Regarding to its effect on antipsychotic effect, several studies has associated the Met allele with greater improvement in negative symptoms and cognition[90-92]. 
				This might be explained by the dopamine hypothesis of schizophrenia that negative symptoms and cognitive impairment are related to dopamine hypoactivity in prefrontal cortex, therefore, reduced metabolism of dopamine associated with the Met allele might help by restoring dopamine level[71]. 
				This is consistent with the finding of a meta-analysis that the Met/Met genotype is associated with higher intelligence scores[93]. 
				However, other studies have reported no association or association of the opposite allele (Val allele) with better response[94]. 
				These inconsistent results indicate that the relationship between this SNP and antipsychotic response is not fully understood, hence, further investigation is required.
				 
				The brain-derived neurotrophic factor (\gene{BDNF}) is an important regulator of neuronal development synaptic plasticity, and plays a major role in the regulation of expression of dopamine receptors[95].
				In drug-naive patients, serum \gene{BDNF} levels are lower than control and altered after administering antipsychotic treatment[96]. 
				This suggests that variations in \gene{BDNF} activity might be involved in the therapeutic response to antipsychotic drugs.  
				The most extensively studied is rs6265, a missense variant Val66Met at codon 66 of the \gene{BDNF} gene.
				The Met allele was shown to disrupt \gene{BDNF} mRNA targeting to dendritic cell, \gene{BDNF} protein packaging and secretion, resulting in reduced synaptic plasticity[97, 98]. 
				With regard to antipsychotic response, the Met allele has been reported to be associated with poorer treatment response in two independent studies[99, 100]. 
				However, other studies cannot replicate this finding[101, 102]. 
				
			\paragraph{Others}
				Although the SNP rs1344706 in the zinc finger 804A gene \gene{ZNF804A} was the first detected gene in genome-wide association study with schizophrenia and a broader psychosis phenotype, its biological function of \gene{ZNF804A} has been entirely unknown. 
				Based on its sequence domain with homology to the Cys2his2 family of zinc diner-binding transcription factors, \gene{ZNF804} is predicted to act as a transcription factor[103]. 
				The differentially expressed genes in \gene{ZNF804A} knock-down are involved in cell adhesion, suggesting the functional role of \gene{ZNF804A} in processes such as neural migration, neurite outgrowth and synapse formation[104]. 
				The SNP rs1344706 is a functional variant of which the A allele conferring increased expression of \gene{ZNF804A}[105, 106]. 
				In respect of antipsychotic response, the A allele of rs1344706 has been reported to be associated with less improvement in positive symptoms[105, 107]. 
				Yet, the mechanism remains to be elucidated. 
			\paragraph{Summary of candidate gene studies}
				Despite some inconsistent literature on the genetic association of candidate genes, research from the past two decades has provided emerging evidence that several genetic variants area capable to predict antipsychotic treatment outcome. 
				There has been accumulating evidence of genetic variants in \gene{CYP2D6}, \gene{DRD2} and \gene{HTR2A} in association with antipsychotic response. 
				Notably, their effect sizes appear to be small or modest.
				Most of the other genetic variants have not been replicated or remained significant after meta-analysis, more well-designed studies with larger sample sizes are clearly needed to validate their role.  
				
			\subsection{Genome-wide association studies} 
				While candidate gene studies provide insight into the role of genes that have been already known to be pharmacokinetic and pharmacodynamic targets in response to antipsychotic treatment, these studies are restricted to our limited knowledge about antipsychotic mechanism of action. 
				Moreover, none of these studies revealed markers robustly associated with antipsychotic response as a result of inconsistent findings across the studies. 
				An alternative way to conduct pharmacogenetic study is to examine variations across the entire human genome through a hypothesis-free approach. 
				With the advancement of in high-throughput SNPs genotyping technology, GWAS allows us to examine association across the entire genome with antipsychotic response. 
				Therefore, new molecular target and pathways implicated in antipsychotic drug action could be potentially discovered.  
				In the past years, increasing number of GWAS has been applied in the field of psychiatry, yet there are only a few GWAS conducted on antipsychotic response. 
				To date, there have been two GWASs of antipsychotic response published.  
				
			\subsubsection{GWAS of response to iloperidone}
				The first GWAS of antipsychotic response was conducted in a phase 3 randomized controlled trial of a new antipsychotic drug, iloperidone, which was approved for use treating schizophrenia by Food and Drug Administration in 2009 in the U.S[108]. 
				In this GWAS, it consists of 407 patients from different ethnical groups(17 Asian, 105 African-American, 80 white and 8 other), PANSS was assessed at baseline and after 4 weeks of treatment as a measure of response. 
				Six SNPs were reported to be associated with iloperiodone response[109].
				Of these, rs11851892, located in an intron of the neuronal PAS domain protein 3 (\gene{NPAS3}) gene, is intragenic.
				Prior association with schizophrenia and other neuropsychiatric diseases have been reported for \gene{NPAS} [110]. 
				Four of the other SNPs were located in the intergenic regions of genes that are involved in biologically plausible schizophrenia pathway. 
				These include the tenascin-R gene (\gene{TNR}), encoding for an extracellular matrix glycoprotein involved in patterning of central nervous system[111]; the glutamate receptor, inotropic, AMPA 4 gene (\gene{GRIA4}), encoding for the iontropic type of glutamate receptor implicated in the pathophysiology of schizophrenia[82]; the XK, Kell blood group complex subunit-related family, member 4 gene (\gene{XKR4}), which has been implicated in risk of Attention-Deficit Hyperactivity Disorder[112]; and the glial cell line-derived neurotrophic factor receptor-alpha2 gene (\gene{GFRA2}), which is implicated in risk of schizophrenia and response variability to clozapine[113]. 
				However, it should be noted that the study population is heterogeneous in terms of ethnicity and the correction for population stratification was inadequate in this study. Additionally, none of identified SNPs reached genome-wide significance (p $< 5\times10^{-8}$). 
				
			\subsubsection{GWAS of treatment response in the CATIE study}
				A GWAS was conducted on CATIE participants to detect genetic variants underlying difference in response to antipsychotic treatment with olanzapine, quetiapine, risperidone, ziprasidone and perphenazine. 
				This study included 738 subjects of various ethnic groups and used multimensional scaling approach for substructure correction. 
				Treatment response was measured using the total PANSS as well as PANSS subscales derived by the five-factor model. 
				No SNPs showed association with PANSS total. 
				With a false-discovery rate of $le$ 0.5, they identified a SNP rs17390445 on chromosome 4p15 significantly associated with improvement in positive symptoms on ziprasidone (p = 9.82 × 10−8).
				However, this SNP is located in an intergenic region and the function of the variant is unknown. 
				In addition, they identified SNPs in the Contactin-Associated Protein-Like 5 gene (\gene{CNTNAP5}) and the Transient Receptor Potential Cation Channel, subfamily M, member 1 (\gene{TRPM1}) to be nominally associated with improvement in negative symptoms on risperidone, and a SNP in the Ankyrin Repeat and Sterile Alpha Motif Domain-Containing Protein 1B (\gene{ANKS1B}) gene to be nominally associated with improvement in negative symptoms on olanzapine. 
				Interestingly, all of these genes are expressed in the brain and potentially involved in neuronal development or neurotransmission in the central nervous system. 
				\gene{CNTNAP5} encodes for a member of neurexin family of multidomain transmembrane proteins which function in the cell adhesion and intercellular communication in the central nervous system and impact risk of autism [114, 115]. 
				\gene{TRPM1} is a member of the Transient Receptor Potential family of Ca2+ permeable cation channel[116].
				Lastly, \gene{ANKS1B} encodes for a tyrosine kinase signal transduction gene that interacts with amyloid-b precursor protein and may involve in the brain development [117, 118].
				Moreover, the top SNPs in \gene{CNTNAP5} and \gene{TRPM1} are both nonsynonymous variant, suggesting their functional relevance to clinical drug response.
				 
				The second study focused on the neurocognition as the outcome measure of antipsychotic response using the same dataset[119]. 
				The neurocognition was assessed using five neurocongition domains and a composite neurocognition score. 
				With a false-discovery rate of $<$ 0.1, six SNPs were found to be associated with treatment response.
				Of these, five of these SNPs are located in the intragenic regions of genes that do not have obvious pharmacokinetic or pharmacodynamic roles. 
				By contrast, the SNP rs11214606 located in an intron of \gene{DRD2} was reported to mediate the effect of olanzapine on working memory (p-value = $4.84 \times 10^{-7}$). 
				Given this SNP is in LD with the Taq1A variant and \gene{DRD2} is a well-known candidate for antipsychotic response, this was an encouraging finding. 
				Nevertheless, this association was not corroborated by the neighboring SNPs, indicating the possibility of being false positive. 
				
				The third study used both the clinician-rated CGI-Severity scale and patient-rated patient global impression (PGI) scale as the outcome measure of antipsychotic response [120]. 
				With a false-discovery rate of $<$ 0.1, six SNPs were significantly associated with clinician-rated CGI and seven SNPs were associated with patient-rated PGI. 
				In addition, they investigated whether the SNPs that were associated with one global impression scale were also nominally associated with the other rater global impression scale. 
				In total, four SNPs showed cross-outcome association. 
				Notably, a SNP rs6688363 located in the promoter region of ATPase, Na+/K+ transporting, alpha 2 polypeptide gene (\gene{ATP1A2}) is associated with response to olanzapine (CGI-Severity p-value $= 1.59 \times 10^{-7}$ ; PGI p-value $=$ 0.019). 
				This gene encodes for a subunit of a sodium/potassium transporting ATPase that functions in electrical conduction of nerve impulses and mutations in this gene is implicated in the familial form of migraine[121], suggesting its potential role in response to drug effect.
				 
				To date, the CATIE subjects remains to be the largest study cohort in term of the published GWAS of antipsychotic treatment response. 
				It is worth noting that there are certain limitations of the CATIE cohort including 
				\begin{enumerate*}[label=\arabic*)]
					\item low follow-up rates,
					\item inadequate study duration, 
					\item idiosyncratic sample characteristics, 
					\item unusual outcome measures, 
					\item exclusion of patient with  Tardive dyskinesia from the randomization, 
					\item choice of study drugs and doses,
					\item biasing differences in treatments used before randomization, and
					\item insufficient statistical power[122].
				\end{enumerate*} 
				
			\subsubsection{Summary of GWAS of antipsychotic response}
				In summary, a number of novel genes have been identified to contribute to antipsychotic treatment response through GWAS approach. 
				Unfortunately, it appears the findings from GWAS do not emerge in the well-replicated candidate gene associations, leaving inconsistency between studies unexplained. 
				The non-overlap of findings between candidate gene studies and GWAS might be due to the fact that the published GWAS are limited by sample size, therefore, under power to detect variant with small to modest effect. 
				Nonetheless, the pattern of GWAS findings is encouraging in the context of biological relevance. 
				How these identified genes are involved in antipsychotic response is far from clear and these findings warrant replication in independent studies to confirm the associations. 
				However, replication in pharmacogenetic study is not easy. 
				The difficulties and challenges will be discussed in the later section. 
			\subsection{Next-generation sequencing}
				Given GWAS have been the main enablers for pharmacogenomics studies in the past few years, most variants that are implicated in drug response still remain to be elucidated. 
				One of possible explanation is that novel or rare variants cannot to be identified by GWAS. 
				Providing that the role of rare and/or novel variants in neuropsychiatric diseases have broadly been recognized, they might also play an important role in response to drug treatment.
				For example, rare and/or novel variants of functional impact in pharmacogenetic candidate genes might render an individual nonresponse to certain drugs, and they remain undetected in the commercial genotyping arrays[123].
				Indeed, a recent study reported the presence of an abundance of rare functional variants in genes encoding drug target, and these variants might play a potential role to drug response [124]. 
				 
				Recent progress in next-generation sequencing (NGS), or massively parallel sequencing technology, enables the sequencing of whole genome or exome.
				These technologies were initially applied to identify the causative mutations of inherited diseases with unknown genetic mutations. 
				The sharp decrease in the cost of sequencing has allowed whole-genome or exome sequencing to be more frequently used as methods of analysis, enabling us to analyze genome variation comprehensively and with a high degree of accuracy[125, 126]. 
				By examining the entire spectrum of genetic variants present in gene networks and pathways, we might be able to obtain a more comprehensive overview of the genetic factors that contribute to antipsychotic treatment response[127]. 
				One study has recently adopted whole-exome sequencing for studying the rare functional variants in antipsychotic treatment response in African population.  
				
				Drogemoller et al. studied the functional variants that influence the response to antipsychotic treatment (flupentixol decanoate) in South African first-episode schizophrenia patients[128].
				The treatment outcome was assessed by PANSS. 
				Subjects falling at the extreme ends of the treatment response were classified as responder and non-responder and selected for whole exome sequencing. 
				In total, five responders and six non-responders were selected. 
				All five responders exhibited at least 40$\%$ improvement in total PANSS after 12 months of antipsychotic treatment, while the non-responders exhibited less than 20$\%$ improvement or end point PANSS total score greater than 70. 
				Investigation of coding variants revealed that slightly more loss of function variants in the non-responder, although these difference were not significant. 
				One INDEL variant rs11368509 located on the splice site of the uridine phosphorylase 2 gene (\gene{UPP2}) was enriched in non-responders. 
				\gene{UPP2} encodes for an enzyme which catalyzes the phosphorylytic cleavage of uridine to uracil and uridine has been shown to enhance the antagonism of the dopamine system by haloperidol[129, 130]. 
				In addition, this enzyme has been shown the affect the metabolism of an anticancer drug fluorouracil[131], suggesting its potential role in pharmacogenetics. 
				However, further examination in two replication cohorts showed this variant was weakly associated with better treatment outcome (p = 0.057 ; p = 0.016), which was of the opposite direction to the result obtained in the exome data. 
				
			\subsubsection{Summary}
				Although the rise of NGS technology is a promising tool to facilitate the search of response-related genes, the use of NGS in the pharmacogenomic field is still in its infancy with only one small study published to date. 
				Given drug response is a multidimensional and complex trait, larger sample size will help to uncover novel proteins and pathways involved in antipsychotic treatment response. 
			\section{Difficulties and challenges in pharmacogenomic studies}
				Lack of replication among stuides is the major issues in the field of pharmacogenetics/pharmacogenomics of antipsychotic response. 
				The challenge of identifying replicable and robust genetic markers in pharmacogenetic study of antipsychotic response is the result of a number of factors. In addition to clinical factors that might differ across studies, some other factors such as methodological and statistical issues might be the possible explanation for the inconsistent findings.  
			\subsection{Methodological issues}
			\subsubsection{Biased selection}
				Most of the pharmacogentic study utilized data from ongoing clinical trials. 
				Although it is convenient to recruit samples from available resources, there are a number of limitations to this approach. 
				These samples collected tend to be chronic schizophrenia patients with lengthy prior treatment histories and exposure to multiple antipsychotics. 
				Since highly responsive and stable patients are underrepresented for these trials, they need not to seek for change in treatment. 
				As a result of that, study subjects recruited from trials may systematically biased towards inclusion of severe patients who are less responsive to antipsychotic treatment and/or nonadherent with treatment, and therefore not representing the full spectrum of symptoms[132]. 
				Furthermore, prior antipsychotic treatment may be an important confounding factor in the study of pharmacokinetic variants, as the effect of genetic variations in drug metabolizing enzyme activity has already been diluted by dosage adjustment[67].
				What’s more, chronic schizophrenia patients are reported to abuse alcohol or drugs, experience longer duration of psychotic symptoms, suffer from social disabilities, all of which are may contribute to drug response and introduce increased variance into data analyzes[133]. 
				Hence, recruitment of patients with early phase of schizophrenia or first-episode schizophrenia in pharmacogenetic study may increase power for detecting association with drug efficacy. 
				
				Indeed, a meta-analysis has demonstrated that a 50$\%$ greater effect size for the -141C Ins/Del of \gene{DRD2} when compared studies consisting first episode patient to studies of chronic patients[78].
				An example is from the study of \gene{HTR2C} and antipsychotic-induced weight gain. 
				They reported that the odd ratio of the C-759T polymorphism in the \gene{HTR2C} significantly differ between first-episode sample and chronic sample with the odd ratio of 5.40 and 1.64 in first-episode and chronic schizophrenia patients respectively[134]. 
				These findings suggest the focusing on first episode or early phase of schizophrenia patient only could help minimize heterogeneity associated with prior treatment, but also they also represent unique advantage in pharmacogenetic study of antipsychotic response.      
			\subsubsection{Medication non-adherence }
				Treatment adherence has been recognized as an important contributor to variability of response. 
				A recent review reported that non-adherence ranges from 20$\%$ to 72$\%$ for schizophrenia[135].
				In the CATIE trial, up to three quarters of patients stopped their initially assigned antipsychotic treatment within 18 months, and one third of them stopped mediation due to patient’s own decision[40].
				These data strongly suggest the non-adherence is common among patients with schizophrenia. 
				Medication non-adherence not only results in symptoms relapse, it also weakens signal and reduces statistical power in pharmacogenetic studies of treatment response[133]. 
				If medication adherence is not assessed properly in a pharmacogenetic study, non-adherent patients would be mislabeled as non-response to the given drug treatment. 
				As a result of misclassification of subject, it would be difficult to detect a significant genotype-phenotype relationship regardless the effect size of the genetic marker. 
				Nevertheless, this important factor is frequently underestimated in neglected. 
				
				A stimulation study showed that, in a typical scenario of modest effect size and 20$\%$ frequency of risk allele, the statistical power for detecting significant association reduces rapidly as the non-adherence increases[136]. 
				Even with a sample size of 400 which is considered to be large in pharmacogenetic study, the power drops below 0.7 when the non-adherence rate reaches 50$\%$[137].
				On the other hand, by decreasing the non-adherence rate from 50$\%$ to 10$\%$, the sample size can be reduced to 200 to obtain adequate power. 
				Therefore, the sample size required can be reduced by maximizing treatment adherence in any given pharmacogenetic study of drug response. 
				
				One way to ensure medication adherence is to collect plasma antipsychotic drug levels at each visit[138], and exclude any subject with undetected levels from the study. 
				There has been a GWAS success story of antipsychotic-induced weight gain, in which a near genome wide significant result was obtained using a relatively small sample of 139 adherent patients[139]. 
				And the result was replicated in three independent samples with monitored adherence. 
				These studies highlight the significance of treatment adherence in pharmacogenetic study by increasing statistical power and reducing the overall cost. 
			\subsubsection{Heterogeneity in the choice and definition of outcome}
				Antipsychotic drug response is a complex, multidimensional and fluctuating trait, making it difficult to quantify objectively. 
				Most studies have replied on a various severity rated scales such as PANSS, BPRS and CGI to define response, in which the reduction in scales reflects symptom improvement. 
				However, owing to these rating scales differ largely in definition and in their degree of comprehensiveness, there is a risk of heterogeneity in response definition. 
				For example, PANSS contains more items and assesse more extensively on negative symptoms compared to BPRS. 
				Hence, careful interpretation of these scales is necessary before comparison can be made. 
				Numerous studies have attempted to find out the relationship between PANSS, BRPS and CGI. 
				It was found that a BPRS score of 31 and 41 approximately corresponds to ``mildly ill'' and ``modestly ill'' in CGI respectively[140].
				A 10/15 point reduction of the BPRS/PANSS corresponds to one severity level decrease in CGI-severity scale[141]. 
				Since there is no standard criteria to define response, rate scale cutoffs for response is often chosen arbitrary and varies from a 20$\%$ to 50$\%$ reduction in score[142].  
				Moreover, some studies use overall remission or recovery to define response. 
				Remission indicates that the symptoms have been mostly alleviated while recovery focuses on patients' social and vocational functioning rather than symptoms[142]. 
				Taken together, these differences in choice and definition of response outcome make the comparison across studies difficult, leading to failure of replication.  
			\subsubsection{Population stratification}
				Like any other genetic association study, population stratification is a confounding factor in pharmcogenetic study. 
				Population stratification arises when ethnic subpopulations within a study differ in terms of allele frequency and incidence of study outcome, resulting in spurious association due to sampling difference rather than the trait of interest[143]. 
				A classic example of population stratification in the aspect of pharmacogenetics is the HLA allele B*1502. 
				This allele is strongly associated with high risk of carbamazepine induced severe cutaneous drug reaction in Han Chinese and most Southeast Asians. 
				However, such association cannot be found in Caucasian. 
				It is because the HLA allele B*1502 is common in Hans Chinese and other Asian population but not present in Caucasian population[144, 145].   
				
				Several methods are established to eliminate or correct for bias of population stratification.
				Population stratification can be minimized by matching cases (non-responders) and controls (responder) for ethnicity during sample collection stage[146]. 
				Self-reported race/ethnicity is frequently used. 
				For GWAS, the most commonly used method is to apply principal component analysis (PCA) to explicitly model population structure along continuous axes of variation and correct for substructure[147]. 
				An alternative method is multidimensional scaling implemented in PLINK[148]. 
				Another approach to adjust for stratification is the use of genomic control. 
			\subsection{Statistical issues}
			\subsubsection{Study power}
				The power to detect significant association depends on the sample size, effect size, frequency of causal allele, and frequency of marker allele and its correlation with the causal variant[149]. 
				Given a common variant with large effect in complex trait such as drug response is unlikely, pharmacogenetic study should be powered to detect common variant with modest effect or rare variants with large effect[150]. 
				Hence, a large sample size is required to provide adequate power in genetic association study.
				 
				To illustrate, Table X shows the approximate sample size required under different effect size and allele frequency (assuming power = 0.8 and alpha = 0.05). 
				These estimates are based on the assumption that the variant allele is causal. 
				Hence, if the variant allele is in LD with causal variant, even larger sample size is needed. 
				Move over, pharmacogenetic studies typically investigate more than one genetic variant in which case larger sample size is again required[143]. 
				Unfortunately, it is not common for pharmacogenetic studies of antipsychotic response to have sample size as large as those in Table X. 
				This suggests the majority of pharmacogenetic studies are often underpowered, which may lead to false-positive or false-negative findings. 
				
				The need of large sample size to obtain adequate statistical power has led to the formation of international consortia where data can be shared across studies and meta-analyzed. 
				In era of schizophrenia genetics, identification of susceptibility genes had been a disappointment in the past decades. 
				However, this has begun to change after success has been achieved through the collaborative science in recent years. 
				In the latest publication of PGC schizophrenia using 36,989 cases and 113.075 controls, 108 loci were identified to be associated with schizophrenia[151]. 
				This highlights the importance of collaboration and data-sharing. 
				Unfortunately, there is no large consortium on pharmacogenomics of antipsychotic response. 
			\subsubsection{Multiple testing}
				As the number of tested SNPs increases, data analysis becomes a statistical challenge due to multiple testing problem. 
				The p-value threshold of 0.05 is typically used in biomedical research to control type I error.
				However, this standard p-value is no longer appropriate because the frequency of type I increases with increasing number of tests[152]. 
				Since GWAS involved testing millions of SNPs in a single study, multiple testing correction is probably one of the most important ways to reduce false positive. 
				
				One of the simplest ways to correct for multiple testing is Bonferroni correction in which the critical p-value is divided by the number of tests performed. 
				Bonferroni correction is considered to be very conservation because it assumes each association test is independent to all other tests. 
				However, this is untrue in the content of GWAS because many SNPs are in LD.
				Moreover, since many of the current studies are already underpowered to detect pharmacogenetic variants, this over-conservative correction may null the study results[146]. 
				Another approach to correct for multiple testing is false discovery rate (FDR) which is less conservative than Bonferroni correction. 
				The FDR computes q-value which is an estimate of the expected proportion of significant associations that are false positive[153].  
	
				
						
						
				
	\chapter*{Appendix}
\end{document}