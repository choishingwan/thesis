	\chapter{Introduction}
	% Disease background
	\section{Schizophrenia}
	\Gls{scz} is a devastating psychiatric disorder affecting approximately $0.3\sim0.7\%$ of the population  worldwide \citep{AmericanPsychiatricAssociation2013}.
	According to one of the current standard classification manual \gls{dsm}-\rom{5}, a diagnosis of \glng{scz} (F20.9) can only be reached if the patient suffered from 2 or more of the following symptoms for a significant portion of time during a 1-month period: 
	\begin{enumerate*}[label=\arabic*\upshape)]
		\item delusion; \label{ls:delusion}
		\item hallucinations;\label{ls:hallucinations}
		\item disorganized speech;\label{ls:disorganizedSpeech}
		\item grossly disorganized or catatonic behaviour; and\label{ls:catatonicBehavior}
		\item negative symptoms such as diminished emotional expression,\label{ls:negativeSymptoms}
	\end{enumerate*}  where one of the symptom must be either (\ref{ls:delusion}, (\ref{ls:hallucinations} or (\ref{ls:disorganizedSpeech}.
	Signs of disturbance also need to persist for at least 6-month before the patient can be diagnosed with \glng{scz}.
	
	Because of the detrimental symptoms and the lack of effective treatments, \glng{scz} imposes a long lasting health, social and financial burden to the patients and their families \citep{Knapp2004}. 
	\Glng{scz} patient also have a higher tendency to suicide  \citep{Saha2007}, leading to a higher mortality.
	Based on the \gls{who} report, \glng{scz} is one of the top 20 leading cause of \gls{yld} in 2012, ranking 16 among all possible causes (\cref{tab:whoYLD}), demonstrating the extent of impact from \glng{scz} to patients.
	\begin{table}[ht]
		\centering
		\caption[Top 20 leading cause of \glng{yld}]{Top 20 leading cause of \gls{yld} calculated by \gls{who} in year 2012.
			\Glng{scz} was considered as one of the top 20 leading cause of \gls{yld}\citep{Geneva2013}}
		\begin{tabular}{rrrrr}
			\toprule
			Rank  & Cause & \gls{yld} (000s) & \% \gls{yld} & \specialcell[b]{\gls{yld} per \\100k population}\\
			\midrule
			0     & All Causes & 740,545 & 100   & 10466 \\
			1     & Unipolar depressive disorders & 76,419 & 10.3  & 1080 \\
			2     & Back and neck pain & 53,855 & 7.3   & 761 \\
			3     & Iron-deficiency anaemia & 43,615 & 5.9   & 616 \\
			4     & Chronic obstructive pulmonary disease & 30,749 & 4.2   & 435 \\
			5     & Alcohol use disorders & 27,905 & 3.8   & 394 \\
			6     & Anxiety disorders & 27,549 & 3.7   & 389 \\
			7     & Diabetes mellitus & 22,492 & 3     & 318 \\
			8     & Other hearing loss & 22,076 & 3     & 312 \\
			9     & Falls & 20,409 & 2.8   & 288 \\
			10    & Migraine & 18,538 & 2.5   & 262 \\
			11    & Osteoarthritis & 18,096 & 2.4   & 256 \\
			12    & Skin diseases & 15,744 & 2.1   & 223 \\
			13    & Asthma & 14,134 & 1.9   & 200 \\
			14    & Road injury & 13,902 & 1.9   & 196 \\
			15    & Refractive errors & 13,498 & 1.8   & 191 \\
			16    & Schizophrenia & 13,408 & 1.8   & 189 \\
			17    & Bipolar disorder & 13,271 & 1.8   & 188 \\
			18    & Drug use disorders & 10,620 & 1.4   & 150 \\
			19    & Endocrine, blood, immune disorders & 10,495 & 1.4   & 148 \\
			20    & Gynecological diseases & 10,227 & 1.4   & 145 \\
			\bottomrule
		\end{tabular}%
		\label{tab:whoYLD}%
	\end{table}%
	Due to the severity of \glng{scz}, it has drawn much attention from the research community, hoping to delineate the disease mechanics and to identify risk factors associated with \glng{scz}.
	Ultimately, the goal of \glng{scz} research is to identify effective treatment(s) to help improving the quality of life of the patients.
	
	\section{Understanding the Disease Mechanism}
	An important first step in \glng{scz} research is to understand whether if it is a genetic or environmental disorder. 
	For example, if \glng{scz} is a genetic disorder, then one should focus on collecting genetic data and identify genetic variants that might associate with \glng{scz}.
	Yet if \glng{scz} is an environmental disorder, one should instead focus on how the environmental factors affect the normal functioning of the patients.
	In order to study the relative contribution of genetic and environmental influence to individual differences in \glng{scz}, one will need to calculate the \emph{heritability} of \glng{scz}.
	\subsection{Broad Sense Heritability}
	
	Heritability is defined as the \emph{proportion} of total variance of a trait in a population explained by variation of genetic factors in the population.
	One can partition observed phenotype into a combination of genetic and environmental components \citep{Falconer1996}
	$$
	\text{Phenotype (P)}=\text{Genotype (G)}+\text{Environment (E)}
	$$
	where the variance of the observed phenotype ($\sigma_P^2$) can be expressed as variance of genotype ($\sigma_G^2$) and variance of environment ($\sigma_E^2$)
	$$
	\sigma_P^2=\sigma_G^2+\sigma_E^2
	$$
	The broad sense heritability can then be defined as the ratio between the variance of the observed phenotype and the variance of the genetic effects
	$$
	H^2=\frac{\sigma_G^2}{\sigma_P^2}
	$$
	
	One key feature of heritability is that it is a \emph{ratio} of \emph{population} measurement at a specific time point.
	As a result of that, the heritability estimation might differ from one population to another due to difference in \gls{maf} and one might obtain a different heritability estimate if the method or time-point of measurement of the trait differs because of different environmental factors coming into play.
	A classic example was the study of \gls{iq} where the heritability estimation increases with age \citep{Bouchard2013}.
	It was hypothesize that the shared environment has a larger effect on individuals when they were young, and as they become more independent, the effect of shared environment diminishes, leading to an \emph{increased portion} of variance in \gls{iq} explained by the variance in genetic \citep{Bouchard2013}. 
	
	\subsection{Narrow Sense Heritability}
	In reality, the problem of heritability was more complicated for there were different forms of genetic effects. 
	For example, one can partition the genetic variance into variance of additive genetic effects ($\sigma_A^2$), variance of dominant genetic effects ($\sigma_D^2$) and other epistatic genetic effects ($\sigma_I^2$) such that
	$$
	\sigma_G^2=\sigma_A^2+\sigma_D^2+\sigma_I^2
	$$
	where additive genetic variance was the variance explained by the average effects of all loci involved in the determination of the trait, whereas dominant genetic effects and epistatic genetic effects were the interaction between alleles at the \emph{same} locus or \emph{different} loci respectively.
	
	As individuals only transmit one copy of each allele to their offspring, relatives other than full siblings and identical twins will only share a maximum of one copy of the allele.
	Considering that dominance and non-additive genetic effects were the interactive effect, which usually involve more than one copy of the alleles, these effects are unlikely to contribute to the resemblance between relatives \citep{Visscher2008}.
	On the other hand, the additive genetic effects is usually transmitted from parent to offspring, thus it is more useful to consider the narrow sense heritability ($h^2$) which only consider the additive genetic effects:
	\begin{align}
	h^2&=\frac{\sigma_A^2}{\sigma_P^2} \notag\\
	h^2&=\frac{\sigma_A^2}{\sigma_G^2+\sigma_E^2}
	\label{eq:narrowHeritability}
	\end{align}
	
	To obtain the additive genetic effect, we can first consider the genetic effect of parents to be $G_p=A+D$. 
	As only half of the additive effect were transmitted to their offspring, the child will have a genetic effect of $G_c=\frac{1}{2}A+\frac{1}{2}A'+D'$ where $A'$ is the additive genetic effect obtained from another parent by random and $D'$ is the non-additive genetic effect in the offspring.
	If we then consider the parent offspring covariance, we will get
	\begin{align}
	\mathrm{Cov_{OP}}&= \sum(\frac{1}{2}A+\frac{1}{2}A'+D')(A+D)\notag\\
	&=\frac{1}{2}\sum A^2+\frac{1}{2}\sum AD + \frac{1}{2}\sum A'(A+D) +D'(A+D) \notag\\ 
	&=\frac{1}{2}V_A+ \frac{1}{2}\mathrm{Cov}_{AD} + \frac{1}{2}\mathrm{Cov}_{A'A} + \frac{1}{2}\mathrm{Cov}_{A'D} +\mathrm{Cov}_{D'A} +\mathrm{Cov}_{D'D}  
	\label{eq:halfCompletedCovOP}
	\end{align} 
	Under the assumption of random mating,  $A'$ should be independent from $A$ and $D$. 
	On the other hand, as $D'$ was specific to the child, it should be independent from $A$ and $D$.
	Moreover, the covariance between the additive genetics and non-additive genetics should be zero \citep{Falconer1996}.
	Thus, \cref{eq:halfCompletedCovOP} becomes
	\begin{align}
	\mathrm{Cov_{OP}} &= \frac{1}{2}V_A+\mathrm{Cov}_{AD} \notag\\
	&= \frac{1}{2}V_A
	\label{eq:covOP}
	\end{align}
	Now if we assume the variance of phenotype of the parent and offspring were the same, then using \cref{eq:covOP}, we can obtain the narrow-sense heritability as
	\begin{align}
	h^2 &= \frac{1}{2}\frac{V_A}{\sigma_P^2}
	\label{eq:narrowHerit}
	\end{align}
	If we consider the simple linear regression equation $Y=X\beta+\epsilon$, its slope can be calculated as 
	\begin{equation}
	\beta_{XY} = \frac{\mathrm{Cov}_{XY}}{\sigma_{X}{Y}}
	\end{equation}
	which resemble \cref{eq:narrowHerit}. 
	Therefore,  we can calculate the narrow sense heritability as
	\begin{equation}
	h^2 = 2\beta_{OP}
	\label{eq:narrowSenseHerit}
	\end{equation}
	where $\beta_{OP}$ is the slope of the simple linear regression regressing the phenotype of an offspring to the phenotype of \emph{one} of its parents.
	We can further generalize \cref{eq:narrowSenseHerit} to all possible relativeness 
	\begin{equation}
	h^2=\frac{\beta_{XY}}{r}
	\label{eq:finalNarrow}
	\end{equation}
	where $r$ is the relativeness of $X$ and $Y$.
	
	A key assumption in this calculation was that the relatives does not share anything other than the additive genetic factors.
	However, this was usually not the case as relatives does tends to be in the same cultural group and might have similar socio-economic status which might all contribute to the variance of the trait.
	This might therefore lead to bias in \cref{eq:finalNarrow} and we shall discuss the partitioning of variance in the later sections.
	
	Nonetheless, \cref{eq:finalNarrow} was still useful for the understanding of the calculation of heritability.
	However, in the case of discontinuous trait (e.g. disease status) the calculation becomes more complicated because the variance of the phenotype was dependent on the population prevalence.
	As \cref{eq:finalNarrow} does not account for the trait prevalence, it cannot be directly applied to discontinuous traits.
	In order to perform heritability estimation, we will need the concept of liability threshold model popularized by \cite{Falconer1965}.
	
	\subsection{Liability Threshold}
	\label{sec:liability}
	According to the central limit theorem, if a phenotype is determined by a multitude of genetics and environmental factors with relatively small effect, then its distribution will likely follow a normal distribution as is the case of many quantitative traits \citep{Visscher2008}. % No, what if there is interaction between variables? Then it will break the CLT
	The variance of phenotype can therefore be calculated as the variance under the normal distribution.
	However, such is not the case for disease such as \glng{scz} where instead of having a continuous distribution of phenotype, only a dichotomous labeling of ``affected'' and ``normal'' were obtained.
	The variance of these phenotype were therefore more difficult to obtain.
	
	\citet{Falconer1965} proposed the liability threshold model, which suggesting that these discontinuous traits also follow a continuous distribution with an additional parameter called the ``liability threshold''.
	Under the liability threshold model, the discontinuous traits were affected by combination of multitude of genetics and environmental factors, each with a small effects, as in the case of the continuous traits.
	The main difference was that the phenotype of an individual is determined by whether if the combined effects of these factors (``liability'') were above a particular threshold (``liability threshold'').
	So for example, in the case of \glng{scz}, only when an individual has a liability above the liability threshold will he/she be affected.
	
	One can then estimate the heritability of the discontinuous by comparing the mean liability of the general population when compared to the relatives of the affected individuals.	
	For example, if we consider a single threshold model of a dichotomous trait, where 
	\begin{align}
	T_G &= \text{Liability threshold of the general population}\notag\\
	T_R &= \text{Liability threshold of relatives of the index case} \notag\\
	q_G &= \text{Prevalence in the general population}\notag\\
	q_R &= \text{Prevalence in relatives of the index case}\notag\\
	L_a &= \text{Mean Liability of the index case} \notag
	\end{align}
	by assuming both the liability distribution of the general population and the relative of the index case both follows the standard normal distribution, we can align the two distribution with respect to $T_G$ and $T_R$. 
	We can then calculate the mean liability of the index case $L_a$ as $L_a=\frac{z_G}{q_G}$ where $z_G$ is the density of the normal distribution at the liability threshold $T_G$.
	Then we can express the regression of relatives' liability on the liability of the index case as
	\begin{align}
	\beta &= \frac{T_G-T_R}{L_a}
	\label{eq:liability}
	\end{align}
	
	Thus, by applying \cref{eq:liability} to \cref{eq:finalNarrow}, we get
	\begin{align}
	h^2 =\frac{T_G-T_R}{rL_a}
	\end{align}
	
	\subsection{Adoption Study}
	% Need to go deeper into twin studies
	The key limitation of \cref{eq:finalNarrow} was its inability to discriminate the genetic factors from the shared environmental factors.
	Such problem arise as family not only shared some of their genes, but they also tends to share some of the environmental factors such as diet. 
	In fact, this was the main reason for researchers to discord the argument that \glng{scz} is a genetic disorder.
	
	A classical adoption study carried out by \citet{HESTON1966} in 1966 set off to discriminate whether if the increased risk of \glng{scz} in relatives of \glng{scz} was caused by the shared environmental factors or the shared genetic factors. 
	An advantages of adoption studies was that if the child was separated from their family early after birth, then the shared environmental factors should be minimized, thus any resemblance between the parent and child should be driven mainly by the shared genetic factors.
	\citet{HESTON1966} collected data of 47 individuals born from a schizophrenic mother during the period from 1915 to 1947. 
	They were separated from their mother within three day of birth and were sent to a foster family. 
	50 matched control were also recruited to the study.
	It was observed that there was an increased risk of \glng{scz} in individual born to schizophrenic mother when compared to the control group even-though they were brought up in a different environment as that of their mother.
	This result suggested that \glng{scz} was likely driven by the shared genetic factors instead of the shared environmental factors.
	
	\subsection{Twin Studies}
	Despite the usefulness of adoption studies in delineating the effect of shared environment from the genetic factors, collection of adoption data were difficult. 
	Moreover, any prenatal influence such as alcohol abuse during pregnancy might confound the results.
	Therefore, an alternative way would be the twin studies using the relationship between the \gls{mz} and \gls{dz} twins.
	
	Theoretically, \gls{mz} twins should share all their genetic components (both additive ($A$) and non-additive ($D$) genetic factors) and also their common environmental factors ($C$) where the only difference between a twin pair would be the non-shared environmental factors ($E$). 
	As for the \gls{dz} twins, they also share the same common environmental factors yet they only share $\frac{1}{2}$ of their additive genetic factors and $\frac{1}{4}$ of their non-additive genetic factors. 
	The non-shared environmental was also by definition not shared among the twins \citep{Rijsdijk2002}.
	Based on these assumptions, \cite{Falconer1996} derived the heritability as
	\begin{equation}
	h^2 = 2(\rho_{MZ}-\rho_{DZ})
	\end{equation}
	where $\rho_{MZ}$ and $\rho_{DZ}$ were the phenotype correlation between the \gls{mz} twins and \gls{dz} twins respectively.
	
	By combining Falconer's formula and the concept of liability threshold model, \citet{Gottesman01071967} estimated that the heritability of \glng{scz} to be $>60\%$ based on previously collected twin data, strongly suggest \glng{scz} as a genetic disorder.
	The result was further supported by one of the landmark meta-analysis study conducted by \citet{Sullivan2003}.
	Based on data obtained from 12 published \glng{scz} twin studies, \citet{Sullivan2003} found that although there was a non-zero contribution of environmental influence on liability of \glng{scz} ($11\%$, \gls{ci}=$3\%-19\%$), there was a much larger contribution from genetics ($81\%$, \gls{ci}=$73\%-90\%$), further supporting that \glng{scz} was largely mediated by the genetic factors.
	
	Such findings were not limited to twin-studies but were also reported in large scale population based studies.
	A recent large scale population based study in Sweden population \citep{Lichtenstein2009} also found that there was a large genetic contribution in \glng{scz} ($64\%$).
	Although the estimated heritability (64\% \citep{Lichtenstein2009} vs 81\% \citep{Sullivan2003}) differs between the two studies, there is no doubt that \glng{scz} is highly heritable, leading to the initiative of genetic research in \glng{scz}.
	
	\section{Schizophrenia Genetics}
	The results from the twin studies strongly support \glng{scz} as a genetic disorder.
	However, little was known about the mechanism of \glng{scz} nor the genetic architecture of the disorder. 
	All data from adoption studies, twin studies and family studies shown that \glng{scz} does not follow the Mendelian framework \citep{Gottesman01071967,Gottesman1982}.
	Specifically, shall \glng{scz} be a Mendelian disorder, then we would expect all \gls{mz} siblings of the proband to also suffer from \glng{scz}.
	However, the life time morbid risk of monozyogitc twins were only $48\%$ (\cref{fig:lifeMRscz}) \citep{gottesman1991schizophrenia}, making it unlikely for \glng{scz} to follow a Mendelian pattern.
	\begin{figure}[t]
		\centering
		\includegraphics[width=0.6\textwidth]{figure/lifeTimeMorbidRisk.png}
		\caption[Lifetime morbid risks of \glng{scz} in various classes of relatives of a proband]{Lifetime morbid risks of \glng{scz} in various classes of relatives of a proband.
			It was noted that the morbid risk of monozygotic (MZ) twins were only $48\%$, much lower than one would expect if \glng{scz} follows a Mendelian pattern.
			Reproduced with permission from journal \citep{Riley2006}. \label{fig:lifeMRscz}}
	\end{figure}
	
	
	Based on these observations, \citet{Gottesman1967} proposed that \glng{scz} follows a polygenic model where disease phenotype were determined by the additive effects from multiple genes.
	Thus, \glng{scz} is likely to be a complex genetic disorder with complicated pattern of inheritance. 
	Their hypothesis was supported by the calculation of \citet{Risch1990a}.
	
	Not only does \citet{Risch1990a} supports the polygenic model for schizophrenia, \citet{Risch1990a} also estimated the possible effect size of individual locus in schizophrenia. 
	By comparing the observed life time morbid risk and the expected risk from different models, \citet{Risch1990a} proposed that genetic models with a single locus with risk of 3.0 and with all other loci of small effect or models with two or three loci with risk of 2.0 were most consistent with the observed life time morbid risk of \glng{scz} \citep{Risch1990}.
	
	\citet{Risch1990a}'s calculation provided an explanation for the early inconsistent findings of linkage studies in \glng{scz} \citep{Harrison2005}.
	As linkage studies were aimed to identify genetic variation of large effect size they failed to capture genetic loci with small effect size.
	It was therefore tempting to suggest that \glng{scz} only follows the ``common disease-common variant'' model, which stated that \glng{scz} is mediated by large amount of common variants such as \glng{SNP}, each carries a small effect size.
	
	However, another possible hypothesis was that the variation mediating \glng{scz} were rare, therefore require a large sample size to detect and the inconsistent results of early linkage studies might be due to the inadequate sample size. 
	This lead to some researchers suggesting the ``common disease-rare variant'' hypothesis, which propose that \glng{scz} was mediated by a small amount of rare variants, each with a large effect size \citep{McClellan2007}.
	
	Nevertheless, success in genetic research of \glng{scz} remains limited.
	Only until the initiation of Human Genome Project and technological advance resulted from that does genetic research of \glng{scz} entered an era of success.
	
	\subsection{The Human Genome Project and HapMap Project}
	\glsreset{SNP}
	\glsreset{LD}
	In 1990, the Human genome project was initiated, aiming at constructing the first physical map of the human genome at per nucleotide resolution \citep{Lander2001}.
	The completion of the human genome project has opened up a new era of genetic research, allowing researchers to identify \glspl{SNP} on the human genome, which is one of the major source of genetic variation.
	
	Soon after the completion of the human genome project, the HapMap Project was initiated \citep{Consortium2005}, aiming to provide a genome-wide database of common human sequence variation such as \glspl{SNP} with \gls{maf} $\ge0.05$.
	More importantly was that the HapMap Project also provided a detailed \gls{LD} map of the human genome.
	
	\gls{LD} was of particular importance to genetic research for it was the non-random correlation of genotypes between 2 genetic loci. 
	\glspl{SNP} in high \gls{LD} were usually observed together in the human genome.
	When a large amount of \glspl{SNP} were in high \gls{LD} together, they form what was known as a \gls{LD} block.
	By performing association testing on \glspl{SNP} representing a \gls{LD} block (``tagging''), one can avoid the need of performing association on the whole genome, therefore reducing the cost of the experiment.
	This was the fundamental concept of \gls{GWAS} which was now extensively used in the genetic research.
	
	\subsection{Genome Wide Association Study}
	In \gls{GWAS}, genome-wide genotyping array were commonly used to systematically detect common genetic variants such as \gls{SNP} and \gls{cnv}.
	For quantitative traits, the association between the trait and frequency of the variants were calculated using methods such as linear regression.
	On the other hand, for dichotomous traits such as \glng{scz}, the frequency of the variants were compared between the case and control samples using methods such as chi-square test or logistic regression.
	Because of the problem of multiple testing, only variants with a p-value passing a genome wide threshold (p-value $\le5\times10^{-8}$) were considered significant.
	Another possible method to decide the significant threshold was to consider the ``effective number'' of tests \citep{Li2011}, which reduced the genome wide threshold according to the \gls{LD} structure.
	When designing a \gls{GWAS}, one need to take into account of the magnitude of effect, sample size, and required level of statistical significance (the false-positive, or type I, error rate) in order to have a powerful study \citep{Purcell2003}.
	
	\subsubsection{The Success of Psychiatric Genomic Consortium} 
	Despite the great promise from \gls{GWAS}, early \gls{GWAS} in \glng{scz} remain largely disappointing and were unable to identify any robust genetic markers associated with \glng{scz}.
	The failure of early \gls{GWAS} in \glng{scz} were mainly due to the relative small sample size of the studies, which result in low detection power.
	
	To overcome the problem of small sample size, large consortium were formed such that data from different research groups from different countries were combined, which provides a large sample size for the analysis.
	By 2014, the \Glng{scz} Working group of the \gls{pgc} has collected 34,241 \glng{scz} samples and 45,604 controls \citep{Ripke2014}.
	By combining the samples with those obtained by deCODE genetics, a total of 36,989 \glng{scz} samples and 113,075 controls were used for the largest meta-analysis of \glng{scz}.
	In their study \citep{Ripke2014}, 128 linkage-disequilibrium-independent \glspl{SNP} were found to  exceeded the genome-wide significance (p-value $\le 5\times10^{-8}$), corresponding to 108 genetic loci.
	75\% of these loci contain protein coding genes and a further 8\% of these loci were within 20\gls{kb} of a gene. 
	It was found that genes involved in glutamatergic neurotransmission (e.g. \textit{GRM3}, \textit{GRIN2A} and \textit{GRIA1}), synaptic plasticity and genes encoding the voltage-gated calcium channel subunits (e.g. \textit{CACNA1C}, \textit{CACNB2} and \textit{CACNA1I}) were among the genes associated within these loci.
	Importantly, \textit{DRD2}, the target of all effective anti-psychotic drug were also associated with \glng{scz}.
	This result converges with existing knowledge of \textit{DRD2} being involved in the pathology of \glng{scz}, supported by multiple lines of research \citep{Talkowski2007}.
	\begin{figure}
		\centering
		\caption[Enrichment of enhancers of SNPs associated with Schizophrenia]{Enrichment of enhancers of SNPs associated with \glng{scz}. 
			It was observed that the largest enrichment were in cell lines related to the brain and in tissues with important immune functions. 
			Graphs reproduced with permission from the journal \citep{Ripke2014}.}
		\includegraphics[height=\textwidth]{figure/pgc_enrichment_tissue.jpg}
		\label{fig:pgcEnrich}
	\end{figure}
	It was further demonstrated that \glng{scz} association were significantly enriched at enhancers active in brain and enriched at enhancers active in tissues with important immune functions (\cref{fig:pgcEnrich})\citep{Ripke2014}.
	
	The enrichment of immune related enhancers remains significant even after the removal of \gls{mhc} region from the analysis, provided further genetic support of the involvement of the immune system in the etiology of \glng{scz}.
	Because of its role in neural development \citep{Zhao1998,Deverman2009}, it is likely that the perturbation in the immune system might disrupt the brain development, therefore increasing the risk of \glng{scz}.
	%Indeed, studies on \gls{mia} has demonstrated that cytokine imbalance might predispose individual to \glng{scz} \citep{Meyer2009}. 
	
	Although the \gls{pgc} \glng{scz} \gls{GWAS} is very successful, it is uncertain whether if all common variants associated with \glng{scz} has been captured. 
	With the unknown number of causal loci with moderate-to-small effect size, many \glspl{SNP} associated with \glng{scz} may be left undetected given the current sample size. 
	However, it is also possible that the \gls{pgc} \glng{scz} \gls{GWAS} has already captured all or near most of the \glspl{SNP} associated with the disease. 
	Therefore, estimating the contribution of these common \glspl{SNP} to \glng{scz} has important implications for future research strategy.
	
	\subsection{Contribution of Common SNPs}
	In a typical \gls{GWAS}, a stringent genome wide significant threshold were usually employed to avoid false positive findings. 
	However, if individual \glspl{SNP} have a small effect on the trait, the real association might be missed.
	Therefore, to estimate the true contribution of common \glspl{SNP} to a disease (\gls{SNP}-heritability), one should try to use all \glspl{SNP} in the estimation.
	
	\subsubsection{Genome-wide Complex Trait Analysis}
	Currently, the most popular algorithm used for the estimation of \gls{SNP}-heritability is \gls{gcta}, which uses information from the \gls{grm} \citep{Yang2011}.
	The \gls{grm} is represents the ``genetic distance'' between all individuals within the \gls{GWAS}.
	Genetic relationship between individual $j$ and $k$ is estimated as 
	\begin{equation}
	A_{jk} = \frac{1}{N}\sum^N_{i=1}\frac{(x_{ij}-2p_i)(x_{ik}-2p_i)}{2p_i(1-p_i)}
	\end{equation}
	where $x_{ij}$ is the number of copies of the reference allele for the $i^{th}$ \gls{SNP} of the $j^{th}$ individual and $p_i$ is the frequency of the reference allele.
	This is based on the fact that genotypes were usually coded as 0, 1 or 2 (homozygous reference, heterozygous and homozygous alternative respectively) and should follow the binomial distribution.
	From the binomial distribution, the expected mean and variance of the genotype $i$ will be $2p_i$ and $2p_i(1-p_i)$ respectively.
	Thus $A_{jk} = \frac{1}{N}\sum^N_{i=1}z_{ij}z_{ik}$ where $z_{ij}$ is the standardized genotype for the $i^{th}$ \gls{SNP} of the $j^{th}$ individual.
	
	Using the information from the \gls{grm}, \citet{Yang2011} then fit the effects of all the \glspl{SNP} as random effects by a \gls{mlm}
	\begin{align}
	\boldsymbol{y} &= \boldsymbol{X\beta}+\boldsymbol{g}+\epsilon\\
	\mathrm{Var}(\boldsymbol{y}) &= \boldsymbol{A}\sigma_g^2+\boldsymbol{I}\sigma_\epsilon^2
	\end{align}
	where $\boldsymbol{y}$ is an $n\times 1$ vector of phenotypes with $n$ samples, $\boldsymbol{\beta}$ is a vector of fixed effects such as sex and age, $\boldsymbol{g}$ is an $n\times 1$ vector of the total genetic effects of the individuals, $\sigma_g^2$ is the variance explained by all the \glspl{SNP} and finally, $\sigma_\epsilon^2$ is the variance explained by residual effects.

	The main concept of \gls{gcta} is that instead of testing the associations for individual \glspl{SNP}, one fit the effects of all \glspl{SNP} as random effects in a \gls{mlm} and estimate a single parameter, i.e. the variance explained by all \glspl{SNP} or \gls{SNP}-heritability.
	Given the information of the \gls{grm}, \citet{Yang2011} implemented the \gls{reml} using the average information algorithm to estimates the $\sigma_g^2$ and $\sigma_\epsilon^2$where the \gls{reml} is a form of maximum likelihood estimation that allows unbiased estimates of variance and covariance parameters.
	The \gls{SNP}-heritability of the trait is then defined as $\frac{\sigma_g^2}{\sigma_g^2+\sigma_e^2}$.

	Based on the above concept, \citet{Yang2010a} were able to estimate the variance in height explained by \glspl{SNP} from the height \gls{GWAS} to be around 45\%, much larger than previously reported 5\%.
	The main difference in the estimates was because the \gls{mlm} \gls{reml} were able to consider all \glspl{SNP} simultaneously without limited on significant \glspl{SNP}.
	Although the estimates was still less than 80\% which was the expected heritability of height, \citet{Yang2010a} was able to demonstrated that one possible source of ``missing heritability'' might be due to incomplete \gls{LD}.
	By taking into consideration of incomplete \gls{LD}, it was estimated that the proportion of variance explained by causal variants can be as high as 0.84 with \gls{se} of 0.16 \citep{Yang2010a}, close to the expected heritability.
	Together, \citet{Yang2011} provide a possible method for the estimation of the variance explained by \glspl{SNP} in \gls{GWAS} data and the method is now implemented in \gls{gcta} which is wildly adopted.
	
	The problem with \gls{gcta} was that genotype data are required to calculate the \gls{grm}.
	For complex disease like \glng{scz}, the data were usually obtained from multiple data source where the raw genotypes were unavailable due to privacy concerns.
	Instead, summary statistics were usually provided.
	Therefore estimation of variance explained by \glspl{SNP} in these \gls{GWAS} can only rely on the summary statistics. 
	
	\subsubsection{\glng{ldsc}}
	In large scale \gls{GWAS} studies, a general inflation of summary statistics can sometimes be observed.
	It was usually considered to be contributed by the presence of confounding factors such as population stratification, under the assumption that most of the \glspl{SNP} should have no association to the disease.
	It was therefore a common practice for one to perform the \gls{gc} on the \gls{GWAS} results \citep{Zheng2006}.
	
	The problem of \gls{gc} was that the basic assumption of a small number of causal \glspl{SNP} might not be true, especially in complex disease like \glng{scz}.
	Through careful simulation, \citet{Yang2011b} demonstrated that in the absence of population stratification and other form of technical artifacts, the presence of polygenic inheritance can inflate the summary statistic \citep{Yang2011b}.
	More importantly, they observed that the magnitude of inflation was determined by the \emph{heritability}, the \gls{LD} structure, sample size and the number of causal \glspl{SNP} of the trait.
	
	The observation of \citet{Yang2011b} provide important foundation for the estimation of \gls{SNP} heritability based on summary statistics where a possible method will be to elucidate the heritability based on the magnitude of inflation of the summary statistics. 
	However, when confounding factors such as population stratification and cryptic relatedness are presented, they can also inflate the summary statistics.
	Therefore, in order to estimate the \gls{SNP}-heritability, one must be able to delineate the confounding factors from the polygenicity of the trait.
	
	Based on the work of \citet{Yang2011b}, \citet{Bulik-Sullivan2015} hypothesized that strength of ``tagging'' of a \gls{SNP} should be correlated with the probability of it to ``tag'' the causal \gls{SNP} yet should be independent to confounding factors such as population stratification and cryptic relatedness.
	\citet{Bulik-Sullivan2015} then define the strength of ``tagging'' of a \gls{SNP} as the \gls{LD} score, which is the sum of $r^2$ of $k$ \glspl{SNP} within a 1cM window of \gls{SNP}$_j$:
	\begin{equation}
	l_j = \sum_kr^2_{jk}
	\label{eq:ldScore}
	\end{equation}
	
	Based on their hypothesis, the expected $\chi^2$ of association of \gls{SNP}$_j$ with the trait can be defined as a function of the \gls{LD} score ($l_j$), the number of samples ($N$), the number of \glspl{SNP} in the analysis($M$) and most importantly, the \gls{SNP} heritability ($h^2$):
	\begin{equation}
	\mathrm{E}[\chi^2_j | l_j] = \frac{Nh^2}{M}l_j+1
	\label{eq:fixedLDSC}
	\end{equation}
	%TODO understand the exact calculation of LDSC
	
	When confounding factors were present in the study (e.g. population stratification), \cref{eq:fixedLDSC} can instead be defined as
	\begin{equation}
	\mathrm{E}[\chi^2_j | l_j] = \frac{Nh^2}{M}l_j+Na+1
	\label{eq:fullLDSC}
	\end{equation}
	where $a$ is the contribution of confounding bias.
	
	By considering \cref{eq:fullLDSC} as a regression model, \citet{Bulik-Sullivan2015} observed that the contribution of common variants (the \gls{SNP} heritability $h^2$) will be the slope of the regression and the intercept minus one will represent the mean contribution of the confounding bias such as those of population stratification. 
	The \gls{ldsc} was implemented by \citet{Bulik-Sullivan2015}, hoping to use \cref{eq:fullLDSC} to delineate the contribution from confounding factors and common genetic variants.
	
	To test their hypothesis, \citet{Bulik-Sullivan2015} simulated multiple \gls{GWAS} where the trait can have a polygenic architecture or where confounding factors can present.
	When the simulated trait is polygenic and no confounding factors were presented, the average \gls{ldsc} intercept was close to one and the estimates were unbiased in all situation.
	Only when the number of causal variants was small will the standard error of the estimates become very large.
	On the other hand, when the \gls{GWAS} was simulated with only the confounding factors such as population stratification, the intercept estimated was approximately equal to the \gls{gc} inflation factor with only a small positive bias in the regression slope.
	
	Moreover, when a polygenic trait was simulated with confounding factors, the intercept of \gls{ldsc} was approximately equal to the mean $\chi^2$ statistic among the null \glspl{SNP}, providing strong evidence that \gls{ldsc} can partition the inflation in test statistic even in the presence of both bias and polygenicity.
	
	Given  the success of the simulation, \citet{Bulik-Sullivan2015} estimated the \gls{SNP} heritability of \glng{scz} using the summary statistics from the \gls{pgc} \glng{scz} \gls{GWAS} \citep{Ripke2014}.
	By applying the liability threshold adjustment, \citet{Bulik-Sullivan2015} estimated the \gls{SNP}-heritability of \glng{scz} should be 0.555 with \gls{se} of 0.008.
	The estimated \gls{SNP} heritability was lower than the heritability estimated from population based study (64\% \citep{Lichtenstein2009}) and twin studies (81\% \citep{Sullivan2003}) suggesting that it is possible for variants other than common \glspl{SNP} to account for variations in \glng{scz}.
	
	\subsubsection{Partitioning of Heritability}
	Another implication of \gls{ldsc} is that it allows the partitioning of heritability, which allow one to identify pathways that were associated with a trait.
	
	Traditionally, functional enrichment analysis in \gls{GWAS} only take into account of \glspl{SNP} that passed the genome wide significance threshold. 
	However, for complex traits such as that of \glng{scz}, much of the heritability might lies in \glspl{SNP} that do not reach genome wide significance threshold at the current sample size.
	For example, in 2013, only 13 risk loci were detected using 13,833 \glng{scz} samples and 18,310 controls \citep{Ripke2013}. 
	When the sample size increased to 34,241 \glng{scz} samples and 45,604 controls in 2014, 108 risk loci were identified \citep{Ripke2014}. 
	Thus, if one only consider the significant loci, risk loci that have not reach genome wide significance threshold might be ignored from the analysis, decreasing the power of the functional enrichment analysis.

	In order to estimate whether if a functional categories was associated with the trait, \gls{ldsc} takes into consideration of the summary statistic of all the \glspl{SNP}.
	The partitioning of the heritability is then calculated as 
	\begin{equation}
	\mathrm{E}[\chi^2_j] = N\sum_C\tau_Cl(j,C)+Na+1
	\label{eq:partitionH}
	\end{equation}
	
	The main difference between \cref{eq:partitionH} and \cref{eq:fullLDSC} is that $\frac{h^2}{M}l_j$ is substituted by $\sum_C\tau_Cl(j,C)$ where $l(j,C)$ is the \gls{LD} Score of \gls{SNP} $j$ with respect
	to category $C$ and $\tau C$ is the per-\gls{SNP} heritability in category $C$.
	
	Using data from \citet{Ripke2014} and functional categories derived from the ENCODE annotation \citep{ENCODEProjectConsortium2012}, the NIH Roadmap Epigenomics Mapping Consortium annotation \citep{Bernstein2010} and other studies, \citep{Finucane2015} tried to identify functional categories that were most enriched in \glng{scz}.
	In their study, it was found that brain cell types were most enriched in \glng{scz}, especially those related to the \gls{cns}.
	Of all the functional categories, the most enriched category in \glng{scz} was the H3K4me3 mark in the fetal brain(\cref{tab:cellTypeScz}). 
	As H3K4me3 was mostly linked to active promoters, this suggest that genes that were activated in fetal brain (e.g. genes related to brain development) were associated with \glng{scz}, supporting the idea of \glng{scz} as a neuro-developmental disorder. 
		
	Moreover, it was also observed that the second most enriched cell types were those related to immunity.
	Undoubtedly, the \gls{cns} and the immune system have an important role in the disease etiology of \glng{scz}. 
		
	\begin{singlespace}
		\begin{longtable}{p{6cm}rrr}
			%\begin{tabular}{rrrr}
			\toprule
			Cell type & cell-type group & Mark  & P-value \\
			\midrule
			Fetal brain** & CNS   & H3K4me3 & $3.09\times 10^{-19}$ \\
			Mid frontal lobe** & CNS   & H3K4me3 & $3.63\times 10^{-15}$ \\
			Germinal matrix** & CNS   & H3K4me3 & $2.09\times 10^{-13}$ \\
			Mid frontal lobe** & CNS   & H3K9ac & $5.37\times 10^{-12}$ \\
			Angular gyrus** & CNS   & H3K4me3 & $1.29\times 10^{-11}$ \\
			Inferior temporal lobe** & CNS   & H3K4me3 & $1.70\times 10^{-11}$ \\
			Cingulate gyrus** & CNS   & H3K9ac & $5.37\times 10^{-11}$ \\
			Fetal brain** & CNS   & H3K9ac & $5.75\times 10^{-11}$ \\
			Anterior caudate** & CNS   & H3K4me3 & $2.19\times 10^{-10}$ \\
			Cingulate gyrus** & CNS   & H3K4me3 & $4.57\times 10^{-10}$ \\
			Pancreatic islets** & Adrenal/Pancreas & H3K4me3 & $2.24\times 10^{-09}$ \\
			Anterior caudate** & CNS   & H3K9ac & $3.16\times 10^{-9}$ \\
			Angular gyrus** & CNS   & H3K9ac & $4.68\times 10^{-9}$ \\
			Mid frontal lobe** & CNS   & H3K27ac & $7.94\times 10^{-9}$ \\
			Anterior caudate** & CNS   & H3K4me1 & $1.20\times 10^{-8}$ \\
			Inferior temporal lobe** & CNS   & H3K4me1 & $3.72\times 10^{-8}$ \\
			Psoas muscle** & Skeletal Muscle & H3K4me3 & $4.17\times 10^{-8}$ \\
			Fetal brain** & CNS   & H3K4me1 & $6.17\times 10^{-8}$ \\
			Inferior temporal lobe** & CNS   & H3K9ac & $9.33\times 10^{-8}$ \\
			Hippocampus middle** & CNS   & H3K9ac & $9.33\times 10^{-7}$ \\
			Pancreatic islets** & Adrenal/Pancreas & H3K9ac & $1.62\times 10^{-6}$ \\
			Penis foreskin melanocyte primary** & Other & H3K4me3 & $2.09\times 10^{-6}$ \\
			Angular gyrus** & CNS   & H3K27ac & $2.34\times 10^{-6}$ \\
			Cingulate gyrus** & CNS   & H3K4me1 & $2.82\times 10^{-6}$ \\
			Hippocampus middle** & CNS   & H3K4me3 & $2.82\times 10^{-6}$ \\
			CD34 primary** & Immune & H3K4me3 & $4.68\times 10^{-6}$ \\
			Sigmoid colon** & GI    & H3K4me3 & $5.01\times 10^{-6}$ \\
			Fetal adrenal** & Adrenal/Pancreas & H3K4me3 & $6.31\times 10^{-6}$ \\
			Inferior temporal lobe** & CNS   & H3K27ac & $8.32\times 10^{-6}$ \\
			Peripheralblood mononuclear primary** & Immune & H3K4me3 & $9.33\times 10^{-6}$ \\
			Gastric** & GI    & H3K4me3 & $1.17\times 10^{-5}$ \\
			Substantia nigra* & CNS   & H3K4me3 & $1.95\times 10^{-5}$ \\
			Fetal brain* & CNS   & H3K4me3 & $2.63\times 10^{-5}$ \\
			Hippocampus middle* & CNS   & H3K4me1 & $3.31\times 10^{-5}$ \\
			Ovary* & Other & H3K4me3 & $6.46\times 10^{-5}$ \\
			CD19 primary (UW)* & Immune & H3K4me3 & $7.08\times 10^{-5}$ \\
			Small intestine* & GI    & H3K4me3 & $8.51\times 10^{-5}$ \\
			Lung* & Cardiovascular & H3K4me3 & $1.17\times 10^{-4}$ \\
			Fetal stomach* & GI    & H3K4me3 & $1.29\times 10^{-4}$ \\
			Fetal leg muscle* & Skeletal Muscle & H3K4me3 & $1.51\times 10^{-4}$ \\
			Spleen* & Immune & H3K4me3 & $1.70\times 10^{-4}$ \\
			Breast fibroblast primary* & Connective/Bone & H3K4me3 & $2.04\times 10^{-4}$ \\
			Right ventricle* & Cardiovascular & H3K4me3 & $2.14\times 10^{-4}$ \\
			CD4+ CD25- Th primary* & Immune & H3K4me3 & $2.19\times 10^{-4}$ \\
			CD4+ CD25- IL17- PMA Ionomycin stim MACS Th sprimary* & Immune & H3K4me1 & $2.19\times 10^{-4}$ \\
			CD8 naive primary (UCSF-UBC)* & Immune & H3K4me3 & $2.24\times 10^{-4}$ \\
			Pancreas* & Adrenal/Pancreas & H3K4me3 & $2.34\times 10^{-4}$ \\
			CD4+ CD25- Th primary* & Immune & H3K4me1 & $2.75\times 10^{-4}$ \\
			CD4+ CD25- CD45RA+ naive primary* & Immune & H3K4me1 & $2.75\times 10^{-4}$\\
			Colonic mucosa* & GI    & H3K4me3 & $3.24\times 10^{-4}$ \\
			Right atrium* & Cardiovascular & H3K4me3 & $3.31\times 10^{-4}$ \\
			Fetal trunk muscle* & Skeletal Muscle & H3K4me3 & $3.39\times 10^{-4}$ \\
			CD4+ CD25int CD127+ Tmem primary* & Immune & H3K4me3 & $3.47\times 10^{-4}$ \\
			Substantia nigra* & CNS   & H3K9ac & $3.63\times 10^{-4}$ \\
			Placenta amnion* & Other & H3K4me3 & $4.17\times 10^{-4}$ \\
			Breast myoepithelial* & Other & H3K9ac & $5.50\times 10^{-4}$ \\
			CD8 naive primary (BI)* & Immune & H3K4me1 & $5.75\times 10^{-4}$ \\
			Substantia nigra* & CNS   & H3K4me1 & $6.61\times 10^{-4}$ \\
			Cingulate gyrus* & CNS   & H3K27ac & $7.94\times 10^{-4}$ \\
			CD4+ CD25- CD45RA+ naive primary* & Immune & H3K4me3 & $8.71\times 10^{-4}$ \\
			\bottomrule
				%\end{tabular}%
			\caption[Enrichment of Top Cell Type of Schizophrenia]{Enrichment of Top Cell type of Schizophrenia.
				* = significant at False Discovery Rate $<$ 0.05.
				** = significant at p $<$ 0.05 after correcting for multiple hypothesis. 
				Reproduce with permission from Journal.\citep{Finucane2015}}
			\label{tab:cellTypeScz}%
		\end{longtable}%
	\end{singlespace}
		
	\subsection{Rare Variants in Schizophrenia}
	\glsreset{cnv}
	The estimated \gls{SNP}-heritability using the common variants captured by the \gls{pgc} \glng{scz} \gls{GWAS} suggest that variants other than common \glspl{SNP} were accounting for the variation in \glng{scz}.
	Based on the ``common disease-rare variant'' hypothesis, another interesting direction of \glng{scz} research will be to identify rare variants associated with \glng{scz}.
	
	\subsubsection{Copy Number Variation}
	A possible source of rare variants can be \glspl{cnv}.
	\gls{cnv} were classified as segment of DNA that is 1kb or larger and that is present at a different copy number when compared to the reference genome, usually in the form of insertion, deletion or duplication \citep{Feuk2006}.
	Due to the length of these variants, the \gls{cnv} might contain the entire genes and their regulatory regions which might in turn contribute to significant phenotypic differences \citep{Feuk2006}.
	
	Recently, \citet{Szatkiewicz2014} conducted a \gls{GWAS} for \gls{cnv} association with \glng{scz} used the Swedish national sample (4,719 \glng{scz} samples and 5,917 controls).
	In their study, they were able to association between \glng{scz} and \gls{cnv} such as 16p11.2 duplications, 22q11.2 deletions, 3q29 deletions and 17q12 duplications were identified.
	Through the gene set association analysis, calcium channel signaling and binding partners of the fragile X mental retardation protein were found to be associated with these \gls{cnv} \citep{Szatkiewicz2014}.
	Interestingly, the calcium channel signaling were also enriched in the \gls{pgc} \gls{GWAS} on \gls{SNP} association, suggesting that the variants were converging on similar set of pathway or gene sets. 
	
	Similarly, \citet{Walsh2008} also found that genes disrupted by structure variants in their cases were significantly overrepresented in pathways important for brain development, including neuregulin signaling, extracellular signal-regulated kinase/\gls{mapk} signaling, 
	synaptic long-term po-tentiation, axonal guidance signaling, integrin signaling, and glutamate receptor signaling \citep{Walsh2008}.
	
	An important observation in these \gls{cnv} studies was that the \gls{cnv}  were generally rare ($\le12$ in 4,719 samples \citep{Szatkiewicz2014}) and has a relative large effect (e.g. odd ratio $>2$ \citep{Szatkiewicz2014,Walsh2008}), following the ``common disease-rare variant'' model.
	
	\subsubsection{Rare Single Nucleotide Mutation}
	Unlike \gls{cnv} which affects a large region, it is difficult to capture rare \gls{SNP} using current genotyping chips.
	Therefore, large scale association of rare \glspl{SNP} was unavailable until the development of the \gls{ngs} technology.
	The \gls{ngs} generates high-throughput sequencing data with per base resolution, allow one to investigate the whole human genome or the human exome without relying on ``tagging''.
	
	Using exome sequencing, \citet{Purcell2014} sequenced the exome of 2,536 \glng{scz} cases and 2,543 normal controls. 
	They were able to identify a common missense allele in \textit{CCHCR1} in the \gls{mhc} that were associated with \glng{scz}.
	Although none of the genes showed a significant burden of rare mutation in cases, a significant increased burden of rare nonsense and disruptive variants was observed in cases in gene sets likely to be associated with \glng{scz} such as voltage-gated calcium ion channel, genes affected by \textit{de novo} mutations in \glng{scz} \citep{Fromer2014} and the postsynaptic density.

	The overlaps between the rare variant studies and the common variant studies suggest that both rare and common variants are likely to be acting upon the same pathway and are complementary to each other.
	
	\section{Environmental Risk Factors of Schizophrenia}
	On top of rare variants, another possible source of ``missing'' heritability can comes from interaction between the genetic and environmental risk factors.
	Although previous studies \citep{Gottesman01071967} suggested that the non-additive genetic factors were unlikely to contribute to \glng{scz}, the possibility of involvement of gene-environmental interaction ($G\times E$) were not ruled out.
	Indeed, in the adoption study conducted by \citet{Tienari2004}, it was found that individuals with higher genetic risk were significantly more sensitive to ``adverse'' vs ``healthy'' rearing patterns in adoptive families than are adoptees at low genetic risk \citep{Tienari2004}.
	Moreover, using the national registers in Finland, \citet{Clarke2009} found that the effect of prenatal infection was five times greater in those who had a family history of psychosis when compared to those who did not. 
	Together, these findings support a mechanism of gene-environment interaction in the causation of schizophrenia.
	
	In order to understand the $G\times E$ interaction, one might need to first understand how environmental factors, especially that of prenatal infection, participate in the development of \glng{scz}.
	
	\subsection{Prenatal Infection}
	\begin{figure}
		\centering
		\includegraphics[width=\textwidth]{figure/risk_factors_of_schizophrenia.png}
		\caption[Risk factors of \glng{scz}]{Risk factors of \glng{scz}.
			It was observed that family history of \glng{scz} was the largest risk factors.
			Risk of \glng{scz} can be more than 9 times higher than the general population for individual with a family history of \glng{scz}}
		\label{fig:riskfactors}
	\end{figure}
	Prenatal infection has always been an important risk factor of \glng{scz}, being the single largest non-genetic risk factor of \glng{scz} (\cref{fig:riskfactors})\citep{Sullivan2005}.
	Initial clues indicated that births during the winter and spring months and in urban areas were related to an increased risk of the disorder \citep{Brown2010}.
	It was also observed that there was an increased risk of \glng{scz} in individuals who were fetuses during the 1957 influenza epidemic \citep{Mednick1958}.
	As the chance of getting infectious disease varies by season and infectious disease can spread more quickly in urban regions due to higher population density, these evidence suggest that prenatal infection might be associated with \glng{scz}.
	
	Early studies of prenatal infection in \glng{scz} mainly relies on ecological data such as influenza epidemics in the population to define the exposure status \citep{Brown2010}.
	The problem of these studies was that the exposure status was based solely on whether an individual was in gestation at the time of the epidemic without any confirmation of maternal infection during pregnancy.
	This leads to difficulties in replication of the findings.
	Subsequently, researchers uses birth cohorts where infection was documented using different biomarkers during pregnancies to provide a better labeling of the exposure status \citep{Brown2010}.
	Through these rigorous studies it was found that the risk of \glng{scz} increases as long as an individual's mother was infected by different form of infectious agents such as influenza, HSV-2 and \textit{T.gondii} during gestation \citep{Brown2010}.
	As different infectious agents all increase the risk of \glng{scz}, it leads to the hypothesis of \gls{mia} \citep{Brown2010} where it was suggested that instead of a particular infectious agents, it was the maternal immune response that disrupt the brain development in the offspring, thus leading to an elevated risk of \glng{scz}.
	
	To really understand how \gls{mia} increase the risk of \glng{scz}, it is important to understand the molecular mechanism.
	A great challenge in the study of \gls{mia} was that one cannot carry out empirical experiment in human samples due to ethical concerns.
	Thus a popular alternative is to employ rodent models.
	However, unlike physiological traits, psychiatric disorder such as that of \glng{scz} often contain symptoms related to higher level functioning such as hallucinations, delusion, disorganized speech etc \citep{AmericanPsychiatricAssociation2013} that are not readily detectable in rodents.
	This raises challenge in diagnosing whether if the rodent has demonstrated the symptoms of \glng{scz} for not only it was difficult to check whether if the high level functioning of the rodent is disrupted, there were no available biomarkers for \glng{scz}.
	Therefore instead of labeling whether if the rodent is ``schizophrenic'' or ``normal'', one would rather consider whether if the rodent demonstrate any ``schizophrenia-like'' behaviours such as impaired prepulse inhibition, impaired working memory and reduced social interaction \citep{Meyer2007a}.
	An important point to note here is that as autism and \glng{scz} shares most of these behavioral abnormality, and that risk of autism is also increased by \gls{mia} \citep{Brown2012}, studies using these rodent models were usually non-specific to \glng{scz} or autism. 
	Rather, autism and \glng{scz} were usually considered together in these models.
	However, the discussion of the etiology of autism and the similarity and difference between autism and \glng{szc} is beyond the scope of the current thesis.
	Therefore, for the simplicity and focus of the current thesis, we would limit our discussion to \glng{scz}.
	
	A common rodent model in the study of effect of \gls{mia} is to use the viral analogue \gls{polyic} to induce the maternal immune response during pregnancy in rodents.
	It was found that offspring exposed to \gls{polyic} displays phenotypes mirrors that observed in schizophrenia \citep{Li2009c,Meyer2009b,Li2010a} such as deficiency in prepulse inhibition \citep{Cadenhead2000}.
	Because \gls{polyic} only induce the \gls{mia} without infecting the fetuses, the \gls{polyic} model provide strong evidence that \gls{mia}, instead of the specific infection, contributes to the increased risk of \glng{scz}.	
	
	\citet{Smith2007} were able to demonstrate that a single injection of \gls{il6} to the pregnant mouse can induce \glng{scz}-like behaviour in the adult offspring. 
	What was most interesting was by eliminating the \gls{il6} from the maternal immune response using either genetic methods (\gls{il6} knock out) or with blocking antibodies, the behaviour deficits associated with \gls{mia} were not present in the adult offspring, suggesting that \gls{il6} is central to the process by which \gls{mia} causes long-term behavioral changes.
	
	Further studies of global gene expression patterns in \gls{mia}-exposed rodent fetal brains \citep{Oskvig2012,Garbett2012a} suggest that the post-pubertal onset of schizophrenic and other psychosis-related phenotypes might stem from attempts of the brain to counteract the environmental stress induced by \gls{mia} during its early development \citep{Garbett2012a}.
	For example, genes with neuroprotective function such as crystallins might also have additional roles in neuronal differentiation and axonal growth \citep{Garbett2012a}. 
	By over-expressing these genes to counteract the environmental stress, the balance between neurogenesis and differentiation in the embryonic brain maybe disrupted. 
	Based on these observations, \citet{Garbett2012a} propose that once the immune activation disappears, the normal brain development programme resumes with a time lag, result in permanent changes in connectivity and neurochemistry that might ultimately leads to \glng{scz}-like behaviours.
	\begin{figure}
		\centering
		\includegraphics[width=\textwidth]{figure/mia_impact.jpg}
		\caption[Hypothesized model of the impact of prenatal immune challenge on fetal brain development]{Hypothesized model of the impact of prenatal immune challenge on fetal brain development.
			Maternal infection in early/mid pregnancy may affect early neurodevelopmental events in the fetal brain, thereby influencing the differentiation of neural precursor cells (grey) into particular neuronal phenotype (yellow or brown).
			This may predispose the developing fetal nervous system to additional failures leading to multiple structural and functional brain abnormalities in later life.
			Figure used with permission from Journal \citep{Meyer2007a}}
		\label{fig:miaEffect}
	\end{figure}
	
	On the other hand, an age dependent structural abnormalities in the mesoaccumbal and nigrostriatal dopamine systems were also found to be induced by \gls{mia} \citep{Vuillermot2010}.
	Specifically, \gls{mia} induces an early abnormality in specific dopaminergic systems such as those in the striatum and midbrian region \citep{Vuillermot2010}.
	Based on these observations, \citet{Meyer2007a} hypothesize that inflammation in the fetal brain during early gestation not only can disrupt neurodavelopmental processes such as cell proliferation and differentiation, it also predispose the developing nervous system to additional failures in subsequent cell migration, target selection, and synapse maturation (\cref{fig:miaEffect}) \citep{Meyer2007a}.
		
	In a separate study by \citet{Giovanoli2013}, mice were exposed to a lower dosage of \gls{polyic} during early gestation.
	Offspring born were then left undisturbed or exposed to unpredictable stress during peripubertal development.
	It was observed that offspring exposed to \gls{polyic} has an increased level of dopamine in the nucleus accumbens independent to whether if they were exposed to postnatal stress whereas serotonin (5-HT) were decreased in the medial prefrontal cortex when exposed to postnatal stress regardless of prenatal exposure.
	Only when the offspring were exposed to both \gls{polyic} and postnatal stress will they have an increased dopamine levels in the hippocampus or will sensorimotor gating and psychotomimetic drug sensitivity be affected \citep{Giovanoli2013}.
	\citet{Giovanoli2013} therefore suggest that the prenatal insult serves as a ``disease primer'' that increase offspring's vulnerability to subsequent insults.
	
	Together, these results supports the involvement of \gls{mia} in the development of \glng{scz}.
	It was even estimated that one third of all \glng{scz} cases could have been prevented shall all infection were prevented from the entire pregnant population \citep{Brown2010}.
	
	\subsection{RNA Sequencing}
	%	Although the \gls{GWAS} in \glng{scz} seems to return a lot of interesting results, the question remains: How much of the genetic variations of \glng{scz} were captured by \gls{GWAS}?
	To answer the question, we need to estimate the heritability based on the \gls{GWAS} data. 
	The challenge however, was that in order to acquire sufficient participants for the association studies, participants were randomly sampled from the population where they were usually not related to each other.
	
	\subsection{\glng{gcta}}
	As the participants were unrelated, one will need to estimate the relativeness between individuals to estimate the heritability.
	Given the genotypes of each individuals were known, one can estimate the ``genetic distance'' between two individual, which can be used to represent their relativeness.
	By calculating the ``genetic distance'' between all individuals within the \gls{GWAS}, one can obtain the \gls{grm} \citep{Yang2011}.
	
	Given that the genotypes were usually coded as 0, 1 or 2 (homozygous reference, heterozygous and homozygous alternative respectively), the genotypes follow the binomial distribution.
	The expected mean and variance of a genotype $i$ will be $2p_i$ and $2p_i(1-p_i)$ respectively where $p_i$ is the frequency of the reference allele.
	Based on this information, \citet{Yang2011} estimates the genetic relationship between individuals $j$ and $k$ as:
	\begin{equation}
	A_{jk} = \frac{1}{N}\sum^N_{i=1}\frac{(x_{ij}-2p_i)(x_{ik}-2p_i)}{2p_i(1-p_i)}
	\end{equation}
	where $x_{ij}$ is the number of copies of the reference allele for the $i^{th}$ \gls{SNP} of the $j^{th}$ individual.
	The effects of all the \glspl{SNP} were then fitted as random effects by a \gls{mlm}
	\begin{align}
	\boldsymbol{y} &= \boldsymbol{X\beta}+\boldsymbol{g}+\epsilon\\
	\mathrm{Var}(\boldsymbol{y}) &= \boldsymbol{A}\sigma_g^2+\boldsymbol{I}\sigma_\epsilon^2
	\end{align}
	where $\boldsymbol{y}$ is an $n\times 1$ vector of phenotypes with $n$ samples, $\beta$ is a vector of fixed effects such as sex and age, $\boldsymbol{g}$ is an $n\times 1$ vector of the total genetic effects of the individuals, $\sigma_g^2$ is the variance explained by all the \glspl{SNP} and finally, $\sigma_\epsilon^2$ is the variance explained by residual effects.
	
	The main concept of the method is that instead of testing the associations for individual \glspl{SNP}, one fit the effects of all SNPs as random effects in a \gls{mlm} and estimate a single parameter, i.e. the variance explained by all \glspl{SNP} or \gls{SNP}-heritability.
	Given the information of the \gls{grm}, \citet{Yang2011} implemented the \gls{reml} using the average information algorithm to estimates the $\sigma_g^2$ and $\sigma_\epsilon^2$where the \gls{reml} is a form of maximum likelihood estimation that allows unbiased estimates of variance and covariance parameters.
	The heritability of the trait is then defined as $\frac{\sigma_g^2}{\sigma_g^2+\sigma_e^2}$.
	
	Based on the above concept, \citet{Yang2010a} were able to estimate the variance in height explained by \glspl{SNP} from the height \gls{GWAS} to be around 45\%, much larger than previously reported 5\%.
	The main difference in the estimates was because the \gls{mlm} \gls{reml} were able to consider all \glspl{SNP} simultaneously without limited on significant \glspl{SNP}.
	Although the estimates was still less than 80\% which was the expected heritability of height, \citet{Yang2010a} was able to demonstrated that one possible source of ``missing heritability'' might be due to incomplete \gls{LD}.
	By taking into consideration of incomplete \gls{LD}, it was estimated that the proportion of variance explained by causal variants can be as high as 0.84 with \gls{se} of 0.16 \citep{Yang2010a}, close to the expected heritability.
	Together, \citet{Yang2011} provide a possible method for the estimation of the variance explained by \glspl{SNP} in \gls{GWAS} data and the method is now implemented in \gls{gcta} which is wildly adopted.
	
	The problem with \gls{gcta} was that genotype data are required to calculate the \gls{grm}.
	For complex disease like \glng{scz}, the data were usually obtained from multiple data source where the raw genotypes were usually unavailable due to privacy concerns.
	Instead, summary statistics were usually provided.
	Therefore estimation of variance explained by \glspl{SNP} in these \gls{GWAS} can only rely on the summary statistics. 
	  
	\subsection{\glng{ldsc}}
	An important observation of \gls{GWAS} results was that sometimes a general inflation of test statistics can be observed. 
	It was usually considered to be contributed to the presence of confounding factors such as population stratification, under the assumption that most of the \glspl{SNP} should have no association to the disease.
	It was therefore a common practice for one to perform the \gls{gc} on the \gls{GWAS} results \citep{Zheng2006}.
	
	The problem of the \gls{gc} was that the basic assumption of a small number of causal \glspl{SNP} might not be true. 
	Through careful simulation, \citet{Yang2011b} demonstrated that in the absence of population stratification and other form of technical artifacts, the presence of polygenic inheritance can also inflate the test statistic \citep{Yang2011b}.
	More importantly, they observed that the magnitude of inflation was determined by the \emph{heritability}, the \gls{LD} structure, sample size and the number of causal \glspl{SNP} of the trait.

	Following on this observation, \citet{Bulik-Sullivan2015} developed the \gls{ldsc}.
	The fundamental concept of \gls{ldsc} was that the more genetic variants a \gls{SNP} tag (e.g. in high \gls{LD} with many other \glspl{SNP}), the more likely for it to be able to tag a causal variant; 
	whereas population stratification and cryptic relatedness should not be associated with \gls{LD}. 
	The number of genetic variants tagged by a \gls{SNP}$_j$ ($l_j$)(\gls{LD} score) was then defined as the sum of $r^2$ of $k$ \glspl{SNP} within a 1cM window of \gls{SNP}$_j$:
	\begin{equation}
	l_j = \sum_kr^2_{jk}
	\label{eq:ldScore}
	\end{equation}
	
	When there is no confounding factors, the expected $\chi^2$ of \gls{SNP}$_j$ can be defined as a function of the \gls{LD} score ($l_j$), the number of samples ($N$), the number of \glspl{SNP} in the analysis($M$) and most importantly, the heritability ($h^2$):
	\begin{equation}
	\mathrm{E}[\chi^2_j | l_j] = \frac{Nh^2}{M}l_j+1
	\label{eq:fixedLDSC}
	\end{equation}
	%TODO understand the exact calculation of LDSC

	Interestingly, when confounding factors were present in the study (e.g. population stratification), \cref{eq:fixedLDSC} simply becomes
	\begin{equation}
	\mathrm{E}[\chi^2_j | l_j] = \frac{Nh^2}{M}l_j+Na+1
	\label{eq:fullLDSC}
	\end{equation}
	where $a$ is the contribution of confounding bias.
	%The equation becomes more complicated when one want to incorporate the contribution of confounding factors to the equation.
	%For example, large scale \gls{GWAS} can often suffer from population stratification.
	%To account for the population stratification, \citet{Bulik-Sullivan2015} introduced the genetic drift term $f$ as $f\sim N(0, F_{ST}V)$ where $V$ is a correlation matrix and $F_{ST}$ is Wright's Fixation index ($F_{ST}$).
	%They made the assumption where the LD Score of variant $j$ in the populations are the same.
	%Although this assumption is not reasonable when the population is very different (e.g. they are from different continents), the authors argues that they were only interested in modeling the the population stratification that might remain \emph{after} principal components analysis in \gls{GWAS} and thus their assumption are reasonable.
	%Furthermore, it was noted that there was a large correlation observed between the \gls{LD} score for all pairs of 1000 Genomes European subpopulations. 
	% I disagree with them here. It is uncertain how the correction of the population stratification affects the LD pattern or the relationship between the test statistics, thus one require a more rigid study before such a strong assumption can be made.
	
%	If one express the \gls{LD} score and the $\chi^2$ as vectors ($\boldsymbol{L}$ and $\boldsymbol{\chi^2}$) respectively, \cref{eq:fullLDSC} becomes a regression of the $\chi^2$ against the \gls{LD} score:
%	\begin{equation}
%	\boldsymbol{\chi^2}= \frac{N}{M}\boldsymbol{L}h^2+Na+1
%	\label{eq:ldReg}
%	\end{equation}
	
	As a result of that, the heritability $h^2$ will be the slope of the regression and the intercept minus one will represent the mean contribution of the confounding bias such as those of population stratification. 
	Thus, \cref{eq:fullLDSC} can be used for the estimation of heritability given that only the summary statistics and population \gls{LD} were provided. 
	
	
	Although \gls{ldsc} can be used for heritability estimation, the main focus of \citet{Bulik-Sullivan2015}'s paper was to delineating the confounding factors from the 	polygenicity of a trait.
	To test whether if \gls{ldsc} can delineate the confounding factors such as cryptic relationship and population stratification from the polygenicity of a trait, \citet{Bulik-Sullivan2015} simulated multiple \gls{GWAS} where the trait can have a polygenic architecture or where confounding factors can present.
	When the simulated trait is polygenic and no confounding factors were presented, the average \gls{ldsc} intercept was close to one, as expected by the authors.
	Only when the number of causal variants was small will the standard error of the estimates become very large.
	However, even with the enlarged variance, the estimates remained unbiased.
	On the other hand, when the \gls{GWAS} was simulated with only the confounding factors such as population stratification, the intercept estimated was approximately equal to the \gls{gc} inflation factor with only a small positive bias in the regression slope.
	
	The most important simulation was to investigate the performance of \gls{ldsc} in \gls{GWAS} of a polygenic trait where confounding factors were present. 
	It was found that even when both polygenicity and confounding factors were presented in the \gls{GWAS}, the intercept of \gls{ldsc} was approximately equal to the mean $\chi^2$ statistic among null \glspl{SNP}, providing strong evidence that \gls{ldsc} can partition the inflation in test statistic even in the presence of both bias and polygenicity.
	
	Although the main focus of \citet{Bulik-Sullivan2015} was not the estimation of \gls{SNP}-heritability in their paper, they did provide an estimate of variance explained by the \glspl{SNP} in the \gls{pgc} \glng{scz} \gls{GWAS} \citep{Ripke2014}.
	By applying the liability threshold adjustment, \citet{Bulik-Sullivan2015} estimated the \gls{SNP}-heritability of \glng{scz} should be 0.555 with \gls{se} of 0.008.
	The estimated \gls{SNP} heritability was lower than the heritability estimated from population based study (64\% \citep{Lichtenstein2009}) and twin studies (81\% \citep{Sullivan2003}) suggesting that it is possible for variants other than common \glspl{SNP} were accounting for the variation in \glng{scz}.
	An example will be \gls{cnv} which was also found to be associated with \glng{scz} \citep{Szatkiewicz2014}.
	
	Another possibility of the``missing'' heritability can be due to interaction between the genetic and environmental factors. 
	Although previous studies \citep{Gottesman01071967} suggested that the non-additive genetic factors were unlikely to contribute to \glng{scz}, the possibility of involvement of gene-environmental interaction $G\times E$ were not ruled out.
	Indeed, in the adoption study conducted by \citet{Tienari2004}, it was found that individuals with higher genetic risk were significantly more sensitive to ``adverse'' vs ``healthy'' rearing patterns in adoptive families than are adoptees at low genetic risk \citep{Tienari2004}, providing support to a possible interaction between genetic and environmental factors.
	Therefore, in order to account for the ``missing'' heritability, one might need to consider genetic variations other than \glspl{SNP} and might need to take into consideration of the $G\times E$ interaction.
	On the other hand, as demonstrated by \citet{Yang2010a}, the ``missing'' heritability might simply because of incomplete \gls{LD} between the \glspl{SNP} and the causal variants. 
	A possible method would be to perform sequencing studies variants across the whole genome can be detected at the same time.
	
	Nonetheless, the \gls{SNP}-heritability estimation from \citet{Ripke2014} were still encouraging, as for the first time in genetic research of \glng{scz}, a large portion of heritability of \glng{scz} were finally identified.
	This permit the genetic research of \glng{scz} to move beyond statistical association and focus on the functional basis of the genetic susceptibility locus of \glng{scz}.
	
	\subsection{Partitioning of Heritability of Schizophrenia}
	\subsectionmark{Partitioning of Heritability}
	Traditionally, functional enrichment analysis in \gls{GWAS} only take into account of \glspl{SNP} that passed the genome wide significance threshold. 
	However, for complex traits such as that of \glng{scz}, much of the heritability might lies in \glspl{SNP} that do not reach genome wide significance threshold at the current sample size.
	For example, in 2013, only 13 risk loci were detected using 13,833 \glng{scz} samples and 18,310 controls \citep{Ripke2013}. 
	When the sample size increased to 34,241 \glng{scz} samples and 45,604 controls in 2014, 108 risk loci were identified \citep{Ripke2014}. 
	Thus, if one only consider the significant loci, risk loci that have not reach genome wide significance threshold might be ignored from the analysis, decreasing the power of the functional enrichment analysis.
	
	Unlike traditional functional enrichment analysis, \gls{ldsc} uses information from all \glspl{SNP} and taking into account of the \gls{LD} structure to partition heritability into different functional categories. 
	It should therefore be more powerful when compared to traditional analysis and should help to provide useful insight into the disease etiology of \glng{scz}.

	\citet{Finucane2015} used data from \citet{Ripke2014} and functional categories derived from the ENCODE annotation \citep{ENCODEProjectConsortium2012}, the NIH Roadmap Epigenomics Mapping Consortium annotation \citep{Bernstein2010} and other studies \citep{Finucane2015} to identify functional categories that were most enriched in \glng{scz}.
	In their study, it was found that brain cell types were most enriched in \glng{scz}, especially those related to the \gls{cns}.
	Of all the functional categories, the most enriched category in \glng{scz} was the H3K4me3 mark in the fetal brain(\cref{tab:cellTypeScz}). 
	As H3K4me3 was mostly linked to active promoters, this suggest that genes that were activated in fetal brain (e.g. genes related to brain development) were associated with \glng{scz}, supporting the idea of \glng{scz} as a neuro-developmental disorder. 
	
	Moreover, it was also observed that the second most enriched cell types were those related to immunity.
	Undoubtedly, the \gls{cns} and the immune system have an important role in the disease etiology of \glng{scz}. 

	\begin{singlespace}
	\begin{longtable}{p{6cm}rrr}
		%\begin{tabular}{rrrr}
			\toprule
			Cell type & cell-type group & Mark  & P-value \\
			\midrule
			Fetal brain** & CNS   & H3K4me3 & $3.09\times 10^{-19}$ \\
			Mid frontal lobe** & CNS   & H3K4me3 & $3.63\times 10^{-15}$ \\
			Germinal matrix** & CNS   & H3K4me3 & $2.09\times 10^{-13}$ \\
			Mid frontal lobe** & CNS   & H3K9ac & $5.37\times 10^{-12}$ \\
			Angular gyrus** & CNS   & H3K4me3 & $1.29\times 10^{-11}$ \\
			Inferior temporal lobe** & CNS   & H3K4me3 & $1.70\times 10^{-11}$ \\
			Cingulate gyrus** & CNS   & H3K9ac & $5.37\times 10^{-11}$ \\
			Fetal brain** & CNS   & H3K9ac & $5.75\times 10^{-11}$ \\
			Anterior caudate** & CNS   & H3K4me3 & $2.19\times 10^{-10}$ \\
			Cingulate gyrus** & CNS   & H3K4me3 & $4.57\times 10^{-10}$ \\
			Pancreatic islets** & Adrenal/Pancreas & H3K4me3 & $2.24\times 10^{-09}$ \\
			Anterior caudate** & CNS   & H3K9ac & $3.16\times 10^{-9}$ \\
			Angular gyrus** & CNS   & H3K9ac & $4.68\times 10^{-9}$ \\
			Mid frontal lobe** & CNS   & H3K27ac & $7.94\times 10^{-9}$ \\
			Anterior caudate** & CNS   & H3K4me1 & $1.20\times 10^{-8}$ \\
			Inferior temporal lobe** & CNS   & H3K4me1 & $3.72\times 10^{-8}$ \\
			Psoas muscle** & Skeletal Muscle & H3K4me3 & $4.17\times 10^{-8}$ \\
			Fetal brain** & CNS   & H3K4me1 & $6.17\times 10^{-8}$ \\
			Inferior temporal lobe** & CNS   & H3K9ac & $9.33\times 10^{-8}$ \\
			Hippocampus middle** & CNS   & H3K9ac & $9.33\times 10^{-7}$ \\
			Pancreatic islets** & Adrenal/Pancreas & H3K9ac & $1.62\times 10^{-6}$ \\
			Penis foreskin melanocyte primary** & Other & H3K4me3 & $2.09\times 10^{-6}$ \\
			Angular gyrus** & CNS   & H3K27ac & $2.34\times 10^{-6}$ \\
			Cingulate gyrus** & CNS   & H3K4me1 & $2.82\times 10^{-6}$ \\
			Hippocampus middle** & CNS   & H3K4me3 & $2.82\times 10^{-6}$ \\
			CD34 primary** & Immune & H3K4me3 & $4.68\times 10^{-6}$ \\
			Sigmoid colon** & GI    & H3K4me3 & $5.01\times 10^{-6}$ \\
			Fetal adrenal** & Adrenal/Pancreas & H3K4me3 & $6.31\times 10^{-6}$ \\
			Inferior temporal lobe** & CNS   & H3K27ac & $8.32\times 10^{-6}$ \\
			Peripheralblood mononuclear primary** & Immune & H3K4me3 & $9.33\times 10^{-6}$ \\
			Gastric** & GI    & H3K4me3 & $1.17\times 10^{-5}$ \\
			Substantia nigra* & CNS   & H3K4me3 & $1.95\times 10^{-5}$ \\
			Fetal brain* & CNS   & H3K4me3 & $2.63\times 10^{-5}$ \\
			Hippocampus middle* & CNS   & H3K4me1 & $3.31\times 10^{-5}$ \\
			Ovary* & Other & H3K4me3 & $6.46\times 10^{-5}$ \\
			CD19 primary (UW)* & Immune & H3K4me3 & $7.08\times 10^{-5}$ \\
			Small intestine* & GI    & H3K4me3 & $8.51\times 10^{-5}$ \\
			Lung* & Cardiovascular & H3K4me3 & $1.17\times 10^{-4}$ \\
			Fetal stomach* & GI    & H3K4me3 & $1.29\times 10^{-4}$ \\
			Fetal leg muscle* & Skeletal Muscle & H3K4me3 & $1.51\times 10^{-4}$ \\
			Spleen* & Immune & H3K4me3 & $1.70\times 10^{-4}$ \\
			Breast fibroblast primary* & Connective/Bone & H3K4me3 & $2.04\times 10^{-4}$ \\
			Right ventricle* & Cardiovascular & H3K4me3 & $2.14\times 10^{-4}$ \\
			CD4+ CD25- Th primary* & Immune & H3K4me3 & $2.19\times 10^{-4}$ \\
			CD4+ CD25- IL17- PMA Ionomycin stim MACS Th sprimary* & Immune & H3K4me1 & $2.19\times 10^{-4}$ \\
			CD8 naive primary (UCSF-UBC)* & Immune & H3K4me3 & $2.24\times 10^{-4}$ \\
			Pancreas* & Adrenal/Pancreas & H3K4me3 & $2.34\times 10^{-4}$ \\
			CD4+ CD25- Th primary* & Immune & H3K4me1 & $2.75\times 10^{-4}$ \\
			CD4+ CD25- CD45RA+ naive primary* & Immune & H3K4me1 & $2.75\times 10^{-4}$\\
			Colonic mucosa* & GI    & H3K4me3 & $3.24\times 10^{-4}$ \\
			Right atrium* & Cardiovascular & H3K4me3 & $3.31\times 10^{-4}$ \\
			Fetal trunk muscle* & Skeletal Muscle & H3K4me3 & $3.39\times 10^{-4}$ \\
			CD4+ CD25int CD127+ Tmem primary* & Immune & H3K4me3 & $3.47\times 10^{-4}$ \\
			Substantia nigra* & CNS   & H3K9ac & $3.63\times 10^{-4}$ \\
			Placenta amnion* & Other & H3K4me3 & $4.17\times 10^{-4}$ \\
			Breast myoepithelial* & Other & H3K9ac & $5.50\times 10^{-4}$ \\
			CD8 naive primary (BI)* & Immune & H3K4me1 & $5.75\times 10^{-4}$ \\
			Substantia nigra* & CNS   & H3K4me1 & $6.61\times 10^{-4}$ \\
			Cingulate gyrus* & CNS   & H3K27ac & $7.94\times 10^{-4}$ \\
			CD4+ CD25- CD45RA+ naive primary* & Immune & H3K4me3 & $8.71\times 10^{-4}$ \\
			\bottomrule
		%\end{tabular}%
		\caption[Enrichment of Top Cell Type of Schizophrenia]{Enrichment of Top Cell type of Schizophrenia.
			* = significant at False Discovery Rate $<$ 0.05.
			** = significant at p $<$ 0.05 after correcting for multiple hypothesis. 
			Reproduce with permission from Journal.\citep{Finucane2015}}
		\label{tab:cellTypeScz}%
	\end{longtable}%
	\end{singlespace}
	
%	\subsection{Genetic Correlation}
%	Another very important application of \gls{ldsc} is that it allow one to identify the genetic correlation between traits\citep{Bulik-Sullivan2015a}. 
%	The genetic correlation can be used as a genetic analogue to co-morbidity, thus allowing deeper understanding to the etiology of the traits.
%	Above all, genetic correlation was important in studying the treatment response. 
%	It has been observed that there was an increased prevalence of anxiety, depression and substance abuse in \glng{scz} \citep{Buckley2009}. 
%	These co-morbidity were generally associated with more severe psychopathology and with poorer outcome \citep{Buckley2009}.
%	A deeper understanding of possible co-morbidity between different traits and \glng{scz} might provide insight not only to the disease etiology of \glng{scz}, it might even provide important information in possible treatment options for \glng{scz}. 
%	Using breast cancer as an example, it was found that patients with comorbidity had poorer survival than those without comorbidity \citep{Sogaard2013} and it was suggested that by treating the comorbid diseases, one might be able to delay mortality in breast cancer patients \citep{Ording2013}.
%		
%	By applying their method to 25 different phenotypes, \citet{Bulik-Sullivan2015a} shown that \glng{scz} has significant genetic correlation with bipolar disorder, major depression and more surprisingly, anorexia nervosa.
%	Previous studies have always suggest there to be a co-morbidity between \glng{scz} and bipolar disorder \citep{Lichtenstein2009,Purcell2009,Buckley2009}.
%	Similarly, it was not uncommon for \glng{scz} to display depressive symptoms \citep{Buckley2009}. 
%	It was even observed that individuals at high risk and ultrahigh risk for developing \glng{scz} have generally demonstrated a significant degree of depressive symptoms prior to and during the emergence of psychotic symptoms, suggesting a close relationship between \glng{scz} and depression. 
%	
%	On the other hand, the genetic correlation between \glng{scz} and anorexia nervosa were slightly unexpected for there has been a lack of study in the co-morbidity between eating disorder and \glng{scz}. 
%	Nonetheless, this finding raises the possibility of similarity between anorexia and nervosa.
	% Serotonergic system have been implicated in depression, negative symptoms of sczhiophrneia and eating disorders \citep{Arranz2007}.
	%	\section{Antipsychotics}
	Despite the success in the genetic research of \glng{scz}, an effective cure of \glng{scz} was yet to be found.
	Currently, the main treatment method for \glng{scz} was the use of antipsychotic drugs to reduce symptoms and prevent relapse. 
	However, there was a large variability between individuals in their response to treatment, some might even suffer from adverse side effects such as agranulocytosis and \gls{td}.
	Thus, it is vital to administrate the antipsychotics according to individual conditions.
	Unfortunately, there was a lack of understanding of the factors influencing the drug response, forcing clinicians to administrate antipsychotics on a trial and error process.
	There is a therefore a pressing need for better understanding treatment response in \glng{scz} such that an optimal treatment can be provided for the patients.
	
	\subsection{History of Antipsychotic}
	Early research in treatment of \glng{scz} largely follows a random trial and error process where methods such as prolonged sleep treatment, insulin coma therapy and pharmacoconvulsive treatment were proposed\citep{Lehmann1997}.
	The first antipsychotic drug Chlorpromazine, a phenothiazine, were developed in early 1950s.
	Subsequently within a period of less than 10 years, 20 other antipsychotic phenothiazine were in development.
	Collectively, they were considered as the \glspl{fga}.
	
	\glspl{fga} were found to be extremely effective in reducing the positive symptoms of schizophrenia such as delusions, hallucinations and disorganized thinking.
	However, the \glspl{fga} were found to be ineffective against negative and cognitive symptoms, and might even cause acute \gls{eps} such as parkinsonism, dysphoria and tardive dyskinesia, making them unpopular among patients\citep{Tandon2007}.
	
	In 1966, a new drug, name Clozapine was introduced\citep{Lehmann1997}. 
	Clozapine has been shown to be more effective when compared to \glspl{fga} and was less likely to cause \gls{eps} and tardive dyskinesia.
	Moreover, it was shown to reduce suicidality and was more effective in reducing negative and cognitive symptoms\citep{Lehmann1997,Tandon2007}.
	Despite the superior performance of Clozapine, it found to be associated with the severe and potentially lethal adverse side effect, agranulocytosis\citep{Alvir1993}, limiting its use as a first line treatment of \glng{scz}\citep{Remington2013}.
	Subsequently, a number of antipsychotic were developed in hope of a ``safe clozapine'' which have the same level of effectiveness as clozapine and not having the adverse side effects.
	These were considered as the \glspl{sga} which includes risperidone, olanzapine and quetiapin.
	Although the \glspl{sga} tends to have lower risk for \gls{eps}, they tends to be associated with significant metabolic side effects such as weight gain, diabetes mellitus and hyperlipidemia\citep{UCOK2008}.
	
	\subsection{Mechanism of Action of Antipsychotic}
	The difference between the \glspl{fga} and \glspl{sga} provides valuable information on possible mechanisms associated with treatment response and adverse side effects such as \gls{eps}.
	It was first demonstrated on 1963 that \glspl{fga} tends to block the dopamine receptors\citep{Lehmann1997} and it was hypothesized that the binding of dopamine receptors, especially the D$_2$ receptors were required for reduction of positive symptoms\citep{Arranz2007}. 
	Indeed, it was found that dopamine receptor blockade was not unique to \glspl{fga} but was also required for \glspl{sga} and there has yet been any successful antipsychotic drugs that works without dopamine D$_2$ blockade\citep{Zhang2011}.
	However, it was observed that there were significant differences between \glspl{fga} and \glspl{sga} affinities. 

	When compared to \glspl{fga}, \glspl{sga} have a lower affinity for and occupancy at the D$_2$ receptors and tends to have a more diverse receptor binding profiles.
	For example, the ratio between affinity of serotonin receptor (5-HT$_2$) to that of the D$_2$ receptor were significantly greater (15.8 times) for \glspl{sga} when compared to \glspl{fga}\citep{Meltzer1991}.
	These leads to two competing hypothesis of antipsychotic action: 
	the serotonin-dopamine hypothesis, which stated that the ratio of serotonin 5-HT$_2$ to dopamine D$_2$ affinity was the main mechanism accounting for the superior performance of \glspl{sga};
	and the dopamine hypothesis which stated that the modulation of the dopamine D$_2$ receptor was the single most important factor affecting the performance of the antipsychotic\citep{Kapur2003}.
	
	One common characteristics for most \glspl{sga} except amisulpride was their affinity to the serotonin receptors such as 5-HT$_2$. 
	It was therefore suggested that the reduction of negative symptoms were resulting from the serotonin blockade and the serotonin-dopamine interactions were important to the antipsychotic drug actions\citep{Meltzer1999}.
	However, amisulpride serves as a counter example to the serotonin-dopamine hypothesis.
	Amisulpride is a \gls{sga} that does not have any affinity for serotonin receptors yet have comparable performance in reduction of negative response when compared to olanzapine\citep{Kumar2014}.
	Thus, serotonin receptor blockade might not be required for the reduction of negative symptoms.
	
	% Serotonin -> SGA all bind better at serotonin when compared to Dopamine
	Moreover, \gls{pet} studies have shown that a minimum occupancy of 60\%-65\% of striatal D$_2$ like receptors is required to obtain clinical response whereas D$_2$ occupancy of above 80\% is considered as the main cause of \gls{eps}\citep{Arranz2007,Kapur2003}.
	Upon further investigation, it was found that clozapine preferentially target the mesolimbic dopamine system while sparing the nigrostriatal dopamine system\citep{Gardner1993}.
	This raise the possibility that the main difference between \glspl{fga} and \glspl{sga} was the preferential blockade of cortical dopamine D$_2$ receptors compared with striatal dopamine D$_2$ receptors\citep{Kapur2003}.
	Based on these observation, it was now hypothesized that \glng{scz} was a result of both ``hypodopaminergia'' in the prefrontal cortex and ``hyperdopaminergia'' in the straitum, with a possible involvement of the glutamate system\citep{Howes2009}.
	
	It was worth noting that most clinical studies of \glspl{sga} were sponsored by industry, leading to questions of their validity. 
	Two government lead clinical trial, \gls{catie}\citep{Lieberman2005} and CUtLASS\citep{Jones2006}, were therefore performed to provide unbiased comparison between \glspl{fga} and \glspl{sga}.
	Unfortunately, the superior performance of \glspl{sga} over \glspl{fga} were not observed nor were the \glspl{sga} associated with better cognitive or social outcomes.
	It therefore seems like the only advantages of \glspl{sga} over \glspl{fga} were the reduced risk of adverse side effects such as \gls{eps} and \gls{td}.
	
	\subsection{Antipsychotic Response}
	Although the government lead studies does not support \glspl{sga}'s role in reducing negative and cognitive symptoms of \glng{scz}, there is without doubt that \glspl{sga} were better in terms of reduced risk of \gls{eps} and \gls{td}. 
	There is no question that a better treatment is required yet it is just as important to learn how to better utilize currently available antipsychotics. 
	Simply a better understanding of factors behind the variation in individual responses to different antipsychotic drugs will be extremely beneficial. 
	It will allow researchers to categorize people by their personal profile and provide the most optimal antipsychotic drug for their treatment. 
	
	It is worth noting that the antipsychotic drug response is a multidimensional problem which not only focus in the reduction of symptoms, but the instance of adverse drug effect is also an important research focus. 
	However, due to limited scope of the current thesis, we will focus only on studies on the reduction of symptoms.
	
	\subsubsection{Positive and Negative Symptom Scale (PANSS)}		
	In order to study the response of antipsychotic, it is important to have an objective scale to quantify the reduction of symptoms.
	The \gls{panss}\citep{Kay1987} were among one of the most commonly scale used to measure the core symptoms of \glng{scz} and is composed of 3 subscales: positive, negative and general psychopathology.
	There were a total of 30 different symptoms included in \gls{panss} and each symptoms were rated from 1 to 7, thus the minimal score for \gls{panss} is 30.
	To calculate the percentage reduction of \gls{panss}, which represent a reduction in severity of symptoms, the reduction of \gls{panss} will then be divided by the original \gls{panss} minus 30:
	$$
		\%\text{improvement} = \frac{\text{PANSS}_{after}-\text{PANSS}_{before}}{\text{PANSS}_{before}-30}\times 100\%
	$$
	
	\subsubsection{Factors Associated with Antipsychotic Responses}
	Factors such as diet, smoking and concomitant medications were known to significantly affect metabolic enzyme activity rates, thus have an impact to antipsychotic treatment response\citep{Arranz2011}.
	On the other hand, clinical features such as treatment adherence and duration of illness; individual variation such as gender and ethnicity all influence the treatment efficacy\citep{Arranz2011}.
	 
	Considering the heritability of \glng{scz} were up to 80\%, genetic variations can explain much of the variation in \glng{scz}. 
	Therefore, people hypothesize that the genetic variations might also be able to explain much of the variation in antipsychotic drug response.
	However, although there were incidence report of concordance of response in \gls{mz} twin data\citep{Vojvoda1996,Mata2001}, the sample size were not enough for heritability estimation(studies usually consist of only one pair of twins).
	Nonetheless, these studies shades lights on the possibility that variation in antipsychotic response might be able to be explained by genetic variations of individuals.
	

	\subsection{Pharmacogenetics and Pharmacogenomics}
	Given that genetic variations might be able to explain the variation in antipsychotic drug response, it was therefore compelling to study the association between genetic variations and antipsychotic drug response.
	The terms ``pharmacogenetics'' and ``pharmacogenomics'' were introduced to define study of variability in drug response due to genetic variations and can be used interchangeably\citep{Pirmohamed2001}. 
	Before the popularity of \gls{GWAS}, pharmacogenetic studies were mainly conducted based on the candidate gene approach.
	Genes targeted by antipsychotic drugs such as genes coding for dopamine receptors and serotonin receptors were among the major target of research.
	Similarly, genes involve in the metabolizing the antipsychotic drugs such as the P450 family of enzymes were extensively studied.
	\subsubsection{Dopamine Receptors}
	The dopamine D$_2$ receptor plays a critical role in antipsychotic drug action, with D$_2$ receptor antagonism considered to be necessary and sufficient for antipsychotic drug efficacy\citep{Kapur2003}.
	As such, polymorphisms on the \textit{DRD2} gene, which codes for the D$_2$ receptor, were extensively studied. 
	The \gls{SNP}(rs1799732) representing a deletion at position -141, which was located in the 5' promoter region of \textit{DRD2} were found to be able to influence the density of D$_2$ receptor density in the striatum in healthy samples unexposed to antipsychotic drug treatment\citep{Arinami1997}.
	A significant difference in response rate between deletion carrier and patients with homozygous insertion genotype were observed (odds ratio = 0.65, 95\% \gls{ci}: 0.43-0.97), indicating patients who carry one or two deletion allele were more likely to have less favorable antipsychotic drug responses.

	Other than the D$_2$ receptor, most antipsychotics also shown similar affinity for the dopamine D$_3$ receptor\citep{Sokoloff2006}, leading to pharmacogenetic studies of variants on the \textit{DRD3} gene, which codes for the D$_3$ receptors.
	Much of the research were focused on the \gls{SNP}(rs6280) coding for the serine to glycine substitution at amino acid position 9 in the N-terminal extracellular domain of the D$_3$ receptor protein.
	It was suggested that the dopamine has 4-5 times higher affinity to the glycine-9 variant when compared to the serine-9 variant\citep{Jeanneteau2006}, thus it was hypothesized that the serine to glycine substitution might modulate the antipsychotic drug response. 
	Interestingly, it was found that the serine allele was associated with better response to \glspl{fga} but was associated with non-response to clozapine treatment\citep{Zhang2011}. 
	However, this finding was not replicated and there was yet any consistent evidence of the association of the serine to glycine substitution with antipsychotic response\citep{Zhang2011}.
	
	\subsubsection{Serontonin Receptors}
	Serontonin receptors, especially the 5-HT$_{2A}$ receptors first gain attention because of it critical involvement in the pathophysiology of hallucinations\citep{Aghajanian1999}, leading to speculation of its role in the etiology of \glng{scz} where hallucinations is one of the main symptoms.
	Although there are debates on the importance of serontonin receptors in antipsychotic drug responses\citep{Kapur2003}, pharmacogenetic studies on serotonin receptors such as the 5-HT$_{2A}$ receptors remains popular.
	
	Polymorphisms on \textit{HTR2A} gene, which codes for the 5-HT$_{2A}$ receptors, were extensively studied. 
	The synonymous \gls{SNP} at codon 10(T102C,rs6313) and the A-1438G \gls{SNP}(rs6311) in the promoter region of \textit{HTR2A} are in complete \gls{LD}. 
	It was found that the C allele of the T102C \gls{SNP}, together with the G allele of the A-1438G \gls{SNP} might cause lower promoter activities of \textit{HTR2A} and may decrease the 5-HT$_{2A}$ densities in some brain areas\citep{Zhang2011}.
	However, results from studies on the association of T102C and A-1438G have not reach an agreement\citep{Zhang2011}.
	
	\subsubsection{Cytochrome P450 enzymes}
	There are many other factors that might affect the antipsychotic drug response.
	For example, the time course of the absorption, the bioavailability, the distribution of the drug in the body, the excretion of the drugs and the metabolism of the drugs all influences the efficacy of antipsychotics.
	Genetic variants in enzymes mediating these factors are therefore interesting target for pharmacogenetic studies.
	
	The Cytochrome P450 enzyme family, including CYP1A1, CYP2A6, CYP2C8,
	CYP2C9, CYP2C19, CYP2D6, CYP2E1, CYP3A5 and many others,  in the liver is one of the major target of pharmacogenetic studies because of its role in metabolizing many of the antipsychotic drugs\citep{Cacabelos2011}.
	Around 40\% of antipsychotics are major substrate for CYP2D6\citep{Cacabelos2011} making it an ideal target to study.
	There are more than 100 genetic variations observed on the \textit{CYP2D6} gene and by combining different alleles, the CYP2D enzyme can be categorized based on the degrees of the enzymatic activities: poor metabolizer, intermediate metabolizer, extensive metabolizer(normal) and ultra-rapid metabolizer\citep{Zhang2011}.
	It was hypothesized that individuals' CYP2D enzymatic activities can affect the level of drugs in their blood. 
	For example, people with \textit{CYP2D} alleles from the poor metabolizer categories were expected to have a higher drug levels in the blood when compared to people with \textit{CYP2D} alleles from the ultra-rapid metabolizer categories.
	
	Although there were data suggesting the association of poor metabolizer with higher rate of drug induced side effects\citep{Ravyn2013}, most studies to date have been unable to provide sufficient evidence to support the use of Cytochrome 450 genotype testing to improve therapeutic
	efficacy in the use of antipsychotic medications\citep{Ravyn2013}.
	However, \citet{Ravyn2013} do agree that the use of cytochrome 450 genotype testing might help to prevent adverse side effects in patients receiving some antipsychotics such as Risperidone and Aripiprazole.
	
	\subsubsection{Genome Wide Association of Antipsychotic Drug response}
	Despite the usefulness of the candidate gene approach, it was restricted by our limited knowledge regarding the mechanism behind antipsychotic response.
	With the popularization of \gls{GWAS} and advancement of sequencing technology, we now have the ability to perform association on variants across the whole human genome, allowing a hypothesis free approach.
	
	The \gls{catie} project conducted a total of four \gls{GWAS}, on phenotype such as antipsychotic treatment response\citep{McClay2011}, antipsychotic-induced
	Parkinsonism\citep{Alkelai2009}, movement related adverse antipsychotic effect\citep{Aberg2010} and metabolic side effects\citep{Adkins2011}.
	For the study of antipsychotic treatment response\citep{McClay2011}, a total of 738 subjects from the \gls{catie} project, each from different ethnic background, were genotyped.
	\Gls{pca} were performed to control for subtle and extensive variation due to both genomic and experimental features.
	Based on \citet{VandenOord2009}, it was assumed that it takes on average about 30 days for treatment to exert an effect. 
	Therefore the total \gls{panss} score change within a 30 days period, along with change of the five scale \gls{panss}, including positive symptoms, negative symptoms, disorganization symptoms, excitement and emotional distress within a 30 days period were used to represent the treatment effect.
	Because of variation in efficacy for different antipsychotic drugs, the treatment effect of the five antipsychotic used(olanzapine, quetiapine, isperidone, ziprasidone and perphenazine) were estimated independently. 
	In total, there were 30 \gls{panss} outcome measured (5 drugs $\times$ 6 \gls{panss} scales) and were associated with the genotypes.
	
	Unfortunately, none of the \glspl{SNP} passed through the genome wide significance threshold(p-value$\le5\times10^{-8}$).
	When considering the \gls{fdr} instead, rs17390445 was found to be significantly associated with change in positive symptoms score when Ziprasidone were administrated(q-value$=0.049$).
	The rs17390445 is located in the intergenic region of chromosome 4p15 and does not associate with any genes. 
	On the other hand, \gls{SNP} in the Ankrin Repeat and Sterile Alpha Motif Domain-Containing Protein 1B gene(\textit{ANKS1B}) was found to be associated in change in negative symptoms when Olanzapine was administrated and \gls{SNP} in the Contactin-Associated Protein-Like 5 gene(\textit{CNTNAP5}) was found to be associated with change in negative symptoms when Risperidone were administrated.

	Despite being the largest \gls{GWAS} on antipsychotic treatment response, the sample size per drug group were relatively small ($\sim150$ samples per drug group) compared to other \gls{GWAS} on psychiatric phenotypes.
	Similar to \glng{scz}, it was hypothesized that the antipsychotic treatment response is more likely to be affected by rare variants with large effect or common variants with modest effect\citep{Jorgensen2008}. 
	As such, a large number of samples will be required to provide adequate power in genetic association study on antipsychotic treatment outcome.
	Given the current sample size of \gls{catie} and only by assuming all antipsychotic drugs efficacy were influenced by the same genetic variant, one can at best detect a common (\gls{maf}$\le$ 0.2) genetic variant with odd ratio of 2 or above\citep{Jorgensen2008}.	
	However, the calculation in \citet{Jorgensen2008} did not take into account of \gls{LD} and the calculated power was likely to be over-estimated.
	Therefore, the \gls{catie} \gls{GWAS} is likely to be under-powered and might contain large amount of false negative results.
	
	Another problem faced by the \gls{catie} \gls{GWAS} was the large non-adherence rate. 
	74\% of patients discontinued the study medication before the 18 months period ends\citep{Lieberman2005} with almost 30\% stopped medication because of `patient's decision'.
	It was estimated that with a sample size of 400 and non-adherence rate of 50\%, the power of the study will be less than 0.7 and the power might further drop to below 0.4 when the non-adherence rate reaches 70\%\citep{Malhotra2012}.
	Considering that the non-adherence rate in \glng{scz} ranges from 20 to 70\%\citep{Malhotra2012}, it was more than likely that majority of the samples were not adhered to their medication, thus decreasing the power of the study.
	
	On the other hand, chronic \glng{scz} patients were recruited for the \gls{catie} study, which were associated with an increased duration of psychotic symptoms, increased likelihood of substance abuse, and functional/social disabilities that may influence drug response rates and confound the result of association\citep{Zhang2013}.
	Previous treatment of antipsychotic might also confound the results for a better dose can be given to patients based on previous treatments.
	
	Nonetheless, although no genome-wide significant \gls{SNP} was identified in the \gls{catie} \gls{GWAS}, a number of \gls{SNP} were marginally significant. 
	By increasing the study power, we might start to identify some of the genetic variants that are associated with antipsychotic treatment response.
	With the increased knowledge in \glng{scz} with the success of \gls{pgc}, we might soon be welcoming a better clinical application of the genetic data in treatment of \glng{scz}, helping the \glng{scz} patients to have a better quality of life.
	
	
	\section{Summary}
	To conclude, \glng{scz} is a complex disorder affecting approximately 1\% of the population worldwide. 
	It is now known that the disease is affected by a combination of genetic and environmental factors. 
	Therefore, to fully understand the disease mechanism for the development of proper treatments, it is important not only to examine how certain genetic polymorphisms can predispose individuals to the disease development, but also how environmental factors can act as a trigger for the disorder in apparently healthy individuals.  
	
	In this thesis, we would like to develop an algorithm for the estimation of \gls{SNP} heritability from \gls{GWAS} summary statistics that is robust to traits with different genetic architectures. 
	We would also like to investigate the effect of case control sampling and extreme phenotype sampling on the performance of \gls{ldsc}.
	On the other hand, as prenatal infection is the largest environmental risk for \glng{scz}, we would like to understand how prenatal infection triggers \glng{scz} through studying the change in global gene expressions in mice cerebellum.
	
	First, in \Cref{heritabilityChapter}, we performed a series of empirical simulations to assess the performance of \gls{ldsc} in the estimation of \gls{SNP} heritability. 
	We also proposed an alternative approach for the estimation of \gls{SNP}-heritability from \gls{GWAS} summary statistics that is robust to different genetic architectures.
	
	In \Cref{omegaProject}, a hypothesis generation study was performed to study the effect of \gls{mia} on the gene expression pattern of mouse cerebellum. 
	On top of that, as recent study suggested that n-3 \gls{pufa} rich diet can help to reduce the \glng{scz}-like behaviour observed mouse exposed to early \gls{mia} \citep{Li2015}, we also investigated the effect of n-3 \gls{pufa} rich diet on the gene expression pattern of mouse cerebellum.
	
	Lastly, we summarize and conclude all findings in \Cref{conclusionChapter} and give future perspectives on the \glng{szc} research.
	