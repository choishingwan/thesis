\documentclass[12pt]{scrbook}
\usepackage{amsmath}
\usepackage{bm}
\usepackage{colortbl}
\usepackage{graphicx}
\usepackage{caption}
\usepackage{fullpage}
\usepackage{afterpage}
\usepackage{float}
\usepackage{multirow}
\usepackage[nodisplayskipstretch]{setspace}
\usepackage{booktabs}
\usepackage{textcomp}
\usepackage{gensymb}
\usepackage[utf8]{inputenc} 
%\usepackage{parskip}
\usepackage{xr}
\usepackage{pdflscape}
\usepackage[labelfont=bf,tableposition=top]{caption}
\usepackage[inline]{enumitem}
\usepackage{wrapfig}
\usepackage{longtable}
\usepackage{subfig}
\usepackage[hidelinks]{hyperref}
\usepackage{geometry}
\usepackage{fancyhdr}
\usepackage[acronym,nomain,toc,makeindex ]{glossaries}
\usepackage{graphicx}
\usepackage[section]{placeins}
\usepackage{eufrak}
\usepackage[table]{xcolor} 
\usepackage{cmap}
\usepackage{cleveref} %This need to be the last package for it to work properly 
\usepackage[hyperref=true,maxcitenames=3,maxbibnames=3,url=false,doi=false,isbn=false,backref=true,natbib=true,backend=biber,sorting=nyvt,defernumbers=false, style=authoryear]{biblatex}
\setcounter{secnumdepth}{3}
%
%
%	TITLE PAGE DEFINITION
%
%
\renewcommand*\contentsname{Contents}

\newcommand{\rom}[1]{\uppercase\expandafter{\romannumeral #1\relax}}
\newcommand*{\glng}{\glsentrylong}
\newcommand*{\Glng}{\Glsentrylong}
\newcommand{\beginsupplement}{%
	\setcounter{table}{0}
	\renewcommand{\thetable}{S\arabic{table}}%
	\setcounter{figure}{0}
	\renewcommand{\thefigure}{S\arabic{figure}}%
}
\newcommand{\specialcell}[2][c]{\begin{tabular}[#1]{@{}c@{}}#2\end{tabular}}

%
%	GLOSSARY SECTION
%
\addbibresource[datatype=bibtex]{citation/citation.bib}
\makeglossary
\newacronym{scz}{SCZ}{schizophrenia}
\newacronym[longplural={Genome Wide Association Studies}]{GWAS}{GWAS}{Genome Wide Association Study}
\newacronym{wgs}{WGS}{Whole Genome Sequencing}
\newacronym{SNP}{SNP}{Single Nucleotide Polymorphism}
\newacronym{LD}{LD}{Linkage Disequilibrium}
\newacronym{PGS}{PGS}{Polygenic Risk Score}
\newacronym{tSVD}{tSVD}{Truncated Singular Value Decomposition}
\newacronym{SVD}{SVD}{Singular Value Decomposition}
\newacronym{shrek}{SHREK}{SNP HeRitability Estimation Kit}
\newacronym{gcta}{GCTA}{Genome-wide Complex Trait Analysis}
\newacronym{ldsc}{LDSC}{LD SCore regression}
\newacronym{CEU}{CEU}{Northern Europeans from Utah}
\newacronym{se}{SE}{standard error}
\newacronym{maf}{MAF}{Minor Allele Frequency}
\newacronym{isc}{ISC}{International Schizophrenia Consortium}
\newacronym{mgs}{MGS}{Molecular Genetic of Schizophrenia}
\newacronym{sgene}{SGENE}{Schizophrenia Genetics Consortium}
\newacronym{pgc}{PGC}{Psychiatric Genomics Consortium}
\newacronym{rin}{RIN}{RNA integrity number}
\newacronym{gd}{GD}{Gestation Day}
\newacronym{rpkm}{RPKM}{Reads Per Kilobase per Million mapped reads}
\newacronym{wgcna}{WGCNA}{Weighted Gene Co-expression Network Analysis}
\newacronym{pc}{PC}{Principle Component}
\newacronym{GO}{GO}{Gene Ontology}
\newacronym{MAGMA}{MAGMA}{Multi-marker Analysis of GenoMic Annotation}
\newacronym{ngs}{NGS}{next generation sequencing}
\newacronym{dsm}{DSM}{Diagnostic and Statistical Manual of Mental Disorders}
\newacronym{mse}{MSE}{mean squared error}
\newacronym{who}{WHO}{World Health Organization}
\newacronym{yld}{YLD}{years lost due to disability}
\newacronym{ci}{CI}{confidence interval}
\newacronym{iq}{IQ}{Intelligence Quotient}
\newacronym{polyic}{PolyI:C}{polyriboinosinic-polyribocytidilic acid}
\newacronym{lps}{LPS}{lipopolysaccharide}
\newacronym{il6}{IL-6}{Interleukin-6}
\newacronym{mz}{MZ}{monozygotic}
\newacronym{dz}{DZ}{dizygotic}
\newacronym{mia}{MIA}{maternal immune activation}
\newacronym{ncp}{NCP}{non-centrality parameter}
\newacronym{cnv}{CNV}{copy number variation}
\newacronym{grm}{GRM}{Genetic Relationship Matrix}
\newacronym{gc}{GC}{Genomic Control}
\newacronym{mhc}{MHC}{major histocompatibility complex}
\newacronym{cns}{CNS}{central nervous system}
\newacronym{panss}{PANSS}{Positive and Negative Symptom Scale}
\newacronym{fga}{FGA}{First Generation Antipsychotic}
\newacronym{sga}{SGA}{Second Generation Antipsychotic}
\newacronym{nmda}{NMDA}{N-methyl-D-aspartate}
\newacronym{eps}{EPS}{extrapyramidal motor symptoms}
\newacronym{td}{TD}{tardive dyskinesia}
\newacronym{pet}{PET}{positron emission tomography}
\newacronym{catie}{CATIE}{Clinical Antipsychotic Trials of Intervention Effectiveness}
\newacronym{pca}{PCA}{principle component analysis}
\newacronym{fdr}{FDR}{false discovery rate}
\newacronym{pcgc}{PCGC}{Phenotype correlation - genotype correlation regression}
\newacronym{scidp}{SCID-P}{Structured Clinical Interview for DSM-IV-TR Axis I Disorders, Research Version, Patient Edition}
\newacronym{ibd}{IBD}{identity by descent}
\newacronym{bp}{bp}{base pair}
\newacronym{mb}{mb}{megabase}
\newacronym{kb}{kb}{kilobase}
\newacronym{deg}{DEG}{differentially expressed gene}
\newacronym{qc}{QC}{quality control}
\newacronym{hku}{HKU}{the University of Hong Kong}
\newacronym{pcr}{PCR}{polymerase chain reaction}
\newacronym{ercc}{ERCC}{External RNA Controls Consortium}
\newacronym{go}{GO}{Gene Ontology}
\newacronym{rtpcr}{rt-PCR}{real time PCR}
\newacronym{ct}{CT}{cycle threshold}
\newacronym{geo}{GEO}{Gene Expression Omnibus}
\newacronym{mlm}{MLM}{mixed linear model}
\newacronym{reml}{REML}{restricted maximum likelihood}
\newacronym{pufa}{PUFA}{polyunsaturated fatty acid}
\newacronym{lrt}{LRT}{likelihood ratio test}
\newacronym{msigdb}{MSigDB}{Molecular Signatures Database}
\newacronym{fgf}{FGF}{fibroblast growth factor}
\newacronym{egf}{EGF}{epidermal growth factor}
\newacronym{pi3k}{PI3K}{phosphatidylinositol 3-kinase}
\newacronym{mapk}{MAPK}{mitogen-activated protein kinase}
\newacronym{gaba}{GABA}{$\gamma$-aminobutyric acid}
\newacronym{ecm}{ECM}{extracellular matrix}
\newacronym{mmp}{MMP}{matrix metalloproteinase}
\newacronym{qqplot}{QQ-plot}{quantile-quantile Plot}
\newacronym{cm}{cM}{centiMorgan}
\newacronym{mrna}{mRNA}{messenger RNA}
\newacronym{psd}{PSD}{postsynaptic density}
\newacronym{kegg}{KEGG}{Kyoto Encyclopedia of Genes and Genomes}
\newacronym{arc}{ARC}{neuronal activity-regulated cytoskeleton-associated protein}
\newacronym{cb1}{CB1}{cannabinoid receptor type 1}
%
%
%	FORMATING SECTION
%
%
\pagestyle{fancy}
\fancyhf{}
\fancyfoot[LE,RO]{\thepage}
\renewcommand{\footrulewidth}{1pt}
\fancyhead[LE]{\leftmark}
\fancyhead[RO]{\rightmark}

\geometry{
	top=1in,            % <-- you want to adjust this
	inner=1in,
	outer=3.5cm,
	bottom=3.5cm,
	headheight=3ex,       % <-- and this
	headsep=2ex,          % <-- and this
}


\raggedbottom %Remove it before printing as this is something to do with global settings. Can make each page look uneven but more dense. 
%\onehalfspacing
\doublespacing
%TODO this is the include space
%\includeonly{heritability_estimation}
%\includeonly{omega_project}
%\includeonly{heritability_estimation,omega_project}
%\includeonly{literature_review}Z


%\extrafloats{100}






\makeindex
\begin{document}\thispagestyle{empty}
\pagestyle{empty}

%\maketitle
\begin{singlespace}
	\begin{titlepage}
		\begin{center}
			\vspace*{1cm}
			
			\Huge
			\textbf{The Genetics of Schizophrenia - The Contribution of Common Variants and Genes Implicated in a Maternal Immune Activation Model}
			
			\vspace{0.5cm}
			\LARGE
			
			\vspace{1.5cm}
			
			\textbf{\href{mailto:choishingwan@gmail.com}{Choi Shing Wan}}
			
			\vfill
			
			A thesis submitted in partial fulfillment of the requirements for \\
			the Degree of Doctor of Philosophy
			
			\vspace{0.8cm}
			
			\includegraphics[width=0.4\textwidth]{figure/hkuLogo.jpg}
			
			\Large
			Department of Psychiatry\\
			University of Hong Kong\\
			Hong Kong\\
			\today
		\end{center}
	\end{titlepage}
\end{singlespace}


\frontmatter 
	\cleardoublepage
	\phantomsection
	\addcontentsline{toc}{chapter}{Abstract}
	\chapter*{\centerline{Abstract}}
	\vspace{0.1cm}
	\Glng{scz} is a disabling disorder affecting approximately 1\% of the population worldwide.
	To fully understand disease mechanisms for the development of effective treatments, it is important not only to examine how certain genetic polymorphisms can predispose individuals to develop, but also how environmental factors triggers onset of the disorder in apparently healthy individuals. 
	
	\gls{GWAS} is now a standard approach for investigating associations of common genetic variations (mainly \glspl{SNP}) with \glng{scz}. 
	A recent meta-analysis of \gls{GWAS} of \glng{scz} has identified 108 loci significantly associated with \glng{scz}. 
	However, it is possible for other genetic factors such as rare variants to contribute to the disease etiology of \glng{scz}. 
	Estimating the overall contribution of common \glspl{SNP} to \glng{scz} therefore has important implications for future research strategy.
	
	In this thesis, we propose a novel approach for estimating the contribution of \glspl{SNP} to \glng{scz} (\gls{SNP}-heritability) from \gls{GWAS} summary statistics, called the \gls{shrek}.
	Our simulation results suggest that, when compared to an existing method, \gls{ldsc}, \gls{shrek} provided a more robust estimate for oligo-gentic traits and in case-control designs where no confounding variables was present. 
	Using the summary statistics from the latest meta-analysis of \gls{GWAS} on \glng{scz}, we estimated that \glng{scz} has a \gls{SNP}-heritability of 0.185 (SD=0.00450), which is similar to the estimate of 0.198 (SD=0.0057) by \gls{ldsc}.
	The result indicate that common \glspl{SNP} have a relatively small contribution to the genetic predisposition to \glng{scz} when compared to the overall heritability estimated from twin and family studies. 
	Therefore, alternative strategies like whole-genome sequencing may be necessary for identifying additional \glng{scz} genes.
	
	Previous studies have reported the interaction between genetic variation and prenatal infection in the etiology of \glng{scz}.
	There are evidences that the effect of prenatal infection is mediated by maternal immune response, thus it is likely for the perturbation induced by \gls{mia} to interact with genetic variations in the development of \glng{scz}.
	
	We therefore performed a RNA-sequencing study to investigate whether there are any genetic overlaps between differential genes induced by \gls{mia} and genetic variations detected by \glng{scz} \gls{GWAS} using the \gls{polyic} mouse model.
	We found that the functional gene sets associated with \glng{scz} are also enriched in \gls{mia}.
	In addition, when investigating the treatment effect of n-3 \gls{pufa} rich diet in \gls{mia}, we found that the gene expression of \textit{Sgk1}, a gene that regulates the glutamatergic system, is affected by the n-3 \gls{pufa} rich diet in the \gls{polyic} exposed mice. 
	It is therefore possible for \textit{Sgk1} to mediate the treatment effect of n-3 \gls{pufa} rich diet in the \gls{mia} model. 
	In conclusion, our results suggested that genes related to neural function and calcium ion signaling, as well as glutamate-related genes such as \textit{Sgk1}, are the potential targets for future \glng{scz} research.
	
	\begin{flushright}
		(467 words)
	\end{flushright}
	\glsresetall
	\cleardoublepage
	\phantomsection
	\addcontentsline{toc}{chapter}{Declaration}
	\newgeometry{inner=2in, outer=2in}
	\chapter*{\centerline{Declaration}}
	\vspace{1cm}
	I declare that this thesis represents my own work, except where due acknowledgments	is made, and that it has not been previously included in a thesis, dissertation or report submitted to this University or to any other institution for a degree, diploma or other qualification.
	\vspace{1.5cm}
	
	\centerline{Signed....................................................................}
	\centerline{Choi Shing Wan}
	\restoregeometry
	\cleardoublepage
	\phantomsection
	\addcontentsline{toc}{chapter}{Acknowledgments}
	\chapter*{\centerline{Acknowledgements}}
	\vspace{1cm}
	I would like to express my deepest gratitude to Professor Pak Sham.
	I am eternally grateful for his trust, supervision, patience and support in the course of my study.
	I would also like to thanks Dr Stacey Cherny and Dr Wanling Yeung for giving me valuable advice for my projects.
	My special thanks go to Dr Johnny Kwan.
	He has provided critical advices on my projects and has taught me a great deal in the field of statistic.
	
	The past 4 years has been a blast and I really enjoy my time in this department.
	This is only possible because of all the great people here.
	Thank you Beatrice Wu, Dr Li Qi, Tomy Hui, Vicki Lin, Nick Lin, John Wong, Dr Clara Tang, Dr Amy Butler, Dr Emily Wong, Dr Allen Gui, Dr Sylvia Lam, Yung Tse Choi, Oi Chi Chan, Pui King Wong and Dr Miaoxin Li, without you everything will be much different.
	I will forever cherish the time I spent with you. 
	
	Words alone cannot express my gratitude to Beatrice Wu and my family.
	Their support and encouragement have been my greatest source of energy and have helped me to continue on with my study. 
	
	\begin{center}
	\large THANK YOU!
	\end{center}
	\cleardoublepage
	\phantomsection
	\begin{singlespace}
	\printglossary[title=Abbreviations,toctitle=Abbreviations,style=long,nonumberlist]
	\cleardoublepage
	\phantomsection
	%To generate the correct abbreviations, use alt+shift+F1 twice before using F1
	
	\cleardoublepage
	\phantomsection
	\addcontentsline{toc}{chapter}{Contents}
	
		\setcounter{tocdepth}{3}
		\tableofcontents
		\listoffigures
		\listoftables
	\end{singlespace}
\mainmatter
\pagestyle{fancy}

\setlength{\parindent}{4em}
\setlength{\parskip}{0.75em}
	
		\chapter{Introduction}
	% Disease background
	\section{Schizophrenia}
	\Gls{scz} is a devastating psychiatric disorder affecting approximately 0.3--0.7\% of the population worldwide \citep{AmericanPsychiatricAssociation2013}.
	According to the \gls{dsm}-\rom{5}, which is one of the standard diagnostic tools in psychiatry, a diagnosis of \glng{scz} (F20.9) can only be reached if the patient has suffered from 2 or more of the following symptoms for a significant portion of time during a 1-month period: 
	\begin{enumerate*}[label=\arabic*\upshape)]
		\item delusion; \label{ls:delusion}
		\item hallucination;\label{ls:hallucinations}
		\item disorganized speech;\label{ls:disorganizedSpeech}
		\item grossly disorganized or catatonic behaviour; and\label{ls:catatonicBehavior}
		\item negative symptoms such as diminished emotional expression,\label{ls:negativeSymptoms}
	\end{enumerate*}  where one of the symptom must be either (\ref{ls:delusion}, (\ref{ls:hallucinations} or (\ref{ls:disorganizedSpeech}, which are known as positive symptoms.
	Signs of disturbance need to persist for at least 6-month before the patient can be diagnosed with \glng{scz}.
	Current medical treatment of \glng{scz}, based on dopamine D2 receptor blockage, is effective only for the amelioration of positive symptoms in approximately 2/3 of patients.
	
	Because of its disabling symptoms and the lack of entirely effective treatments, \glng{scz} imposes a serious and long lasting health, social and financial burden to patients and their families \citep{Knapp2004}. 
	\Glng{scz} patients also have an increased tendency to commit suicide \citep{Saha2007}.
	Based on an \gls{who} report, \glng{scz} was one of the top 20 leading cause of \gls{yld} in 2012, ranking 16 among all possible causes (\cref{tab:whoYLD}).
	\begin{table}[ht]
		\centering
		\caption[Top 20 leading causes of \glng{yld}]{Top 20 leading causes of \gls{yld} calculated by \gls{who} in year 2012.
			\Glng{scz} was considered as one of the top 20 leading causes of \gls{yld} \citep{Geneva2013}.}
		\begin{tabular}{rp{5cm}rrr}
			\toprule
			Rank  & Cause & \gls{yld} (000s) & \% \gls{yld} & \specialcell[b]{\gls{yld} per \\100k population}\\
			\midrule
			0     & All Causes & 740,545 & 100   & 10466 \\
			1     & Unipolar depressive disorders & 76,419 & 10.3  & 1080 \\
			2     & Back and neck pain & 53,855 & 7.3   & 761 \\
			3     & Iron-deficiency anaemia & 43,615 & 5.9   & 616 \\
			4     & Chronic obstructive pulmonary disease & 30,749 & 4.2   & 435 \\
			5     & Alcohol use disorders & 27,905 & 3.8   & 394 \\
			6     & Anxiety disorders & 27,549 & 3.7   & 389 \\
			7     & Diabetes mellitus & 22,492 & 3     & 318 \\
			8     & Other hearing loss & 22,076 & 3     & 312 \\
			9     & Falls & 20,409 & 2.8   & 288 \\
			10    & Migraine & 18,538 & 2.5   & 262 \\
			11    & Osteoarthritis & 18,096 & 2.4   & 256 \\
			12    & Skin diseases & 15,744 & 2.1   & 223 \\
			13    & Asthma & 14,134 & 1.9   & 200 \\
			14    & Road injury & 13,902 & 1.9   & 196 \\
			15    & Refractive errors & 13,498 & 1.8   & 191 \\
			16    & Schizophrenia & 13,408 & 1.8   & 189 \\
			17    & Bipolar disorder & 13,271 & 1.8   & 188 \\
			18    & Drug use disorders & 10,620 & 1.4   & 150 \\
			19    & Endocrine, blood, immune disorders & 10,495 & 1.4   & 148 \\
			20    & Gynecological diseases & 10,227 & 1.4   & 145 \\
			\bottomrule
		\end{tabular}%
		\label{tab:whoYLD}%
	\end{table}%
	In view of its severity, \glng{scz} has drawn much attention from the research community to delineate disease etiology and mechanisms, and identify risk factors associated with \glng{scz}.
	Ultimately, the goal of \glng{scz} research is to identify effective treatment(s) to improve the quality of life of patients.
	
	\section{Understanding Disease Etiology}
	An important first step in \glng{scz} research is to understand whether if genetic or environmental variation contribute more to the disease etiology.
	A measure of the relative contribution of genetic and environmental influences to individual differences in the liability to a disorder is \emph{heritability}.
	There are two definitions of heritability: the broad sense heritability and the narrow sense heritability.
	Broad sense heritability is defined as the \emph{proportion} of total variance of a trait in a population explained by the \emph{total} variation of genetic factors in the population, whereas the narrow sense heritability only takes into account of the variation of \emph{additive} genetic factors in the population instead of the total variation of genetic factors.
	
	\subsection{Broad Sense Heritability}
	For any phenotype, one can partition it into a combination of genetic and environmental components \citep{Falconer1996}
	$$
	\text{Phenotype (P)}=\text{Genotype (G)}+\text{Environment (E)}
	$$
	In the absence of gene-environmental correlation or interaction, the variance of the observed phenotype ($\sigma_P^2$) can be expressed as the sum of the variance of genotype ($\sigma_G^2$) and variance of environment ($\sigma_E^2$)
	$$
	\sigma_P^2=\sigma_G^2+\sigma_E^2
	$$
	The ratio between the variance of the observed phenotype and the variance of the genetic effects is then defined as the broad sense heritability:
	$$
	H^2=\frac{\sigma_G^2}{\sigma_P^2}
	$$
	
	One key feature of heritability is that it is a \emph{ratio} of \emph{population} measurements at a specific time point.
	As a result, the heritability of a trait can differ in different strata of the same population (because of differences in the environment), and in different populations (because of differences in both genes and environment).
	A classic example is \gls{iq}, which increases in heritability with increasing age \citep{Bouchard2013}.
	It was hypothesized that the shared environment has a relatively larger effect on individuals when they were young, and gradually diminishes when they grow older and become more independent.
	The reduction in shared environmental influences results in an \emph{increased portion} of variance in \gls{iq} explained by genetic differences \citep{Bouchard2013}. 
	
	The definition of heritability becomes more complicated when we take into account different forms of genetic effects; this leads to the concept of narrow sense heritability.
	
	\subsection{Narrow Sense Heritability}
	The effects of genes are not always additive but can differ depending on the other gene at the same locus (dominance) or genes at different loci (epistasis).
	As a result, one can partition the total genetic variance into variance due to additive genetic effects ($\sigma_A^2$), variance due to dominant genetic effects ($\sigma_D^2$), and variance due to other epistatic genetic effects ($\sigma_I^2$), as follows:
	$$
	\sigma_G^2=\sigma_A^2+\sigma_D^2+\sigma_I^2
	$$
	
	As individuals only transmit one copy of each gene at a single genetic locus to their offspring, relatives other than full siblings and identical twins will only share a maximum of one gene for each locus.
	Considering that dominance and epistatic genetic effects are interactive effect, which usually involve more than one gene, these effects are unlikely to contribute substantially to the resemblance between relatives other than monozygotic twins and full siblings \citep{Visscher2008}.
	On the other hand, the additive genetic effects are usually transmitted from parent to offspring, thus it is more useful to consider the narrow sense heritability ($h^2$) which only includes the additive genetic effects, when predicting parent-offspring resemblance:
	\begin{align}
	h^2&=\frac{\sigma_A^2}{\sigma_P^2} \notag\\
	h^2&=\frac{\sigma_A^2}{\sigma_G^2+\sigma_E^2}
	\label{eq:narrowHeritability}
	\end{align}
	
	To obtain the additive genetic effect, we can first consider the genetic effect of a parent to be $G_p=A+D$. 
	As only half of the additive effect were transmitted to their offspring, the child will have a genetic effect of $G_c=\frac{1}{2}A+\frac{1}{2}A'+D'$ where $A'$ is the additive genetic effect obtained from another parent by random and $D'$ is the non-additive genetic effect in the offspring.
	If we then consider the parent offspring covariance, we will get
	\begin{align}
	\mathrm{Cov_{OP}}&= \sum(\frac{1}{2}A+\frac{1}{2}A'+D')(A+D)\notag\\
	&=\frac{1}{2}\sum A^2+\frac{1}{2}\sum AD + \frac{1}{2}\sum A'(A+D) +D'(A+D) \notag\\ 
	&=\frac{1}{2}V_A+ \frac{1}{2}\mathrm{Cov}_{AD} + \frac{1}{2}\mathrm{Cov}_{A'A} + \frac{1}{2}\mathrm{Cov}_{A'D} +\mathrm{Cov}_{D'A} +\mathrm{Cov}_{D'D}  
	\label{eq:halfCompletedCovOP}
	\end{align} 
	Under the assumption of random mating,  $A'$ should be independent from $A$ and $D$. 
	Moreover, as $D'$ was specific to the child, it should be independent from $A$ and $D$, with the covariance between the additive genetics and non-additive genetics being zero \citep{Falconer1996}.
	Thus, \cref{eq:halfCompletedCovOP} becomes
	\begin{align}
	\mathrm{Cov_{OP}} &= \frac{1}{2}V_A+\mathrm{Cov}_{AD} \notag\\
	&= \frac{1}{2}V_A
	\label{eq:covOP}
	\end{align}
	Now if we assume the variance of phenotype of the parent and offspring were the same, then using \cref{eq:covOP}, we can obtain the narrow-sense heritability as
	\begin{align}
	h^2 &= \frac{1}{2}\frac{V_A}{\sigma_P^2}
	\label{eq:narrowHerit}
	\end{align}

	In the simple linear regression equation $Y=X\beta+\epsilon$, the regression slope can be calculated as 
	\begin{equation}
	\beta_{XY} = \frac{\mathrm{Cov}_{XY}}{\sigma_{X}{Y}}
	\end{equation}
	which resemble \cref{eq:narrowHerit}. 
	Therefore,  we can calculate the narrow sense heritability as
	\begin{equation}
	h^2 = 2\beta_{OP}
	\label{eq:narrowSenseHerit}
	\end{equation}
	where $\beta_{OP}$ is the slope of the simple linear regression regressing the phenotype of an offspring to the phenotype of \emph{one} of its parents.
	We can further generalize \cref{eq:narrowSenseHerit} to all possible relativeness 
	\begin{equation}
	h^2=\frac{\beta_{XY}}{r}
	\label{eq:finalNarrow}
	\end{equation}
	where $r$ is the relativeness of $X$ and $Y$.
	
	A key assumption in this calculation is that only additive genetic factors are shared among relatives.
	However, this is very unlikely to be entirely true as relatives do tends to be in the same cultural group and might have similar socio-economic status.
	These might all contribute to the variance of the trait, thus lead to bias in \cref{eq:finalNarrow} and we shall discuss the partitioning of variance in the later sections.
	
	Nonetheless, \cref{eq:finalNarrow} provide a simple example to the calculation of the narrow sense heritability.
	However, in the case of discontinuous trait (e.g. disease status) the calculation becomes more complicated because the variance of the phenotype was dependent on the population prevalence.
	As \cref{eq:finalNarrow} does not account for the trait prevalence, it cannot be directly applied to discontinuous traits.
	In order to perform heritability estimation on discontinuous trait, the concept of liability threshold model proposed by \cite{Falconer1965} is necessary with the calculation.
	
	\subsection{Liability Threshold}
	\label{sec:liability}
	According to the central limit theorem, if a phenotype is determined by a multitude of genetics and environmental factors with relatively small effect, then its distribution will likely follow a normal distribution as is the case of many quantitative traits \citep{Visscher2008}. % No, what if there is interaction between variables? Then it will break the CLT
	The variance of phenotype can therefore be calculated as the variance under the normal distribution.
	However, such is not the case for disease like \glng{scz} where only a dichotomous disease status (``affected'' and ``normal'') are obtained.
	The variance of these phenotypes are therefore more difficult to obtain.
	
	\citet{Falconer1965} proposed the liability threshold model, which suggests that these discontinuous traits also follow a continuous distribution with an additional parameter called the ``liability threshold''.
	Under the liability threshold model, the discontinuous traits are assumed to be affected by combination of multitude of genetics and environmental factors, each with small effects.
	The main difference is that the phenotype of an individual is determined by whether if the combined effects of these factors (``liability'') are above a particular threshold (``liability threshold'') (\cref{fig:liability}), e.g. only when an individual has a liability above the liability threshold will he/she be affected.
	\begin{figure}
		\centering
		\includegraphics[width=0.5\textwidth]{figure/liability.png}
		\caption[Liability Threshold Model]{
			The liability threshold model.
			Only when an individual has a liability above the liability threshold will he/she be affected.
			}
			\label{fig:liability}
	\end{figure}
	One can then estimate the heritability of the discontinuous trait by comparing the mean liability of the general population when compared to the relatives of the affected individuals.	
	For example, if we consider a single threshold model of a dichotomous trait, where 
	\begin{align}
	T_G &= \text{Liability threshold of the general population}\notag\\
	T_R &= \text{Liability threshold of relatives of the index case} \notag\\
	q_G &= \text{Prevalence in the general population}\notag\\
	q_R &= \text{Prevalence in relatives of the index case}\notag\\
	L_a &= \text{Mean Liability of the index case} \notag
	\end{align}
	by assuming both the liability distribution of the general population and the relative of the index case both follows the standard normal distribution, we can align the two distributions with respect to $T_G$ and $T_R$. 
	We can then calculate the mean liability of the index case $L_a$ as $L_a=\frac{z_G}{q_G}$ where $z_G$ is the density of the normal distribution at the liability threshold $T_G$.
	Then we can express the regression of relatives' liability on the liability of the index case as
	\begin{align}
	\beta &= \frac{T_G-T_R}{L_a}
	\label{eq:liability}
	\end{align}
	
	Thus, by applying \cref{eq:liability} to \cref{eq:finalNarrow}, we get
	\begin{align}
	h^2 =\frac{T_G-T_R}{rL_a}
	\end{align}
	
	\subsection{Adoption Study}
	% Need to go deeper into twin studies
	One key limitation of \cref{eq:finalNarrow} is its inability to discriminate the genetic factors from the shared environmental factors.
	Such problem arises as family not only share some of their alleles, but they also tends to share some of the environmental factors such as diet and socio-economic status. 
	In fact, this was the main reason for researchers to discord the argument that \glng{scz} is a genetic disorder.
	
	A classical adoption study carried out by \citet{HESTON1966} in 1966 investigated whether if the increased risk of \glng{scz} in relatives of \glng{scz} was caused by the shared environmental factors or the shared genetic factors. 
	An advantage of adoption studies is that if the child was separated from their family early after birth, then the shared environmental factors should be minimized, thus any resemblance between the parent and child should be driven mainly by the shared genetic factors.
	\citet{HESTON1966} collected data of 47 individuals who were born to schizophrenic mothers during the period from 1915 to 1947. 
	They were separated from their mother within three days of birth and sent to a foster family. 
	50 matched controls were also recruited in this study.
	It was observed that there was an increased risk of \glng{scz} in individuals born to schizophrenic mothers when compared to the control groups even-though they were brought up in a different environment as that of their mother.
	This result suggested that \glng{scz} is likely driven by the shared genetic factors instead of the shared environmental factors.
	
	\subsection{Twin Studies}
	Despite the usefulness of adoption studies in delineating the effect of shared environment from the genetic factors, collection of adoption data are extremely difficult. 
	Moreover, any prenatal influence such as alcohol abuse and malnutrition during pregnancy might confound the results.
	Therefore, an alternative method would be the twin studies which utilize the relationship between the \gls{mz} and \gls{dz} twins.
	
	Theoretically, \gls{mz} twins share all their genetic components (both additive ($A$) and non-additive ($D$) genetic factors) and also their common environmental factors ($C$) where the only difference between a twin pair is the non-shared environmental factors ($E$). 
	As for the \gls{dz} twins, they also share the same common environmental factors yet they only share $\frac{1}{2}$ of their additive genetic factors and $\frac{1}{4}$ of their non-additive genetic factors. 
	The non-shared environment is also, by definition, not shared among the twins \citep{Rijsdijk2002}.
	Based on these assumptions, \cite{Falconer1996} derived the heritability as
	\begin{equation}
	h^2 = 2(\rho_{MZ}-\rho_{DZ})
	\end{equation}
	where $\rho_{MZ}$ and $\rho_{DZ}$ are the phenotype correlation between the \gls{mz} twins and \gls{dz} twins respectively.
	
	By combining Falconer's formula and the concept of liability threshold model, \citet{Gottesman1967} estimated that the heritability of \glng{scz} to be $>60\%$ based on previously collected twin data, provide strong evidence that the genetic variation contributes more to the variance of \glng{scz}.
	The result was further supported by one of the landmark meta-analysis study conducted by \citet{Sullivan2003}.
	Based on data obtained from 12 published \glng{scz} twin studies, \citet{Sullivan2003} found that although there is a non-zero contribution of environmental influence on liability of \glng{scz} ($11\%$, \gls{ci}=$3\%-19\%$), there is a much larger contribution from genetics ($81\%$, \gls{ci}=$73\%-90\%$), further supporting that \glng{scz} is largely mediated by the genetic factors.
	
	Such findings are not only limited to twin-studies but were also reported in large scale population based studies.
	A recent large scale population based study in Sweden population \citep{Lichtenstein2009} also found that there is a large genetic contribution in \glng{scz} ($64\%$).
	Although the estimated heritability (64\% \citep{Lichtenstein2009} vs 81\% \citep{Sullivan2003}) differs between the two studies, there is no doubt that \glng{scz} is highly heritable.
	
	\section{Schizophrenia Genetics}
	Although population-based, adoption and twin studies suggest that \glng{scz} is highly heritable, little was known about the disease mechanism of \glng{scz} nor the genetic architecture of the disorder. 
	All data from adoption studies, twin studies and family studies shown that \glng{scz} does not follow the Mendelian framework \citep{Gottesman1967,Gottesman1982}.
	Specifically, shall \glng{scz} be a Mendelian disorder, then we would expect all \gls{mz} siblings of the proband to also suffer from \glng{scz}.
	However, the life time morbid risk of monozyogitc twins were only $48\%$ (\cref{fig:lifeMRscz}) \citep{gottesman1991schizophrenia}, making it unlikely for \glng{scz} to follow a Mendelian pattern.
	\begin{figure}[t]
		\centering
		\includegraphics[width=0.6\textwidth]{figure/lifeTimeMorbidRisk.png}
		\caption[Lifetime morbid risks of \glng{scz} in various classes of relatives of a proband]{Lifetime morbid risks of \glng{scz} in various classes of relatives of a proband.
			It was noted that the morbid risk of monozygotic (MZ) twins were only $48\%$, much lower than one would expect if \glng{scz} follows a Mendelian pattern.
			Reproduced with permission from journal \citep{Riley2006}. \label{fig:lifeMRscz}}
	\end{figure}
	
	Based on these observations, \citet{Gottesman1967} proposed that \glng{scz} follows a polygenic model where disease phenotype were determined by the additive effects from multiple genes.
	Thus, \glng{scz} is likely to be a complex genetic disorder with complicated pattern of inheritance. 
	
	By comparing the observed life time morbid risk and the expected risk from different models, \citet{Risch1990} proposed that the cause variants of \glng{scz} are more likely to have a risk less than 2 with no loci with risk larger than 3, suggesting a relatively small effect size.
	Very large samples are therefore required to detect these susceptibility loci through linkage studies \citep{Risch1990}.
	
	This might explain the early inconsistent findings of linkage studies in \glng{scz} \citep{Harrison2005}.
	As linkage studies were aimed to identify genetic variation of large effect size and usually have a relatively small sample size, it lacks detection power to identify the susceptibility loci with small effect size.
	The early failure of linkage studies in \glng{scz} were disappointing and not until the initiation of Human Genome Project, and technological advance resulted from that does genetic research of \glng{scz} began to enter an era of success.
	
	\subsection{The Human Genome Project and HapMap Project}
	\glsreset{SNP}
	\glsreset{LD}
	In 1990, the Human genome project was initiated, aiming at constructing the first physical map of the human genome at per nucleotide resolution \citep{Lander2001}.
	The completion of the human genome project has opened up a new era of genetic research, allowing researchers to identify \glspl{SNP}, which is one of the major source of genetic variation in the human genome.
	
	Soon after the completion of the human genome project, the HapMap Project was initiated \citep{Consortium2005}, aiming to provide a genome-wide database of common human sequence variation such as \glspl{SNP} with \gls{maf} $\ge0.05$.
	
	More importantly, the HapMap Project provided a detailed \gls{LD} map of the human genome.
	\gls{LD} is of particular importance to genetic research for it is the non-random correlation of genotypes between 2 genetic loci. 
	\glspl{SNP} in high \gls{LD} are usually observed together in the human genome.
	When a large amount of \glspl{SNP} are in high \gls{LD} together, they form what is known as a \gls{LD} block.
	By performing association testing on \glspl{SNP} representing a \gls{LD} block (``tagging''), one can avoid the need of performing association on the whole genome, therefore reducing the cost of the experiment.
	This was the fundamental concept of \gls{GWAS} which is now extensively used in the genetic research.
	
	\subsection{Genome Wide Association Study}
	In \gls{GWAS}, genome-wide genotyping array are commonly used to systematically detect common genetic variants such as \gls{SNP} and \gls{cnv} in genome-wide scale.
	For quantitative traits, the association between the trait and frequency of the variants are calculated using methods such as linear regression.
	On the other hand, for dichotomous traits such as \glng{scz}, the frequency of the variants are compared between the case and control samples using methods such as chi-square test or logistic regression.
	Because of the problem of multiple testing, only variants with a p-value passing a genome wide threshold (p-value $\le5\times10^{-8}$) are considered to be significant in \gls{GWAS}.
	Another possible method to decide the significant threshold is to consider the ``effective number'' of tests \citep{Li2011}, which reduced the genome-wide threshold according to the \gls{LD} structure.
	When designing a \gls{GWAS}, one need to take into account of the magnitude of effect, sample size, and required level of statistical significance (the false-positive, or type I, error rate) in order to have a powerful study \citep{Purcell2003}.
	
	\subsubsection{The Success of Psychiatric Genomic Consortium} 
	Despite the great promise from \gls{GWAS}, early \gls{GWAS} in \glng{scz} remain largely disappointing and were unable to identify any robust genetic markers associated with \glng{scz}.
	The failure of early \gls{GWAS} in \glng{scz} were mainly due to the relative small sample size of the studies, which result in low statistical power.
	
	To overcome the problem of small sample size, large consortium were formed such that genetic data from different research groups from different countries were combined and analyzed.
	By 2014, the \Glng{scz} Working group of the \gls{pgc} has conducted a multi-stage \glng{scz} \gls{GWAS} of up to 36,989 \glng{scz} samples and 113,075 controls.
	In their study \citep{Ripke2014}, 128 linkage-disequilibrium-independent \glspl{SNP} were found to exceeded the genome-wide significance (p-value $\le 5\times10^{-8}$), corresponding to 108 independent genetic loci.
	75\% of these loci contain protein coding genes and a further 8\% of these loci were within 20\gls{kb} of a gene. 
	It was found that genes involved in glutamatergic neurotransmission (e.g. \textit{GRM3}, \textit{GRIN2A} and \textit{GRIA1}), synaptic plasticity and genes encoding the voltage-gated calcium channel subunits (e.g. \textit{CACNA1C}, \textit{CACNB2} and \textit{CACNA1I}) were among the genes associated within these loci.
	Moreover, \textit{DRD2}, the target of all effective anti-psychotic drug were also found to be associated with \glng{scz}.
	This result converges with existing knowledge of \textit{DRD2} being involved in the pathology of \glng{scz}, supported by multiple lines of research \citep{Talkowski2007}.
	\begin{figure}
		\centering
		\caption[Enrichment of enhancers of SNPs associated with Schizophrenia]{Enrichment of enhancers of SNPs associated with \glng{scz}. 
			It was observed that the largest enrichment were in cell lines related to the brain and in tissues with important immune functions. 
			Graphs reproduced with permission from the journal \citep{Ripke2014}.}
		\includegraphics[height=\textwidth]{figure/pgc_enrichment_tissue.jpg}
		\label{fig:pgcEnrich}
	\end{figure}
	It was further demonstrated that \glng{scz} association were significantly enriched at enhancers active in brain and enriched at enhancers active in tissues with important immune functions (\cref{fig:pgcEnrich})\citep{Ripke2014}.
	
	Additionally, the enrichment of immune related enhancers remains significant even after the removal of \gls{mhc} region from the analysis, providing further genetic support of the involvement of the immune system in the etiology of \glng{scz}.
	Because of its role in neural development \citep{Zhao1998,Deverman2009}, it is likely that the perturbation in the immune system might disrupt the brain development, therefore increasing the risk of \glng{scz}.
	%Indeed, studies on \gls{mia} has demonstrated that cytokine imbalance might predispose individual to \glng{scz} \citep{Meyer2009}. 
	
	Although the \gls{pgc} \glng{scz} \gls{GWAS} is very successful, it is uncertain whether if all common variants associated with \glng{scz} has been captured. 
	With the unknown number of causal loci with moderate-to-small effect size, many \glspl{SNP} associated with \glng{scz} may be left undetected given the current sample size. 
	However, it is also possible that the \gls{pgc} \glng{scz} \gls{GWAS} has already captured all or near most of the \glspl{SNP} associated with the disease. 
	Therefore, estimating the contribution of these common \glspl{SNP} to \glng{scz} has important implications for future research strategy.
	
	\subsection{Contribution of Common SNPs}
	In a typical \gls{GWAS}, a stringent genome wide significant threshold were usually employed to avoid false positive findings. 
	However, if individual \glspl{SNP} have a small effect on the trait, the real association might be missed.
	Therefore, to estimate the true contribution of common \glspl{SNP} to a disease (\gls{SNP}-heritability), one should try to consider all \glspl{SNP} in the estimation.
	
	\subsubsection{Genome-wide Complex Trait Analysis}
	Currently, the most popular algorithm for the estimation of \gls{SNP}-heritability is \gls{gcta}, which uses information from the \gls{grm} \citep{Yang2011}.
	The \gls{grm} represents the ``genetic distance'' between all individuals within the \gls{GWAS}.
	Genetic relationship between individual $j$ and $k$ is estimated as 
	\begin{equation}
	A_{jk} = \frac{1}{N}\sum^N_{i=1}\frac{(x_{ij}-2p_i)(x_{ik}-2p_i)}{2p_i(1-p_i)}
	\end{equation}
	where $x_{ij}$ is the number of copies of the reference allele for the $i^{th}$ \gls{SNP} of the $j^{th}$ individual and $p_i$ is the frequency of the reference allele.
	This is based on the fact that genotypes are usually code as 0, 1 or 2 (homozygous reference, heterozygous and homozygous alternative respectively) and should follow the binomial distribution where the expected mean and variance of the genotype $i$ will be $2p_i$ and $2p_i(1-p_i)$ respectively.
	Thus $A_{jk} = \frac{1}{N}\sum^N_{i=1}z_{ij}z_{ik}$ where $z_{ij}$ is the standardized genotype for the $i^{th}$ \gls{SNP} of the $j^{th}$ individual.
	
	Using the information from the \gls{grm}, \citet{Yang2011} then fit the effects of all the \glspl{SNP} as random effects by a \gls{mlm}
	\begin{align}
	\boldsymbol{y} &= \boldsymbol{X\beta}+\boldsymbol{g}+\epsilon\\
	\mathrm{Var}(\boldsymbol{y}) &= \boldsymbol{A}\sigma_g^2+\boldsymbol{I}\sigma_\epsilon^2
	\end{align}
	where $\boldsymbol{y}$ is an $n\times 1$ vector of phenotypes with $n$ samples, $\boldsymbol{\beta}$ is a vector of fixed effects such as sex and age, $\boldsymbol{g}$ is an $n\times 1$ vector of the total genetic effects of the individuals, $\sigma_g^2$ is the variance explained by all the \glspl{SNP} and finally, $\sigma_\epsilon^2$ is the variance explained by residual effects.

	The main concept of \gls{gcta} is that instead of testing the associations for individual \glspl{SNP}, one fit the effects of all \glspl{SNP} as random effects in a \gls{mlm} and estimate a single parameter, i.e. the variance explained by all \glspl{SNP} or \gls{SNP}-heritability.
	Given the information of the \gls{grm}, \citet{Yang2011} implemented the \gls{reml} using the average information algorithm to estimate the $\sigma_g^2$ and $\sigma_\epsilon^2$ where the \gls{reml} is a form of maximum likelihood estimation that allows unbiased estimates of variance and covariance parameters.
	The \gls{SNP}-heritability of the trait is then defined as $\frac{\sigma_g^2}{\sigma_g^2+\sigma_e^2}$.

	Based on the above concept, \citet{Yang2010a} were able to estimate the variance in height explained by \glspl{SNP} from the height \gls{GWAS} to be around 45\%, much larger than previously reported 5\%.
	The main difference in the estimates was because the \gls{mlm} \gls{reml} were able to consider all \glspl{SNP} simultaneously without thresholding the significant \glspl{SNP}.
	Although the estimates was still less than 80\% which is the expected heritability of height, \citet{Yang2010a} was able to demonstrated that one possible source of ``missing heritability'' might be due to incomplete \gls{LD}.
	By taking into consideration of incomplete \gls{LD}, it was estimated that the proportion of variance explained by causal variants can be as high as 0.84 with \gls{se} of 0.16 \citep{Yang2010a}, close to the expected heritability.
	Together, \citet{Yang2011} provide a possible method for the estimation of the variance explained by \glspl{SNP} in \gls{GWAS} data and the method is now implemented in \gls{gcta} which is wildly adopted.
	
	One limitation of \gls{gcta} is that genotype data are required to calculate the \gls{grm}.
	For complex disease like \glng{scz}, the data were usually obtained from multiple data source where the raw genotypes are unavailable.
	Instead, summary statistics are usually provided.
	Therefore estimation of variance explained by \glspl{SNP} in these \gls{GWAS} can only rely on the summary statistics. 
	
	\subsubsection{\glng{ldsc}}
	In large scale \gls{GWAS} studies, a general inflation of summary statistics can sometimes be observed.
	It was usually considered to be contributed by the presence of confounding factors such as population stratification, under the assumption that most of the \glspl{SNP} should have no association to the disease.
	It was therefore a common practice for one to perform the \gls{gc} on the \gls{GWAS} results \citep{Zheng2006}.
	
	The problem of \gls{gc} was that the basic assumption of a small number of causal \glspl{SNP} might not be true, especially in complex disease like \glng{scz}.
	Through careful simulation, \citet{Yang2011b} demonstrated that in the absence of population stratification and other form of technical artifacts, the presence of polygenic inheritance can inflate the summary statistic \citep{Yang2011b}.
	More importantly, they observed that the magnitude of inflation was determined by the \emph{heritability}, the \gls{LD} structure, sample size and the number of causal \glspl{SNP} of the trait.
	
	The observation of \citet{Yang2011b} provided important foundation for the estimation of \gls{SNP} heritability based on summary statistics where a possible method will be to elucidate the heritability based on the magnitude of inflation of the summary statistics. 
	However, when confounding factors such as population stratification and cryptic relatedness are presented, they can also inflate the summary statistics.
	Therefore, in order to estimate the \gls{SNP}-heritability, one must delineate the confounding factors from the polygenicity of the trait.
	
	Based on the work of \citet{Yang2011b}, \citet{Bulik-Sullivan2015} hypothesized that strength of ``tagging'' of a \gls{SNP} should be correlated with the probability of it to ``tag'' the causal \gls{SNP} and should be independent to confounding factors such as population stratification and cryptic relatedness.
	\citet{Bulik-Sullivan2015} then defined the strength of ``tagging'' of a \gls{SNP} as the \gls{LD} score, which is the sum of $r^2$ of $k$ \glspl{SNP} within a 1\gls{cm} window of \gls{SNP}$_j$:
	\begin{equation}
	l_j = \sum_kr^2_{jk}
	\label{eq:ldScore}
	\end{equation}
	
	Based on their hypothesis, the expected $\chi^2$ of association of \gls{SNP}$_j$ with the trait can be defined as a function of the \gls{LD} score ($l_j$), the number of samples ($N$), the number of \glspl{SNP} in the analysis($M$) and most importantly, the \gls{SNP} heritability ($h^2$):
	\begin{equation}
	\mathrm{E}[\chi^2_j | l_j] = \frac{Nh^2}{M}l_j+1
	\label{eq:fixedLDSC}
	\end{equation}
	
	When confounding factors presents in the study (e.g. population stratification), \cref{eq:fixedLDSC} can instead be defined as
	\begin{equation}
	\mathrm{E}[\chi^2_j | l_j] = \frac{Nh^2}{M}l_j+Na+1
	\label{eq:fullLDSC}
	\end{equation}
	where $a$ is the contribution of confounding bias.
	
	By considering \cref{eq:fullLDSC} as a regression model, \citet{Bulik-Sullivan2015} observed that the contribution of common variants (the \gls{SNP} heritability $h^2$) will be the slope of the regression and the intercept minus one will represent the mean contribution of the confounding bias such as those of population stratification. 
	The \gls{ldsc} was implemented by \citet{Bulik-Sullivan2015}, using \cref{eq:fullLDSC} to delineate the contribution from confounding factors and common genetic variants.
	
	To test their hypothesis, \citet{Bulik-Sullivan2015} simulated multiple \gls{GWAS} where the trait can have a polygenic architecture or where confounding factors can present.
	When the simulated trait is polygenic and no confounding factors were presented, the average \gls{ldsc} intercept was close to one and the estimates were unbiased in all situation.
	Only when the number of causal variants was small will the standard error of the estimates become very large.
	On the other hand, when the \gls{GWAS} was simulated with only the confounding factors such as population stratification, the intercept estimated was approximately equal to the \gls{gc} inflation factor with only a small positive bias in the regression slope.
	
	Moreover, when a polygenic trait was simulated with confounding factors, the intercept of \gls{ldsc} was approximately equal to the mean $\chi^2$ statistic among the null \glspl{SNP}, providing strong evidence that \gls{ldsc} can partition the inflation in test statistic even in the presence of both bias and polygenicity.
	
	Given the success of the simulation, \citet{Bulik-Sullivan2015} estimated the \gls{SNP} heritability of \glng{scz} using the summary statistics from the \gls{pgc} \glng{scz} \gls{GWAS} \citep{Ripke2014} to be 0.555 with \gls{se} of 0.008 after adjusting for ascertainment bias.
	The estimated \gls{SNP} heritability was lower than the heritability estimated from population based study (64\% \citep{Lichtenstein2009}) and twin studies (81\% \citep{Sullivan2003}) suggesting that it is possible for variants other than common \glspl{SNP} also account for variations in \glng{scz}.
	
	\subsubsection{Partitioning of Heritability}
	Another implication of \gls{ldsc} is that it allows the partitioning of heritability, which helps to identify pathways that are associated with a trait.
	
	Traditionally, functional enrichment analysis in \gls{GWAS} only take into account of \glspl{SNP} that passed the genome wide significance threshold. 
	However, for complex traits such as \glng{scz}, much of the heritability might lies in \glspl{SNP} that do not reach genome wide significance threshold at the current sample size.
	For example, in 2013, only 13 risk loci were detected using 13,833 \glng{scz} samples and 18,310 controls \citep{Ripke2013}. 
	When the sample size increased to 34,241 \glng{scz} samples and 45,604 controls in 2014, 108 risk loci were identified \citep{Ripke2014}. 
	Thus, if one only consider the significant loci, risk loci that have not reach genome wide significance threshold might be ignored from the analysis, decreasing the power of the functional enrichment analysis.

	In order to estimate whether if a functional categories is associated with the trait, \gls{ldsc} takes into consideration of the summary statistic of all the \glspl{SNP} including in the \gls{GWAS}.
	The partitioning of the heritability is then calculated as 
	\begin{equation}
	\mathrm{E}[\chi^2_j] = N\sum_C\tau_Cl(j,C)+Na+1
	\label{eq:partitionH}
	\end{equation}
	
	The main difference between \cref{eq:partitionH} and \cref{eq:fullLDSC} is that $\frac{h^2}{M}l_j$ is substituted by $\sum_C\tau_Cl(j,C)$ where $l(j,C)$ is the \gls{LD} Score of \gls{SNP} $j$ with respect to category $C$ and $\tau C$ is the per-\gls{SNP} heritability in category $C$.
	
	Using data from \citet{Ripke2014} and functional categories derived from the ENCODE annotation \citep{ENCODEProjectConsortium2012}, the NIH Roadmap Epigenomics Mapping Consortium annotation \citep{Bernstein2010} and other studies, \citet{Finucane2015} attempted to identify functional categories that were most enriched in \glng{scz}.
	In their study, it was found that brain cell types and immune related cell types were most enriched in \glng{scz}.
	Among the functional categories, the most enriched category in \glng{scz} was the H3K4me3 mark in the fetal brain(\cref{tab:cellTypeScz}). 
	As H3K4me3 is mostly linked to active promoters, this suggests that genes that are activated in fetal brain (e.g. genes related to brain development) are associated with \glng{scz}, supporting the idea of \glng{scz} as a neuro-developmental disorder. 
	Undoubtedly, the \gls{cns} and the immune system have an important role in the disease etiology of \glng{scz}. 
		
	\begin{singlespace}
		\begin{longtable}{p{6cm}rrr}
			%\begin{tabular}{rrrr}
			\toprule
			Cell type & cell-type group & Mark  & P-value \\
			\midrule
			Fetal brain** & CNS   & H3K4me3 & $3.09\times 10^{-19}$ \\
			Mid frontal lobe** & CNS   & H3K4me3 & $3.63\times 10^{-15}$ \\
			Germinal matrix** & CNS   & H3K4me3 & $2.09\times 10^{-13}$ \\
			Mid frontal lobe** & CNS   & H3K9ac & $5.37\times 10^{-12}$ \\
			Angular gyrus** & CNS   & H3K4me3 & $1.29\times 10^{-11}$ \\
			Inferior temporal lobe** & CNS   & H3K4me3 & $1.70\times 10^{-11}$ \\
			Cingulate gyrus** & CNS   & H3K9ac & $5.37\times 10^{-11}$ \\
			Fetal brain** & CNS   & H3K9ac & $5.75\times 10^{-11}$ \\
			Anterior caudate** & CNS   & H3K4me3 & $2.19\times 10^{-10}$ \\
			Cingulate gyrus** & CNS   & H3K4me3 & $4.57\times 10^{-10}$ \\
			Pancreatic islets** & Adrenal/Pancreas & H3K4me3 & $2.24\times 10^{-09}$ \\
			Anterior caudate** & CNS   & H3K9ac & $3.16\times 10^{-9}$ \\
			Angular gyrus** & CNS   & H3K9ac & $4.68\times 10^{-9}$ \\
			Mid frontal lobe** & CNS   & H3K27ac & $7.94\times 10^{-9}$ \\
			Anterior caudate** & CNS   & H3K4me1 & $1.20\times 10^{-8}$ \\
			Inferior temporal lobe** & CNS   & H3K4me1 & $3.72\times 10^{-8}$ \\
			Psoas muscle** & Skeletal Muscle & H3K4me3 & $4.17\times 10^{-8}$ \\
			Fetal brain** & CNS   & H3K4me1 & $6.17\times 10^{-8}$ \\
			Inferior temporal lobe** & CNS   & H3K9ac & $9.33\times 10^{-8}$ \\
			Hippocampus middle** & CNS   & H3K9ac & $9.33\times 10^{-7}$ \\
			Pancreatic islets** & Adrenal/Pancreas & H3K9ac & $1.62\times 10^{-6}$ \\
			Penis foreskin melanocyte primary** & Other & H3K4me3 & $2.09\times 10^{-6}$ \\
			Angular gyrus** & CNS   & H3K27ac & $2.34\times 10^{-6}$ \\
			Cingulate gyrus** & CNS   & H3K4me1 & $2.82\times 10^{-6}$ \\
			Hippocampus middle** & CNS   & H3K4me3 & $2.82\times 10^{-6}$ \\
			CD34 primary** & Immune & H3K4me3 & $4.68\times 10^{-6}$ \\
			Sigmoid colon** & GI    & H3K4me3 & $5.01\times 10^{-6}$ \\
			Fetal adrenal** & Adrenal/Pancreas & H3K4me3 & $6.31\times 10^{-6}$ \\
			Inferior temporal lobe** & CNS   & H3K27ac & $8.32\times 10^{-6}$ \\
			Peripheralblood mononuclear primary** & Immune & H3K4me3 & $9.33\times 10^{-6}$ \\
			Gastric** & GI    & H3K4me3 & $1.17\times 10^{-5}$ \\
			Substantia nigra* & CNS   & H3K4me3 & $1.95\times 10^{-5}$ \\
			Fetal brain* & CNS   & H3K4me3 & $2.63\times 10^{-5}$ \\
			Hippocampus middle* & CNS   & H3K4me1 & $3.31\times 10^{-5}$ \\
			Ovary* & Other & H3K4me3 & $6.46\times 10^{-5}$ \\
			CD19 primary (UW)* & Immune & H3K4me3 & $7.08\times 10^{-5}$ \\
			Small intestine* & GI    & H3K4me3 & $8.51\times 10^{-5}$ \\
			Lung* & Cardiovascular & H3K4me3 & $1.17\times 10^{-4}$ \\
			Fetal stomach* & GI    & H3K4me3 & $1.29\times 10^{-4}$ \\
			Fetal leg muscle* & Skeletal Muscle & H3K4me3 & $1.51\times 10^{-4}$ \\
			Spleen* & Immune & H3K4me3 & $1.70\times 10^{-4}$ \\
			Breast fibroblast primary* & Connective/Bone & H3K4me3 & $2.04\times 10^{-4}$ \\
			Right ventricle* & Cardiovascular & H3K4me3 & $2.14\times 10^{-4}$ \\
			CD4+ CD25- Th primary* & Immune & H3K4me3 & $2.19\times 10^{-4}$ \\
			CD4+ CD25- IL17- PMA Ionomycin stim MACS Th sprimary* & Immune & H3K4me1 & $2.19\times 10^{-4}$ \\
			CD8 naive primary (UCSF-UBC)* & Immune & H3K4me3 & $2.24\times 10^{-4}$ \\
			Pancreas* & Adrenal/Pancreas & H3K4me3 & $2.34\times 10^{-4}$ \\
			CD4+ CD25- Th primary* & Immune & H3K4me1 & $2.75\times 10^{-4}$ \\
			CD4+ CD25- CD45RA+ naive primary* & Immune & H3K4me1 & $2.75\times 10^{-4}$\\
			Colonic mucosa* & GI    & H3K4me3 & $3.24\times 10^{-4}$ \\
			Right atrium* & Cardiovascular & H3K4me3 & $3.31\times 10^{-4}$ \\
			Fetal trunk muscle* & Skeletal Muscle & H3K4me3 & $3.39\times 10^{-4}$ \\
			CD4+ CD25int CD127+ Tmem primary* & Immune & H3K4me3 & $3.47\times 10^{-4}$ \\
			Substantia nigra* & CNS   & H3K9ac & $3.63\times 10^{-4}$ \\
			Placenta amnion* & Other & H3K4me3 & $4.17\times 10^{-4}$ \\
			Breast myoepithelial* & Other & H3K9ac & $5.50\times 10^{-4}$ \\
			CD8 naive primary (BI)* & Immune & H3K4me1 & $5.75\times 10^{-4}$ \\
			Substantia nigra* & CNS   & H3K4me1 & $6.61\times 10^{-4}$ \\
			Cingulate gyrus* & CNS   & H3K27ac & $7.94\times 10^{-4}$ \\
			CD4+ CD25- CD45RA+ naive primary* & Immune & H3K4me3 & $8.71\times 10^{-4}$ \\
			\bottomrule
				%\end{tabular}%
			\caption[Enrichment of Top Cell Type of Schizophrenia]{Enrichment of Top Cell type of Schizophrenia.
				* = significant at False Discovery Rate $<$ 0.05.
				** = significant at p $<$ 0.05 after correcting for multiple hypothesis. 
				Reproduce with permission from Journal.\citep{Finucane2015}}
			\label{tab:cellTypeScz}%
		\end{longtable}%
	\end{singlespace}
		
	\subsection{Rare Variants in Schizophrenia}
	\glsreset{cnv}
	The estimated \gls{SNP}-heritability using the common variants captured by the \gls{pgc} \glng{scz} \gls{GWAS} suggest that variants other than common \glspl{SNP} are accounting for the variation in \glng{scz}.
	Based on the ``common disease-rare variant'' hypothesis, another interesting direction of \glng{scz} research will be to identify rare variants associated with \glng{scz}.
	
	\subsubsection{Copy Number Variation}
	A possible source of rare variants can be \glspl{cnv}.
	\gls{cnv} are classified as segment of DNA that is 1\gls{kb} or larger and that is present at a different copy number when compared to the reference genome, usually in the form of insertion, deletion or duplication \citep{Feuk2006}.
	Due to the length of these variants, the \gls{cnv} might contain the entire genes and their regulatory regions which might in turn contribute to significant phenotypic differences \citep{Feuk2006}.
	
	Recently, \citet{Szatkiewicz2014} conducted a \gls{GWAS} for \gls{cnv} association with \glng{scz} using the Swedish national sample (4,719 \glng{scz} samples and 5,917 controls).
	In their study, they were able to identify association between \glng{scz} and \glspl{cnv} such as 16p11.2 duplications, 22q11.2 deletions, 3q29 deletions and 17q12 duplications.
	Through the gene set association analysis, calcium channel signaling and binding partners of the fragile X mental retardation protein were found to be associated with these \gls{cnv} \citep{Szatkiewicz2014}.
	Interestingly, the calcium channel signaling were also enriched in the \gls{pgc} \gls{GWAS} on \gls{SNP} association, suggesting that the variants were converging the same set of pathways or gene sets. 
	
	Similarly, \citet{Walsh2008} also found that genes disrupted by structure variants in their cases were significantly overrepresented in pathways important for brain development, including neuregulin signaling, extracellular signal-regulated kinase/\gls{mapk} signaling, 
	synaptic long-term potentiation, axonal guidance signaling, integrin signaling, and glutamate receptor signaling \citep{Walsh2008}.
	
	An important observation in these \gls{cnv} studies was that the \glspl{cnv}  were generally rare ($\le12$ in 4,719 samples \citep{Szatkiewicz2014}) and has a relative large effect (e.g. odd ratio $>2$ \citep{Szatkiewicz2014,Walsh2008}), following the ``common disease-rare variant'' model.
	
	\subsubsection{Rare Single Nucleotide Mutation}
	Unlike \gls{cnv} which affects a large region, rare \glspl{SNP} cannot be captured using current genotyping chips.
	Therefore, large scale association of rare \glspl{SNP} was unavailable until the development of the \gls{ngs} technology.
	The \gls{ngs} generates high-throughput sequencing data with per base resolution, allow one to investigate the whole human genome or the human exome without relying on ``tagging''.
	
	Using exome sequencing, \citet{Purcell2014} sequenced the exome of 2,536 \glng{scz} cases and 2,543 normal controls. 
	They were able to identify a common missense allele in \textit{CCHCR1} in the \gls{mhc} that were associated with \glng{scz}.
	Although none of the genes showed a significant burden of rare mutation in cases, a significant increased burden of rare nonsense and disruptive variants was observed in cases in gene sets such as voltage-gated calcium ion channel, genes affected by \textit{de novo} mutations in \glng{scz} \citep{Fromer2014} and the postsynaptic density, all of which have been reported to be associated with \glng{scz} in previous genetic studies \citep{Ripke2014}.

	The overlaps between the rare variant studies and the common variant studies suggest that both rare and common variants are likely to be acting upon the same pathway and are complementary to each other.
	
	\section{Environmental Risk Factors of Schizophrenia}
	Apart from genetic variants, another possible source of ``missing'' heritability can come from interaction between the genetic and environmental risk factors.
	Although previous studies \citep{Gottesman01071967} suggested that the non-additive genetic factors were unlikely to contribute to \glng{scz}, the possibility of involvement of gene-environmental interaction ($G\times E$) were not ruled out.
	Indeed, in the adoption study conducted by \citet{Tienari2004}, it was found that individuals with higher genetic risk were significantly more sensitive to ``adverse'' vs ``healthy'' rearing patterns in adoptive families than are adoptees at low genetic risk \citep{Tienari2004}.
	Moreover, using the national registers in Finland, \citet{Clarke2009} found that the effect of prenatal infection was five times greater in those who had a family history of psychosis when compared to those who did not. 
	Together, these findings support a mechanism of gene-environment interaction in the causation of \glng{scz}.
	
	Many environmental factors have been associated with \glng{scz}, including prenatal infection \citep{Brown2010}, winter birth \citep{OCallaghan1991}, tobacco consumption \citep{Kelly1999} and socio economic status \citep{McGrath2008a}.
	They are therefore potential targets for the study of $G\times E$ interaction.
	However, by and large, the prenatal infection is the largest environmental risk factor of \glng{scz} and existing evidence suggest that there are indeed an interaction between prenatal infection and genetic variations \citep{Clarke2009}.
	It is therefore interesting to investigate how prenatal infection trigger \glng{scz} and how it interacts with genetic variations in the development of \glng{scz}.	
	
	\subsection{Prenatal Infection}
	\begin{figure}
		\centering
		\includegraphics[width=\textwidth]{figure/risk_factors_of_schizophrenia.png}
		\caption[Risk factors of \glng{scz}]{Risk factors of \glng{scz}.
			It was observed that family history of \glng{scz} was the largest risk factors.
			Risk of \glng{scz} can be more than 9 times higher than the general population for individual with a family history of \glng{scz}}
		\label{fig:riskfactors}
	\end{figure}
	Among all the environmental factors, prenatal infection has been considered to be an important risk factor of \glng{scz}, being the single largest non-genetic risk factor of \glng{scz} (\cref{fig:riskfactors})\citep{Sullivan2005}.
	Initial clues indicated that births during the winter and spring months and in urban areas were related to an increased risk of the disorder \citep{Brown2010}.
	It was also observed that there was an increased risk of \glng{scz} in individuals who were fetuses during the 1957 influenza epidemic \citep{Mednick1958}.
	As the chance of getting infectious disease varies by season and infectious disease can spread more quickly in urban regions due to higher population density, these evidences suggest that prenatal infection might be associated with \glng{scz}.
	
	Early studies of prenatal infection in \glng{scz} mainly relies on ecological data such as influenza epidemics in the population to define the exposure status \citep{Brown2010}.
	The problem of these studies was that the exposure status was based solely on whether an individual was in gestation at the time of the epidemic without any confirmation of maternal infection during pregnancy, leads to difficulties in replication of the findings.
	Subsequently, researchers uses birth cohorts where infection was documented using different biomarkers during pregnancies to provide a better labeling of the exposure status \citep{Brown2010}.
	Through these rigorous studies, it was found that the risk of \glng{scz} increases as long as an individual's mother was infected by any form of infectious agents such as influenza, HSV-2 and \textit{T.gondii} during gestation \citep{Brown2010}.
	As various infectious agents increase the risk of \glng{scz}, it leads to the hypothesis that \gls{mia} \citep{Brown2010} rather than a particular infectious agents, is the source of risk factor. 
	It was suggested that the maternal immune response disrupt the brain development in the fetus, thus leading to an elevated risk of \glng{scz} \citep{Garbett2012a}.
	
	A great challenge in the study of \gls{mia} is that one cannot carry out empirical experiment in human samples due to ethical concerns.
	Thus a popular alternative is to employ rodent models.
	However, unlike physiological traits, psychiatric disorder such as \glng{scz} are characterized by symptoms related to higher level functioning such as hallucinations, delusion, disorganized speech etc \citep{AmericanPsychiatricAssociation2013}, these traits are not readily detectable in rodents.
	This raises challenge in diagnosing whether if the rodent has demonstrated the symptoms of \glng{scz} for not only it was difficult to check whether if the high level functioning of the rodent is disrupted, there were no available biomarkers for \glng{scz}.
	Therefore instead of labeling whether if the rodent is ``schizophrenic'' or ``normal'', one would rather consider whether if the rodent demonstrate any ``schizophrenia-like'' behaviours such as impaired prepulse inhibition, impaired working memory and reduced social interaction \citep{Meyer2007a}.
	An important point to note here is that as autism and \glng{scz} shares most of these behavioral abnormality, and that risk of autism is also increased by \gls{mia} \citep{Brown2012}, studies using these rodent models were usually non-specific to \glng{scz} or autism. 
	Rather, autism and \glng{scz} are usually considered together in these rodent models.
	However, the discussion of the etiology of autism and the similarity and difference between autism and \glng{szc} is beyond the scope of the current thesis.
	Therefore, for the simplicity and focus of the current thesis, we will limit our discussion to \glng{scz}.
	
	A common rodent model in the study of effect of \gls{mia} is to use the viral analogue \gls{polyic} to induce the maternal immune response during pregnancy in rodents.
	It was found that offspring exposed to \gls{polyic} displays phenotypes mirrors that observed in schizophrenia \citep{Li2009c,Meyer2009b,Li2010a} such as deficiency in prepulse inhibition \citep{Cadenhead2000}.
	Because \gls{polyic} only induce the \gls{mia} without infecting the fetuses, the \gls{polyic} model provide strong evidence that \gls{mia}, instead of the specific infection, contributes to the increased risk of \glng{scz}.	
	
	\citet{Smith2007} were able to demonstrate that a single injection of \gls{il6} to the pregnant mouse can induce \glng{scz}-like behaviour in the adult offspring. 
	What was most interesting was by eliminating the \gls{il6} from the maternal immune response using either genetic methods (\gls{il6} knock out) or with blocking antibodies, the behaviour deficits associated with \gls{mia} were not present in the adult offspring, suggesting that \gls{il6} is central to the process by which \gls{mia} causes long-term behavioral changes.
	
	Further studies of global gene expression patterns in \gls{mia}-exposed rodent fetal brains \citep{Oskvig2012,Garbett2012a} suggest that the post-pubertal onset of schizophrenic and other psychosis-related phenotypes might stem from attempts of the brain to counteract the environmental stress induced by \gls{mia} during its early development \citep{Garbett2012a}.
	For example, genes with neuroprotective function such as crystallins might also have additional roles in neuronal differentiation and axonal growth \citep{Garbett2012a}. 
	By over-expressing these genes to counteract the environmental stress, the balance between neurogenesis and differentiation in the embryonic brain maybe disrupted. 
	Based on these observations, \citet{Garbett2012a} propose that once the immune activation disappears, the normal brain development programme resumes with a time lag, result in permanent changes in connectivity and neurochemistry that might ultimately leads to \glng{scz}-like behaviours.
	\begin{figure}
		\centering
		\includegraphics[width=\textwidth]{figure/mia_impact.jpg}
		\caption[Hypothesized model of the impact of prenatal immune challenge on fetal brain development]{Hypothesized model of the impact of prenatal immune challenge on fetal brain development.
			Maternal infection in early/mid pregnancy may affect early neurodevelopmental events in the fetal brain, thereby influencing the differentiation of neural precursor cells (grey) into particular neuronal phenotype (yellow or brown).
			This may predispose the developing fetal nervous system to additional failures leading to multiple structural and functional brain abnormalities in later life.
			Figure used with permission from Journal \citep{Meyer2007a}}
		\label{fig:miaEffect}
	\end{figure}
	
	On the other hand, an age dependent structural abnormalities in the mesoaccumbal and nigrostriatal dopamine systems were also found to be induced by \gls{mia} \citep{Vuillermot2010}.
	Specifically, \gls{mia} induces an early abnormality in specific dopaminergic systems such as those in the striatum and midbrian region \citep{Vuillermot2010}.
	Based on these observations, \citet{Meyer2007a} hypothesized that inflammation in the fetal brain during early gestation not only can disrupt neurodavelopmental processes such as cell proliferation and differentiation, it also predispose the developing nervous system to additional failures in subsequent cell migration, target selection, and synapse maturation (\cref{fig:miaEffect}) \citep{Meyer2007a}.
		
	In a separate study by \citet{Giovanoli2013}, mice were exposed to a lower dosage of \gls{polyic} during early gestation.
	%TODO might want to explain the impact of a lower dosage of PolyIC
	Offspring born were then left undisturbed or exposed to unpredictable stress during peripubertal development.
	It was observed that offspring exposed to \gls{polyic} has an increased level of dopamine in the nucleus accumbens independent to whether if they were exposed to postnatal stress whereas serotonin (5-HT) were decreased in the medial prefrontal cortex when exposed to postnatal stress regardless of prenatal exposure.
	Only when the offspring were exposed to both \gls{polyic} and postnatal stress will they have an increased dopamine levels in the hippocampus or will sensorimotor gating and psychotomimetic drug sensitivity be affected \citep{Giovanoli2013}.
	\citet{Giovanoli2013} therefore suggest that the prenatal insult serves as a ``disease primer'' that increase offspring's vulnerability to subsequent insults.
	
	Together, these results supports the involvement of \gls{mia} in the development of \glng{scz}.
	It was even estimated that one third of all \glng{scz} cases could have been prevented shall all infection were prevented from the entire pregnant population \citep{Brown2010}.
	
	One of the critical consideration in the study of \gls{mia} is the specific gestation period of vulnerability to infection-mediated disturbance \citep{Meyer2007a}.
	Early epidemiological studies have suggested that the second trimester of human pregnancy might have been the vulnerability period.
	However, in the birth cohorts such as the Prenatal Determinants of \Glng{scz}, it was found that the time window with maximal risk for infection-mediated disturbance in brain development is earlier than the second trimester of human pregnancy and can be as early as the first trimester \citep{Meyer2007a}.
	Through the review of existing \gls{mia} studies on rodent models, \citet{Meyer2007a} suggests that effect of \gls{mia} during late pregnancy can be restricted to the late developmental programmes, thus have a more restricted pathological phenotype in the grown offspring compared to \gls{mia} during early pregnancy \citep{Meyer2007a}.
	Subsequent \gls{mia} studies using the \gls{polyic} mouse model also support the hypothesis proposed by \citet{Meyer2007a}, where it was observed that \gls{mia} early in gestation event might exert a more extensive impact on the phenotype of offspring \citep{Li2009c,Li2010a}.

	Despite the more severe impact of \gls{mia} during early gestation, most \gls{mia} studies have been focusing on the mid-gestation period and the understanding of the full molecular implication of early \gls{mia} events in adult brain were lacking.
	As technology advances, we can now employ the RNA Sequencing technique to examine the global \gls{mrna} expression changes in the brain of the adult offspring exposed to \gls{mia} during early gestation.
	
	\subsection{RNA Sequencing}
	Before the development of the \gls{ngs}, one can only inspect the global expression changes using the microarray which is based on probe hybridization.
	As \gls{ngs} developed, one can now use poly-T probes to ``extract'' the \gls{mrna} fragments and sequence them.
	The depth of coverage of each gene then provide a general representation of the concentration of the \gls{mrna} in the cell.
	When compared to microarray, the RNA Sequencing has a number of advantages, most notably, because RNA Sequencing does not rely on specific probe hybridization, it does not suffer from bias introduced by probe performances such as signal saturation, cross-hybridization, background noises and non-specific hybridization \citep{Zhao2014}.
	Moreover, RNA Sequencing has the additional advantage that one can perform not only the differential expression analysis, but also detect alternative splicing events and de novo transcripts.
	
	However, the analysis of RNA Sequencing is more complicated when compared to microarray.
	The first hurdle in the analysis of RNA Sequencing data is the sequence alignment.
	RNA sequencing will typically generate sequence reads from the \gls{mrna} transcripts and one will need to align these reads to either the genome or the transcriptome in order to be able to calculate the depth of coverage for each genes, thus allowing the differential expression analysis.
	The different alignment strategies have their own pros and cons.
	
	Alignment to transcriptomes are most straightforward as the reads were originated from the transcripts and should have sequence composition similar to the transcriptome. 
	The problem of transcriptome alignment is that multiple isoform can share the same exon, leading to read mapping uncertainties \citep{Li2011e}.
	Without taking into consideration of the uncertainties, the downstream analysis might be biased and inaccurate. 
	When one is only interested in analyzing the gene level expression difference, this complication might be unnecessary.
	
	On the other hand, alignment to the genome should help to reduce the problem of multiple mapping yet it will require a splice aware aligner such as TopHat2 \citep{Kim2013}, STAR \citep{Dobin2013} and MapSplice \citep{Wang2010}.
	The reason behind is that as the reads were originated from the \gls{mrna} where alternative splicing might have occurred, the reads might span multiple exons which are separated by intronic regions. 
	The splicing algorithm will be able to ``split'' the reads and correctly align them onto the exons. 
	With the accurate alignment, one can then quantify the ``expression'' of each individual genes.
	
	The expression of a gene is usually represented in terms of number of reads aligned to the gene. 
	Given this information, statistic analysis can then be performed on the count data. 
	Unlike microarray, where the signal usually follows a normal distribution \citep{Hoyle2002,Giles2003}, the distribution of the RNA Sequencing count data are more complicated.
	Early RNA Sequencing experiment assumes the gene expression counts follows the Poisson distribution \citep{Marioni2008} where the variance is equal to the mean of the expression.
	However, it was found that the assumption of Poisson distribution is too restrictive where an over-dispersion was typically observed in RNA Sequencing data \citep{Anders2010}.
	Therefore, to overcome the problem of over-dispersion, modern RNA Sequencing statistical package usually models the RNA Sequencing counts using the negative binomial distribution \citep{Anders2010,Robinson2010} or the beta negative binomial distribution \citep{Trapnell2012} instead of the Poisson distribution.
	\begin{figure}
		\centering
		\includegraphics[width=0.5\textwidth]{figure/overdispersion.png}
		\caption[Over-dispersion observed in RNA Sequencing Count Data]{
			Over-dispersion observed in RNA Sequencing Count Data.
			If the RNA Sequencing count data follows the Poisson distribution, then the mean and variance of the data should be equal (follow the diagonal). 
			However, it was observed that as the mean increases, the variance increases even more, suggesting that there is an over-dispersion in the data. 
		}
	\end{figure}
	
	Nonetheless, as our knowledge with RNA Sequencing advances, we are getting better in utilizing the information provided by RNA Sequencing and it should serves as an important tool for the analysis of gene expression changes induced by \gls{mia} event.
	
	%TODO check till here
	\section{Summary}
	In this thesis, we would like to first perform a series of empirical simulations to the effect of different genetic architectures and sampling strategies in \gls{GWAS} to the performance of \gls{ldsc}, for example, the effect of extreme phenotype samplings.
	On the other hand, as suggested by \citet{Bulik-Sullivan2015}, under certain conditions such as when the trait is oligogenic, the performance of \gls{ldsc} might be subpar. 
	Thus we would also like to develop an alternative algorithm for the estimation of \gls{SNP} heritability that is robust to different genetic architecture.
	Ultimately, we would like to repeat the analysis by \citet{Bulik-Sullivan2015} to estimate the true contribution of common \glspl{SNP} to the variance in \glng{scz}.

	Currently, there are evidences suggesting that there might be interaction between prenatal infection and genetic variations in the development of \glng{scz} \citep{Tienari2004,Clarke2009}.
	We therefore hypothesize that the differential gene expression induced by \gls{mia} and genetic mutation might have act upon the same functional pathway.
	To test this hypothesis, we performed a hypothesis generation RNA Sequencing study to capture gene expression changes induced by early \gls{mia} events (\gls{gd}9) in the cerebellum of mouse using the \gls{polyic} mouse model.
	Based on the gene expression changes, we hope to identify functional pathways perturbed by early \gls{mia} events.
	Most importantly, we would like to test whether if these pathways contribute disproportionately to the heritability of \glng{scz}.
	As a result of that, we would also perform the partitioning of heritability using \gls{ldsc} on the pathways affected by \gls{mia}.
	
	Moreover, recent study from our lab suggested that n-3 \gls{pufa} rich diet might help to reduce the \glng{scz}-like behaviour in mice exposed to early \gls{mia} insults \citep{Li2015}. 
	Therefore we would also like to take this opportunity to assess the effect of n-3 \gls{pufa} rich diet on the gene expression pattern in the brain of the adult offspring.
	
	This thesis will be divided into three parts.
	First, in \Cref{heritabilityChapter}, we performed a series of empirical simulations to assess the performance of \gls{ldsc} in the estimation of \gls{SNP} heritability. 
	We also proposed an alternative approach for the estimation of \gls{SNP}-heritability from \gls{GWAS} summary statistics that is robust to different genetic architectures.
	
	In \Cref{omegaProject}, a hypothesis generation study was performed to study the effect of \gls{mia} on the gene expression pattern of mouse cerebellum. 
	On top of that, as recent study suggested that n-3 \gls{pufa} rich diet can help to reduce the \glng{scz}-like behaviour observed mouse exposed to early \gls{mia} \citep{Li2015}, we also investigated the effect of n-3 \gls{pufa} rich diet on the gene expression pattern of mouse cerebellum.
	
	Lastly, we summarize and conclude all findings in \Cref{conclusionChapter} and give future perspectives on the \glng{scz} research.
	
	\chapter{Heritability Estimation}

% Need to stress that we are only calculating the narrow sense heritability
	\section{Introduction}
	The development of \glng{ldsc} has brought great prospect in estimating the heritability of complex disease for one can now estimate the heritability of a trait without requiring the rare genotype. 
	However, as noted by the author of \gls{ldsc}, when the number of causal variants were small, or when working on targeted genotype array, \gls{ldsc} tends to have a larger standard error or might produce funky results\citep{Bulik-Sullivan2015}.
	Ideally, we would like to be able to robustly estimate the heritability for all traits, disregarding the genetic architecture (e.g. number of causal \glspl{SNP}).
	
	On the other hand, it has been shown that there can be huge bias in the heritability estimation of \gls{gcta} when prevalence of a dichotomous trait is low\citep{Golan2014}.
	Although \citet{Golan2014} developed the \gls{pcgc}, which can provide robust estimation of heritability for traits with different prevalence, it still relies on the relationship matrix and therefore require the raw genotype of the samples. 
	
	Herein, we would like to develop an alternative algorithm to \gls{ldsc} for heritability estimation using only the test statistics. 
	We would also like to inspect whether if \gls{ldsc}'s heritability estimation is robust to prevalence of a trait. 
	A number of simulations were performed to compare the performance of \gls{ldsc} and our algorithm under different conditions.
	
	The work in this chapter were done in collaboration with my colleagues who have kindly provide their support and knowledges to make this piece of work possible.
	Dr Johnny Kwan, Dr Miaxin Li and Professor Sham have helped to laid the framework of this study. 
	Dr Timothy Mak has derived the mathematical proof for our heritability estimation method. 
	Miss Yiming Li, Dr Johnny Kwan, Dr Miaxin Li, Dr Timothy Mak and Professor Sham have helped with the derivation of the standard error of the heritability estimation. 
	Dr Henry Leung has provided critical suggestions on the implementation of the algorithm.
	
	\section{Methodology}	
		The overall aims of this study is to develop a robust algorithm for the estimation of the narrow sense heritability using only the summary statistic from a \gls{GWAS}.
		In \gls{GWAS}, the test statistic of a particular \gls{SNP} should be proportional to its effect size and the effect size from all the other \glspl{SNP} in \gls{LD} with it.
		Based on this property, we may use the information from the \gls{LD} matrix and the test statistic of the \gls{GWAS} \gls{SNP} the estimate the narrow sense heritability.
		
		
		\subsection{Heritability Estimation}
			Remember that the narrow-sense heritability is defined as 
			$$
				h^2 = \frac{\mathrm{Var}(X)}{\mathrm{Var}(Y)}
			$$
			where $\mathrm{Var}(X)$ is the variance of the genotype and $\mathrm{Var}r(Y)$ is the variance of the phenotype.
			In a \gls{GWAS}, regression were performed between the \glspl{SNP} and the phenotypes, giving
			\begin{equation}
				Y=\beta X+\epsilon
				\label{eq:standardRegress}
			\end{equation}
			where $Y$ and $X$ are the standardized phenotype and genotype respectively. 
			$\epsilon$ is then the error term, accounting for the non-genetic elements contributing to the phenotype (e.g. Environment factors).
			Based on \cref{eq:standardRegress}, one can then have
			\begin{align}
				\mathrm{Var}(Y) = \mathrm{Var}(\beta X)+ \mathrm{Var}(\epsilon) \nonumber\\
				\mathrm{Var}(Y) = \beta^\mathrm{Var}(X) \nonumber\\
				\beta^2\frac{\mathrm{Var}(X)}{\mathrm{Var}(Y)}= 1
				\label{eq:betaHeri}
			\end{align}
			$\beta^2$ is then considered as the portion of phenotype variance explained by the variance of genotype, which can also be considered as the narrow-sense heritability of the phenotype.
					
			A challenge in calculating the heritability from \gls{GWAS} data is that usually only the test-statistic or p-value were provided and one will not be able to directly calculate the heritability based on \cref{eq:betaHeri}. 
			In order to estimation the heritability of a trait from the \gls{GWAS} test-statistic, we first observed that when both $X$ and $Y$ are standardized, $\beta^2$ will be equal to the coefficient of determination ($r^2$). 
			Then, based on properties of the Pearson product-moment correlation coefficient:
			\begin{equation}
				r = \frac{t}{\sqrt{n-2+t^2}}
				\label{eq:pearsonProduct}
			\end{equation}
			where $t$ follows the student-t distribution and $n$ is the number of samples, one can then obtain the $r^2$ by taking the square of \cref{eq:pearsonProduct}
			\begin{equation}
				r^2 = \frac{t^2}{n-2+t^2}
				\label{eq:oriRSquared}
			\end{equation}
			It is observed that $t^2$ will follow the F-distribution.
			When $n$ is big, $t^2$ will converge into $\chi^2$ distribution.
			
			Furthermore, when the effect size is small and $n$ is big, $r^2$ will be approximately $\chi^2$ distributed with mean $\sim 1$. 
			We can then approximate \cref{eq:oriRSquared} as
			\begin{equation}
				r^2= \frac{\chi^2}{n}
				\label{eq:approxChi}
			\end{equation}
			and define the \emph{observed} effect size of each \gls{SNP} to be
			\begin{equation}
			f=\frac{\chi^2-1}{n}
			\label{eq:observedEffect}
			\end{equation}
			
			When there are \gls{LD} between each individual \glspl{SNP}, the situation will become more complicated as each \glspl{SNP}' observed effect will contains effect coming from other \glspl{SNP} in \gls{LD} with it:
			\begin{equation}
			f_{observed} = f_{true}+f_{LD}
			\label{eq:conceptF}
			\end{equation}
			
			To account for the \gls{LD} structure, we first assume our phenotype $\boldsymbol{Y}$ and genotype $\boldsymbol{X}=(X_1,X_2,\dots,X_m)^t$ are standardized and that
			\begin{align*}
				\boldsymbol{Y}\sim f(0,1) \\
				\boldsymbol{X}\sim f(0,\boldsymbol{R})
			\end{align*}
			Where $\boldsymbol{R}$ is the \gls{LD} matrix between \glspl{SNP}.
			
			We can then express \cref{eq:standardRegress} in matrix form:
			\begin{align}
				\boldsymbol{Y}=\boldsymbol{\beta}^t\boldsymbol{X}+\epsilon
				\label{eq:matrixRegress}
			\end{align}
			Because the phenotype is standardized with variance of 1, the narrow sense heritability can then be expressed as
			\begin{align}
				Heritability& = \frac{\mathrm{Var}(\boldsymbol{\beta}^t\boldsymbol{X})}{\mathrm{Var}(\boldsymbol{Y})} \nonumber\\
				&=\mathrm{Var}(\boldsymbol{\beta}^t\boldsymbol{X})
			\end{align}
			If we then assume now that $\boldsymbol{\beta} = (\beta_1, \beta_2,\dots,\beta_m)^t$ has distribution
			\begin{align*}
				\boldsymbol{\beta}&\sim f(0,\boldmath{H})\\
				\boldsymbol{H}&=diag(\boldsymbol{h})\\
				\boldsymbol{h}&=(h_1^2,h_2^2,\dots,h_m^2)^t
			\end{align*}
			where $\boldsymbol{H}$ is the variance of the ``true'' effect. 
			It is shown that heritability can be expressed as %The later part was gone because that will contains E(\beta) which = 0
			\begin{align}
			\mathrm{Var}(\boldsymbol{\beta}^t\boldsymbol{X}) &= \mathrm{E}_X\mathrm{Var}_{\beta|X}(\boldsymbol{X}^t\boldsymbol{\beta})+\mathrm{Var}_X\mathrm{E}_{(\beta|X)}(\boldsymbol{\beta}^2\boldsymbol{X}) \nonumber\\
			&=\mathrm{E}_X(\boldsymbol{X}^t\boldsymbol{\beta\beta}^T\boldsymbol{X}) \nonumber\\ 
			&= \mathrm{E}_X(\boldsymbol{X}^t\boldsymbol{HX}) \nonumber\\
			&= \mathrm{E}(\boldsymbol{X})^t\boldsymbol{H}\mathrm{E}(\boldsymbol{X})+\mathrm{Tr}(\mathrm{Var}(\boldsymbol{X}\boldsymbol{H})) \nonumber\\
			&=\mathrm{Tr}(\mathrm{Var}(\boldsymbol{X}\boldsymbol{H})) \nonumber\\
			&=\sum_ih_i^2
			\label{eq:proveHerit}
			\end{align}
			
			Now if we consider the covariance between \gls{SNP} i ($\boldsymbol{X_i}$) and $\boldsymbol{Y}$, we have
			\begin{align}
			 \mathrm{Cov}(\boldsymbol{X}_i,\boldsymbol{Y}) &= \mathrm{Cov}(\boldsymbol{X}_i,\boldsymbol{\beta}^t\boldsymbol{X}+\epsilon) \nonumber\\
			 &=\mathrm{Cov}(\boldsymbol{X}_i,\boldsymbol{\beta}^t\boldsymbol{X}) \nonumber\\
			 &=\sum_j{\mathrm{Cov}(\boldsymbol{X}_i,\boldsymbol{X}_j)\boldsymbol{\beta}_j} \nonumber\\
			 &=\boldsymbol{R}_i\boldsymbol{\beta}_j
			 \label{eq:covPhenoTrue}
			\end{align}
			
			As both $\boldsymbol{X}$ and $\boldsymbol{Y}$ are standardized, the covariance will equal to the correlation and we can define the correlation between \gls{SNP} i and $Y$ as
			\begin{equation}
				\rho_i = \boldsymbol{R}_i\boldsymbol{\beta}_j
				\label{eq:corPhenoTrue}
			\end{equation}
			In reality, the \emph{observed} correlation usually contains error. 
			Therefore we define the \emph{observed} correlation between SNP$_i$ and the phenotype($\hat{\rho_i}$) to be
			\begin{equation}
			\hat{\rho_i} = \rho_i+\frac{\epsilon_i}{\sqrt{n}}
			\label{eq:obsPheno}
			\end{equation}
			for some error $\epsilon_i$. 
			The distribution of the correlation coefficient about the true correlation $\rho$ is approximately
			$$
				\hat{\rho_i}\sim f(\rho_i, \frac{(1-\rho^2)^2}{n})
			$$
			By making the assumption that $\rho_i$ is close to 0 for all $i$, we have 
			\begin{align*}
				\mathrm{E}(\epsilon_i|\rho_i)&\sim 0\\
				\mathrm{Var}(\epsilon_i|\rho_i)&\sim 1
			\end{align*}
			We then define our $z$-statistic and $\chi^2$-statistic as
			\begin{align*}
				z_i &= \hat{\rho_i}\sqrt{n} \\
				\chi^2 &= z_i^2\\
				&=\hat{\rho_i}^2n
			\end{align*}
			From \cref{eq:obsPheno} and \cref{eq:corPhenoTrue}, $\chi^2$ can then be expressed as
			\begin{align*}
			\chi^2&=\hat{\rho}^2n\\
			&=n(\boldsymbol{R}_i\boldsymbol{\beta}_j+\frac{\epsilon_i}{\sqrt{n}})^2
			\end{align*}
			The expectation of $\chi^2$ is then
			\begin{align*}
			\mathrm{E}(\chi^2) &= n(\boldsymbol{R}_i\boldsymbol{\beta\beta}^t\boldsymbol{R}_i+2\boldsymbol{R}_i\boldsymbol{\beta}\frac{\epsilon_i}{\sqrt{n}}+\frac{\epsilon_i^2}{n}) \\
			&= n\boldsymbol{R}_i\boldsymbol{H}\boldsymbol{R}_i+1
			\end{align*}
			To derive least square estimates of $h_i^2$, we need to find $\hat{h_i^2}$ which minimizes
			\begin{align*}
				\sum_i(\chi_i^2-\mathrm{E}(\chi_i^2))^2&=\sum_i(\chi_i^2-(n\boldsymbol{R}_i\boldsymbol{H}\boldsymbol{R}_i+1))^2 \\
				&=\sum_i(\chi_i^2-1-n\boldsymbol{R}_i\boldsymbol{H}\boldsymbol{R}_i)^2 
			\end{align*}
			If we define 
			\begin{equation}
			f_i= \frac{\chi_i^2-1}{n}
			\label{eq:defineF}
			\end{equation}
			we got
			\begin{align}
			\sum_i(\chi_i^2-\mathrm{E}(\chi_i^2))^2&=\sum_i(f_i-\boldsymbol{R}_i\boldsymbol{H}\boldsymbol{R}_i)^2 \nonumber\\
			&=\boldsymbol{ff}^t-2\boldsymbol{f}^t\boldsymbol{R_{sq}\hat{h}}+\boldsymbol{\hat{h}}^t\boldsymbol{R_{sq}}^t\boldsymbol{R_{sq}\hat{h}}
			\label{eq:leastSquareH}
			\end{align}
			where $\boldsymbol{R_{sq}} = \boldsymbol{R}\circ\boldsymbol{R}$.
			By differentiating \cref{eq:leastSquareH} w.r.t $\hat{h}$ and set to 0, we get
			\begin{align}
				2\boldsymbol{R_{sq}}^t\boldsymbol{R_{sq}}\boldsymbol{\hat{h^2}}-2\boldsymbol{R_{sq}f}&=0 \nonumber\\
				\boldsymbol{R_{sq}}\boldsymbol{\hat{h^2}} &=\boldsymbol{f}
				\label{eq:shrekEq}
			\end{align}
			And the heritability is then defined as 
			\begin{equation}
			\hat{Heritability} = \boldsymbol{1}^t\boldsymbol{R_{sq}}^{-1}\boldsymbol{f}
			\label{eq:fullShrek}
			\end{equation}
		\subsection{Calculating the \Glng{se}}
			From \cref{eq:fullShrek}, we can derive the variance of heritability $H$ as 
			\begin{align}
				\mathrm{Var}(H) &= \mathrm{E}[H^2]-\mathrm{E}[H]^2\nonumber\\
				&=\mathrm{E}[(\boldsymbol{1}^t\boldsymbol{R_{sq}}^{-1}\boldsymbol{f})^2]-\mathrm{E}[\boldsymbol{1}^t\boldsymbol{R_{sq}}^{-1}\boldsymbol{f}](\mathrm{E}[\boldsymbol{1}^t\boldsymbol{R_{sq}}^{-1}\boldsymbol{f}])^t \nonumber \\
				&=\mathrm{E}[\boldsymbol{1}^t\boldsymbol{R_{sq}}^{-1}\boldsymbol{ff}^t\boldsymbol{R_{sq}}^{-1}\boldsymbol{1}]-\mathrm{E}[\boldsymbol{1}^t\boldsymbol{R_{sq}}^{-1}\boldsymbol{f}](\mathrm{E}[\boldsymbol{1}^t\boldsymbol{R_{sq}}^{-1}\boldsymbol{f}])^t \nonumber \\
				&=\boldsymbol{1}^t\boldsymbol{R_{sq}}^{-1}\mathrm{E}[\boldsymbol{ff}^t]\boldsymbol{R_{sq}}^{-1}\boldsymbol{1}-\mathrm{E}[\boldsymbol{1}^t\boldsymbol{R_{sq}}^{-1}\boldsymbol{f}](\mathrm{E}[\boldsymbol{1}^t\boldsymbol{R_{sq}}^{-1}\boldsymbol{f}])^t \nonumber \\
				&=\boldsymbol{1}^t\boldsymbol{R_{sq}}^{-1}\mathrm{Var}(\boldsymbol{f})\boldsymbol{R_{sq}}^{-1}\boldsymbol{1}+\mathrm{E}[\boldsymbol{1}^t\boldsymbol{R_{sq}}^{-1}\boldsymbol{f}](\mathrm{E}[\boldsymbol{1}^t\boldsymbol{R_{sq}}^{-1}\boldsymbol{f}])^t-\mathrm{E}[\boldsymbol{1}^t\boldsymbol{R_{sq}}^{-1}\boldsymbol{f}](\mathrm{E}[\boldsymbol{1}^t\boldsymbol{R_{sq}}^{-1}\boldsymbol{f}])^t \nonumber\\
				&=\boldsymbol{1}^t\boldsymbol{R_{sq}}^{-1}\mathrm{Var}(\boldsymbol{f})\boldsymbol{R_{sq}}^{-1}\boldsymbol{1}
				\label{eq:varHvarf}
			\end{align}
			Therefore, to obtain the variance of $H$, we first need to calculate the variance covariance matrix of $\boldsymbol{f}$.
			
			We first consider the standardized genotype $X_i$ with standard normal mean $z_i$ and non-centrality parameter
			$\mu_i$, we have
			\begin{align*}
				\mathrm{E}[X_i]&=\mathrm{E}[z_i+\mu_i]\\
				&=\mu_i\\
				\mathrm{Var}(X_i) &=\mathrm{E}[(z_i+\mu_i)^2]+\mathrm{E}[(z_i+\mu_i)]^2\\
				&=\mathrm{E}[z_i^2+\mu_i^2+2z_i\mu_i]+\mu_i^2\\
				&=1 \\
				\mathrm{Cov}(X_i,X_j)&=\mathrm{E}[(z_i+\mu_i)(z_j+\mu_j)]-\mathrm{E}[z_i+\mu_i]\mathrm{E}[z_j+\mu_j]\\
				&=\mathrm{E}[z_iz_j+z_i\mu_j+\mu_iz_j+\mu_i\mu_j]-\mu_i\mu_j\\
				&=\mathrm{E}[z_iz_j]+\mathrm{E}[z_i\mu_j]+\mathrm{E}[z_j\mu_i]+\mathrm{E}[\mu_i\mu_j]-\mu_i\mu_j\\
				&=\mathrm{E}[z_iz_j]
			\end{align*}
			As the genotypes are standardized, therefore $\mathrm{Cov}(X_i,X_j)==\mathrm{Cor}(X_i,X_j)$, we can obtain
			$$
				\mathrm{Cov}(X_i,X_j)=\mathrm{E}[z_iz_j]=R_{ij}
			$$
			where $R_{ij}$ is the \gls{LD} between \gls{SNP}$_i$ and \gls{SNP}$_j$.
			Given these information, we can then calculate $\mathrm{Cov}(\chi_i^2,\chi_j^2)$ as:
			\begin{align*}
				\mathrm{Cov}(X_i^2,X_j^2)=&\mathrm{E}[(z_i+\mu_i)^2(z_j+\mu_j)^2]-\mathrm{E}[z_i+\mu_i]\mathrm{E}[z_j+\mu_j]\\
				=&\mathrm{E}[(z_i^2+\mu_i^2+2z_i\mu_i)(z_j^2+\mu_j^2+2z_j\mu_j)] \\
				&-\mathrm{E}[z_i^2+\mu_i^2+2z_i\mu_i]\mathrm{E}[z_j^2+\mu_j^2+2z_j\mu_j]\\
				=&\mathrm{E}[(z_i^2+\mu_i^2+2z_i\mu_i)(z_j^2+\mu_j^2+2z_j\mu_j)]\\
				&-(\mathrm{E}[z_i^2]+\mathrm{E}[\mu_i^2]+2\mathrm{E}[z_i\mu_i])(\mathrm{E}[z_j^2]+\mathrm{E}[\mu_j^2]+2\mathrm{E}[z_j\mu_j])\\
				=&\mathrm{E}[z_i^2(z_j^2+\mu_j^2+2z_j\mu_j)+\mu_i^2(z_j^2+\mu_j^2+2z_j\mu_j)+2z_i\mu_i(z_j^2+\mu_j^2+2z_j\mu_j)]\\
				&-(1+\mu_i^2)(1+\mu_j^2)\\
				=&\mathrm{E}[z_i^2(z_j^2+\mu_j^2+2z_j\mu_j)]+\mu_i^2\mathrm{E}[z_j^2+\mu_j^2+2z_j\mu_j]\\
				&+2\mu_i\mathrm{E}[z_i(z_j^2+\mu_j^2+2z_j\mu_j)]-(1+\mu_i^2)(1+\mu_j^2)\\
				=&\mathrm{E}[z_i^2z_j^2+z_i^2\mu_j^2+2z_i^2z_j\mu_j]+\mu_i^2+\mu_i^2\mu_j^2\\
				&+2\mu_i\mathrm{E}[z_iz_j^2+z_i\mu_j^2+2z_iz_j\mu_j]-(1+\mu_i^2)(1+\mu_j^2)\\
				=&\mathrm{E}[z_i^2z_j^2]+\mu_j^2+\mu_i^2+\mu_i^2\mu_j^2+4\mu_i\mu_j\mathrm{E}[z_iz_j]-(1+\mu_i^2+\mu_j^2+\mu_i\mu_j)\\
				=&\mathrm{E}[z_i^2z_j^2]+4\mu_i\mu_j\mathrm{E}[z_iz_j]-1
			\end{align*}
			Remember that $\mathrm{E}[z_iz_j] = R_{ij}$, we then have
			$$
				\mathrm{Cov}(X_i^2, X_j^2)=\mathrm{E}[z_i^2z_j^2]+4\mu_i\mu_jR_{ij}-1
			$$
			By definition, 
			$$
				z_i|z_j\sim N(\mu_i+R_{ij}(z_j-\mu_j),1-R_{ij}^2)
			$$
			We can then calculate $\mathrm{E}[z_i^2z_j^2]$ as
			\begin{align*}
				\mathrm{E}[z_i^2z_j^2]&=\mathrm{Var}[z_iz_j]+\mathrm{E}[z_iz_j]^2\\
				&=\mathrm{E}[\mathrm{Var}(z_iz_j|z_i)]+\mathrm{Var}[\mathrm{E}[z_iz_j|z_i]]+R_{ij}^2\\
				&=\mathrm{E}[z_j^2\mathrm{Var}(z_i|z_j)]+\mathrm{Var}[z_j\mathrm{E}[z_i|z_j]]+R_{ij}^2\\
				&=(1-R_{ij}^2)\mathrm{E}[z_j^2]+\mathrm{Var}(z_j(\mu_i+R_{ij}(z_j-\mu_j)))+R_{ij}^2\\
				&=(1-R_{ij}^2)+\mathrm{Var}(z_j\mu_i+R_{ij}z_j^2-\mu_jz_jR_{ij})+R_{ij}^2\\
				&=1+\mu_i^2\mathrm{Var}(z_j)+R_{ij}^2\mathrm{Var}(z_j^2)-\mu_j^2R_{ij}^2\mathrm{Var}(z_j)\\
				&=1+2R_{ij}^2
			\end{align*}
			As a result, the variance covariance matrix of the $\chi^2$ variances represented as
			\begin{equation}
				\mathrm{Cov}(X_i^2,X_j^2) = 2R_{ij}^2+4R_{ij}\mu_i\mu_j
				\label{eq:finalChi}
			\end{equation}
			As we only have the \emph{observed} expectation, we should re-define \cref{eq:finalChi} as
			\begin{equation}
				\mathrm{Cov}(X_i^2,X_j^2) = \frac{2R_{ij}^2+4R_{ij}\mu_i\mu_j}{n^2}
				\label{eq:finalChiCov}
			\end{equation}
			where $n$ is the sample size.
			
			By substituting \cref{eq:finalChiCov} into \cref{eq:varHvarf}, we will get
			\begin{align}
				\mathrm{Var}(H) &=\boldsymbol{1}^t\boldsymbol{R_{sq}}^{-1}\frac{2\boldsymbol{R_{sq}}+4\boldsymbol{R}\circ \boldsymbol{zz}^t}{n^2}\boldsymbol{R_{sq}}^{-1}\boldsymbol{1}
				\label{eq:covH}
			\end{align}
			where $\boldsymbol{z} = \sqrt{\boldsymbol{\chi^2}}$ from \cref{eq:defineF}, with the direction of effect as its sign and $\circ$ is the element-wise product (Hadamard product).
			 
			The problem with \cref{eq:covH} is that it requires the direction of effect. 
			Without the direction of effect, the estimation of \gls{se} will be inaccurate. 
			If we consider that $\boldsymbol{f}$ is approximately $\chi^2$ distributed, we might view \cref{eq:shrekEq} as a decomposition of a vector of $\chi^2$ distributions with degree of freedom of 1. 
			Replacing the vector $\boldsymbol{f}$ with a vector of 1, we can perform the decomposition of the degree of freedom, getting the ``effective number''($e$) of the association\citep{Li2011}. 
			%The problem of this effective number is that they uses the eigenvalue instead of this multiplication.
			%So either we have to explain why we don't follow it (therefore explaining the slidding windows) or we should just avoid mentioning the effective number
			Substituting $e$ into the variance equation of non-central $\chi^2$ distribution will yield
			\begin{equation}
			\mathrm{Var}(H) = \frac{2(e+2H)}{n^2}
			\label{eq:effectiveChi}
			\end{equation}
			\cref{eq:effectiveChi} should in theory gives us an heuristic estimation of the \gls{se}. 
			Moreover, the direction of effect was not required for \cref{eq:effectiveChi}, reducing the number of input required from the user.
		\subsection{Case Control Studies}	 
		%Discuss on the liability threshold model. Then the apply orange paper. Then explain how to get the results. 
			When dealing with case control data, as the phenotype were usually discontinuous, we cannot directly use \cref{eq:fullShrek} to estimate the heritability.
			Instead, we will need to employ the concept of liability threshold model from \cref{sec:liability}. 
			
			Based on the derivation of \citet{Yang2010}, the approximate ratio between the \gls{ncp} obtained from case control studies ($NPC_{CC}$) and quantitative trait studies($NCP_{QT}$) were
		
			\begin{equation}
			\frac{NCP_{CC}}{NCP_{QT}} = \frac{i^2v(1-v)N_{CC}}{(1-K)^2N_{QT}}
			\label{eq:originNCPTransform}
			\end{equation}
			where
			\begin{align*}
			 K &= \text{Population Prevalence} \\
			 v &= \text{Proportion of Cases}\\
			 N &= \text{Total Number of Samples}\\
			 i &= \frac{z}{K}\\
			 z &= \text{height of standard normal curve at truncation pretained to K}
			\end{align*}
			
			Using this approximation deviated by \citet{Yang2010}, we can directly transform the \gls{ncp} between the case control studies and quantitative trait studies.
			As we were transforming the \gls{ncp} of a single study, the $N_{CC}$ and $N_{QT}$ will be the same, therefore \cref{eq:originNCPTransform} became
			\begin{equation}
			NCP_{QT} = \frac{NCP_{CC}(1-K)^2}{i^2v(1-v)}
			\label{eq:transform}
			\end{equation}
			
			By combining \cref{eq:transform} and \cref{eq:defineF}, we can then have
			\begin{equation}
			f = \frac{(\chi^2_{CC}-1)(1-K)^2}{ni^2v(1-v)}
			\end{equation}
			where $\chi^2_{CC}$ is the test statistic from the case control association test.
			Finally, the heritability estimation of case control studies can be simplified to 
			\begin{equation}
			\hat{Heritability} =\frac{(1-K)^2}{i^2v(1-v)} \boldsymbol{1}^t\boldsymbol{R_{sq}}^{-1}\boldsymbol{f}
			\label{eq:caseControlHerit}
			\end{equation}
			
		\subsection{Extreme Phenotype Selections}
			%Explain why we perform extreme phenotype selections. Explain how that affect the variance of the estimation. Finally, explain how to perform heritability estimation on extreme phenotype. 
			When extreme phenotype selection were performed, the variance of the selected phenotype will not be representative of that in the population.
			Most notably, the variance of the post selection phenotype will tends to increase.
			Thus, to adjust for this bias, one can multiple the estimated heritability $\hat{h^2}$ by the ratio between the variance before $V_P$ and after $V_{P'}$ the selection process\citep{Sham2014}:
			
			\begin{equation}
			\hat{Heritability} = \frac{V_{P'}}{V_P}\boldsymbol{1}^t\boldsymbol{R_{sq}}^{-1}\boldsymbol{f}
			\label{eq:extremeShrek}
			\end{equation}
			
		\subsection{Calculating the \glsentrylong{LD} matrix}
			% Might want to remove this section as we no longer use this correction
			To estimate the heritability, the population \gls{LD} matrix is required.
			In reality, one can only obtain the \gls{LD} matrix based on a subset of the population (e.g. the 1000 genome project\citep{Project2012} or the HapMap project\citep{Altshuler2010}).
			There are therefore sampling errors among the \gls{LD} elements. 
			
			Now if we consider \cref{eq:fullShrek}, the $\boldsymbol{R_{sq}}$ matrix is required.
			As the squared \gls{LD} is used, a positive bias is induced into our $\boldsymbol{R_{sq}}$ matrix. 
			
			Based on \citet{Shieh2010}, one can correct for bias in the Pearson correlation $\rho$ using
			\begin{equation}
			\rho = \rho\{1+\frac{1-\rho^2}{2(N-4)}\}
			\label{eq:rhoCorrect}
			\end{equation}
			where $N$ is the number of sample used in the calculation of $\rho$. 
			Similarly, there exists a bias correction equation for $\rho^2$:
			\begin{equation}
				\rho^2=1-\frac{N-3}{N-2}(1-\rho^2)\{1+\frac{2(1-\rho^2)}{N-3.3}\}
				\label{eq:rho2Correct}
			\end{equation}
			Therefore, we corrected the $\boldsymbol{R_{sq}}$ based on \cref{eq:rho2Correct} such that the bias in estimation can be minimized. 
		\subsection{Inverse of the \glsentrylong {LD} matrix}
			In order to obtain the heritability estimation, we will require to solve \cref{eq:fullShrek}. 
			If $\boldsymbol{R_{sq}}$ is of full rank and positive semi-definite, it will be straight-forward to solve the matrix equation.
			However, more often than not, the \gls{LD} matrix are rank-deficient and suffer from multicollinearity, making it ill-conditioned, therefore highly sensitive to changes or errors in the input.
			To be exact, we can view \cref{eq:fullShrek} as calculating the sum of $\boldsymbol{\hat{h^2}}$ from  \cref{eq:shrekEq}.
			This will involve solving for
			\begin{equation}
			\boldsymbol{\hat{h^2}} = \boldsymbol{R_{sq}}^{-1}\boldsymbol{f}
			\label{eq:shrekInverse}
			\end{equation}
			where an inverse of $\boldsymbol{R_{sq}}$ is observed. 
			
			In normal circumstances (e.g. when $\boldsymbol{R_{sq}}$ is full rank and positive semi-definite), one can easily solve \cref{eq:shrekInverse} using the QR decomposition or LU decomposition.
			However, when $\boldsymbol{R_{sq}}$ is ill-conditioned, the traditional decomposition method will fail.
			Even if the decomposition is successfully performed, the result tends to be a meaningless approximation to the true $\boldsymbol{\hat{h^2}}$. 
			
			Therefore, to obtain a meaningful solution, regularization techniques such as the Tikhonov Regularization (also known as Ridge Regression) and \gls{tSVD} has to be performed\citep{Neumaier1998}. 
			There are a large variety of regularization techniques, yet the discussion of which is beyond the scope of this study. 
			In this study, we will focus on the use of \gls{tSVD} in the regularization of the \gls{LD} matrix.
			This is because the \gls{SVD} routine has been implemented in the EIGEN C++ library \citep{eigenweb}, allowing us to implement the \gls{tSVD} method without much concern with regard to the detail of the algorithm. 
			
			To understand the problem of the ill-conditioned matrix and regularization method, we consider the matrix equation $\boldsymbol{Ax}=\boldsymbol{B}$ where $\boldsymbol{A}$ is ill-conditioned or singular with $n\times n$ dimension.
			The \gls{SVD} of $\boldsymbol{A}$ can be expressed as 
			\begin{align}
				\boldsymbol{A} = \boldsymbol{U\Sigma V}^t
				\label{eq:svd}
			\end{align}
			where $\boldsymbol{U}$ and $\boldsymbol{V}$ are both orthogonal matrix and $\boldsymbol{\Sigma}=\mathrm{diag}(\sigma_1,\sigma_2,\dots,\sigma_n)$ is the diagonal matrix of the \emph{singular values}($\sigma_i$) of matrix $\boldsymbol{A}$.
			Based on \cref{eq:svd}, we can get the inverse of $\boldsymbol{A}$ as 
			\begin{align}
				\boldsymbol{A}^{-1}= \boldsymbol{V\Sigma}^{-1}\boldsymbol{U}^t
				\label{eq:svdInverse}
			\end{align}
			Where $
			\boldsymbol{\Sigma}^{-1} = \mathrm{diag}(\frac{1}{\sigma_1},\frac{1}{\sigma_2},\dots,\frac{1}{\sigma_n})$.
			Now if we consider there to be error within $\boldsymbol{B}$ such that
			\begin{equation}
				\boldsymbol{\hat{B_i}} = \boldsymbol{B_i}+\epsilon_i
				\label{eq:errorB}
			\end{equation}
			we can then represent $\boldsymbol{Ax}=\boldsymbol{B}$ as
			\begin{align}
				\boldsymbol{Ax}&=\boldsymbol{\hat{B}} \nonumber\\
				\boldsymbol{U\Sigma V}^t\boldsymbol{x}&=\boldsymbol{\hat{B}} \nonumber\\
				\boldsymbol{x}&=\boldsymbol{V\Sigma}^{-1}\boldsymbol{U}^t\boldsymbol{\hat{B}}
				\label{eq:solveBwithError}
			\end{align}
			A matrix $\boldsymbol{A}$ is considered as ill-condition when its condition number $\kappa(\boldsymbol{A})$ is large or singular when its condition number is infinite. 
			One can represent the condition number as $\kappa(\boldsymbol{A})=\frac{\sigma_1}{\sigma_n}$.
			Therefore it can be observed that when $\sigma_n$ is tiny, $\boldsymbol{A}$ is likely to be ill-conditioned and when $\sigma_n=0$, $\boldsymbol{A}$ will be singular. 
			
			One can also observe from \cref{eq:solveBwithError} that when the singular value $\sigma_i$ is small, the error $\epsilon_i$ in \cref{eq:errorB} will be drastically magnified by a factor of $\frac{1}{\sigma_i}$. 
			Making the system of equation highly sensitive to errors in the input.
			
			To obtain a meaningful solution from this ill-conditioned/singular matrix $\boldsymbol{A}$, we may perform the \gls{tSVD} method to obtain a pseudo inverse of $\boldsymbol{A}$.
			Similar to \cref{eq:svd}, the \gls{tSVD} of $\boldsymbol{A}$ can be represented as 
			\begin{alignat}{2}
				&\boldsymbol{A}^+ = \boldsymbol{U\Sigma}_k\boldsymbol{V}^t  &\qquad\text{and}\qquad  &\boldsymbol{\Sigma}_k=\mathrm{diag}(\sigma_1,\dots,\sigma_k,0,\dots,0)
				\label{eq:tsvd}				
			\end{alignat}
			where $\boldsymbol{\Sigma}_k$ equals to replacing the smallest $n-k$ singular value replaced by 0 \citep{Hansen1987}. 
			Alternatively, we can define
			\begin{equation}
			\sigma_i=\begin{cases}
			\sigma_i\qquad\text{for}\qquad\sigma_i\ge t\\
			0\qquad\text{for}\qquad\sigma_i<t
			\end{cases}
			\end{equation}
			where $t$ is the tolerance threshold. 
			Any singular value $\sigma_i$ less than the threshold will be replaced by 0. 
			
			By selecting an appropriate $t$, \gls{tSVD} can effectively regularize the ill-conditioned matrix and help to find a reasonable approximation to $x$. 
			A problem with \gls{tSVD} however is that it only work when matrix $\boldsymbol{A}$ has a well determined numeric rank\citep{Hansen1987}.
			That is, \gls{tSVD} work best when there is a large gap between $\sigma_k$ and $\sigma_{k+1}$.
			If a matrix has ill-conditioned rank, then $\sigma_k-\sigma_{k+1}$ will be small.
			For any threshold $t$, a small error can change whether if $\sigma_{k+1}$ and subsequent singular values should be truncated, leading to unstable results. 
			
			According to \citet{Hansen1987}, matrix where its rank has meaning will have well defined rank. 
			As \gls{LD} matrix is the correlation matrix between each individual \glspl{SNP}, the rank of the \gls{LD} matrix is the maximum number of linear independent \glspl{SNP} in the region, therefore likely to have a well-defined rank. 
			The easiest way to test whether if the threshold $t$ and whether if the matrix $\boldsymbol{A}$ has well-defined rank is to calculate the ``gap'' in the singular value:
			\begin{equation}
			gap = \sigma_k/\sigma_{k+1}
			\label{eq:gapSingular}
			\end{equation}
			a large gap usually indicate a well-defined gap. 
			
			In this study, we adopt the threshold as defined in MATLAB, NumPy and GNU Octave: $t=\epsilon\times\mathrm{max}(m,n)\times\mathrm{max}(\boldsymbol{\Sigma})$ where $\epsilon$ is the machine epsilon (the smallest number a machine can define as non-zero). 
			And we perfomed a simulation study to investigate the performance of \gls{tSVD} under the selected threshold.
			Ideally, if the ``gap'' is large under the selected threshold, then \gls{tSVD} will provide a good regularization to the equation. 
			
			1,000 samples were randomly simulated from the HapMap\citep{Altshuler2010} \acrshort{CEU} population with
			1,000 \glspl{SNP} randomly select from chromosome 22. 
			The \gls{LD} matrix and its corresponding singular value were calculated. 
			The whole process were repeated 50 times and the cumulative distribution of the ``gap'' of singular values were plotted (\cref{fig:singularValueDist}). 
			It is clearly show that the \gls{LD} matrix has a well-defined rank with a mean of maximum ``gap'' of 466,198,939,298.
			Therefore the choice of \gls{tSVD} for the regularization is appropriate.
			%\begin{wrapfigure}{L}{3in}
			\begin{figure}
				\caption[Cumulative Distribution of ``gap'' of the LD matrix]{Cumulative Distribution of ``gap'' of the LD matrix, the vertical line indicate the full rank. It can be observed that there is a huge increase in ``gap'' before full rank is achieved. Suggesting that the rank of the LD matrix is well defined}
				\centering
				\includegraphics[width=0.5\textwidth]{figure/singular_value_distribution.png}
				\label{fig:singularValueDist}
				\vspace{-20pt}
			\end{figure}
			%\end{wrapfigure}
			
			By employing the \gls{tSVD} as a method for regularization, we were able to solve the ill-posed \cref{eq:shrekEq}, and obtain the estimated heritability.
						
		\subsection{Comparing with \glsentrylong{ldsc}}
			% main difference 
			Conceptually, the fundamental hypothesis of \gls{ldsc} and our algorithm were quite different.
			\gls{ldsc} were based on the ``global'' inflation of test statistic and its relationship to the \gls{LD} pattern.
			\gls{ldsc} hypothesize that the larger the \gls{LD} score, the more likely will the \gls{SNP} be able to ``tag'' the causal \gls{SNP} and the heritability can then be estimated through the regression between the \gls{LD} score and the test statistic.
			
			On the other hand, our algorithm focuses more on the per-\gls{SNP} level.
			Our main idea was that the individual test statistic of each \glspl{SNP} is a combination of its own effect and effect from \glspl{SNP} in \gls{LD} with it. 
			Thus, based on this concept, our algorithm aimed to ``remove'' the inflation of test statistic introduced through the \gls{LD} between \glspl{SNP} and the heritability can be calculated by adding the test statistic of all \glspl{SNP} after ``removing'' the inflation. 
			
			Mathematically, the calculation of \gls{ldsc} and our algorithm were also very different. 
			\gls{ldsc} take the sum of all $R^2$ within a 1cM region as the LD score and regress it against the test statistic to obtain the slope and intercept which represent the heritability and amount of confounding factors respectively. 
			In their model, \gls{ldsc} assume that each \glspl{SNP} will explain the same portion of heritability
			\begin{align}
			 \mathrm{Var}(\beta)&=\frac{h^2}{M}\boldsymbol{I}\\
			 M &= \text{number of SNPs}\notag\\
			 \beta &= \text{vector containing per normalized genotype effect sizes}\notag\\
			 I &= \text{identity matrix}\notag\\
			 h^2 &= \text{heritability}\notag
			\end{align}
			
			As for our algorithm, the whole \gls{LD} matrix were used and inverted to decompose the \gls{LD} from the test statistic. 
			There were no assumption of the amount of heritability explained by each \glspl{SNP}. 
			However, our algorithm does assumed that the null should be 1 and therefore cannot detect the amount of confounding factors. 
					
	\section{Simulation}
		First, we would like to test how well our algorithm works for heritability estimation under different scenarios.
		To account for different genetic architecture, we varies the heritability of the trait, the number of causal \glspl{SNP} and the genotypes(therefore varies the \gls{LD} pattern) during the quantitative trait simulation.
		
		\subsection{Sample Size}
		One important consideration in our simulation was the number of sample simulated. 
		The sample size was the most important parameter in determining the standard error of the heritability estimation. 
		As sample size increases, study will be more representative of the true population. 
		The increased number of information also means a better estimation of parameters, therefore a smaller \acrfull{se}.
		% awk -F "\t" '{print $2"\t"$9}' full | uniq | sed -e 's/[^0-9[:space:]]//g' | awk '{for(i=2;i<=NF;++i)j+=$i; print $1" "j; j=0}' | sort | uniq  %script for text mining
		Based on information from \gls{GWAS} catalog\citep{Welter2014}, we calculate the sample size distribution using simple text mining and exclude studies with conflicting sample size information in multiple entries. 
		The average sample size for all \gls{GWAS} recorded on the \gls{GWAS} catalog was 7,874, with a median count of 2,506 and a lower quartile at 940 (\cref{fig:gwasCata}). 
		We argue that if the algorithm works for studies with a small sample size (e.g lower quartile sample size), then it should perform even better when the sample size is larger. 
		Thus, we only simulate 1,000 samples in our simulation, which roughly represent the lower quartile sample size range.
		
		\begin{wrapfigure}{R}{8cm}
			\centering
			\includegraphics[width=0.5\textwidth]{figure/gwasSampleSize.png}
			\caption[GWAS Sample Size distribution]{
				\gls{GWAS} sample size distribution.
				}
			\label{fig:gwasCata}
		\end{wrapfigure}
		
		\subsection{Number of SNPs in Simulation}
		Another consideration in the simulation was the number of \glspl{SNP} included.
		In a typical \gls{GWAS} study, there are usually a larger number of \glspl{SNP} when compared to the sample size. 
		Fr example, in the \gls{pgc} \glng{scz} \gls{GWAS}, more than 9 million \glspl{SNP} were included, with around 700,000 \glspl{SNP} on chromosome 1.
		Although it would be idea to simulate 700,000 \glspl{SNP} in our simulation, the time required for simulating the samples will become unrealistic.
		
		As the number of \glspl{SNP} simulated grow, more time were required for the simulation of samples and more calculation will be required.
		Moreover, the increasing number of \glspl{SNP} will lead to increased size of the \gls{LD} matrix, requiring a long time for the inverse of the matrix.
		In reality, this should not be a real problem as one typically only calculate the heritability of the data set once and the speed of the algorithm is still relatively fast. 
		However, in the case of simulation where we would like to repeat the same analysis many times, the small increment of time will lead to an escalation in total simulation time, making the simulation infeasible. 
		To compromise, we simulate a total of 50,000 \glspl{SNP} from chromosome 1 as a balance between run time of simulation and the total \glspl{SNP} simulated.
		
		\subsection{Genetic Architecture}
		Of all simulation parameter, the genetic architecture was the most complicated and important parameter. 
		The \gls{LD} pattern, the number of causal \glspl{SNP}, the effect size of the causal \glspl{SNP} and the heritability of the trait were all important factors contribute to the genetic architecture of a trait. 
		
		First and foremost, because the aim of the algorithm was to estimating the heritability of the trait, it is important that the algorithm works for traits from different heritability spectrum.
		We therefore simulate traits with heritability ranging from 0 to 0.9, with increment of 0.1.
		
		Secondly, in real life scenario, the ``causal'' variant might not be readily included on the \gls{GWAS} chip and were only ``tagged'' by \glspl{SNP} included on the \gls{GWAS} chip.
		However, to simplify our simulation, all ``causal'' variants were included in our simulation (e.g. perfectly ``tagged'')
		
		Thirdly, to obtain a realistic \gls{LD} pattern, we simulate the genotypes using the HAPGEN2 programme\citep{Su2011}, using the 1000 genome \gls{CEU} haplotypes as an input.
		In short, HAPGEN2 simulate new haplotypes as an imperfect mosaic of haplotpyes from a reference panel and the haplotypes that have already been simulated using the \textit{Li and Stephens} (LS) model of \gls{LD} \citep{Li2003}.
		In a typical \gls{GWAS} , one usually only have power in detecting ``common variants'', usually defined as variants with \gls{maf} $\ge 0.01$.
		We therefore only consider scenario with ``common'' variants and only use \glspl{SNP} with \gls{maf} $\ge0.1$ in the \gls{CEU} haplotypes as an input to HAPGEN2. 
		This will reduce the probability of having \glspl{SNP} with \gls{maf} $<0.01$ in the final simulated sample sets.
		
		Finally, we would like to simulate traits with different inheritance model such as oligogenic traits and polygenic traits.
		We therefore varies the number of causal \glspl{SNP} ($k$) with $k\in\{5, 10, 50, 100, 250, 500\}$.
		
		
		To assess how well our algorithm performs for narrow sense heritability estimation in comparison to other current methods, we performed series of systematic simulation.
		In these simulations, performance of our algorithm, \gls{gcta} \citep{Yang2011} and \gls{ldsc} \citep{Bulik-Sullivan2015} with and without the intercept estimation function (-{}-no-intercept) were tested.
		Through simulation, we can obtain the sample distribution of the heritability estimate under different study designs (e.g. Quantitativat traits, Case-Control studies or extreme phenotype selection). 
		Factors considered in our simulations were as follow:		
		
		
		
		\subsection{Genetic Architecture} % Should contain the effect size distribution, the causal SNP number and the heritability spectrum
		Of all the simulation parameter, the genetic architecture was arguably the most complicated parameter. 
		It involves the \gls{LD} pattern, the distribution of effect size, the number of causal \glspl{SNP}, the \gls{maf} of the causal \glspl{SNP} and most importantly, the heritability of the trait ($h^2$).
		
		Because we would like to cover most of the heritability spectrum, we will simulate traits with $h^2$ ranging from 0 to 0.9, with increment of 0.1 such that $h^2 \in \{0,\allowbreak 0.1,\allowbreak 0.2,\allowbreak 0.3,\allowbreak 0.4,\allowbreak 0.5,\allowbreak 0.6,\allowbreak 0.7,\allowbreak 0.8,\allowbreak 0.9\}$.
		To simplify the condition, all ``causal'' variants were included in the simulation (e.g. perfect tagging).
		We also try to obtain realistic \gls{LD} pattern by using HAPGEN2\citep{Su2011} to simulate genotype based on the \gls{LD} pattern of the 1000 genome \gls{CEU} samples. 
		
		
		First, we only consider situation of common \glspl{SNP} and only simulate \glspl{SNP} with \gls{maf} $> 0.1$.
		In these simulation, we varies the number of causal \glspl{SNP} ($k$) and the effect size distribution.
		We consider $k\in\{5, 10, 50, 100, 250, 500\}$ such that we can cover different disease spectrum (Oligogenic to Polygenic diseases). 
		As for effect size distribution, we considered two conditions:
		\begin{enumerate}
			\item Equal Effect Size
			\item Random Effect Size
		\end{enumerate}
		The simplest situation was when all casual \glspl{SNP} have the same effect size 
		\begin{equation}
		\beta_s=\pm\sqrt{\frac{h^2}{k}}
		\label{eq:stableEffect}
		\end{equation}
		The direction of effect should be randomly simulated.
		As for the random effect size scenario, we consider the effect size to be 
		\begin{equation}
		\beta_r=\pm\sqrt{\frac{\gamma \times h^2}{\sum \gamma}}
		\label{eq:randomEffect}
		\end{equation}
		with $\gamma\sim exp(\lambda=1)$ and a random direction of effect.
		Rationale behind the choice of exponential distribution was based on \citet{Orr1998}, which suggested that exponential distribution with $\lambda=1$ may serve as a heuristic expectation the genetic architecture of adaptation.

		Once the effect size was calculated, we can then randomly assign the effect size to $k$ random \glspl{SNP} with normalized genotype $\boldsymbol{X}$, the phenotype can then be calculated as 
		\begin{align}
		\epsilon_i&\sim N(0,\sqrt{\mathrm{Var}(\boldsymbol{X\beta})\frac{1-h^2}{h^2}} )\notag\\
		\boldsymbol{\epsilon} &= (\epsilon_1,\epsilon_2,...,\epsilon_n)^t\notag\\
		\boldsymbol{y} &= \boldsymbol{X\beta}+\boldsymbol{\epsilon}
		\label{eq:simulationOfPhenotype}
		\end{align}
		
		For each batch of simulated samples, we calculate the estimated heritability using our algorithm, \gls{gcta}, \gls{ldsc} with intercept fixed at 1 and \gls{ldsc} allowing for intercept estimation for each $h^2$.
		In each iteration, the sample genotype was provided to \gls{gcta} for the calculation of genetic relationship matrix (GRM) whereas for our algorithm and \gls{ldsc}, 500 independent samples were simulated based on the 1000 genome project \gls{CEU} samples\parencite{Project2012} to construct the \gls{LD} matrix and calculate the \gls{LD} score respectively.
		This was because both \gls{ldsc} and our algorithm were designed to work in situation where the raw genotype were not provided and the \gls{LD} structure was usually obtained form the public data base instead.
		Therefore to provide a realistic simulation, an independent set of reference samples were provided for our algorithm and \gls{ldsc}.
		
		The whole process were repeated 50 times with the same \glspl{SNP} set, the same causal \glspl{SNP} and the same effect size for each $h^2$ such than an empirical variance can be obtained.
		We then repeat the process 10 times on different \glspl{SNP} sets, with different causal \glspl{SNP} but the same effect size for each $h^2$.
		This should introduce a slight variation in the \gls{LD} structure and should demonstrate the robustness of the programmes under different \gls{LD} construct.
		To summarize, the simulation procedure follows:
		\begin{enumerate}
			\item Randomly select 50,000 \glspl{SNP} with \gls{maf}$>0.1$ from chromosome 1
			\item Randomly generate $k$ effect size with $k \in \{5,10,50,100,250,500\}$ following either \cref{eq:stableEffect} or \cref{eq:randomEffect}
			\item Randomly assign the effect size to $k$ \glspl{SNP}
			\item Simulate 1,000 samples using HAPGEN2 and calculate their phenotype according to \cref{eq:simulationOfPhenotype}
			\item Perform heritability estimation using our algorithm, \gls{ldsc} and \gls{gcta}
			\item Repeat step 4-5 50 times
			\item Repeat step 1-6 10 times
		\end{enumerate}
		
		\subsubsection{Extreme Effect Size}
		Another condition we were interested in was the performance of the tools when there is a small amount of \glspl{SNP} that explain a large portion of effect e.g. 50\%.
		Similarly, we only consider 1,000 samples, with 50,000 common \glspl{SNP}(\gls{maf} $>0.1$).
		We hypothesize that under the polygenic model, such extreme distribution in effect size should have much larger effect when compared to that in oligogenic condition. 
		Thus we only consider the polygenic condition where the number of causal \gls{SNP} ($k$) were limited to 100 or 250. 
		
		We simulate $m$ \glspl{SNP} accounting for 50\% of all the effect where $m\in\{1,5,10\}$.
		The effect size was then calculated as
		\begin{align}
		\beta_{eL} &= \pm\sqrt{\frac{0.5h^2}{m}} \notag\\
		\beta_{eS} &= \pm\sqrt{\frac{0.5h^2}{k-m}} \notag\\
		\beta &= \{\beta_{eL}, \beta_{eS}\}
		\label{eq:extremEffect}
		\end{align}
		the effect size were then randomly assigned to $k$ causal \glspl{SNP} and phenotype was calculated as in \cref{eq:simulationOfPhenotype}.
		The simulation procedure then becomes
		\begin{enumerate}
			\item Randomly select 50,000 \glspl{SNP} with \gls{maf}$>0.1$ from chromosome 1
			\item Randomly generate $k$ effect size with $k \in \{100,250\}$ and $m$ extreme effect, following \cref{eq:extremEffect} where $m\in{1,5,10}$
			\item Randomly assign the effect size to $k$ \glspl{SNP}
			\item Simulate 1,000 samples using HAPGEN2 and calculate their phenotype according to \cref{eq:simulationOfPhenotype}
			\item Perform heritability estimation using our algorithm, \gls{ldsc} and \gls{gcta}
			\item Repeat step 4-5 50 times
			\item Repeat step 1-6 10 times
		\end{enumerate}
		
		\subsection{Case Control Studies}
		The simulation of case control studies was very much like that of the simulation of quantitative trait. 
		However, there were two additional parameters to consider: the population prevalence and the observed prevalence.
		These parameters allow us to simulate the samples under a liability model, therefore simulating the case control studies.

		Although there were only two additional parameter, the computational challenge for the simulation of case control was significantly bigger than that for the simulation of quantitative trait.
		Take for example, if one like to simulate a trait with population prevalence of $p$ and observed prevalence  of $q$ and would like to have $n$ cases in total, one will have to simulate $\min(\frac{n}{p}, \frac{n}{q})$ samples.
		Considering the scenario where the observed prevalence is 50\%, the population prevalence is 1\% and 1,000 cases,a minimum of 100,000 samples will be required.
		
		% Maybe instead of chromosome 1, use chromosome 22 and reduce the number of SNPs, that will be better.
		Given limited computer resources, we only simulate 1,000 cases, with an observed prevalence of 0.5 and population prevalence $p\in\{0.5, 0.1, 0.05, 0.01\}$.
		Most importantly, we reduce the number of \glspl{SNP} simulated to 5,000 and used chromosome 22 instead, such that the \glspl{SNP} density remains more or less unchanged by the number of processes required were largely reduced.
		We acknowledged that the current simulation was relatively brief, however, it should be able to serve as a prove of concept simulation to study the performance of the tools under the case control scenario.
		
		\subsection{Extreme Phenotype Selection}
		The simulation of extreme phenotype selection was the same as the quantitative trait simulation. 
		The only difference being that instead of using all samples for heritability estimation, we only use the extreme 10\% of samples among the population for the heritability estimation.
		In brief, instead of simulating 1,000 samples, we simulate 5,000 samples following the exact procedure in the quantitative trait simulation with random effect size.
		However, after simulation of the phenotype using \cref{eq:simulationOfPhenotype}, we standardize the phenotype and only select the top 10\% and bottom 10\% samples (500 samples each) from the sample distribution.
		We then perform the same simulation procedure as in the quantitative trait simulation with random effect size.
		
		It was noted that the extreme phenotype selection were not supported by the \gls{ldsc} and \gls{gcta}.
		To allow comparison in such scenario, we apply the extreme phenotype adjustment from \citet{Sham2014} to the estimation obtained from \gls{ldsc} and \gls{gcta}.
		
	\section{Result}
		The heritabilibty estimation were implemented in \gls{shrek} and is available on \url{https://github.com/choishingwan/shrek}.  
		
		To study the performance of \gls{shrek} and \gls{ldsc} in comparison to \gls{gcta}, we performed a variety of simulations to model scenarios with different number of causal \glspl{SNP}, different effect size distribution and different type of traits. 
		
		First, we examined the performance of the programmes under the quantitative trait scenario. 
		In the quantitative trait scenario, we varies the number of causal \glspl{SNP} and either assigned an equal effect size to each causal \glspl{SNP} or assigned a per-allele effect sizes drawn from the squared root of the exponential distribution with $\lambda=1$.
		
		\subsection{Quantitative Trait Simulation with Equal Effect Size}
		% QT Equal Effect
		
		\begin{figure}
			\centering
			\subfloat[SHREK]{
				\scalebox{.4}{\includegraphics{figure/he_summary/equal/shrek_Qt_Equal_mean.png}}
				\label{fig:shrekQtEqualMean}
			}
			\subfloat[GCTA]{
				\scalebox{.4}{\includegraphics{figure/he_summary/equal/gcta_Qt_Equal_mean.png}}
				\label{fig:gctaQtEqualMean}
			}\\
			\subfloat[LDSC with fix intercept]{
				\scalebox{.4}{\includegraphics{figure/he_summary/equal/ldsc_Qt_Equal_mean.png}}
				\label{fig:ldscQtEqualMean}
			}
			\subfloat[LDSC with intercept estimation]{
				
				\scalebox{.4}{\includegraphics{figure/he_summary/equal/ldscIn_Qt_Equal_mean.png}}
				\label{fig:ldscInQtEqualMean}
			}
			\caption[Quantitative Trait with Equal Effect Size Simulation Result(Mean)]
			{Mean of results from quantitative trait simulation with equal effect size simulation.
				\gls{shrek} was found to be less biased of all the tools whereas there was a slight upward bias for \gls{ldsc} when the intercept was fixed, especially when the number of causal \glspl{SNP} was small.} 
			\label{fig:QtEqualMean}
		\end{figure}
		
		\begin{figure}
			\centering
			\subfloat[SHREK]{
				\scalebox{.4}{\includegraphics{figure/he_summary/equal/shrek_Qt_Equal_sd.png}}
				\label{fig:shrekQtEqualVar}
			}
			\subfloat[GCTA]{
				\scalebox{.4}{\includegraphics{figure/he_summary/equal/gcta_Qt_Equal_sd.png}}
				\label{fig:gctaQtEqualVar}
			}\\
			\subfloat[LDSC with fix intercept]{
				\scalebox{.4}{\includegraphics{figure/he_summary/equal/ldsc_Qt_Equal_sd.png}}
				\label{fig:ldscQtEqualVar}
			}
			\subfloat[LDSC with intercept estimation]{
				
				\scalebox{.4}{\includegraphics{figure/he_summary/equal/ldscIn_Qt_Equal_sd.png}}
				\label{fig:ldscInQtEqualVar}
			}
			\caption[Quantitative Trait with Equal Effect Size Simulation Result(Variance)]
			{Variance of results from quantitative trait simulation with equal effect size simulation.
				Of all the programmes, \gls{gcta} was found to have the lowest variance, follow by \gls{ldsc} with fixed intercept.
				The variance of \gls{shrek} was slightly higher than that of \gls{ldsc} with fixed intercept and is lower than that of \gls{ldsc} with intercept estimation.
				Unlike \gls{ldsc}, the variance of \gls{shrek} was less sensitive to change in total heritability.} 
			\label{fig:QtEqualVar}
		\end{figure}
		
		\begin{figure}
			\centering
			\subfloat[SHREK]{
				\scalebox{.4}{\includegraphics{figure/he_summary/equal/shrek_Qt_Equal_sdCom.png}}
				\label{fig:shrekQtEqualVarCom}
			}
			\subfloat[GCTA]{
				\scalebox{.4}{\includegraphics{figure/he_summary/equal/gcta_Qt_Equal_sdCom.png}}
				\label{fig:gctaQtEqualVarCom}
			}\\
			\subfloat[LDSC with fix intercept]{
				\scalebox{.4}{\includegraphics{figure/he_summary/equal/ldsc_Qt_Equal_sdCom.png}}
				\label{fig:ldscQtEqualVarCom}
			}
			\subfloat[LDSC with intercept estimation]{
				
				\scalebox{.4}{\includegraphics{figure/he_summary/equal/ldscIn_Qt_Equal_sdCom.png}}
				\label{fig:ldscInQtEqualVarCom}
			}
			\caption[Quantitative Trait with Equal Effect Size Simulation Result(Estimated Variance)]
			{Estimated variance of results from quantitative trait simulation with equal effect size simulation compared to the empirical variance.
				The estimated variances of all the tools were rather sensitive to the number of causal \glspl{SNP}, where \gls{ldsc} tends to over-estimate the variance as the number of causal \glspl{SNP} decreases and \gls{shrek} and \gls{gcta} tends to under-estimate the variance.} 
			\label{fig:QtEqualVarCom}
		\end{figure}
		The simulation of equal effect size serves as a simplistic baseline model for the performance of the programmes.
		The first thing to look at is the mean estimation of heritability of the programmes.
		If there is any bias in the estimation of the programmes, one can easily visualize it by plotting the mean estimated heritability against the simulated heritability(\cref{fig:QtEqualMean}).
		
		From the graph, it is clear that there was a slight over estimation for \gls{ldsc} with fixed intercept(\cref{fig:ldscQtEqualMean}).
		The over estimation seems to be a function of the simulated heritability, where a large inflation was observed when a larger heritability was simulated.
		On the other hand, when allow for the estimation of intercept, less bias was observed for \gls{ldsc} except for the scenario where only 5 causal \glspl{SNP} was simulated where the estimation was downwardly biased.
		
		Comparing to \gls{ldsc}, \gls{shrek} has a smaller bias and tends to slightly under-estimate(\cref{fig:shrekQtEqualMean}).
		However, the bias of \gls{shrek} is insensitive to the simulated heritability, making it robust to traits with different heritability.
		Similarly, the bias of \gls{gcta} is also smaller than that of \gls{ldsc}(\cref{fig:shrekQtEqualMean}), with a slight upward bias in the estimation except when 5 causal \glspl{SNP} was simulated.
		Again, the estimate of \gls{gcta} is also relatively insensitive to the simulated heritability.
		Overall, there was no clear pattern as to how the number of causal \glspl{SNP} affects the mean estimation. 
		
		Next, we examine the empirical variance of the programmes(\cref{fig:QtEqualVar}).
		As can be seen from the graph, there is a clear pattern where the decrease in number of causal \gls{SNP} generally increases the variance for all the programmes, with \gls{shrek} least affected.
		For \gls{ldsc}, the simulated heritability also have a large impact to its empirical variance, with the empirical variance increases as the simulated heritability increases.
		In general, \gls{ldsc} with fixed intercept(\cref{fig:ldscQtEqualVar}) has a lower variance when compared to \gls{ldsc} with intercept estimation(\cref{fig:ldscInQtEqualVar}). 
		Moreover, when the number of causal \gls{SNP} is large, the variance of \gls{ldsc} with fixed intercept(\cref{fig:ldscQtEqualVar}) is lower than \gls{shrek}(\cref{fig:shrekQtEqualVar}).
		However, \gls{shrek} is more robust to change in the number of causal \glspl{SNP} and simulated heritabiliy when compared ot \gls{ldsc}.

		Of all the programmes, \gls{gcta} has the best performance(\cref{fig:gctaQtEqualVar}) except when the trait only contains 5 causal \glspl{SNP}. 
		Not only does it has the smallest variance, its empirical variances was almost invariant to change in simulated heritability.
		However, the case with 5 causal \glspl{SNP} serves as an out-lier. 
		It was most obvious when inspecting the relationship between the estimated variance and the empirical variance of \gls{gcta}(\cref{fig:gctaQtEqualVarCom}).
		Comparing the estimated variance and the empirical variance, it was clear that \gls{gcta} can, in most case accurately estimate its variance. 
		In the case of 5 causal \glspl{SNP} however, \gls{gcta} underestimates its variance.
		It was also observed in the case of 10 causal \glspl{SNP}, there was already a slight under-estimation of the variance, suggesting that there might be an increase in empirical variance that was not capture by \gls{gcta}.
		
		In the case of the programmes using the test statistic, it was observed that \gls{shrek} in general under-estimate the empirical variance(\cref{fig:shrekQtEqualVarCom}) for an average of 0.9 fold. 
		On the other hand, \gls{ldsc} over-estimates the variance for roughly 1.5 times when a fixed intercept(\cref{fig:ldscQtEqualVarCom}) was used and roughly 1.2 times when the intercept was estimated(\cref{fig:ldscInQtEqualVarCom}). 
		
		To summarize the results, we calculate the \gls{mse} of the estimation of heritability of the programmes under different simulation condition(\cref{tab:mseQtEqual}). 
		With the exception of the 5 causal \glspl{SNP} scenario, \gls{gcta} has the best performance, has a almost 2 fold smaller \gls{mse} when compared to \gls{shrek}.
		As the number of casual \glspl{SNP} increases, the performance of \gls{ldsc} with fixed intercept and \gls{shrek} converges where in general, \gls{shrek} has a smaller \gls{mse}.
		Interestingly, unlike \gls{ldsc}, \gls{shrek} was insensitive to change in number of causal \glspl{SNP} and its performance were relatively stable.
		
		% Describe the mean
		% Effect of heritability on the mean estimation
		% Effect of causal SNPs on the mean estimation
		% Descript the Variance
		% Effect of heritability on variance of estimate
		% Effect of number of causal SNPs on variance of estimation
		% Describe the estimated variance
		% How the number of causal SNPs affect the estimation of variance?

		\begin{table}
			\centering
			\begin{tabular}{rrrrr}
				\toprule
				Number of Causal SNPs&	SHREK&	LDSC&	LDSC-In&	GCTA \\
				\midrule
				5	&	0.167	&	0.308&	0.526&	0.177	\\
				10	&	0.158	&	0.243&	0.337&	0.0944	\\
				50	&	0.150	&	0.163&	0.354&	0.0749	\\
				100	&	0.154	&	0.161&	0.304&	0.0664	\\
				250	&	0.157	&	0.147&	0.255&	0.0659	\\
				500	&	0.147	&	0.148&	0.247&	0.0661	\\
				\bottomrule
			\end{tabular}
			\caption[Mean Squared Error of Quantitative Trait Simulation with Equal Effect Size]{
				\gls{mse} of quantitative trait simulation with equal effect size.
				It was observed that the overall \gls{mse} of \gls{gcta} is very low, follow by \gls{shrek}.
				As the number of causal \glspl{SNP} decreases, the \gls{mse} increases for all programmes. 
				The performance of \gls{shrek} and \gls{ldsc} with fixed intercept converges as the number of causal \glspl{SNP} increases.}
			\label{tab:mseQtEqual}
		\end{table}
		
		\subsection{Quantitative Trait Simulation with Random Effect Size}
		% QT Random Effect
		\begin{figure}
			\centering
			\subfloat[SHREK]{
				\scalebox{.4}{\includegraphics{figure/he_summary/random/shrek_Qt_Random_mean.png}}
				\label{fig:shrekQtRandMean}
			}
			\subfloat[GCTA]{
				\scalebox{.4}{\includegraphics{figure/he_summary/random/gcta_Qt_Random_mean.png}}
				\label{fig:gctaQtRandMean}
			}\\
			\subfloat[LDSC with fix intercept]{
				\scalebox{.4}{\includegraphics{figure/he_summary/random/ldsc_Qt_Random_mean.png}}
				\label{fig:ldscQtRandMean}
			}
			\subfloat[LDSC with intercept estimation]{
				
				\scalebox{.4}{\includegraphics{figure/he_summary/random/ldscIn_Qt_Random_mean.png}}
				\label{fig:ldscInQtRandMean}
			}
			\caption[Quantitative Trait with Random Effect Size Simulation Result(Mean)]
			{Mean of results from quantitative trait simulation with random effect size simulation.
				The result was very much similar to the condition where a equal effect size was simulated.
				Again, \gls{shrek} has the most accurate mean estimate when compared to other tools, with \gls{ldsc} slightly inflated.} 
			\label{fig:QtRandMean}
		\end{figure}
		
		\begin{figure}
			\centering
			\subfloat[SHREK]{
				\scalebox{.4}{\includegraphics{figure/he_summary/random/shrek_Qt_Random_sd.png}}
				\label{fig:shrekQtRandVar}
			}
			\subfloat[GCTA]{
				\scalebox{.4}{\includegraphics{figure/he_summary/random/gcta_Qt_Random_sd.png}}
				\label{fig:gctaQtRandVar}
			}\\
			\subfloat[LDSC with fix intercept]{
				\scalebox{.4}{\includegraphics{figure/he_summary/random/ldsc_Qt_Random_sd.png}}
				\label{fig:ldscQtRandVar}
			}
			\subfloat[LDSC with intercept estimation]{
				
				\scalebox{.4}{\includegraphics{figure/he_summary/random/ldscIn_Qt_Random_sd.png}}
				\label{fig:ldscInQtRandVar}
			}
			\caption[Quantitative Trait with Random Effect Size Simulation Result(Variance)]
			{Variance of results from quantitative trait simulation with random effect size simulation.
				Again, the variance of the estimate were almost the same as in simulation of equal effect size where \gls{gcta} has the smallest variance, follow by \gls{ldsc}. 
				However, it was observed when the number of causal \glspl{SNP} decreases, the variance of the estimation increases for all programme, with variance of the \gls{shrek} estimate being the least sensitive to change in heritability.
			} 
			\label{fig:QtRandVar}
		\end{figure}
		
		\begin{figure}
			\centering
			\subfloat[SHREK]{
				\scalebox{.4}{\includegraphics{figure/he_summary/random/shrek_Qt_Random_sdCom.png}}
				\label{fig:shrekQtRandVarCom}
			}
			\subfloat[GCTA]{
				\scalebox{.4}{\includegraphics{figure/he_summary/random/gcta_Qt_Random_sdCom.png}}
				\label{fig:gctaQtRandVarCom}
			}\\
			\subfloat[LDSC with fix intercept]{
				\scalebox{.4}{\includegraphics{figure/he_summary/random/ldsc_Qt_Random_sdCom.png}}
				\label{fig:ldscQtRandVarCom}
			}
			\subfloat[LDSC with intercept estimation]{
				
				\scalebox{.4}{\includegraphics{figure/he_summary/random/ldscIn_Qt_Random_sdCom.png}}
				\label{fig:ldscInQtRandVarCom}
			}
			\caption[Quantitative Trait with Random Effect Size Simulation Result(Estimated Variance)]
			{Estimated variance of results from quantitative trait simulation with random effect size simulation when compared to the empirical variance.
				Similar to the simulation with equal effect size, the estimated variance seems to be affected by the number of causal \glspl{SNP}.
				} 
			\label{fig:QtRandVarCom}
		\end{figure}
		Next, we simulate quantitative trait with random effect size assigned to the causal \glspl{SNP}.
		The exponential distribution with $\lambda=1$ was selected because it was suggested that it may serve as a heuristic expectation the genetic architecture of adaptation\citep{Orr1998}.
		There might be many other distribution that can be used, however due to limitation in resources, we will only focus on the exponential distribution with $\lambda=1$.

		Under this simulation condition, it was observed that the mean estimation of heritability from \gls{shrek}(\cref{fig:shrekQtRandMean}) and \gls{gcta}(\cref{fig:gctaQtRandMean}) were similar to what was observed in the equal effect size simulation.
		For \gls{ldsc} with intercept estimation(\cref{fig:ldscInQtRandMean}), less bias was observed with only the 10 causal \glspl{SNP} scenario being under estimated. 
		On the other hand, the performance of \gls{ldsc} with fixed intercept remain more or less the same, with a larger degree of fluctuation when small number of causal \glspl{SNP}(5 or 10) was simulated. 
		The fluctuation in estimate can also be observed in the empirical variance of \gls{ldsc}(\cref{fig:ldscQtRandVar,fig:ldscInQtRandVar}).
		Despite the relative stable performance of \gls{gcta}, the empirical variance of \gls{gcta} also fluctuate when the number of causal \glspl{SNP} was small. 
		Such pattern was not observed in \gls{shrek} suggesting that it might be robust against the change in number of causal \glspl{SNP}.
		
		When inspecting the relationship between the estimated and empirical variance, it was observed all programmes have a less accurate estimation of its variance when there is only 5 causal \glspl{SNP}. 
		The difference was most obvious for \gls{gcta} where the under-estimation of variance under the oligo-genic scenario(5 or 10 causal \glspl{SNP}) was more severe when a random effect size was assigned to the causal \glspl{SNP}(\cref{fig:gctaQtRandVarCom}).    
		On the other hand, the degree of bias in estimating the variance remain more or less unchanged for \gls{shrek}(\cref{fig:shrekQtRandVarCom}) and \gls{ldsc} with intercept estimation(\cref{fig:ldscInQtRandVarCom}), with roughly 0.9 and 1.25 times difference from the empirical variance respectively.
		However, for \gls{ldsc} with fixed intercept(\cref{fig:ldscQtRandVarCom}), the fold difference increased slightly, changed from 1.5 fold difference to 1.65 fold difference.
		
		Overall, simulating the effect size using the exponential distribution only slightly increases the \gls{mse} of the programmes when the number of causal \glspl{SNP} is small and decreases the \gls{mse} when the number of causal \glspl{SNP} is larger. 
		Taking into considering of both the bias and standard error, \gls{shrek} has the better performance over \gls{ldsc} except when the trait is extremely polygenic(e.g. $\ge500$ causal \glspl{SNP}).
		\begin{table}
			\centering
			\begin{tabular}{rrrrr}
				\toprule
				Number of Causal SNPs&	SHREK&	LDSC&	LDSC-In&	GCTA \\
				\midrule
				5	&	0.177	&	0.565	&	0.584	&	0.230\\
				10	&	0.159	&	0.251	&	0.470	&	0.151\\
				50	&	0.153	&	0.179	&	0.378	&	0.0796\\
				100	&	0.157	&	0.166	&	0.305	&	0.0794\\
				250	&	0.152	&	0.144	&	0.266	&	0.0674\\
				500	&	0.143	&	0.134	&	0.247	&	0.0646\\
				\bottomrule
			\end{tabular}
			\caption[Mean Squared Error of Quantitative Trait Simulation with Random Effect Size]{
				\gls{mse} of quantitative trait simulation with random effect size.
				Again, \gls{gcta} has the lowest \gls{mse} except when there is only 5 causal \glspl{SNP} and the performance of \gls{shrek} and \gls{ldsc} with fix intercept converges as number of causal \glspl{SNP} increases. 
				\gls{ldsc} with fix intercept even surpassed \gls{shrek}'s performance when the number of causal \glspl{SNP} was as high as 500.}
			\label{tab:mseQtRandom}
		\end{table}
		% Extreme with 100 causal
		\subsection{Quantitative Trait Simulation with Extreme Effect Size}
		
		\begin{figure}
			\centering
			\subfloat[SHREK]{
				\scalebox{.4}{\includegraphics{figure/he_summary/extreme_100c/shrek_QtE_Extreme_mean.png}}
				\label{fig:shrekQtEx100cMean}
			}
			\subfloat[GCTA]{
				\scalebox{.4}{\includegraphics{figure/he_summary/extreme_100c/gcta_QtE_Extreme_mean.png}}
				\label{fig:gctaQtEx100cMean}
			}\\
			\subfloat[LDSC with fix intercept]{
				\scalebox{.4}{\includegraphics{figure/he_summary/extreme_100c/ldsc_QtE_Extreme_mean.png}}
				\label{fig:ldscQtEx100cMean}
			}
			\subfloat[LDSC with intercept estimation]{
				
				\scalebox{.4}{\includegraphics{figure/he_summary/extreme_100c/ldscIn_QtE_Extreme_mean.png}}
				\label{fig:ldscInQtEx100cMean}
			}
			\caption[Quantitative Trait with Extreme Effect Size Simulation Result(100 causal SNPs, Mean)]
			{Mean of results from quantitative trait simulation with extreme effect size simulation.
				100 causal \glspl{SNP} were simulated.
				It was observed that the mean estimation of heritability of all the tools were relatively unaffected by the number of \glspl{SNP} representing a large portion of effect where \gls{shrek} has the least amount of bias.
				} 
			\label{fig:QtEx100cMean}
		\end{figure}
		
		\begin{figure}
			\centering
			\subfloat[SHREK]{
				\scalebox{.4}{\includegraphics{figure/he_summary/extreme_100c/shrek_QtE_Extreme_sd.png}}
				\label{fig:shrekQtEx100cVar}
			}
			\subfloat[GCTA]{
				\scalebox{.4}{\includegraphics{figure/he_summary/extreme_100c/gcta_QtE_Extreme_sd.png}}
				\label{fig:gctaQtEx100cVar}
			}\\
			\subfloat[LDSC with fix intercept]{
				\scalebox{.4}{\includegraphics{figure/he_summary/extreme_100c/ldsc_QtE_Extreme_sd.png}}
				\label{fig:ldscQtEx100cVar}
			}
			\subfloat[LDSC with intercept estimation]{
				
				\scalebox{.4}{\includegraphics{figure/he_summary/extreme_100c/ldscIn_QtE_Extreme_sd.png}}
				\label{fig:ldscInQtEx100cVar}
			}
			\caption[Quantitative Trait with Extreme Effect Size Simulation Result(100 causal SNPs, Variance)]
			{Variance of results from quantitative trait simulation with extreme effect size simulation.
				100 causal \glspl{SNP} were simulated.
				\gls{gcta} has the smallest variance as with previous simulation.
				When compared to \gls{ldsc} with fixed intercept, although the variance of \gls{shrek} was relatively higher, it was less sensitive to change in heritability and the number of \glspl{SNP} explaining a large portion of effect.
				In situation where 1 \gls{SNP} represent 50\% of the effect, the variance of \gls{shrek} is actually lower than that of \gls{ldsc} with fixed intercept once the heritability was $\ge0.2$.
			} 
			\label{fig:QtEx100cVar}
		\end{figure}
		
		\begin{figure}
			\centering
			\subfloat[SHREK]{
				\scalebox{.4}{\includegraphics{figure/he_summary/extreme_100c/shrek_QtE_Extreme_sdCom.png}}
				\label{fig:shrekQtEx100cVarCom}
			}
			\subfloat[GCTA]{
				\scalebox{.4}{\includegraphics{figure/he_summary/extreme_100c/gcta_QtE_Extreme_sdCom.png}}
				\label{fig:gctaQtEx100cVarCom}
			}\\
			\subfloat[LDSC with fix intercept]{
				\scalebox{.4}{\includegraphics{figure/he_summary/extreme_100c/ldsc_QtE_Extreme_sdCom.png}}
				\label{fig:ldscQtEx100cVarCom}
			}
			\subfloat[LDSC with intercept estimation]{
				
				\scalebox{.4}{\includegraphics{figure/he_summary/extreme_100c/ldscIn_QtE_Extreme_sdCom.png}}
				\label{fig:ldscInQtEx100cVarCom}
			}
			\caption[Quantitative Trait with Extreme Effect Size Simulation Result(100 causal SNPs, Estimated Variance)]
			{Estimated variance of results from quantitative trait simulation with extreme effect size simulation when compared to the empirical variance.
				100 causal \glspl{SNP} were simulated.
				\gls{shrek} generally under-estimate the variance whereas \gls{ldsc} over-estimate the variance.
			} 
			\label{fig:QtEx100cVarCom}
		\end{figure}
		Another condition that we were interested in was in the case where a small portion of \glspl{SNP} has a much larger effect than other \glspl{SNP}.
		In this simulation, we simulated either 100 or 250 causal \glspl{SNP} with 1, 5 or 10 \glspl{SNP} having a much larger effect.
		
		% Extreme with 250 causal
		
		\begin{figure}
			\centering
			\subfloat[SHREK]{
				\scalebox{.4}{\includegraphics{figure/he_summary/extreme_250c/shrek_QtE_Extreme_mean.png}}
				\label{fig:shrekQtEx250cMean}
			}
			\subfloat[GCTA]{
				\scalebox{.4}{\includegraphics{figure/he_summary/extreme_250c/gcta_QtE_Extreme_mean.png}}
				\label{fig:gctaQtEx250cMean}
			}\\
			\subfloat[LDSC with fix intercept]{
				\scalebox{.4}{\includegraphics{figure/he_summary/extreme_250c/ldsc_QtE_Extreme_mean.png}}
				\label{fig:ldscQtEx250cMean}
			}
			\subfloat[LDSC with intercept estimation]{
				
				\scalebox{.4}{\includegraphics{figure/he_summary/extreme_250c/ldscIn_QtE_Extreme_mean.png}}
				\label{fig:ldscInQtEx250cMean}
			}
			\caption[Quantitative Trait with Extreme Effect Size Simulation Result(250 causal SNPs, Mean)]
			{Mean of results from quantitative trait simulation with extreme effect size simulation.
				250 causal \glspl{SNP} were simulated.
				It was observed that the mean estimation of heritability of all the tools were relatively unaffected by the number of \glspl{SNP} representing a large portion of effect, similar to what observed when 100 causal \glspl{SNP} were simulated.
				However, there seems to be an upward bias when \gls{ldsc} was performed with fixed intercept.
			} 
			\label{fig:QtEx250cMean}
		\end{figure}
		
		\begin{figure}
			\centering
			\subfloat[SHREK]{
				\scalebox{.4}{\includegraphics{figure/he_summary/extreme_250c/shrek_QtE_Extreme_sd.png}}
				\label{fig:shrekQtEx250cVar}
			}
			\subfloat[GCTA]{
				\scalebox{.4}{\includegraphics{figure/he_summary/extreme_250c/gcta_QtE_Extreme_sd.png}}
				\label{fig:gctaQtEx250cVar}
			}\\
			\subfloat[LDSC with fix intercept]{
				\scalebox{.4}{\includegraphics{figure/he_summary/extreme_250c/ldsc_QtE_Extreme_sd.png}}
				\label{fig:ldscQtEx250cVar}
			}
			\subfloat[LDSC with intercept estimation]{
				
				\scalebox{.4}{\includegraphics{figure/he_summary/extreme_250c/ldscIn_QtE_Extreme_sd.png}}
				\label{fig:ldscInQtEx250cVar}
			}
			\caption[Quantitative Trait with Extreme Effect Size Simulation Result(250 causal SNPs, Variance)]
			{Variance of results from quantitative trait simulation with extreme effect size simulation.
				250 causal \glspl{SNP} were simulated.
				Compared to the case where 100 causal \glspl{SNP} were simulated, most tools, except \gls{shrek} seems to be more sensitive to the number of \gls{SNP}(s) explaining large portion of effect, where a smaller number can lead to a higher variance.
			} 
			\label{fig:QtEx250cVar}
		\end{figure}
		
		\begin{figure}
			\centering
			\subfloat[SHREK]{
				\scalebox{.4}{\includegraphics{figure/he_summary/extreme_250c/shrek_QtE_Extreme_sdCom.png}}
				\label{fig:shrekQtEx250cVarCom}
			}
			\subfloat[GCTA]{
				\scalebox{.4}{\includegraphics{figure/he_summary/extreme_250c/gcta_QtE_Extreme_sdCom.png}}
				\label{fig:gctaQtEx250cVarCom}
			}\\
			\subfloat[LDSC with fix intercept]{
				\scalebox{.4}{\includegraphics{figure/he_summary/extreme_250c/ldsc_QtE_Extreme_sdCom.png}}
				\label{fig:ldscQtEx250cVarCom}
			}
			\subfloat[LDSC with intercept estimation]{
				
				\scalebox{.4}{\includegraphics{figure/he_summary/extreme_250c/ldscIn_QtE_Extreme_sdCom.png}}
				\label{fig:ldscInQtEx250cVarCom}
			}
			\caption[Quantitative Trait with Extreme Effect Size Simulation Result(250 causal SNPs, Estimated Variance)]
			{Estimated variance of results from quantitative trait simulation with extreme effect size simulation when compared to the empirical variance.
				250 causal \glspl{SNP} were simulated.
				The result of simulation were the same as the previous extreme effect simulation with 100 causal \glspl{SNP}.
			} 
			\label{fig:QtEx250cVarCom}
		\end{figure}
		% CC Rand Effect
		\subsection{Case Control Simulation}
			\begin{figure}
			\centering
			\subfloat[SHREK]{
				\scalebox{.4}{\includegraphics{figure/he_summary/cc_100c/shrek_CC_Rand_mean.png}}
				\label{fig:shrekCCRandMean}
			}
			\subfloat[GCTA]{
				\scalebox{.4}{\includegraphics{figure/he_summary/cc_100c/gcta_CC_Rand_mean.png}}
				\label{fig:gctaCCRandMean}
			}\\
			\subfloat[LDSC with fix intercept]{
				\scalebox{.4}{\includegraphics{figure/he_summary/cc_100c/ldsc_CC_Rand_mean.png}}
				\label{fig:ldscCCRandMean}
			}
			\subfloat[LDSC with intercept estimation]{
				
				\scalebox{.4}{\includegraphics{figure/he_summary/cc_100c/ldscIn_CC_Rand_mean.png}}
				\label{fig:ldscInCCRandMean}
			}
			\caption[Mean of Case Control Simulation Results (100 Causal)]
			{Mean of results from case control simulation with random effect size simulation with 100 causal \glspl{SNP}.
				The bias seems to be unaffected by the number of causal \glspl{SNP} and were the same as what was observed when there were 10 or 50 causal \glspl{SNP}.
				} 
			\label{fig:CCRandMean}
		\end{figure}
		
		\begin{figure}
			\centering
			\subfloat[SHREK]{
				\scalebox{.4}{\includegraphics{figure/he_summary/cc_100c/shrek_CC_Rand_sd.png}}
				\label{fig:shrekCCRandVar}
			}
			\subfloat[GCTA]{
				\scalebox{.4}{\includegraphics{figure/he_summary/cc_100c/gcta_CC_Rand_sd.png}}
				\label{fig:gctaCCRandVar}
			}\\
			\subfloat[LDSC with fix intercept]{
				\scalebox{.4}{\includegraphics{figure/he_summary/cc_100c/ldsc_CC_Rand_sd.png}}
				\label{fig:ldscCCRandVar}
			}
			\subfloat[LDSC with intercept estimation]{
				
				\scalebox{.4}{\includegraphics{figure/he_summary/cc_100c/ldscIn_CC_Rand_sd.png}}
				\label{fig:ldscInCCRandVar}
			}
			\caption[Variance of Case Control Simulation Results (100 Causal)]
			{Variance of results from case control simulation with random effect size simulation with 100 causal \glspl{SNP}.
				As the number of causal \glspl{SNP} increased to 100, the relationship between the population prevalence and the empirical variance of the algorithms become clear where as the population prevalence increases, the empirical variance of all algorithm increases.
				Again, \gls{ldsc} with intercept estimation has the largest variation of all the algorithms and the empirical variance of \gls{ldsc} with fix intercept is only slightly higher than that of \gls{shrek}.
			} 
			\label{fig:CCRandVar}
		\end{figure}
		
		
		\begin{figure}
			\centering
			\subfloat[SHREK]{
				\scalebox{.4}{\includegraphics{figure/he_summary/cc_100c/shrek_CC_Rand_sdCom.png}}
				\label{fig:shrekCCRandVarCom}
			}
			\subfloat[GCTA]{
				\scalebox{.4}{\includegraphics{figure/he_summary/cc_100c/gcta_CC_Rand_sdCom.png}}
				\label{fig:gctaCCRandVarCom}
			}\\
			\subfloat[LDSC with fix intercept]{
				\scalebox{.4}{\includegraphics{figure/he_summary/cc_100c/ldsc_CC_Rand_sdCom.png}}
				\label{fig:ldscCCRandVarCom}
			}
			\subfloat[LDSC with intercept estimation]{
				
				\scalebox{.4}{\includegraphics{figure/he_summary/cc_100c/ldscIn_CC_Rand_sdCom.png}}
				\label{fig:ldscInCCRandVarCom}
			}
			\caption[Estimation of Variance in Case Control Simulation (100 Causal)]
			{Estimated variance of results from case control simulation with random effect size simulation when compared to empirical variance when 100 causal \glspl{SNP} was simulated.
				Once again, \gls{shrek} underestimated its empirical variance and \gls{ldsc} with fixed intercept overestimates its empirical variance. 
				However, the magnitude of overestimation of \gls{ldsc} with fixed intercept decreased when compared to previous conditions. 
			} 
			\label{fig:CCRandVarCom}
		\end{figure}
		
		
		
		
	\section{Discussion}
	
	\section{Supplementary place holder}
	
	%Put these graphs in supplementary instead
	\chapter{n-3 Polyunsaturated Fatty Acid Rich Diet in Schizophrenia}
\label{omegaProject}
\section{Introduction}
\glsreset{mia}
\glsreset{pufa}
\glsreset{polyic}
\glsreset{gd}
In the previous chapter, we have found that the \gls{SNP} heritability of \glng{scz} was much smaller than expected, accounting for only 20\% of the variance in \glng{scz}.
Alternative genetic variants such as rare variants and epigenetic factors might contribute to the heritability of \glng{scz}.
It is also possible for gene-environmental interaction ($G\times E$) to contribute to the heritability of \glng{scz}.

Previous studies have suggested there might be interaction between prenatal infection and genetic variations in the development of \glng{scz} \citep{Tienari2004,Clarke2009}.
Evidences now suggest that the effect of prenatal infection was mainly mediated by maternal immune response instead of the specific infection \citep{Brown2010} therefore it is likely that the perturbation induced by \gls{mia} are interacting with genetic variations in the development of \glng{scz}.
With the development of \gls{ldsc}, we may now perform the partitioning of \gls{SNP} heritability using summary statistics from \gls{GWAS}. 
This allow one to investigate whether if a particular functional pathway perturbed by early \gls{mia} contributes disproportionately to the heritability of \glng{scz}.

On the other hand, one of the main goal in \glng{scz} research is to identify effective treatments for \glng{scz} such that the quality of life of schizophrenic patients can be improved. 
Based on the \gls{mia} model, one possible candidate might be the n-3 \gls{pufa} rich diet. 
It has been suggested that n-3 \gls{pufa} can inhibits the production of \gls{il6} \citep{Trebble2003}, which is a major mediator in \gls{mia} \citep{Smith2007}.
Moreover, n-3 \gls{pufa} such as docosahexaenoic acid (DHA) plays a critical role in the development of central nervous system \citep{Clandinin1999,Kitajka2002} and it has robust anti-inflammatory properties \citep{Trebble2003}.
Therefore it is possible that a n-3 \gls{pufa} rich diet can help to alleviate the symptoms of \glng{scz}.
Indeed, previous study from our lab suggested that an n-3 \gls{pufa} rich diet can help to reduce the schizophrenia-like phenotype in mice exposed to early \gls{mia} insults \citep{Li2015}.

% Cerebellum
Herein, we introduce a hypothesis generation study aiming to investigate the gene expression changes induced by early \gls{mia} exposure in the brain of the adult offspring and also expression changes induced by n-3 \gls{pufa} rich diet using RNA Sequencing.
We would also like to investigate whether if functional pathways perturbed by \gls{mia} or changed by diet contributes more to the heritability of \glng{scz} using \gls{ldsc}.

In this study we selected the cerebellum as the target tissue for our experiment. 
Although hippocampus \citep{Velakoulis2006,Nugent2007} and prefrontal cortex \citep{Knable1997,Perlstein2001} are the two most studied region in \glng{scz}, the cerebellum has also been reported to be related to \glng{scz} \citep{Yeganeh-Doost2011,Andreasen2008}.
Moreover, the cerebellum plays a central role in the cortico-cerebellar-thalamic-cortical neuronal circuit which is important to \glng{scz}.
\Gls{pet} studies have shown that a dysfunction in this circuit can contribute to ``cognitive dysmetria'', e.g. impaired cognition and other symptoms of \glng{scz} \citep{Yeganeh-Doost2011}.
Altogether, this makes the cerebellum an interesting target to investigate.

The work in this chapter were done in collaboration with my colleagues who have kindly provide their support and knowledges to make this piece of work possible.
Dr Li Qi and Dr Basil Paul were responsible for generating the animal model and providing the sample for our study;
Dr Li Qi and Dr Desmond Campbell helped with the experimental design;
Vicki Lin has helped with the RNA extraction; 
Tikky Leung for her high quality sequencing service;
Nick Lin for his help in tackling problems encountered during sequencing quality control; 
Dr Johnny Kwan, Dr Desmond Campbell, Dr Timothy Mak and Professor Sham for their guidance in the statistical analysis.

\section{Methodology}
\subsection{Sample Preparation}
Female and male C57BL6/N mice were bred and mated by The University of Hong Kong, Laboratory Animal Unit. 
Timed-pregnant mice were held in a normal light–dark cycle (light on at 0700 hours), and temperature and humidity-controlled animal vivarium. 
All animal procedures were approved by the Committee on the Use of Live Animals in Teaching and Research (CULATR) at The University of Hong Kong.

The \gls{mia} model was generated following procedures previously reported \citep{Li2009c}. 
A dose of 5mg kg$^{-1}$ \gls{polyic} in an injection volume 5ml kg$^{-1}$, prepared on the day of injection was administered to pregnant mice on \gls{gd} 9 via the tail vein under mild physical constraint. 
Control animals received an injection of 5ml kg$^{-1}$ 0.9\% saline. 
The animals were returned to the home cage after the injection and were not disturbed, except for weekly cage cleaning.
The resulting offspring were weaned and sexed at postnatal day 21. 
The pups were weighed and littermates of the same sex were caged separately, with three to four animal per cage.
Half of the animal were fed on diets enriched with n-3 \glspl{pufa} and half were fed a standard  lab diet until the end of the study.
The latter `n-6 \gls{pufa}' control diet had the same calorific value and total fat content as the n-3 \gls{pufa} diet. 
The diets were custom prepared and supplied by Harlan Laboratories (Madison, WI, USA). 
The n-6 and n-3 \gls{pufa} were derived from corn oil or menhaden fish oil, respectively. 
The n-6 \gls{pufa} control diet, was based on the standard AIN-93G rodent laboratory diet \citep{Reeves1993}, and contained 65 g kg$^{-1}$ corn oil and 5 g kg$^{-1}$ fish oil with an approximate (n6)/(n3) ratio of 13:1. 
The n-3 \gls{pufa} diet contained 35 g kg$^{-1}$ corn oil and 35 g kg$^{-1}$ fish oil with an approximate (n6)/(n3) ratio of 1:1 \citep{Olivo2005}.
To avoid being confounded by sex difference, we only use the male offspring for our analysis.
The male offspring were sacrificed by cervical dislocation on postnatal week 12, which roughly correspond to adulthood in human, and the cerebellum was extracted and stored in -80$^{\circ}$C until RNA extraction.

\subsection{RNA Extraction, Quality Control and Sequencing}
Total RNA was extracted from each cerebellum tissue using RNeasy midi kit (Qiagen) following the manufacturer's instructions.
RNA quality was assayed using the Agilent 2100 Bioanalyzer and RNA was quantified using Qubit 1.0 Flurometer.
Samples with \gls{rin} $<7$ were not included in our study as the RNA are most likely degraded.
As a hypothesis generation study, we select a minimum of 3 samples per group and each samples must come from a different litter to control for littering effect.
The RNA Sequencing library was performed at the Centre for Genomic Sciences, the University of Hong Kong, using the KAPA Stranded mRNA-Seq Kit. 
All samples were sequenced using Illumina HiSeq 1500 at 2 lanes (2$\times$101 \gls{bp} paired end reads).
We distribute the samples such that each lane contain roughly the same amount of samples from different conditions.
\begin{table}
	\centering
	\begin{tabular}{rrrrrrr}
		\toprule
		SampleID & Litter & Diet & Condition & Lane & Batch & Rin\\
		\midrule
		B1&	3&	O3&	POL&	1&	B&	7.7\\
		B2&	6&	O3&	POL&	2&	B&	7.7\\
		F1&	4&	O3&	POL&	1&	F&	7.6\\
		F4&	1&	O3&	SAL&	2&	F&	8.1\\
		B4&	5&	O3&	SAL&	1&	B&	7.8\\
		B5&	14&	O3&	SAL&	2&	B&	7.7\\
		F2&	2&	O6&	POL&	1&	F&	7.5\\
		E3&	11&	O6&	POL&	2&	E&	7.8\\
		C2&	7&	O6&	POL&	2&	C&	7.9\\
		B6&	13&	O6&	SAL&	2&	B&	7.4\\
		E6&	14&	O6&	SAL&	1&	E&	8\\
		C6&	1&	O6&	SAL&	1&	C&	7.8\\
		\bottomrule
	\end{tabular}
	\caption[Sample Information]{
		Sample information.
		O3 = n-3 \gls{pufa} diet; O6 = n-6 \gls{pufa} diet; POL = \gls{polyic} exposed; SAL = Saline exposed.
		We have tried to separate the samples into different lane and batch to control for the lane and batch effect. 
		Samples from different litters were also used with the exception of F4 and C6 which came from the same litter but were given a different diet.
		\label{tab:sampleInfo}
	}
\end{table}
\subsection{Sequencing Quality Control}
\Gls{qc} of the RNA Sequencing read data were rather standardized where FastQC \citep{Andrews2010} is the most widely adopted tools.
It can generate the required per base \gls{qc} and provide a general picture of how well the sequencing were done.

From the FastQC report, it was noted that some adapter sequences remained in the final sequence, by using trim\_glore, a wrapper for cutadapt (version 1.9.1) \citep{Martin2011}, we trim the adapter sequences from the sequence reads and only retain reads that were at least 75 \gls{bp} long for subsequent alignment. 

\subsection{Alignment}
In a recent review by \citet{Engstrom2013}, it was demonstrated that STAR \citep{Dobin2013} has the best performance of all the aligners investigated taking into account of accuracy and speed.
Thus STAR aligner was used in our study.
The RNA Sequencing reads were mapped to the \textit{Mus musculus} reference genome (mm10, Ensembl GRCm38.82) using the STAR aligner (version 2.5.0a) \citep{Dobin2013}.
And the quantification of the gene expression levels were conducted using featureCounts (version 1.5.0) \citep{Liao2014}.

\subsection{Differential Expression Analysis}
There are many statistical tools available for the differential gene expression analysis.
Based on the review of \citet{Seyednasrollah2015}, it was suggested that DESeq2 and limma are the most robust statistical packages for analyzing RNA Sequencing data. 
As the authors of DESeq2 are very active in providing supports for the package, we selected DESeq2 (version 2.1.4.5) \citep{Love2014} as the statistic package for the differential gene expression analysis.

Perhaps one of the most controversial study in RNA Sequencing was the mouse ENCODE paper by \citet{Yue2014} where \citet{Gilad2015} demonstrated that most of the findings from \citet{Yue2014} was confounded by lane and batch effect.
This highlights the importance of lane and batch effect in the design of RNA Sequencing.
To avoid batch and lane effect, the whole sampling collection procedure and sequencing was performed in a way where we minimize the batch and lane difference between conditions (\cref{tab:sampleInfo}). 
However, because of the sample quality differs across different batches, we were unable to fully balance out the batch effect. 
Therefore, in our analysis, we must control for the batch effect.

In our study, we were interested in the following comparisons:
\begin{enumerate}
	\item Saline exposed samples with n-3 \gls{pufa} rich diet vs Saline exposed samples with n-6 \gls{pufa} rich diet 
	\item PolyI:C exposed samples with n-3 \gls{pufa} rich diet vs PolyI:C exposed samples with n-6 \gls{pufa} rich diet 
	\item Saline exposed samples with n-6 \gls{pufa} rich diet vs PolyI:C exposed samples with n-6 \gls{pufa} rich diet 
\end{enumerate}
To obtain the desire comparison, and also control for batch effect, we used $\sim Batch+Condition+Diet+Condition:Diet$ as our model of statistical analysis where Condition is the \gls{mia} exposure status.
We did not incorporate the \gls{rin} into our statistic model because it was advised against by the author.

We would also like to see if the batch effect can leads to false positive results.
Therefore we performed the \gls{lrt} to investigate the effect of batch on our result.
The \gls{lrt} examines two models for the counts, a full model with a certain number of terms and a reduced model, in which some of the terms of the full model are removed. 
The test determines if the increased likelihood of the data using the extra terms in the full model is more than expected if those extra terms are truly zero.
Thus we compared the full model $\sim Batch+Condition+Diet+Condition:Diet$ with $\sim Condition+Diet+Condition:Diet$ to understand the effect of batch on our data.

In our analysis, we removed all genes with base mean count $<$ 10  to reduce the noise associated with low expression and the Benjamini and Hochberg method were then used to correct for multiple testing.

\subsection{Functional Annotation}
\label{sec:function}
One of the most important aim of the current study is to investigate whether if functional pathways perturbed by \gls{mia} or affected by diet contribute to larger amount of \gls{SNP} heritability.
It is therefore important to perform functional annotation of the \glspl{deg} in order to identify pathways that were affected (e.g. enriched by the \glspl{deg}).

Because of the limited number of \gls{deg} identified, we performed the Wilcoxon Rank Sum test to test whether if genes within the pathway are more significant the genes outside pathway.
The canonical pathways annotations obtained from the \gls{msigdb} (v5.0 updated April 2015) \citep{Subramanian2005} were used for our analysis.
To avoid testing overly narrow or broad functional pathways, pathways with more than 300 genes or less than 10 genes were removed from our analysis. 
Pathways with adjusted p-value $<0.05$ (using Benjamini and Hochberg adjustment) were considered as significant.

\subsection{Partitioning of Heritability}
We then tried to perform partitioning of heritability using the significant pathways as our annotation.
One problem with the pathways is that they can overlap with each other and can even be a sub-pathway of another pathway.
We therefore calculated the Jaccard distance between each pathways:
\begin{equation}
\text{Jaccard distance} = \frac{Overlapped\ Genes}{Total\ Number\ of\ Unique\ Genes}
\end{equation}
For pathways with a Jaccard distance greater than 0.2, we removed the smaller pathway, thus reducing the number of pathways in the analysis.
Two ``super pathways'' were also generated which included all the genes in the pathways perturbed in \gls{mia} or genes in pathways affected by diet. 

\glspl{SNP} from the 1000 genome was first associated with genes using SnpEFF \citep{Cingolani2012} with the GRCh37.75 annotation, where \glspl{SNP} within $\pm5$\gls{kb} region of a gene was considered to be associated.
The annotation file required by \gls{ldsc} was then generated by identifying \glspl{SNP} participated in the significant pathways from \cref{sec:function}.
If a \gls{SNP} was found to be associated with more than 1 genes, then it was considered to be within all pathways where its associated genes were part of. 

Finally, we performed the partitioning of heritability using \gls{ldsc} \citep{Bulik-Sullivan2015} -{}-annot and -{}-overlap-annot options using 1000\gls{kb} window size and the reference panel generated from \cref{sec:realData}.
We removed the \gls{mhc} region from this analysis.
Pathways with positive proportion of heritability explained and a \gls{fdr} q-value $<0.125$ were considered significant.
We considered the ``super pathways'' as a super test which included all the pathways tested, therefore we did not perform any multiple testing correction on it.

\subsection{Designing the Replication Study}
Another important goal of our current study is to provide information for further replication studies. 
In order to estimate the power and required samples for the replication studies, we performed the power estimation using Scotty \citep{Busby2013}.
We provided the count data from our pilot samples to Scotty to estimate the minimal required samples for our replication study if we would like to detect at least 90\% of the genes that are differentially expressed by a 2$\times$ fold change at p$<0.01$ and that at least 80\% of genes has at least 80\% of the maximum power.

\section{Results} 
\subsection{Sample Quality}
On average, 87 million reads were generated for each sample of which more than 90\% of the read bases has quality score $>30$.
A quality score at 30 represents the probability of having an incorrect base call is less than 1 in 1,000.
After removing the adapter sequences from the reads, more than 97\% of the reads remains.
Over 90\% of the trimmed reads could be uniquely mapped to the \textit{Mus musculus} reference genome (mm10, Ensembl GRCm38.82) using the STAR aligner (version 2.5.0a) \citep{Dobin2013}.
To obtained the expression count, we used the featureCounts (version 1.5.0) \citep{Liao2014} to generate the count matrix required for downstream analysis.

Next, we were interested in whether if there are any contamination of samples or series confounding effect of bath or lane.
We therefore performed an unsupervised clustering on the sample count data.
It was observed that none of the samples were clustered by lane or batch, suggesting that there were no serious batch or lane effect presented in our samples (\cref{fig:distMatrix}).
However, one sample from the n3-\gls{polyic} group was found to be substantially different from all other samples (\cref{fig:distMatrix}).
It was unclear whether if the difference was due to sample contaminations (from other source) or was due to sample mis-label.
To avoid problems in down-stream analysis, we excluded this sample from subsequent analyses.
 \begin{figure}
	\centering
	\includegraphics[width=0.7\textwidth]{figure/DistanceMatrix.png}
	\caption[Sample Clustering]{Sample Clustering results.
		Samples were labeled as $<$Diet$>$-$<$Condition$>$-$<$Lane$>$-$<$Batch$>$ where O3 = n-3 \gls{pufa} rich diet; O6 = n-6 \gls{pufa} rich diet; POL =  \gls{polyic}; SAL = Saline.
		
		No clear clustering for lane or batch effects are observed.
		However, one sample from the n3-\gls{pufa}-\gls{polyic} group is found to be substantially different from all other samples.
		It is unclear whether if the difference is due to sample contaminations or sample mis-label.
		To avoid problems in down-stream analysis, we excluded this sample from subsequent analyses }
	\label{fig:distMatrix}
 \end{figure}
\subsection{Differential Expression Analysis}
\begin{figure}
	\centering
	\subfloat[O6-Saline mice vs O6-PolyI:C mice]{
		\scalebox{.4}{\includegraphics{figure/omega/miaO6_wald_qq.png}}
		\label{fig:miaO6Wald}
	}
	\subfloat[O6-Saline mice vs O3-Saline mice]{
		\scalebox{.4}{\includegraphics{figure/omega/omegaSAL_wald_qq.png}}
		\label{fig:omegaSALWald}
	}\\
	\subfloat[O6-PolyI:C mice vs O3-PolyI:C mice]{
		\scalebox{.4}{\includegraphics{figure/omega/omegaPOL_wald_qq.png}}
		\label{fig:omegaPOLWald}
	}
	\subfloat[Batch Effect]{
		\scalebox{.35}{\includegraphics{figure/omega/nobatch_lrt.png}}
		\label{fig:batchLRT}
	}
	\caption[QQ Plot Statistic Results]
	{QQ Plot of statistic results.
		From the \gls{qqplot}, it was observed that most of the observed p-value was less than what would have been expected. 
		This is likely due to the small sample size of our study which leads to an under powered association.
		The only exception was the analysis of batch effect were a large amount of genes were found to be significant. 
		This demonstrate the importance of adjusting for batch effect} 
	\label{fig:waldQQ}
\end{figure}
After excluding the problematic samples, we performed the DESeq2 analysis.
Of the 16,747 genes that passed through quality control, only one gene, \textit{Sgk1} (p-adjusted=0.00186) was found to be significantly differentiated when we examines the effect of n-3 \gls{pufa} rich diet on the gene expression in the cerebellum of \gls{polyic} exposed mice (\cref{fig:omegaPOLWald}).
No gene was found to be significant for the other two comparisons (\cref{fig:miaO6Wald,fig:omegaSALWald}).

We also performed the \gls{lrt} to compare test the effect of batch on our analysis. 
A total of 178 genes were found to be significant differentiated (\cref{fig:batchLRT}), suggesting that the ``Batch'' is indeed an important factor to consider in our analysis.

\subsection{Functional Annotation}
It is common practice to perform functional annotation to the \glspl{deg}. 
However, in most of our analysis, there were either no \gls{deg} or only 1 \gls{deg}, making it difficult to perform functional annotation.
We therefore used the Wilcox rank sum test to analysis whether if a pathway contain genes that are more significant than genes not within the pathway.

None of the pathways were found to be significant when comparing the effect of the n-3 \gls{pufa} rich diet in Saline exposed mice. 
On the contrary, 17 pathways were found significant when comparing the effect of n-3 \gls{pufa} rich diet in \gls{polyic} exposed samples (\cref{tab:o6polyPath}) where 4 pathways were related to growth factors such as \gls{fgf} or \gls{egf} and 4 others were related to kinases such as \gls{pi3k} or \gls{mapk}.

Finally, 12 pathways were found to be significant when comparing Saline and \gls{polyic} exposed mice given the n-6 \gls{pufa} rich diet (\cref{tab:miaPath}) with pathways such as neuroactive ligand-receptor interaction (p-adj = $1.27\times10^{-3}$), calcium signaling pathway (p-adj = $2.79\times10^{-3}$) and genes involved in Neuronal System (p-adj=0.00153) among the significant pathways.

\subsection{Partitioning of Heritability}
Given the significant pathways, we performed the partitioning of heritability using \gls{ldsc} \citep{Bulik-Sullivan2015}.
In total, 14 unique pathways were included in the analysis were 4 of them were found to have non-negative contribution to the heritability of \glng{scz}, including the pathway related to neuronal system, \gls{ecm} glycoprotein, calcium signaling and \gls{mapk} signaling (\cref{tab:partitioning}).
All of these pathways were affected by \gls{mia} and only the \gls{ecm} pathways were also found to be affected by n-3 \gls{pufa} rich diet in \gls{polyic} exposed mice.
Moreover, the ``super pathway'' for \gls{mia} were found to be significant (p-value=0.0402) yet the ``super pathway'' for diet was found to be insignificant (p-value=0.414).

\subsection{Designing the Replication Study}
Other than generation of hypothesis, we would also like to use the current information to help designing subsequent replication studies.
Using Scotty \citep{Busby2013}, given that we would like to detect at least 90\% of the genes that are differentially expressed by a 2$\times$ fold change at p$<0.01$ and that at least 80\% of genes has at least 80\% of the maximum power, we will need at least 10 samples per group in the replication study given the current sequencing depth.

\begin{landscape}
	\begin{table}
		\begin{tabular}{rrrp{10cm}r}
			\toprule
			ID&	Size&	Source&	Description&	Adjusted P-Value\\
			\midrule
			M508&	78&	REACTOME&	Genes involved in Signaling by SCF-KIT&	0.00671\\
			M570&	44&	REACTOME&	Genes involved in PI3K events in ERBB2 signaling&	0.0242\\
			M3008&	196&	NABA&	Genes encoding structural ECM glycoproteins&	0.0309\\
			M1090&	112&	REACTOME&	Genes involved in Signaling by FGFR&	0.0309\\
			M563&	109&	REACTOME&	Genes involved in Signaling by EGFR in Cancer&	0.0309\\
			M17776&	100&	REACTOME&	Genes involved in Downstream signaling of activated FGFR&	0.0309\\
			M1076&	83&	REACTOME&	Genes involved in Amyloids&	0.0309\\
			M850&	56&	REACTOME&	Genes involved in PI-3K cascade&	0.0309\\
			M10450&	38&	REACTOME&	Genes involved in GAB1 signalosome&	0.0309\\
			M16227&	24&	REACTOME&	Genes involved in Cholesterol biosynthesis&	0.0309\\
			M5872&	17&	KEGG&	Steroid biosynthesis&	0.0309\\
			M16334&	10&	BIOCARTA&	Eph Kinases and ephrins support platelet aggregation&	0.0309\\
			M5884&	275&	NABA&	Ensemble of genes encoding core extracellular matrix including ECM glycoproteins, collagens and proteoglycans&	0.0456\\
			M635&	127&	REACTOME&	Genes involved in Signaling by FGFR in disease&	0.0456\\
			M568&	38&	REACTOME&	Genes involved in PI3K events in ERBB4 signaling&	0.0456\\
			M165&	32&	PID&	Syndecan-4-mediated signaling events&	0.0456\\
			M1262&	15&	REACTOME&	Genes involved in GRB2:SOS provides linkage to MAPK signaling for Intergrins&	0.0456\\
			\bottomrule
		\end{tabular}
		\caption[Significant Pathways When Comparing Effect of Diet in PolyI:C Exposed Mouse]{Significant Pathways when comparing effect of diet in \gls{polyic} exposed mice.
			The pathway IDs are the systematic name from \gls{msigdb}.
			Most of the significant pathways were related to the kinase such as PI3K and MAPK or growth factors such as \gls{fgf} and \gls{egf}.
		}
		\label{tab:o6polyPath}
	\end{table}
	
	\begin{table}
		\begin{tabular}{rrrp{10cm}r}
			\toprule
			ID&	Size&	Source&	Description&	Adjusted P-Value\\
			\midrule
			M13380&	272&	KEGG&	Neuroactive ligand-receptor interaction&	$1.27\times10^{-3}$\\
			M2890&	178&	KEGG&	Calcium signaling pathway&	$2.79\times10^{-3}$\\
			M12289&	188&	REACTOME&	Genes involved in Peptide ligand-binding receptors&	0.00118\\
			M5884&	275&	NABA&	Ensemble of genes encoding core extracellular matrix including ECM glycoproteins, collagens and proteoglycans&	0.00119\\
			M735&	279&	REACTOME&	Genes involved in Neuronal System&	0.00153\\
			M15514&	186&	REACTOME&	Genes involved in Transmission across Chemical Synapses&	0.00401\\
			M4904&	121&	REACTOME&	Genes involved in G alpha (s) signalling events&	0.0127\\
			M3008&	196&	NABA&	Genes encoding structural ECM glycoproteins&	0.0131\\
			M752&	137&	REACTOME&	Genes involved in Neurotransmitter Receptor Binding And Downstream Transmission In The Postsynaptic Cell&	0.0131\\
			M10792&	267&	KEGG&	MAPK signaling pathway&	0.0195\\
			M17&	59&	PID&	Notch signaling pathway&	0.0406\\
			M18437&	184&	REACTOME&	Genes involved in G alpha (q) signalling events&	0.0406\\
			\bottomrule
		\end{tabular}
		\caption[Significant Pathways When Comparing Effect of PolyI:C in Mouse Given n-6 \gls{pufa} Rich Diet]{Significant pathways when comparing effect of PolyI:C in mouse given n-6 \gls{pufa} rich diet.
			The pathway IDs are the systematic name from \gls{msigdb}.
			Interestingly, we observed a lot of neural related pathways and even got significant signal in the calcium signaling pathway, which was reported to be associated with \glng{scz} \citep{Purcell2014}. 
		}
		\label{tab:miaPath}
	\end{table}
	
	\begin{table}
		\begin{tabular}{rrrp{9cm}p{1.7cm}p{1.5cm}p{1.7cm}}
			\toprule
			ID& Size&	Source&	Description&	Proportion of $h^2$&	SE&	Enrichment Q-Value\\
			\midrule
			M735&	279&	REACTOME&	Genes involved in Neuronal System&	0.0287&	0.00627&	0.0456\\
			M5884&	275&	NABA&	Ensemble of genes encoding core extracellular matrix including ECM glycoproteins, collagens and proteoglycans&	0.00363&	0.00342&	0.0456\\
			M2890&	178&	KEGG&	Calcium signaling pathway&	0.0260&	0.00856&	0.127\\
			M10792&	267&	KEGG&	MAPK signaling pathway&	0.0257&	0.008713&	0.127\\	
			\bottomrule
		\end{tabular}
		\caption[Pathways Significantly Contributes to SNP Heritability of Schizophrenia.]{Pathways significantly contributes to \gls{SNP} heritability of \glng{scz}.
		}
		\label{tab:partitioning}
	\end{table}
\end{landscape}

\section{Discussion}
\subsection{Serine/threonine-protein kinase}
In this hypothesis generation study, we demonstrated that the Serine/threonine-protein kinase \textit{Sgk1} might be affected by n-3 \gls{pufa} rich diet in the cerebellum of \gls{mia} exposed mice. 
\textit{Sgk1} is a serine/threonine kinase activated by \gls{pi3k} signals and studies have shown that the expression of \textit{Sgk1} is associated with spatial learning, fear-conditioning learning and recognition learning in rat \citep{Tsai2002,Lee2003}.
For example, \citet{Tsai2002} observed a 4 fold increase of \textit{Sgk1} in the hippocampus of fast learners when compared to slow learners where transfection of \textit{Sgk1} mutant DNA impairs the water maze performance in rat.

On the other hand, it was found that \textit{Sgk1} can regulates the AMPA and kainate glutamate receptors, especially GluR6 which is encoded by \textit{Grik2} \citep{Lang2006,Lang2010}.
The kainate receptors contributes to the excitatory postsynaptic current and are important to the synaptic transmission and plasticity in the hippocampus \citep{Lang2006}.
The upregulation of AMPA and kainate receptors are therefore expected to enhance the excitatory effects of glutamate \citep{Lang2010}.
Moreover, \textit{Sgk1} also up-regulates the glutamate transporters such as EAAT4 \citep{Bohmer2004}.
The glutamate receptors are vital for clearance of glutamate from the synaptic cleft.
This prevents excessive glutamate accumulation and therefore help to prevent the neurotoxic effects of glutamate \citep{Lang2010}.
Considering the complexity of the glutamatergic system and the conflicting role of \textit{Sgk1}, it is likely for more genes to play a role in the tight regulation of the glutamatergic system.
However, it is likely that the disruption of \textit{Sgk1} might be have an impact to the normal functioning of the glutamatergic system.

\begin{figure}
	\centering
	\includegraphics[width=0.6\textwidth]{figure/omega/Sgk1_expression.png}
	\caption[Normalized Expression of \textit{Sgk1}]{
		Normalized Expression of \textit{Sgk1}.
		It was observed that the expression level of \textit{Sgk1} increases after the mice was given a n3-\gls{pufa} rich diet where a significant increase was observed in mice exposed to \gls{polyic}.
	}\label{fig:sgk1Express}
\end{figure}

In our study, it was observed that upon given the n-3 \gls{pufa} rich diet, the \textit{Sgk1} expression in the cerebellum increases (\cref{fig:sgk1Express}).
Although the increase was not significant in the saline mice, a significant up-regulation was observed in the \gls{polyic} exposed mice (\cref{fig:sgk1Express}). 
Additionally, it was also observed that \gls{pi3k} pathways and pathways related to \gls{fgf} receptors and \gls{egf} receptors were significant when studying the effect of n-3 \gls{pufa} rich diet to \gls{polyic} exposed mice.
Upon further investigation, it was found that the \gls{fgf} receptors and \gls{egf} receptors are upstream of the \gls{pi3k}-Akt pathway (\cref{fig:pi3kPathway}) which is responsible for the activation of \textit{Sgk1}.
Although we were unable to provide direct connection between the expression of \textit{Sgk1} and the improve functioning of the \gls{polyic} mice given n-3 \gls{pufa} diet, our results do suggest a possible effect of the n-3 \gls{pufa} rich diet in the expression of genes related to the \gls{pi3k}/Akt pathway and might affect the expression of \textit{Sgk1}.
Further studies are therefore required to understand whether if the change in expression of \textit{Sgk1} can account for the improved functioning of the \gls{polyic} mice. 
A possible design will be to induce the expression of \textit{Sgk1} in \gls{polyic} mice through transfection and examine whether if the \gls{polyic} mice with higher expression of \textit{Sgk1} display a reduction in \glng{scz}-like behaviours.

An important point to note was that previous research of \textit{Sgk1} has been focusing on the hippocampus and not the cerebellum. 
Thus, it is possible that \textit{Sgk1} might have a different function in the cerebellum. 
It is therefore vital for subsequent research to study the effect of \textit{Sgk1} in the cerebellum or for one to study the effect of n-3 \gls{pufa} rich diet on the gene expression in the hippocampus. 

\begin{figure}
	\centering
	\includegraphics[width=0.6\textwidth]{figure/omega/pi3ksignaling.jpg}
	\caption[Schematic of signalling through the PI3K/AKT pathway]{
		Schematic of signalling through the \gls{pi3k}/AKT pathway.
		It was observed that the growth factors were upstream of the \gls{pi3k}/AKT pathway of which \textit{Sgk1} is one of the member of the pathway.
		Figure adopted from \citet{Hennessy2005} with permission from journal.
	}\label{fig:pi3kPathway}
\end{figure}
\subsection{Functional Annotations}
When examine the expression change in mice exposed to \gls{polyic}, none of the genes were significantly differentiated. 
However, we do observe 12 pathways that contains genes that were more significant than genes not within the pathway (\cref{tab:miaPath}).
Interestingly, of the 12 significant pathways, 5 pathways were related to neuronal functions such as neuroactive ligand-receptor interaction (padj=$1.27\times 10^{-3}$), genes involved in neuronal system (padj=0.00153) and genes involved in transmission across chemical synapses (padj=0.00401).
It has long been developed that the neuronal system and the neurotransmitter regulation plays a critical role in \glng{scz}.
For example, the disruption of the GABAergic and glutamtergic neuronal system might leads to excitation/inhibition imbalance which might ultimately lead to \glng{scz} \citep{Wassef2003}.
Moreover, the alteration in balance betweem excitation and inhibition can distort the connectivity patterns between different brain regions, thus leads to developmental and behavioral deficits \citep{Cline2005}.

Additionally, it was found that the calcium signaling pathway was significant when comparing the effect of \gls{mia}. 
The association of the calcium signaling pathway with \glng{scz} was not a new finding \citep{Lidow2003,Purcell2014,Ripke2014}.
Previous exome sequencing study of \glng{scz} by \citet{Purcell2014} has already report the enrichment of non-synonymous variants within the voltage gate calcium ion channel genes in the \glng{scz} cases and the \gls{pgc} \glng{scz} \gls{GWAS} has also found association between genes encoding the calcium channel subunits with \glng{scz}.
As calcium signaling pathway is the key component of the mechanism responsible for regulating neuronal excitability \citep{Berridge2014}, the disruption of the calcium signaling pathway is likely to have a profound effect on the neural function. 
Together, our results suggest that \gls{mia} might have disrupted the normal functioning of the neural system in the cerebellum, thus lead to schizophrenia-like behaviours in the adult mice yet follow up studies are required to validate our findings.

%TODO quick travel
\glsreset{qqplot}
Moreover, we performed the partitioning of heritability hoping to see whether if the significant pathways have contributes disproportionately to the \gls{SNP} heritability of \glng{scz}.
Interestingly, all 4 significant pathways that were found to be contributing a significantly higher portion to the \gls{SNP} heritability were affected by \gls{mia}.
To assess whether if the significance of these pathways were driven by a small number of very significant genes, we compared the \gls{qqplot} of \glspl{SNP} within the pathway and all the \glspl{SNP} included in the \gls{pgc} \gls{GWAS} (\cref{fig:qqAll}).
It is observed that for most of the pathways, there is a general inflation of summary statistics when compared to the full set, suggest that the significance was not driven by a single significant gene.
However, for the \gls{ecm} related pathway, only a small inflation was observed. 
This is therefore likely that the significance was driven by a small number of significant genes. 

\begin{figure}
	\centering
	\subfloat[\gls{ecm}]{
		\scalebox{.4}{\includegraphics{figure/omega/NABA_CORE_MATRISOME.png}}
		\label{fig:ecm}
	}
	\subfloat[MAPK Signaling]{
		\scalebox{.4}{\includegraphics{figure/omega/KEGG_MAPK_SIGNALING_PATHWAY.png}}
		\label{fig:mapk}
	}\\
	\subfloat[Neuronal System]{
		\scalebox{.4}{\includegraphics{figure/omega/REACTOME_NEURONAL_SYSTEM.png}}
		\label{fig:neuronal}
	}
	\subfloat[Calcium Signaling]{
		
		\scalebox{.4}{\includegraphics{figure/omega/KEGG_CALCIUM_SIGNALING_PATHWAY.png}}
		\label{fig:calcium}
	}
	\caption[Comparing the QQ plots with PGC SNPs]
	{Comparing the \gls{qqplot} with \gls{pgc} \glspl{SNP}.
		\glspl{SNP} found within the pathways were colored in blue whereas the all the full set of \glspl{SNP} from the \gls{pgc} \gls{GWAS} was coded in red.
		It is observed that for most of the pathways, there is a general inflation of summary statistics when compared to the full set, suggest that the significance was not driven by a single significant gene.
		However, for the \gls{ecm} related pathway, only a small inflation was observed. 
		This is therefore likely that the significance was driven by a small number of significant genes. 
		} 
	\label{fig:qqAll}
\end{figure}

The ``super pathway'' containing all the genes participating in the \gls{mia} related pathways were also found to be significant (\cref{tab:partitioning}) suggest that the differential gene expression in the cerebellum induced by early \gls{mia} events and the genetic variants might act upon similar pathways in the development of \glng{scz}.

Interestingly, among the 4 pathways found to be significantly contributes to the \gls{SNP} heritability, only the pathway related to the assembly of core \gls{ecm} molecules such as the \gls{ecm} glycoproteins, were also found to be significantly affected by the n-3 \gls{pufa} rich diet in the \gls{polyic} exposed mouse. 
Although the \gls{qqplot} suggest that a small number of genes might have driven the significance of this pathway, it is nonetheless an interesting candidate.

Emerging evidences suggest that the \gls{ecm} abnormality might be associated with \glng{scz} \citep{Berretta2012}.
The \gls{ecm} glycoprotein Reelin has been reported to have a decreased expression in the cerebellum of \glng{scz} patients \citep{Maloku2010} and were found to be accompanied by decreased expression of glutamic acid decarboxylase 67 \citep{Costa2001}.
Studies also suggested that Reelin might have important role in corticogenesis and synaptic maturation and stabilization \citep{Berretta2012}.
Moreover, another \gls{ecm} molecule, Semaphorin 3A has been reported to be increased in the cerebellum of subjects with \glng{scz} \citep{Eastwood2003}.
The Semaphroin 3A protein was found to regulates axonal guidance and has a critical role in the  regulation of tangential migration of cortical GABAergic interneurons \citep{Zimmer2010}.
It was also reported that the elevated Semaphorin 3A is associated with down-regulation of genes involved in synaptic formation and maintenance \citep{Eastwood2003}.
Together, these evidence suggest that the \gls{ecm} molecules might have critical role in the development of \glng{scz}.

It has been reported that the n-3 \gls{pufa} diet can modulate the \gls{mmp} \citep{Derosa2009,Kavazos2015} which can regulates the \gls{ecm} composition \citep{Stamenkovic2003}.
Therefore it is possible that the n-3 \gls{pufa} diet has exerted its effect to the \gls{ecm} through \gls{mmp}.
However, from the \gls{qqplot}, it was noted that only a modest inflation was observed (\cref{fig:ecm}).
This suggest that the significance of the \gls{ecm} pathway might have been driven by a small number of significant genes.
Due to difficulties in delineating the individual \glspl{SNP} effect, it is difficult for us to pin-point the ``driver'' genes of this pathway.
Moreover, none of the \gls{ecm} genes were found to be differentially expressed in either of our condition, therefore we urge that further studies are required to understand how the n-3 \gls{pufa} direct interacts with the \gls{ecm} or the \gls{ecm} related genes and the effect of such interaction in \gls{mia} exposed individuals.

Finally, it is important to note that the current study serves only as a hypothesis generation study and the sample size was modest. 
We therefore like to use the current results to provide an estimation of sample size required for a replication study.
By using Scotty \citep{Busby2013}, we have estimated that the replication study should contain at least 10 samples for each group in order for us to detect at least 80\% of genes has at least 80\% of the maximum power. 
We have also demonstrated that the batch effect can have a big impact to the association (\cref{fig:batchLRT}), therefore one should always control for the batch effect whenever possible.
Given the current resources, one of the preferred design for the follow up study are given in \cref{tab:bestdesign}.

\subsection{Limitations}
We first acknowledge that the sample size of the current study is small and are underpowered.
This is reflected in the \glspl{qqplot} (\cref{fig:waldQQ}) where the observed p-values were generally smaller than would have expected.
A better study design will include more samples yet we were limited by our budget.
However, the importance of a pilot study is to identify potential targets for replications, hypothesis generation or to provide guidance for follow up studies. 
In this study, we have identified \textit{Sgk1} as an interesting candidate gene that might have an important role in the effect of n-3 \gls{pufa} in \gls{polyic} exposed individuals.
Our results provide support for a possible converging functional effect of differential expression induced by early \gls{mia} and genetic variations observed in \glng{scz}.
These provide interesting candidates for follow studies and we were able to estimate and design a better replication study based on the current data. 
Therefore we argue that as a hypothesis generation study, our study is successful.

Second, we examined only the male brains in the current study. 
The decision to direct experimental resources to males was made because there is evidence that the male fetus is more vulnerable to environmental exposures such as inflammation in prenatal life \citep{Bergeron2013,Lein2007}. 
We acknowledge that an interesting follow up study would be to investigate the gender difference in response to \gls{mia} and dietary change.

Third, although RNA Sequencing was performed, we did not performed any analysis on possible alternative splicing events or denovo transcript assembly.
The reason behind such decision is that our sample size is simply too small.
Without sufficient information, denovo transcript assembly can return noisy results.
On the other hand, in order to investigate possible alternative splicing events, we would need to perform the analysis on transcript level instead of gene level. 
This increase the possible candidates from 47,400 genes to 114,083 transcripts.
Combining with the difficulties of the quantification of different isoforms, a much larger power is required for the alternative splicing analysis. 
On top of that, the functional annotation of transcripts is another difficult aspect to tackle.
While there are a lot of information for the annotation of genes, information on functional difference between isoforms of the same gene are generally lacking. 
The lack of annotation leads to difficulties in making sense of the data. 
Thus although we acknowledge the possible importance of alternative splicing and denovo transcripts, we did not perform any alternative splicing analysis or denovo transcripts assembly.
Nonetheless, the use of RNA Sequencing allow us to easily perform these experiments once sufficient samples are obtained.

Forth, it is important to note that a high RNA expression level does not guarantee a high protein concentration \citep{Vogel2012}.
Post transcriptional, translational and degradation regulation can all affect the rates of protein production and turnover, therefore contributes to the determination of protein concentrations, at least as much as transcription itself \citep{Vogel2012}.
The RNA Sequencing thus only provide an approximation to the concentration of a particular protein in the samples.
However, we do argues that RNA Sequencing can help to identify potential targets for protein assays where detail analysis can be performed on the protein level.

Finally, at the time of this thesis, we have yet completed any \gls{rtpcr} or any functional studies to validate our findings.
One of the most vital steps after any RNA Sequencing results is to validate the differential expression findings using the \gls{rtpcr}.
Ideally, not only should one perform the \gls{rtpcr} on the sequenced samples, one should also perform the \gls{rtpcr} on an independent set of samples. 
Moreover, the RNA Sequencing only helps to identify possible candidates that were ``associated'' with a particular trait.
It does not provide any causal linkage between the phenotype and the differential expression.
If one would like to establish a direct linkage between the phenotype and the expression of a gene, one will need to carry out functional studies such as knock-in knock-out mouse design.
For example, in order to understand the functional impact of the differential expression of \textit{Sgk1}, one might try to examine whether if the pure up-regulation of \textit{Sgk1} through transfection can reduce the \glng{scz}-like behavior in \gls{polyic} exposed mice.

Currently, we are planning to perform the \gls{rtpcr} on \textit{Sgk1} on all available samples. 
Shall the results be validated, we can then perform subsequent functional studies. 

\newpage
\section{Supplementary}
% Table generated by Excel2LaTeX from sheet 'Sheet1'
\begin{center}
	\begin{longtable}[H]{rrrrrr}
			\toprule
			Litter & Condition & Diet  & Cage  & Batch & Lane \\
			\midrule
			\endhead
			\hline
			\multicolumn{6}{c}{Continued}\\
			\bottomrule
			\endfoot
			\bottomrule
			\endlastfoot
			    1     & PolyIC & n-3 PUFA & 1     & 1     & 1 \\
			    1     & PolyIC & n-6 PUFA & 2     & 5     & 1 \\
			    2     & PolyIC & n-3 PUFA & 3     & 4     & 2 \\
			    2     & PolyIC & n-6 PUFA & 4     & 3     & 3 \\
			    3     & PolyIC & n-3 PUFA & 5     & 2     & 4 \\
			    3     & PolyIC & n-6 PUFA & 6     & 1     & 1 \\
			    4     & PolyIC & n-3 PUFA & 7     & 5     & 1 \\
			    4     & PolyIC & n-6 PUFA & 8     & 4     & 2 \\
			    5     & PolyIC & n-3 PUFA & 9     & 3     & 3 \\
			    5     & PolyIC & n-6 PUFA & 10    & 2     & 4 \\
			    6     & PolyIC & n-3 PUFA & 1     & 2     & 1 \\
			    6     & PolyIC & n-6 PUFA & 2     & 1     & 2 \\
			    7     & PolyIC & n-3 PUFA & 3     & 5     & 2 \\
			    7     & PolyIC & n-6 PUFA & 4     & 4     & 3 \\
			    8     & PolyIC & n-3 PUFA & 5     & 3     & 4 \\
			    8     & PolyIC & n-6 PUFA & 6     & 2     & 1 \\
			    9     & PolyIC & n-3 PUFA & 7     & 1     & 2 \\
			    9     & PolyIC & n-6 PUFA & 8     & 5     & 2 \\
			    10    & PolyIC & n-3 PUFA & 9     & 4     & 3 \\
			    10    & PolyIC & n-6 PUFA & 10    & 3     & 4 \\
			    11    & Saline & n-3 PUFA & 1     & 3     & 1 \\
			    11    & Saline & n-6 PUFA & 2     & 2     & 2 \\
			    12    & Saline & n-3 PUFA & 3     & 1     & 3 \\
			    12    & Saline & n-6 PUFA & 4     & 5     & 3 \\
			    13    & Saline & n-3 PUFA & 5     & 4     & 4 \\
			    13    & Saline & n-6 PUFA & 6     & 3     & 1 \\
			    14    & Saline & n-3 PUFA & 7     & 2     & 2 \\
			    14    & Saline & n-6 PUFA & 8     & 1     & 3 \\
			    15    & Saline & n-3 PUFA & 9     & 5     & 3 \\
			    15    & Saline & n-6 PUFA & 10    & 4     & 4 \\
			    16    & Saline & n-3 PUFA & 1     & 4     & 1 \\
			    16    & Saline & n-6 PUFA & 2     & 3     & 2 \\
			    17    & Saline & n-3 PUFA & 3     & 2     & 3 \\
			    17    & Saline & n-6 PUFA & 4     & 1     & 4 \\
			    18    & Saline & n-3 PUFA & 5     & 5     & 4 \\
			    18    & Saline & n-6 PUFA & 6     & 4     & 1 \\
			    19    & Saline & n-3 PUFA & 7     & 3     & 2 \\
			    19    & Saline & n-6 PUFA & 8     & 2     & 3 \\
			    20    & Saline & n-3 PUFA & 9     & 1     & 4 \\
			    20    & Saline & n-6 PUFA & 10    & 5     & 4 \\
			\bottomrule
		\caption[Design for Follow Up Study]{
			Design for follow up study.
			This design will allow one to balanced out litter effect, cage effect, batch effect and lane effects such that the confounding effects were minimized.
			One can also include the \gls{ercc} spike in control to serves as an internal standard for additional level of control \citep{Jiang2011a}.
			}
		\label{tab:bestdesign}%
	\end{longtable}%ssss
\end{center}
% End up it is very easy, all you have to do is to first sort the table with the previous condition then e.g. Cage, then repeat the number of condition e.g. 1,2,1,2,1,2 until the end. Keep doing that and you will have a best mixed model
	\chapter{Conclusion}
	\label{conclusionChapter}
	\glsresetall
	It has been a long time since \glng{scz} was recognized as a genetic disorder. 
	After years of failure in identifying robust genetic markers associated with \glng{scz}, the \Glng{scz} Working group of \gls{pgc} has recently identified 108 independent genetic loci associated with \glng{scz} \citep{Ripke2014}.
	
	The positive results from the \gls{pgc} \glng{scz} \gls{GWAS} is highly encouraging, yet unclear whether additional resources should be invested into collecting more \gls{GWAS} samples, or whether other genetic approaches such as exome or whole genome sequencing should be performed to examine the role of rare variants. 
	Therefore, estimating the true contribution of all the \gls{GWAS} \glspl{SNP} to the genetic predisposition of individuals to \glng{scz} (i.e. \gls{SNP} heritability) has important implications for future research strategy.
	% Know whether to increase sample size or just plain right do something else
	
	Problem with the estimation of \gls{SNP} heritability is that the sample genotypes from \gls{pgc} \glng{scz} \gls{GWAS} is unavailable, therefore the \gls{grm} cannot be computed. 
	Methods such as \gls{gcta} \citep{Yang2011} and \gls{pcgc} \citep{Golan2014}, which relies on the \gls{grm} to estimate the \gls{SNP} heritability are therefore not infeasible.
	Recently, \gls{ldsc} \citep{Bulik-Sullivan2015}, which utilize the summary statistics from \gls{GWAS} instead of \gls{grm}, was developed for the estimation of \gls{SNP} heritability.
	Therefore, it is now possible to estimate the \gls{SNP} heritability of \glng{scz}.
	
	Nevertheless, it is important to assess the performance of \gls{ldsc} before applying it to the actual \glng{scz} data.
	Herein, we performed extensive simulations to investigate the impact of different genetic architectures and sampling strategies on the performance of \gls{ldsc}.
	As \citet{Bulik-Sullivan2015} reported that \gls{ldsc} under-perform in certain scenarios, we also developed \gls{shrek}, an alternative algorithm that uses summary statistics from \gls{GWAS} to estimate the \gls{SNP} heritability of a trait. 
	
	Our simulation results suggest that \gls{shrek} provided a more robust estimate for oligogentic traits and in case-control designs in which no confounding variables was present.
	However, it is also observed that estimates from \gls{ldsc} with fixed intercept and \gls{shrek} are inflated in case control designs.
	The magnitude of inflation is proportional to the population prevalence of the trait, where lower population prevalence results in a larger upward bias. 
	Therefore, further development is required in order to obtain an accurate estimation of \gls{SNP} heritability in case-control settings.
	
	Using summary statistics from \gls{pgc} \glng{scz} \gls{GWAS}, we estimated that \glng{scz} has a \gls{SNP} heritability of 0.185 (SD=0.00450), which is similar to the estimate of 0.198 (SD=0.0057) by \gls{ldsc} when intercept estimation was not performed.
	When intercept estimation was perform, \gls{ldsc} estimated the \gls{SNP} heritability of \glng{scz} to be around 0.135 (SD=0.00720), which is lower than the other estimates. 
	This result indicated that common \glspl{SNP} have relatively less contribution to the genetic predisposition of individuals to \glng{scz} as measured by the heritability estimated
	Therefore, alternative strategies like whole genome sequencing, which allows for the identification of rare mutations and structural variation,  would be more efficient for identifying additional susceptibility loci associated with \glng{scz}, compared to \gls{GWAS}.
	
	Another possible source of the ``missing'' heritability is the interaction between genetic and environmental factors ($G\times E$). 
	Previous studies have reported the interaction between genetic variations and prenatal infection \citep{Tienari2004,Clarke2009}. 
	As the effect of prenatal infection was mainly mediated by maternal immune response \citep{Brown2010}, it is likely for the perturbation induced by \gls{mia} to interacts with genetic variations in the development of \glng{scz}.
	
	To investigate whether the perturbation induced by \gls{mia} also affect the same functional gene sets as genetic variations associated with \glng{scz}, we performed an RNA sequencing on the \gls{polyic} \gls{mia} mouse model. 
	As converging evidence from \gls{GWAS}, \gls{cnv} and sequencing studies suggest rare and common variants in genes that related to \gls{psd} \citep{Purcell2014,Consortium2015a} and calcium ion channels \citep{Purcell2014,Ripke2014,Szatkiewicz2014} are contributing to the etiology of \glng{scz}, these gene sets were selected for the gene set enrichment analysis. 
	
	Our results suggest that with the exception of the \gls{psd} gene set obtained from \citet{Purcell2014}, all gene sets related to \gls{psd} and calcium ion channel signalling are found to be significantly perturbed in \gls{mia} samples. 
	Furthermore, the gene set containing genes within the associated \gls{GWAS} LD-intervals from \citet{Purcell2014} is also found to be significant.
	This strongly suggest that the perturbation induced by early \gls{mia} events are acting on the same functional gene sets as the genetic variation associated with \glng{scz}.
	\gls{psd} and calcium ion channel are therefore important target for subsequent research in \glng{scz}.
	
	On the other hand, a number of studies have reported the potential of n-3 \gls{pufa} in the treatment of \glng{scz} \citep{Li2015,Trebble2003}.
	Therefore, we also investigated the effect of n-3 \gls{pufa} rich diet in samples exposed to early \gls{mia} events.
	The analysis allowed us to identify \textit{Sgk1}, a gene that regulates the glutamatergic system, to be significantly differentiated in \gls{polyic} exposed mouse receiving different diet. 
	In addition, the \gls{psd} gene set obtained form the \gls{go} is also found to be significant in \gls{polyic} exposed mouse receiving different diet. 
	We therefore speculate that the n-3 \gls{pufa} rich diet might indirectly enhance the expression of \textit{Sgk1} or increase the expression of the \gls{psd} proteins in \gls{polyic} exposed mice, which may compensating for the reduced neural functioning, therefore reduces the \glng{scz}-like behaviours.
	
	Finally, we estimated the relative contribution of the gene sets to the \gls{SNP} heritability of \glng{scz}. 
	However, only the \gls{psd} gene set from \gls{go}, which was previous reported to be enriched by common variants \citep{Consortium2015a}, and the gene set containing genes within the associated \gls{GWAS} LD-intervals from \citet{Purcell2014} are found to contribute a disproportionate amount of \gls{SNP} heritability in \glng{scz}.
	The relative contribution of the gene sets are only around 1\% of the \gls{SNP} heritability, suggesting that other functional gene sets might be accounting for the remaining \gls{SNP} heritability.
	
	It might be worthwhile to repeat the analysis in \citet{Consortium2015a} using \gls{ldsc}.
	The problem of the partitioning of heritability analysis is that the enrichment can easily be driven by a small number of genes with large effect size. 
	Therefore, further developments might be required in order to prevent the spurious enrichment. 
	
	\section{Schizophrenia: Future Perspectives}
	We are now entering a new era of sequencing, where a wide variety of tools are at our disposal to identify different genetic variations associating with a disease. 
	It is foresee that the ability to investigate the whole genome/exome at a per base resolution will allow for the identification of more genetic variations such as the rare mutations, that are associated with \glng{scz}.
	
	With the sophistication of technologies, whole genome sequencing can now be performed with the HiSeq \RN{10} Ten system, costing less than \$1,000, allowing for more data to be generated. 
	While the technologiy advancement have allowed a massive increase in available sequencing data, the bioinformatic analysis are now becoming the bottleneck of genetic research.
	For example, the alignment of sequence read to low complexity sequence or low-degeneracy repeats remains challenging and error prone, having a negative impact to the quality of the results \citep{Sims2014}. 
	New sequencing technology such as Oxford Nanopore, which can provide extra long-reads, might help to make alignment easier by providing extra information for each individual reads.
	However, the Oxford Nanopore is still under development and has a relatively high error rate \citep{Mikheyev2014}. 
	Only until the error rate is dramatically decreased can the use of Oxford Nanopore system become feasible. 
		
	Even if the reads can be perfectly aligned to the genome, the functional annotation of variants remains challenging.
	For complex disease such as \glng{scz}, a large amount of susceptibility loci can be observed throughout the genome, yet estimation of the functional impact can only be performed on variants located within the exomic regions.
	The development of ENCODE project \citep{ENCODEProjectConsortium2012} and Genotype-Tissue Expression (GTEx) project \citep{Consortium2015} have helped to provide reference point for the annotation of genetic variations in the intergenic regions.
	However, the functional impact of many genetic variation in the genome remains unknown. 
	Only through the tireless effort of the molecular biologist can sufficient information be gained to allow for the detail annotation of the data.

	Moreover, epigenetic studies in \glng{scz} \citep{Wockner2014,Nishioka2012} have identified genes with differential DNA methylation patterns associated with \glng{scz}, whereas etiology studies have reported the possibility of interaction between prenatal infection and genetic variation in risk of developing \glng{scz} \citep{Tienari2004,Clarke2009}, suggesting the possible involvement of epigenetics and $G\times E$ interaction in the etiology of \glng{scz}.
	Therefore, it is important to combine information of multiple dimensions -- including genetic, \gls{cnv}, epigenetic changes, genetic and environmental interaction, expression, structural properties and spatial organization of chromosomes --  can we understand the etiology of \glng{scz}.
		% GxE 
		% Prenatal infection -> only rodant information
		% Need planning (birth cohort)
		
		% Future prospective
		% More functional studies
		% Increase sample size
		% Epigenetic studies

	
%	\begin{figure}
%		\centering
%		\includegraphics[width=0.7\textwidth]{figure/maf_effectSize.png}
%		\caption[Relationship between Effect Size and Allele Frequency]{
%			Relationship between effect size and allele frequency. 
%			It is expected that rare variants with large effect size were actively selected against in the population and therefore should be rare.
%		}
%		\label{fig:effectSize}
%	\end{figure}
	
%	Currently, most of the focus in \glng{scz} was directed to genetic variation yet it is possible that the heritability of \glng{scz} is also transmitted in the form of epigenetic changes such as methylation.
%	It was observed that the risk for individual born from a schizophrenic mother is larger than that from a schizophrenic father. 
%	This suggests that maternal specific elements, such as maternal imprinting and mitochondria might account for part of the risk of \glng{scz}. 
%	Epigenetic studies in \glng{scz} \citep{Wockner2014,Nishioka2012} has identified genes with differential DNA methylation patterns associated with \glng{scz}, suggesting the importance of epigenetics in the etiology of \glng{scz}.
%	
%	As a highly heritable disorder, most of the research of \glng{scz} has been focusing on the genetic factors. 
%	Although the genetic variation accounted for majority of the variations in \glng{scz}, the environmental factors, especially prenatal infection is also an important factor to consider. 
%	It was estimated that prenatl infection accounts for roughly 33\% of all \glng{scz} cases \citep{Brown2010}.
%	The \gls{mia} rodent model has provide vital information on the possible interaction between the immune and neuronal system in the etiology of \glng{scz} \citep{Meyer2007a}.
%	For example, \gls{il6}, a pro-inflammatory cytokine has been found to be an important mediator in generating the schizophrenia-like behaviour in rodent model \citep{Smith2007}.
%	More importantly, there are evidence of the interaction between prenatal infection and genetic variation, supporting a mechanism of gene-environment interaction in the causation of \glng{scz} \citep{Clarke2009}.
%	As the \gls{SNP}-heritability estimation does not take into account of the gene environmental interactions, it is possible that the ``missing'' heritability can be due to gene-environmental interactions. 
%	Efforts is now made by the European network of national \glng{scz} networks studying Gene-Environmental Interaction (EUGEI) to identify possible genetic and environmental interaction that contributes to the disease etiology of \glng{scz}.
%	
%	
%	
%	% This is only the beginning, not the end
%	In conclusion, we have only catch a glimpse of the etiology of \glng{scz} and there are still a lot of questions left unanswered.
%	It is expected that only by combining the study of epigenetic, genomic variation, gene expressions, and gene environmental interaction can provide a deeper understanding of the complex disease mechanism of \glng{scz} be obtained.
%	
	
	
	\backmatter
	\printbibliography[heading=bibintoc,title={Bibliography}]

	%% Table generated by Excel2LaTeX from sheet 'Sheet1'
\begin{table}[htbp]
  \centering
  \caption{Primer Sequences used in real time PCR}
    \begin{tabular}{rr}
    \toprule
    Gene Name & Primer Sequence \\
    \midrule
    \textit{Actb}  & ACTGAGCTGCGTTTTACACCCTTTC \\
    \textit{Akt3}  & CTTCTCAGTGGCAAAATGTCAGTTA \\
    \textit{Eomes} & AATAACATGCAGGGCAATAAGATGT \\
    \textit{Lama5} & ACACGAGCGAGACCAGTGAGAAGAT \\
    \textit{Robo3} & AAGGGAGTCAAGTCCTGCTTTTCCC \\
    \bottomrule
    \end{tabular}%
  \label{suppleTab:rtPCR}%
\end{table}%

\end{document}


