	\begin{figure}
			\centering
			\subfloat[SHREK]{
				\scalebox{.4}{\includegraphics{figure/he_summary/cc_50c/shrek_CC_Rand_mean.png}}
				\label{fig:shrekCC50RandMean}
			}
			\subfloat[GCTA]{
				\scalebox{.4}{\includegraphics{figure/he_summary/cc_50c/gcta_CC_Rand_mean.png}}
				\label{fig:gctaCC50RandMean}
			}\\
			\subfloat[LDSC with fix intercept]{
				\scalebox{.4}{\includegraphics{figure/he_summary/cc_50c/ldsc_CC_Rand_mean.png}}
				\label{fig:ldscCC50RandMean}
			}
			\subfloat[LDSC with intercept estimation]{
				
				\scalebox{.4}{\includegraphics{figure/he_summary/cc_50c/ldscIn_CC_Rand_mean.png}}
				\label{fig:ldscInCC50RandMean}
			}
			\caption[Mean of Case Control Simulation Results (50 Causal)]
			{Mean of results from case control simulation with random effect size simulation with 50 causal \glspl{SNP}.
				In general, the results were similar to the scenario with 10 causal \glspl{SNP} with the only exception that the estimates from \gls{ldsc} with intercept estimates seems to be less affected by the change in prevalence of the trait.
				} 
			\label{fig:CC50RandMean}
		\end{figure}
		
		\begin{figure}
			\centering
			\subfloat[SHREK]{
				\scalebox{.4}{\includegraphics{figure/he_summary/cc_50c/shrek_CC_Rand_sd.png}}
				\label{fig:shrekCC50RandVar}
			}
			\subfloat[GCTA]{
				\scalebox{.4}{\includegraphics{figure/he_summary/cc_50c/gcta_CC_Rand_sd.png}}
				\label{fig:gctaCC50RandVar}
			}\\
			\subfloat[LDSC with fix intercept]{
				\scalebox{.4}{\includegraphics{figure/he_summary/cc_50c/ldsc_CC_Rand_sd.png}}
				\label{fig:ldscCC50RandVar}
			}
			\subfloat[LDSC with intercept estimation]{
				
				\scalebox{.4}{\includegraphics{figure/he_summary/cc_50c/ldscIn_CC_Rand_sd.png}}
				\label{fig:ldscInCC50RandVar}
			}
			\caption[Variance of Case Control Simulation Results (50 Causal)]
			{Variance of results from case control simulation with random effect size simulation with 50 causal \glspl{SNP}.
				For most algorithm except that of \gls{ldsc} with fixed intercept, the empirical variance of the estimates increases as the population prevalence of the trait increases, with the estimations from \gls{ldsc} with intercept estimation display the largest variance.
			} 
			\label{fig:CC50RandVar}
		\end{figure}
		
		
		\begin{figure}
			\centering
			\subfloat[SHREK]{
				\scalebox{.4}{\includegraphics{figure/he_summary/cc_50c/shrek_CC_Rand_sdCom.png}}
				\label{fig:shrekCC50RandVarCom}
			}
			\subfloat[GCTA]{
				\scalebox{.4}{\includegraphics{figure/he_summary/cc_50c/gcta_CC_Rand_sdCom.png}}
				\label{fig:gctaCC50RandVarCom}
			}\\
			\subfloat[LDSC with fix intercept]{
				\scalebox{.4}{\includegraphics{figure/he_summary/cc_50c/ldsc_CC_Rand_sdCom.png}}
				\label{fig:ldscCC50RandVarCom}
			}
			\subfloat[LDSC with intercept estimation]{
				
				\scalebox{.4}{\includegraphics{figure/he_summary/cc_50c/ldscIn_CC_Rand_sdCom.png}}
				\label{fig:ldscInCC50RandVarCom}
			}
			\caption[Estimation of Variance in Case Control Simulation (50 Causal)]
			{Estimated variance of results from case control simulation with random effect size simulation when compared to empirical variance when 50 causal \glspl{SNP} was simulated.
				Again, the estimation of variance from \gls{shrek} tends to be downwardly biased and \gls{ldsc} with fixed intercept tends to be upwardly biased. 
				However, when intercept estimation was performed, the estimation of variance of \gls{ldsc} improved.
			} 
			\label{fig:CC50RandVarCom}
		\end{figure}